\documentclass{article}
\usepackage{amsfonts}
\usepackage{amsthm}
\usepackage{amssymb}
\usepackage{amsmath}
\usepackage{graphicx}
\usepackage{subcaption}
\usepackage{xcolor}

\newcommand{\new}[1]{
    \vspace{2mm}
    \noindent
    \textbf{
    \underline{#1}}
}

\def\calO{{\mathcal{O}}}
\def\th{{\theta}}
\def\_{{\hspace{1mm}}}
\def\<{{\langle}}
\def\>{{\rangle}}


\newcounter{problemcnt}
\setcounter{problemcnt}{0}

\newcommand{\Problem}{{
    \vspace{5mm}
    \stepcounter{problemcnt}
    \noindent
    \arabic{problemcnt}. 
}
}

\newcommand{\nProblem}[1]{
    \vspace{5mm}
    \noindent
    \setcounter{problemcnt}{#1}
    \arabic{problemcnt}. 
}


\newcommand{\Proof}{{
    \vspace{2mm}
    \noindent
    \textbf{
    \underline{Proof}}
}
}

\newcommand{\textOr}{
    {
        \hspace{5mm}
        \textrm{or}
        \hspace{5mm}
    }
}

\newcommand{\textAnd}{
    {
        \hspace{5mm}
        \textrm{and}
        \hspace{5mm}
    }
}

\newcommand{\Ixp}[1]{
    {
        e^{i{#1}}
    }
}



\newcommand{\halfFigure}[1]{
\begin{center}
\includegraphics[width = .5\linewidth]{{#1}}
\end{center}
}

\newcommand{\fullFigure}[1]{
\begin{center}
\includegraphics[width = .9\linewidth]{{#1}}
\end{center}
}

\def\twobytwoMat(#1, #2, #3, #4){
    {
        \begin{bmatrix}
            {#1} & {#2}\\
            {#3} & {#4}
        \end{bmatrix}
    }
}

\def\twobyoneMat(#1, #2){
    {
        \begin{bmatrix}
            {#1}\\
            {#2}
        \end{bmatrix}
    }
}

\def\twobytwoDet(#1, #2, #3, #4){
    {
        \begin{vmatrix}
            {#1} & {#2}\\
            {#3} & {#4}
        \end{vmatrix}
    }
}



\begin{document}
\begin{center}
\LARGE
PHYS 202 HW5

\Large
Daniel Son
\end{center}

\normalsize 

\new{Question 3} Closed-open ended oscillators 
Consider $N$ masses attatched by a string of constant $k$, 
each separated by a distance $a$ apart. One side of the 
string is attatched to a wall, and the other end is free. 
What are the normal modes of this system?

\noindent
a) What are the boundary conditions?

Add two imaginary masses at the wall and at the end of mass $N$. 
Label their displacements to be $x_0, x_{N+1}$. The boundary conditions 
are as follows.

\[
    \boxed{
    x_0(t) = 0 \textAnd x_{N+1}(t) = x_{N}
    }
\]

\noindent
b) Find the normal mode frequencies. 

We know that the solutions are in the form of 
\[
    x_n(t) = 
    \tilde{B} {
        \Ixp{(\omega t + k a n )}
    }
    +
    \tilde{C} {
        \Ixp{(\omega t - k a n)}
    }
\]

Enforce $x_0(t) = 0$, which converts to 
\[
    x_0(t) = \Ixp{
        \omega t
    }
    (\tilde{B} + \tilde{C}) = 0 
    \textAnd \tilde{C} = -\tilde{B}
\]. $x_N(t) - x_{N-1}(t) = 0$ for any $t$. This converts to 
\[
    x_N(t) - x_{N-1}(t) = 
    \tilde{B}\Ixp{(\omega t + kaN)}(
        \Ixp{ka} - 1
    )
    +
    \tilde{C}\Ixp{(\omega t - kaN)}(
        \Ixp{-ka} + 1
    )
\]
\[
    = \tilde{B} \left(
        2i\Ixp{(\omega t + kaN + ka/2)} \sin(ka/2) 
        - 2i\Ixp{(\omega t - kaN - ka/2)} \sin(-ka/2) 
    \right)
\]
\[
    = 2i\tilde{B} \sin(ka/2) 
    \left(
        \Ixp{(\omega t + kaN + ka/2)}
        + \Ixp{(\omega t - kaN - ka/2)}
    \right)
\]
\[
    = 4i \tilde{B} \sin(ka/2) \cos(kaN + ka/2) \Ixp{\omega t} = 0
\]

If the summand inside the cosine function is in the form of 
$\frac{2m - 1}{2} \pi$, the entire function vanishes, and the 
boundary condition is satisfied. Note that $m = {1, 2, 3, ...}$.  Thus, 

\[
    \frac{2m - 1}{2} \pi = 
    \frac{ka(2N + 1)}{2}
\]

Depending on the value of $m$, the value of $k$ varies. Relabel 
$k$ as $k_m$. 
\[
    k_m = \frac{2m - 1}{(2N + 1)a}\pi
\]

We have previously derived the frequency dependant on $k_m$. 

\[
    \boxed{
    \omega = 2\omega_0 \sin(k_m a/2) = 2\omega_0 \sin \left[
        \frac{m - 1/2}{N + 1/2} \frac{\pi}{2}
    \right]
    }
\]


\newpage

\new{Question 4} Driven system of N masses

A system of N masses is connected with strings. One end of the system 
is driven with a displacement of $A\cos(\omega_d t)$. Each of the masses 
are distance $a$ apart. 

\noindent
a) Imagine two imaginary masses at the left of mass zero and right of mass 
n. What boundary conditions can be enforced on these two masses?

Mass zero can be the driver, and mass $N+1$ can be a phantom mass 
that trails the movement of the $N$ th mass. Quantituatively, 

\[
    x_0(t) = A\cos(\omega_d t)
    \textAnd 
    x_{N+1}(t) = x_{N}(t)
\]

\noindent 
b) Find the displacements of each of the masses. 

Complexifying the displacements, we know that the 
solutions must be in the form of 

\[
    \tilde{x}_n(t) = 
    \tilde{B}\Ixp{(\omega t + k z_n)} +
    \tilde{C}\Ixp{(\omega t - k z_n)}
\]. 
We define $z_n$ to be zero on the right $a/2$ position of 
the Nth mass. Algebraically, $z_n$ can be expressed as 
\[
    z_n = \frac{2n - 2N - 1}{2}
\]

Enforce the boundary condition on the $N+1$th mass. 

\[
    \tilde{x}_{N + 1}(t) - \tilde{x}_{N}(t) = 0 
    \textOr 
     \tilde{B}\Ixp{(\omega t + k z_{N + 1})} +
    \tilde{C}\Ixp{(\omega t - k z_{N + 1})}
    -
 \tilde{B}\Ixp{(\omega t + k z_N )} -
    \tilde{C}\Ixp{(\omega t - k z_N)}
    = 0
\]
Divide by $\Ixp{\omega t}$ both sides and collect 
$\tilde{B}, \tilde{C}$
\[
    \tilde{B} (\Ixp{kz_{N + 1}} - \Ixp{kz_{N}})
    + \tilde{C}(
        \Ixp{(-kz_{N + 1})}
    - \Ixp{(-kz_N)}
    )
    = 0
\]
By our placement of the origin, $z_N = -ka/2, z_{N+1} = ka/2$. 
Apply Euler's formula to obtain 

\[
   \tilde{B}\sin(ka/2) + \tilde{C}\sin(-ka/2) = 0 
   \textOr 
   (\tilde{B} - \tilde{C}) \sin(ka/2) = 0
\]

If $\sin(ka/2) = 0$, then all the masses will be stationary. Thus, 
we conclude $\tilde{B} = \tilde{C}$. Returning to our expression of 
the displcaement, we collect $\tilde{B}\Ixp{\omega t}$ to write 
\[
    \boxed{
    \tilde{x}_n(t) = \tilde{B}\Ixp{\omega t} (\Ixp{kz_n} + \Ixp{(-kz_n)})
    = \tilde{D} \Ixp{\omega t}\cos(kz_n)
    }
\]

c) Show that the amplitude diverges if the oscillator 
is driven at a normal mode frequency. 

Now, we know that $\tilde{x_0} = A\Ixp{\omega t}$. Plugging $n = 0$
to our general solution, 

\[
    \tilde{x}_0(t) = \tilde{D}\Ixp{\omega t} \cos(kz_0)
    = \tilde{D}\Ixp{\omega t} \cos(k\frac{- 2N - 1}{2}a) = 
    A\Ixp{\omega t}
\]

The complex amplitude $\tilde{D}$ can be expressed in terms of $A$. 

\[
    \tilde{D} = A / \cos(ka \frac{2N + 1}{2})
\]

The wavenumber $k$ is dependant on the drive frequency by the dispersion 
relation. 

\[
    \omega_d = 2\omega_0 \sin(ka / 2)
    \textOr 
    k = 
    \frac{2}{a}
    \sin^{-1}\left(
        \frac{\omega_d}{2\omega_0}
    \right)
\]

Thus, 
\[
    \tilde{D} = A / \cos\left(
        (2N + 1)\sin^{-1}\left(\frac{\omega_d}{2\omega_0}
        \right)
    \right)
\]

From Q3, we know the ratio $\omega_d / 2\omega_0$ 
for the normal modes. 

\[
    \frac{\omega_d}{2\omega_0} = \sin \left[
        \frac{2m - 1}{2N + 1}
        \frac{\pi}{2}
    \right]
    \textAnd 
    \sin^{-1} \left(
        \frac{\omega_d}{2\omega_0}
    \right)
    = 
    \frac{2m - 1} {2N + 1}\frac{\pi}{2}
\]

At the normal mode frequency, our amplitude simplifies to 
\[
    \tilde{D} = A / \cos\left(
        (2N + 1)
\frac{2m - 1} {2N + 1}\frac{\pi}{2}
    \right)
    = A/\cos\left(
        (2m - 1)\frac{\pi}{2}
    \right) = A/0 = \pm\infty
\]
And the amplitude diverges. \hfill \qed

d) Write out a closed form expression for $x_1(t)$ where there is 
only one mass. 

We define an angle for convinience. 

\[
   \theta := ka/2 
\]

Write the dispersion relation and our amplitude $\tilde{D}$ 
in terms of $\theta$.
\[
    \frac{\omega_d}{2\omega_0} = \sin(\theta)
    \textAnd 
    \tilde{D} = A/\cos((2N+1)\theta) = A/\cos(3\theta)
\]
The displacement $x_1$ is 
\[
    x_1(t) = \tilde{D}\Ixp{\omega t} \cos(k z_1)
    = \tilde{D}\Ixp{\omega t} \cos (\theta (2n - 2N - 1))
    = \tilde{D}\Ixp{\omega t} \cos(\theta)
\]
Plugging in our expression for $\tilde{D}$, 
\[
    x_1(t) = A \frac{\cos(\theta)}{\cos(3\theta)}\Ixp{\omega t}
\]

We pull out an interesting trig identity
\[
    \frac{\cos(\theta)}{\cos(3\theta)}
    = \frac{1}{1- 4sin^2(\theta)}
\]
The derivation can be completed by using Euler's formula. 
We know $\sin(\theta) = \omega_d / 2\omega_0$. Thus 
\[
    \frac{1}{1-4\sin^2(\theta)} = \frac{1}{1-\omega^2_d/\omega^2_0}
    = \frac{\omega^2_0}{\omega^2_0 - \omega^2_d}
\]
Finally, 
\[
    \boxed{
    x_1(t) = 
\frac{\omega^2_0}{\omega^2_0 - \omega^2_d} A \Ixp{\omega t}
    }
\]

\new{Q5} Rod of atoms 


a) Consider an oscillator comprised of n masses where both ends are 
free. Compute the lowest normal mode frequency, and plot it using mathematica. 

We reuse our results from Q4. Use the same coordinate system to 
define the equilibrium positon $z_n$. Enforcing the right boundary 
condition, we have 
\[
    \tilde{x}_n(t) = \tilde{D} \Ixp{\omega t} \cos (kz_n)
\]
The left boundary condition would be that the imaginary mass 0 
shadows the motion of the real mass 1. In symbols, $x_0(t) = x_1(t)$. 
With some messy lines of algebra, this translates to
\[
    \cos(kaN + ka/2) = \cos(kan - ka/2)
\]
Or substituting $ka/2 \mapsto \theta$, we can write 
\[
    \cos(kaN + \theta) = \cos(kaN - \theta)
\]
We reason that $\cos(kaN - x)$ must be an even function. Hence, 
$kaN = m\pi$ for any integer $m$. Thus 
\[
    k = \frac{m\pi} {aN}
\]
And now, plug this into the dispersion relation. 
\[
    \frac{\omega_m}{\omega_0} = 2\sin\left(
        \frac{ka}{2}
    \right)
    = 2\sin \left(
        \frac{m\pi}{2N}
    \right)
\]
We are interested in the lowest frequency, $m = 1$. 
\[
     \frac{\omega_1}{\omega_0} 
    = 2\sin \left(
        \frac{\pi}{2N}
    \right)
\]
Plotting this into mathematica, we get the following result. 
\halfFigure{lowestFreq.png}

b) Approximate the frequency as $N \rightarrow \infty$

The sin function can be approximated as the identity around the origin. 

\[
    \boxed{
     \frac{\omega_1}{\omega_0} 
    = 2\sin \left(
        \frac{\pi}{2N}
    \right)
    \approxeq 
    2\frac{\pi}{2N}
    = 
    \frac{\pi}{N}
    }
\]

c) Plot the relative displacements of the masses for 
$N = 3, 6, 20$. 

\fullFigure{Q5_1.png}
\halfFigure{Q5_2.png}

d) Plot the three lowest normal mode frequencies for 
$N = 3, 6, 20$. Do they agree with Morin 2-84?

\fullFigure{Morin2-84.png}
So morin claims that the normal mode frequencies 
have a linear relationship with the wavenumber $k$. 
We can confirm that this relationship holds for large 
$N$. 
\fullFigure{Q5_3}

The green dots indicate that for $N = 20$, the relationship 
between the wavenumber and the frequency is linear. 


\end{document}