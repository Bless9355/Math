\documentclass{article}
\usepackage{amsfonts}
\usepackage{amsthm}
\usepackage{amssymb}
\usepackage{amsmath}
\usepackage{graphicx}
\usepackage{subcaption}
\usepackage{xcolor}

\newcommand{\new}[1]{
    \vspace{2mm}
    \noindent
    \textbf{
    \underline{#1}}
}

\def\calO{{\mathcal{O}}}
\def\th{{\theta}}
\def\_{{\hspace{1mm}}}
\def\<{{\langle}}
\def\>{{\rangle}}


\newcounter{problemcnt}
\setcounter{problemcnt}{0}

\newcommand{\Problem}{{
    \vspace{5mm}
    \stepcounter{problemcnt}
    \noindent
    \arabic{problemcnt}. 
}
}

\newcommand{\nProblem}[1]{
    \vspace{5mm}
    \noindent
    \setcounter{problemcnt}{#1}
    \arabic{problemcnt}. 
}


\newcommand{\Proof}{{
    \vspace{2mm}
    \noindent
    \textbf{
    \underline{Proof}}
}
}

\newcommand{\textOr}{
    {
        \hspace{5mm}
        \textrm{or}
        \hspace{5mm}
    }
}

\newcommand{\textAnd}{
    {
        \hspace{5mm}
        \textrm{and}
        \hspace{5mm}
    }
}


\newcommand{\Ixp}{
    {
        \textrm{Ixp}
    }
}



\newcommand{\halfFigure}[1]{
\begin{center}
\includegraphics[width = .5\linewidth]{{#1}}
\end{center}
}

\newcommand{\fullFigure}[2]{
\begin{center}
\includegraphics[width = .9\linewidth]{{#1}}
\end{center}
}

\def\twobytwoMat(#1, #2, #3, #4){
    {
        \begin{bmatrix}
            {#1} & {#2}\\
            {#3} & {#4}
        \end{bmatrix}
    }
}

\def\twobyoneMat(#1, #2){
    {
        \begin{bmatrix}
            {#1}\\
            {#2}
        \end{bmatrix}
    }
}

\def\twobytwoDet(#1, #2, #3, #4){
    {
        \begin{vmatrix}
            {#1} & {#2}\\
            {#3} & {#4}
        \end{vmatrix}
    }
}



\begin{document}
\begin{center}
\LARGE
PHYS 202 Formula Sheet

\Large
Daniel Son
\end{center}

\new{Boundary conditions for N masses}
The solutions for the n mass oscillators are in the form of:

\[
    \psi(m, t) = A \Ixp(\omega t + k_ma) + B \Ixp(\omega t - k_ma)
\]

Plugging into the 2nd order DE, we derive an expression for the angular 
frequency.

\[
    \omega_m = 2\omega_0\sin(k_m \frac{a}{2})
\]
$k_m$ is the mth wave number. Recall that $k := \frac{2\pi}{\lambda}$.
$\lambda_n$ can be computed by drawing diagrams.

\halfFigure{Closed_End_Fig.png}

With some algebraic hassle, it is possible to derive the expressions
for $k_m $. 

For closed-closed and open-open ends:
\[
    k_m a = \frac{m\pi}{n+1}
    \textOr 
    k_m = \frac{m\pi}{a(n + 1)} = \frac{m\pi}{L}
\]
Where $m \in \mathbb{Z}^+$

For open-open ends:
\[
    k_m a = \frac{m\pi}{(2n+1)/2}
    \textOr 
    k_m = \frac{m\pi}{L} 
\]
Where $m \in \mathbb{Z}^+$








\new{Transverse waves}
From the geometry of the springs, we use approximation. 
Let theta be the angle between the horizontal axis and the string. 
$\tan(\theta) \approxeq \theta = \Delta y/a$. Consequently, we 
arrive at:
\[
    k \mapsto T/a
\]

\normalsize 

\end{document}