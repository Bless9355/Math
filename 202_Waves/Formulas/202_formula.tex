\documentclass{article}
\usepackage{amsfonts}
\usepackage{amsthm}
\usepackage{amssymb}
\usepackage{amsmath}
\usepackage{graphicx}
\usepackage{subcaption}
\usepackage{xcolor}

\newcommand{\new}[1]{
    \vspace{2mm}
    \noindent
    \textbf{
    \underline{#1}}
}

\def\calO{{\mathcal{O}}}
\def\th{{\theta}}
\def\_{{\hspace{1mm}}}
\def\<{{\langle}}
\def\>{{\rangle}}


\newcounter{problemcnt}
\setcounter{problemcnt}{0}

\newcommand{\Problem}{{
    \vspace{5mm}
    \stepcounter{problemcnt}
    \noindent
    \arabic{problemcnt}. 
}
}

\newcommand{\nProblem}[1]{
    \vspace{5mm}
    \noindent
    \setcounter{problemcnt}{#1}
    \arabic{problemcnt}. 
}


\newcommand{\Proof}{{
    \vspace{2mm}
    \noindent
    \textbf{
    \underline{Proof}}
}
}

\newcommand{\textOr}{
    {
        \hspace{5mm}
        \textrm{or}
        \hspace{5mm}
    }
}

\newcommand{\textAnd}{
    {
        \hspace{5mm}
        \textrm{and}
        \hspace{5mm}
    }
}

\newcommand{\textThen}{
    {
        \hspace{5mm}
        \textrm{then}
        \hspace{5mm}
    }
}


\newcommand{\Ixp}[1]{
    {
        e^{#1}
    }
}

\newcommand{\partialderiv}[2]{
    {
        \frac{\partial #1}{\partial #2}
    }
}



\newcommand{\halfFigure}[1]{
\begin{center}
\includegraphics[width = .5\linewidth]{{#1}}
\end{center}
}

\newcommand{\fullFigure}[2]{
\begin{center}
\includegraphics[width = .9\linewidth]{{#1}}
\end{center}
}

\def\twobytwoMat(#1, #2, #3, #4){
    {
        \begin{bmatrix}
            {#1} & {#2}\\
            {#3} & {#4}
        \end{bmatrix}
    }
}

\def\twobyoneMat(#1, #2){
    {
        \begin{bmatrix}
            {#1}\\
            {#2}
        \end{bmatrix}
    }
}

\def\twobytwoDet(#1, #2, #3, #4){
    {
        \begin{vmatrix}
            {#1} & {#2}\\
            {#3} & {#4}
        \end{vmatrix}
    }
}



\begin{document}
\begin{center}
\LARGE
PHYS 202 Formula Sheet

\Large
Daniel Son
\end{center}

\new{Simple Harmonic Oscillators}
Consider a mass attatched to a spring. Let 
$x(t)$ be the function of displacement of the mass 
from the equilibrium position. Suppose that the 
spring constant is $k$. By Newton's 2nd Law,

\[
    m\ddot{x} = -kx
    \textOr 
    \ddot{x} = -\frac k m x
\]

Where $\dot{x} := \frac{d}{dt}x$

The following function solves the equation. 

\[
    x(t) = Re(\tilde{A}\Ixp{i\omega_0 t})
    = A\cos(\omega_0 t + \phi)
\]

$A, \omega_0$ is referred as the amplitude and 
the natural frequency of the oscillator. Also, 

\[
    A = \sqrt\frac{2E}k 
    \textAnd
    \omega_0 = \sqrt\frac k m
\]

\new{Phase Conventions}
Velocity leads Displacement by a phase of $\pi/2$. 
Acceleration leads Velocity by a phase of $\pi/2$

Say $V \sim \cos(\omega t)$ and $I \sim \cos(\omega t + \phi)$
for some positive phase $\phi \leq \pi$. The current leads 
the voltage and the voltage trails the current. 

\new{Simple RLC}
Consider a circuit where R, L, C is connected in 
parallel. Let the current be $I$. By the loop rule, 
\[
    -\frac q C - R I - L \dot{I} = 0
    \textOr
    \frac q C + R \dot{q} + L \ddot{q} = 0 
\]
Rewrite this in the follwoing form. 
\[
    \ddot{q} + \frac R L \dot{q} + \frac q {LC} = 0
\]
Compare this with the equation for damped driven oscillators.  
\[
    \ddot{x} + \frac b m \dot{x} + \frac k m x = 0
\]
So the following isomorphism holds
\[
    x \mapsto q \textThen
    (m, b, k) \mapsto (L, R, 1/C)
\]

In a parallel circuit, the current through each 
circuit component is identical. By the complexified 
Ohms Law $\tilde V = Z \tilde I$. Recall the 
impedences. 

\[
    Z_R = R\textAnd Z_C = \frac{1}{i\omega C}\textAnd Z_L = i\omega L
\]

Thus, the voltage of the resistor leads the voltage on 
the capacitor by phase $\pi/2$. Likewise, the voltage 
on the resistor trails the voltage on the Inductor by $\pi/2$.

\new{Circuit Filters and Loglog plot}
Using inductors and capacitors, it is possible to filter 
out signals of high or low frequency. Consider the behavior 
of circuit elemnts in high and low frequency. 

\[
    \lim_{\omega \rightarrow 0} Z_L = 
    \lim_{\omega \rightarrow 0} i\omega L = 0
    \textAnd
    \lim_{\omega \rightarrow \infty} Z_L = 
    \lim_{\omega \rightarrow \infty} i\omega L = \infty
\]
\[
    \lim_{\omega \rightarrow 0} Z_C = 
    \lim_{\omega \rightarrow 0} \frac 1{i\omega C} = \infty
    \textAnd
    \lim_{\omega \rightarrow \infty} Z_C = 
    \lim_{\omega \rightarrow \infty} \frac 1{i\omega C} = 0
\]

The inductor will block signals of high frequency 
but allow the passage of signals of low frequency. 
Hence, inductors act as a \textbf{low-pass} filter. On the 
other hand, capacitors block signals of low frequency 
and allow the passage of high frequency signals. 
Hence, capacitors act as a \textbf{high-pass} filter. 


\new{Corner frequency vs Resonant frequency} 
For an RLC circuit, it is possible to tune the frequency 
such that the combined impedence reach zero. This 
frequency is called the resonant frequency. 
For a circuit involving one R, L, C, the resonant frequency 
is computed by the following formula. 

\[
    \omega_r = \frac 1 {\sqrt{LC}}
\]

For and RL or an RC circuit, the imaginary part of the 
impedence cannot reach zero. It is possible to defnine 
a corner frequency where the circuit behavior changes 
drastically. It is the freqnecy where $Im(Z_L) = R$ 
or $Im(Z_C) = R$. In other words, 

\[
    \omega_c = {\frac 1 {RC}}
    \textOr 
    {\frac R L}
\]

\new{Dimensions of C and L}
\[
    [L] = H = \Omega \cdot s 
    \textAnd 
    [C] = \frac s \Omega
\]

\new{Power Resonance Curve}
For an RLC circuit, it is possible to plot 
the relationship between the angular frequency 
of the imput voltave and the power lost 
through the resistor. This curve is 
called the Power Resonance curve. 
\fullFigure{PowerResonance.png}

$P \sim V^2$. 
The Full-Width-Half-Maximum refers to the 
width of the curve at the half maximum point. 
The FWHM of the Power Resonance Curve is 
exactly $\gamma = R/L$, the damping factor. 
Be careful of the factor of $2\pi$ when 
the x-axis is set as frequency (Hz). 

Also, use the dimension $[\omega] = rad/s$
and $[f] = Hz$. The frequency refers to cycles 
per second. 

\new{Q-factor} 
For damped driven oscillators, the Q-factor is an important 
number. 

\[
    Q := \frac{\omega_0}\gamma = \frac {A_{res}} {A_0}
\]


\new{Boundary conditions for N masses}
The solutions for the n mass oscillators are in the form of:

\[
    \psi(m, t) = A \Ixp{(\omega t + k_ma)} + B \Ixp{(\omega t - k_ma)}
\]

Plugging into the 2nd order DE, we derive an expression for the angular 
frequency.

\[
    \omega_m = 2\omega_0\sin(k_m \frac{a}{2})
\]
$k_m$ is the mth wave number. Recall that $k := \frac{2\pi}{\lambda}$.
$\lambda_n$ can be computed by drawing diagrams.

\halfFigure{Closed_End_Fig.png}

With some algebraic hassle, it is possible to derive the expressions
for $k_m $. 

For closed-closed and open-open ends:
\[
    k_m a = \frac{m\pi}{n+1}
    \textOr 
    k_m = \frac{m\pi}{a(n + 1)} = \frac{m\pi}{L}
\]
Where $m \in \mathbb{Z}^+$

For open-open ends:
\[
    k_m a = \frac{m\pi}{(2n+1)/2}
    \textOr 
    k_m = \frac{m\pi}{L} 
\]
Where $m \in \mathbb{Z}^+$




\new{Free end Distance relation}
Always set the origin to be distance $a/2$ apart from the free end. 
This is because the two masses at the boundary must 
be symmetric. 
Refer to pset 5-1. 



\new{Transverse waves}
From the geometry of the springs, we use approximation. 
Let theta be the angle between the horizontal axis and the string. 
$\tan(\theta) \approxeq \theta = \Delta y/a$. Consequently, we 
arrive at:
\[
    k \mapsto T/a
\]

\new{Dispersion Relation}
The differential equation of the masses in the center 
oscillators provide a closed form equation for 
the angular frequency in terms of the wave number $k$. 
\[
    \omega(k) = 2\omega_0 \sin(k a/2)
\]
where a is the distance between the masses. 

\new{Dispersion Relation 2}
The dispersion relation defines the wave number $k$. 
\[
    \partialderiv{^2 \psi}{t^2} = 
    c^2
    \partialderiv{^2 \psi}{x^2}
    \textThen
    \omega^2 = c^2 k^2
\]
And $c$ is called the phase velocity of the wave. 
The wave number depends on the frequency of the wave, which is 
not necessarily the normal mode frequency

\new{Nodes and Antinodes}
In a standing wave, nodes are the points that do not move. 
Max amplitude is achieved at antinodes. 

\new{Waves traveling in different medium}
Consider a transverse wave moving from one string 
of linear mass density $\mu_1$ to another string with linear 
mass density $\mu_2$.
The velocity of the waves on each string is entirely determined 
by the lmd. 

\[
    (v_1, v_2) = \left(
        \frac{T}{\mu_1}
        ,
        \frac{T}{\mu_2}
    \right)
\]

Let the incident wave to be in the form of 
\[
    \psi_i(x, t) := f_i(t - x/v_1) 
\]

We make a natural assumption that the combined 
wavefunction must be smooth with respect to position. 
Let $f_t, f_r$ be the simplified translated wavefunction 
and the simplified reflected wavefunciton. 
We derive 
\[
    \frac{f_r}{f_i} = \frac{v_2 - v_1}{
        v_1 + v_2
    }
    \textAnd 
    \frac{f_t}{f_i} = \frac{2v_2}{v_1 + v_2}
\]

\new{Natural conditions on the string displacements}
We assume that the wavefunction is continuous and 
differentiable at all positions. Also, 
\[
    \dot\psi(x, 0) = 0
\]
allows us to easily complexify the solution. 


\normalsize 



\end{document}