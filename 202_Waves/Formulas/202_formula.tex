\documentclass{article}
\usepackage{amsfonts}
\usepackage{amsthm}
\usepackage{amssymb}
\usepackage{amsmath}
\usepackage{graphicx}
\usepackage{subcaption}
\usepackage{xcolor}


\usepackage{mathtools}



\DeclarePairedDelimiter\bra{\langle}{\rvert}
\DeclarePairedDelimiter\ket{\lvert}{\rangle}
\DeclarePairedDelimiterX\braket[2]{\langle}{\rangle}{#1\,\delimsize\vert\,\mathopen{}#2}



\newcommand{\new}[1]{
    \vspace{2mm}
    \noindent
    \textbf{
    \underline{#1}}
}

\def\calO{{\mathcal{O}}}
\def\th{{\theta}}
\def\_{{\hspace{1mm}}}
\def\<{{\langle}}
\def\>{{\rangle}}


\DeclareMathOperator{\sinc}{sinc}


\newcounter{problemcnt}
\setcounter{problemcnt}{0}

\newcommand{\Problem}{{
    \vspace{5mm}
    \stepcounter{problemcnt}
    \noindent
    \arabic{problemcnt}. 
}
}

\newcommand{\nProblem}[1]{
    \vspace{5mm}
    \noindent
    \setcounter{problemcnt}{#1}
    \arabic{problemcnt}. 
}


\newcommand{\Proof}{{
    \vspace{2mm}
    \noindent
    \textbf{
    \underline{Proof}}
}
}

\newcommand{\textOr}{
    {
        \hspace{5mm}
        \textrm{or}
        \hspace{5mm}
    }
}

\newcommand{\textAnd}{
    {
        \hspace{5mm}
        \textrm{and}
        \hspace{5mm}
    }
}

\newcommand{\textThen}{
    {
        \hspace{5mm}
        \textrm{then}
        \hspace{5mm}
    }
}


\newcommand{\Ixp}[1]{
    {
        e^{i#1}
    }
}

\newcommand{\partialderiv}[2]{
    {
        \frac{\partial #1}{\partial #2}
    }
}



\newcommand{\halfFigure}[1]{
\begin{center}
\includegraphics[width = .5\linewidth]{{#1}}
\end{center}
}

\newcommand{\fullFigure}[2]{
\begin{center}
\includegraphics[width = .9\linewidth]{{#1}}
\end{center}
}

\def\twobytwoMat(#1, #2, #3, #4){
    {
        \begin{bmatrix}
            {#1} & {#2}\\
            {#3} & {#4}
        \end{bmatrix}
    }
}

\def\twobyoneMat(#1, #2){
    {
        \begin{bmatrix}
            {#1}\\
            {#2}
        \end{bmatrix}
    }
}

\def\twobytwoDet(#1, #2, #3, #4){
    {
        \begin{vmatrix}
            {#1} & {#2}\\
            {#3} & {#4}
        \end{vmatrix}
    }
}



\begin{document}
\begin{center}
\LARGE
PHYS 202 Formula Sheet

\Large
Daniel Son
\end{center}

\new{Simple Harmonic Oscillators}
Consider a mass attatched to a spring. Let 
$x(t)$ be the function of displacement of the mass 
from the equilibrium position. Suppose that the 
spring constant is $k$. By Newton's 2nd Law,

\[
    m\ddot{x} = -kx
    \textOr 
    \ddot{x} = -\frac k m x
\]

Where $\dot{x} := \frac{d}{dt}x$

The following function solves the equation. 

\[
    x(t) = Re(\tilde{A}\Ixp{i\omega_0 t})
    = A\cos(\omega_0 t + \phi)
\]

$A, \omega_0$ is referred as the amplitude and 
the natural frequency of the oscillator. Also, 

\[
    A = \sqrt\frac{2E}k 
    \textAnd
    \omega_0 = \sqrt\frac k m
\]

\new{Phase Conventions}
Velocity leads Displacement by a phase of $\pi/2$. 
Acceleration leads Velocity by a phase of $\pi/2$

Say $V \sim \cos(\omega t)$ and $I \sim \cos(\omega t + \phi)$
for some positive phase $\phi \leq \pi$. The current leads 
the voltage and the voltage trails the current. 

\new{Simple RLC}
Consider a circuit where R, L, C is connected in 
parallel. Let the current be $I$. By the loop rule, 
\[
    -\frac q C - R I - L \dot{I} = 0
    \textOr
    \frac q C + R \dot{q} + L \ddot{q} = 0 
\]
Rewrite this in the follwoing form. 
\[
    \ddot{q} + \frac R L \dot{q} + \frac q {LC} = 0
\]
Compare this with the equation for damped driven oscillators.  
\[
    \ddot{x} + \frac b m \dot{x} + \frac k m x = 0
\]
So the following isomorphism holds
\[
    x \mapsto q \textThen
    (m, b, k) \mapsto (L, R, 1/C)
\]

In a parallel circuit, the current through each 
circuit component is identical. By the complexified 
Ohms Law $\tilde V = Z \tilde I$. Recall the 
impedences. 

\[
    Z_R = R\textAnd Z_C = \frac{1}{i\omega C}\textAnd Z_L = i\omega L
\]

Thus, the voltage of the resistor leads the voltage on 
the capacitor by phase $\pi/2$. Likewise, the voltage 
on the resistor trails the voltage on the Inductor by $\pi/2$.

\new{Circuit Filters and Loglog plot}
Using inductors and capacitors, it is possible to filter 
out signals of high or low frequency. Consider the behavior 
of circuit elemnts in high and low frequency. 

\[
    \lim_{\omega \rightarrow 0} Z_L = 
    \lim_{\omega \rightarrow 0} i\omega L = 0
    \textAnd
    \lim_{\omega \rightarrow \infty} Z_L = 
    \lim_{\omega \rightarrow \infty} i\omega L = \infty
\]
\[
    \lim_{\omega \rightarrow 0} Z_C = 
    \lim_{\omega \rightarrow 0} \frac 1{i\omega C} = \infty
    \textAnd
    \lim_{\omega \rightarrow \infty} Z_C = 
    \lim_{\omega \rightarrow \infty} \frac 1{i\omega C} = 0
\]

The inductor will block signals of high frequency 
but allow the passage of signals of low frequency. 
Hence, inductors act as a \textbf{low-pass} filter. On the 
other hand, capacitors block signals of low frequency 
and allow the passage of high frequency signals. 
Hence, capacitors act as a \textbf{high-pass} filter. 


\new{Corner frequency vs Resonant frequency} 
For an RLC circuit, it is possible to tune the frequency 
such that the combined impedence reach zero. This 
frequency is called the resonant frequency. 
For a circuit involving one R, L, C, the resonant frequency 
is computed by the following formula. 

\[
    \omega_r = \frac 1 {\sqrt{LC}}
\]

For and RL or an RC circuit, the imaginary part of the 
impedence cannot reach zero. It is possible to defnine 
a corner frequency where the circuit behavior changes 
drastically. It is the freqnecy where $Im(Z_L) = R$ 
or $Im(Z_C) = R$. In other words, 

\[
    \omega_c = {\frac 1 {RC}}
    \textOr 
    {\frac R L}
\]

\new{Dimensions of C and L}
\[
    [L] = H = \Omega \cdot s 
    \textAnd 
    [C] = \frac s \Omega
\]

\new{Power Resonance Curve}
For an RLC circuit, it is possible to plot 
the relationship between the angular frequency 
of the imput voltave and the power lost 
through the resistor. This curve is 
called the Power Resonance curve. 
\fullFigure{PowerResonance.png}

$P \sim V^2$. 
The Full-Width-Half-Maximum refers to the 
width of the curve at the half maximum point. 
The FWHM of the Power Resonance Curve is 
exactly $\gamma = R/L$, the damping factor. 
Be careful of the factor of $2\pi$ when 
the x-axis is set as frequency (Hz). 

Also, use the dimension $[\omega] = rad/s$
and $[f] = Hz$. The frequency refers to cycles 
per second. 

\new{Q-factor} 
For damped driven oscillators, the Q-factor is an important 
number. 

\[
    Q := \frac{\omega_0}\gamma = \frac {A_{res}} {A_0}
\]


\new{Boundary conditions for N masses}
The solutions for the n mass oscillators are in the form of:

\[
    \psi(m, t) = A \Ixp{(\omega t + k_ma)} + B \Ixp{(\omega t - k_ma)}
\]

Plugging into the 2nd order DE, we derive an expression for the angular 
frequency.

\[
    \omega_m = 2\omega_0\sin(k_m \frac{a}{2})
\]
$k_m$ is the mth wave number. Recall that $k := \frac{2\pi}{\lambda}$.
$\lambda_n$ can be computed by drawing diagrams.

\halfFigure{Closed_End_Fig.png}

With some algebraic hassle, it is possible to derive the expressions
for $k_m $. 

For closed-closed and open-open ends:
\[
    k_m a = \frac{m\pi}{n+1}
    \textOr 
    k_m = \frac{m\pi}{a(n + 1)} = \frac{m\pi}{L}
\]
Where $m \in \mathbb{Z}^+$

For open-open ends:
\[
    k_m a = \frac{m\pi}{(2n+1)/2}
    \textOr 
    k_m = \frac{m\pi}{L} 
\]
Where $m \in \mathbb{Z}^+$




\new{Free end Distance relation}
Always set the origin to be distance $a/2$ apart from the free end. 
This is because the two masses at the boundary must 
be symmetric. 
Refer to pset 5-1. 



\new{Transverse waves}
From the geometry of the springs, we use approximation. 
Let theta be the angle between the horizontal axis and the string. 
$\tan(\theta) \approxeq \theta = \Delta y/a$. Consequently, we 
arrive at:
\[
    k \mapsto T/a
\]

\new{Dispersion Relation}
The differential equation of the masses in the center 
oscillators provide a closed form equation for 
the angular frequency in terms of the wave number $k$. 
\[
    \omega(k) = 2\omega_0 \sin(k a/2)
\]
where a is the distance between the masses. 

\new{Dispersion Relation 2}
The dispersion relation defines the wave number $k$. 
\[
    \partialderiv{^2 \psi}{t^2} = 
    c^2
    \partialderiv{^2 \psi}{x^2}
    \textThen
    \omega^2 = c^2 k^2
\]
And $c$ is called the phase velocity of the wave. 
The wave number depends on the frequency of the wave, which is 
not necessarily the normal mode frequency

\new{Nodes and Antinodes}
In a standing wave, nodes are the points that do not move. 
Max amplitude is achieved at antinodes. 

\new{Waves traveling in different medium, same tension}
Consider a transverse wave moving from one string 
of linear mass density $\mu_1$ to another string with linear 
mass density $\mu_2$.
The velocity of the waves on each string is entirely determined 
by the lmd. 

\[
    (v_1, v_2) = \left(
        \sqrt{\frac{T}{\mu_1}}
        ,
        \sqrt{\frac{T}{\mu_2}}
    \right)
\]

Let the incident wave to be in the form of 
\[
    \psi_i(x, t) := f_i(t - x/v_1) 
\]

We make a natural assumption that the combined 
wavefunction must be smooth with respect to position. 
Let $f_t, f_r$ be the simplified translated wavefunction 
and the simplified reflected wavefunciton. 
We derive 
\[
    \frac{f_r}{f_i} = \frac{v_2 - v_1}{
        v_1 + v_2
    }
    \textAnd 
    \frac{f_t}{f_i} = \frac{2v_2}{v_1 + v_2}
\]

\new{Natural conditions on the string displacements}
We assume that the wavefunction is continuous and 
differentiable at all positions. Also, 
\[
    \dot\psi(x, 0) = 0
\]
allows us to easily complexify the solution. 

\new{Crash-intro to Fourier Transform}
\fullFigure{Fourier.png}

\new{Non-dispersive Waves}
refers to waves in which the variance 
of frequency and wavelength of the wave does not 
affect the velocity of the wave. A good example is 
a traverse wave on a string. 

\[
    v = \sqrt{\frac T \mu}
\]
where $T$ is the tension of the string and $\mu$ refers to mass density. 


\new{Mechanical impedence and transmission across different tension}
We define mechanical impedence of a string as follows. 
\[
    Z = \frac F v = \sqrt{T\mu}
\]

Mechanical impedence is analogous to drag force the string exerts 
on the ends. That is $Z \mapsto b$. Using impedence, we can write out 
the reflection and transmission coefficients. Let $Z_1, Z_2$ denote 
the mechanical impedence of line 1 and line 2 where the wave 
is traveling from line 1 to line 2. The reflection and transmission 
coefficients are written as: 

\[
    R = \frac{Z_1 - Z_2} {Z_1 + Z_2}
    \textAnd 
    T = \frac{2Z_1} {Z_1 + Z_2}
\]

Reflection happens by a small amount, and transmission is dependant 
on the original string. Perfect transmission happens when impedences 
are matched. That is, $Z_1 = Z_2$. 

\new{Space-Time relation of traveling waves}
We know, mathematically, that the solutions to the wave equations 
are in the form of 
\[
    \psi(x, t) = f(x - vt)
\]
Where $v$ is the longditudanal speed of the force. 
By applying the chain rule, we can deduce 
\[
    \dot \psi = -v \psi '
\]
This can also be deduced by drawing the following diagram 
\halfFigure{spacetimewave.png}

\new{Energy and Power of a traveling wave}
Considering the string as a conglomerate of segments, we can retrieve 
a formula for the energy density of the wave. Note that the potential 
energy of the string is compted by $Tdl$ where $dl$ refers to stretch. 
The kinetic energy is related to the time derivative, and the potential energy 
is related to the space derivative. Applying the wave equation, we derive 

\[
    \mathcal{E} = 
    \mu \dot \psi^2 
    = 
    \mu v^2 (\psi ')^2
\]

So the wave equation somewhat coagulates the space and time dependance of energy. 
The energy density of a wave is product of the lmd and square of the time derivative of the wave. 

It makes sense that the power transmission is the negativ product of 
linear energy density and transmission speed. This can be derived 
rigorously using $\dot \psi = -v\psi '$. 
\[
    P(x, t) = -v \mathcal{E} 
\]

\new{Analysis of Standing waves using impedence}
Fixed ends can be considered as a string connected to another 
string with infinite impedence. Free ends likewise can be considered 
as string connected to another spring with zero impedence. Enforcing 
continuity and smoothness we can derive an equation for standing waves. 


\new{Choosing the correct type of wave form}
The solutions to the wavefunction come in the form of 
\[
    \psi(x, t) = A\Ixp{(kx - \omega t)}
    \textOr 
    A\Ixp{(\omega t - kx)}
\]
When imposing the boundary condition which is space-dependant, it is 
more useful to adapt the latter form. 


\new{Crash intro to optics}
The four Maxwell equations govern the behavior of EM waves. 
We are especially interested in the case of waves traveling 
in isotropic media. That means, zero charge density and linear 
current density. The four Maxwell Equations are presented 
as below. 

\begin{align*}
    \triangledown \vec{E} = -\mu_0 \vec{H}
    &&
    \triangledown \vec{H} = -\epsilon_0 \vec{E}
    \\
    \dot{\vec{E}} = 0 
    &&
    \dot{\vec{H}} = 0
\end{align*}

The field $\vec{H}$ is rigorously defined in PM 11.9, 
but for now, just remember 
\[
    \vec{H} = \frac 1 { \mu_0} \vec{B}
\]

$\mu$ refers to permeability and $\epsilon$ refers to 
permittivity. The value of permeability is usually 
fixed to be $\mu_0$ for transparent medium. The value of permittivity depends 
on the medium. The diaelctric constant is defined as 
\[
    \kappa := \frac \epsilon {\epsilon_0}
\]

The velocity of an EM wave, aka light, depends on the traveling 
medium. The index of refraction is defined as 
\[
    n := \frac c v
\]
. Note that the index is defined as the reciprocal of the 
more intuitive definition. 

From E\&M, recall that the energy transmission was governed 
by the Poynting vector, 
\[
    \vec{S} := \vec E \times \vec H
\]
From the Maxwell's equations, we deduce the magnitude of $\vec{H}$. 
\[
    H = \frac {nE}{Z_0}
\]
Where $Z_0 := \sqrt{\mu_0/\epsilon_0} = 377 \Omega$ is defined 
as the impedence of free space. 

We also deduce that irradiance (or intensity) is proportional 
to square of the E field. 
\[
    I = \frac n {2Z_0} E_0^2
\]

\new{Isomorphism btw transmission lines and EM waves} 

Remember from Morin 8.1, we defined how a signal travels through 
a transmission line. Two signals appear, which is the voltage wave 
$V$ and the current wave $E$. Let $L_0, C_0$ denote 
the linear inductance and the capacitance. We obtained the result that 

\[
    v = \sqrt{\frac 1 {L_0 C_0}}
\]

Also, these two waves are isomorphic to the EM waves. The isomorphism 
is given as follows 
\[
    V \mapsto E \hspace{1cm} I \mapsto B / \mu_0 
\]

We also know 
\[
    Z = V/I \textAnd E = cB 
\]
where $c$ is the velocity of the wave given by $1/\sqrt{\epsilon \mu}$

With some algebra, we deduce 
\[
    Z = n Z_{free}
\]
. To obtain the result, use the asusmption that permeability is 
constant for transparent media. 

\newpage

\new{Jones Calculus for polarization}
Consider an EM wave traveling in isotropic media. It 
is a convention to consider the electric field of the 
wave for polarization. The electric field has both 
x and y components which are sinusoids dependant 
on time and the transmission axis z. That is, 
\[
    \vec E = \hat x E_x + \hat y E_y
    \hspace{1cm}
    \textrm{where}
\]
\[
        E_x = E_{x0}\cos(kz - \omega t)
        \textAnd 
        E_y = E_{y0}\cos(kz - \omega t + \phi)
\]
Complexifying $E_x, E_y$, we can write 
\[
    \tilde E_x = E_{x0}\Ixp{(kz - \omega t)}
    \textAnd
    \tilde E_y = E_{y0}\Ixp{(kz - \omega t + \phi)}
\]
So, the vector $\tilde {\vec E}$ can be written as follows 
\[
    \tilde {\vec E }= 
    \twobyoneMat(E_{x0}, E_{y0} \Ixp \phi) \Ixp{(kz - \omega t)}
\]
The exponential term outside the matrix is insignificant 
to describe polarizaton. We define the constant matrix 
as the \textbf{Jones vector} of the polarization. Factoring 
out a global phase factor, a Jones matrix is in the form of 
\[
    L\twobyoneMat(1, \Ixp \phi)
    \textOr
    R\twobyoneMat(\Ixp \phi, 1)
\]
for $\phi \in [0, \pi]$. The first matrix represents a light 
in which the y-axis has a \textit{slower} phase. Shift the y-sinusoid 
of the EM wave to deduce that the light is left polarized. 
Similarly, the second matrix represents a right polarized light. 

The effect of an optical element can be represented as a linear transform 
on the Jones vector. We call this 2x2 matrix a \textbf{Jones Matrix}. 
So far, we have covered two major optical elements, the 
linear polarizer and the waveplate. Using the bra-ket notation, 
their Jones matrix can be written as follows. 

\[
    L_{\vec v} = \ket {\vec v} \bra{ \vec v}
    \textAnd 
    W(\hat f, \hat s, \phi) = \Ixp{\phi} \ket {\hat s} \bra{ \hat s} + \ket {\hat f} \bra{ \hat f}
\]

\new{Some useful implications of the Maxwell equations}
We know that the EM wave equations are in the form of 
\[
    \vec E = \vec E_0 \Ixp(\vec k \vec r - \omega t) 
    \textAnd
    \vec B = \vec B_0 \Ixp(\vec k \vec r - \omega t) 
\]

Applying the Maxwell equations allow us to make the following conclusions. 

The wavevector, or the direction of propagation, is perpendicular 
to both the electic and the magnetic field. 
\[
    \vec k \cdot \vec E = \vec k \cdot \vec B = 0
\]
The magnitude of the Efield and the Bfield are proportional. 
\[
    E = cB  
\]
The cross product of the wavevector and the electric field 
yields the magnetic field multiplied by the angular frequency. 
\[
    \vec k \times \vec E = \omega \vec B
\]
Finally, the Poyntinvg vector provides the energy propagation 
of an EM wave. 
\[
    \vec S = \frac 1 {\mu_0} \vec E \times \vec B
\]

The velocity of the wave is the angular frequency over 
the magnitude of the wavevector. 
\[
    v = \omega / k    
\]

\new{Waves carry Momentum}
We have seen that waves transfer energy. But how can a wave 
have momentum? To answer this question, we consider a free charge 
in the transmission axis. Let $\hat k$ be the transmission axis 
and $q$ be a charge sitting on this axis. The E and B field propagate 
in a way that their vectors are perpendicular to each other and $\hat k$. 

We wish to compute the infinesimal amount of work and momentum exerted 
on the free charge. The electric field exerts force in the same direcion 
of the movement of the particle. Thus, 

\[
    dW = F_E dx = (qE) (v_E dt) 
\]

Assuming the movement of the particle is nearly perpendicular to $\hat k$, 
the magnetic field also exerts a force on the particle. 
\[
    dp = mv_E = ma_E dt = F_B dt = qv_E B dt
\]

Also remember $E = cB$. Thus, $cdp = dW$ and by integration, $cp = E$. 
\[
    p = \frac E c
\]
as desired. 

\hfill 
\qed

\new{The double slit experiment}

A vertical wavefront approaches a wall that has two slits. As 
the wavefront hits the wall, two circular wavefronts are formed 
by the huygens principle. The two wavefronts are in phase by construction. 
\fullFigure{Double_Slit.png}

Remember that for 2D circular waves, the amplitude decreases 
by a factor of $\sqrt{r}$. With some algebra, we deduce the 
irradiance as a function of $x$. 
\[
    \frac {I(x)} {I(0)} = 
    \frac D {\sqrt{x^2 + D^2}}
    \cos^2 \left(
        \frac {kd} 2 
        \frac x {\sqrt{x^2 + D^2}}
    \right)
    = 
    \frac D {\sqrt{x^2 + D^2}}
    \cos^2 \left(
        \frac {d\pi} {2 \lambda} 
        \frac x {\sqrt{x^2 + D^2}}
    \right)
\]

For $x \ll D$,  
\[
    \boxed{
    I_{tot}/I = \cos^2 \left(
        \frac d D \frac x \lambda \pi
    \right)
    }
\]
The argument of the cosine square function can be interpreted 
as the screen distance x projected to the wall space converted to 
some phase factor by $\lambda$. 

\new{Multislit experiment}
Now, consider the same apparatus as the double slit, 
but instead, include $N$ slits. We preceed with the assumption 
that 
\[
    d \ll D \textAnd x \ll D \textAnd \theta \approxeq 0
\]
Also, the phase difference between two consecutive rays are 
\[
    \delta := ka\sin(\theta) := k' a
\]
We also know that the amplitude of the wave decreases by a 
factor of $\sqrt{d}$ for cyllindrical waves, where $d$ is the 
distance travelled. In the slit experiments, the amplitude of 
a single ray depends on the angle $\theta$. Define 
\[
    A(\theta) = \sqrt{\frac {\cos(\theta)} D}A_{src} 
\]
Assuming $\theta$ small, 
\[
    A(\theta) \approxeq A(0)
\]

We first derive the amplitude for angle $\theta$ where there are 
N slits, each a distance $d$ apart. 
\[
    A_{tot}(\theta) = 
    A(\theta)\left| 
        1 + e^{i\delta} + e^{2i\delta} +\cdots
    \right|
    = A(0) \frac {\sin(N\delta/2)} {\sin{\delta / 2}} 
\]

Also note 
\[
    A_{tot}(0) = A(0)\left| 
        1 + 1 + 1 +\cdots 
    \right|
    = n A(0) 
\]

Thus, 
\[
    \frac{A_{tot}(\theta) }{A_{tot}(0)} = 
    \frac{\sin(N\delta/2)}{N\sin(\delta/2)}
\]

Intensity is proportional to the square of the amplitude. 
\[
   \frac { I_{tot}(\theta)} {I_{tot}(0)} 
   =\left( \frac{\sin(N\delta/2)}{N\sin(\delta/2)}\right)^2
\]

\new{Wide Slits}
It is possible to derive the continum case by taking the limit. 
However, the suggested solution is using the integral. Otherwise, 
it is hard to see that the intensity is dependant on the 
slit width. We finally reach, 

\[
    |A| \propto a \sinc\left(
        \frac {k'a} 2
    \right)
    \textAnd
    I
\propto a^2 \sinc^2\left(
        \frac {k'a} 2
    \right)
\]
where $k' = k\sin(\theta)$

\new{Diffraction and Reflection}
The angle of incident and the angle of reflect 
is defined to be the angle between the 
ray and the normal axis. It is not necessary that 
these two angle agree with each other. If any of the 
rays exit the surface, we call this ray the 
\textbf{diffracted} ray. If the incident angle 
matches the reflected angle, we refer to 
the phenomenon as \textbf{reflection}. 

To compute the phase difference of a diffracted 
ray, consider the change of difference traveled. 

\end{document}