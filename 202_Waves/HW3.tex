\documentclass{article}
\usepackage{amsfonts}
\usepackage{amsthm}
\usepackage{amssymb}
\usepackage{amsmath}
\usepackage{graphicx}
\usepackage{subcaption}

\newcommand{\new}[1]{
    \vspace{2mm}
    \noindent
    \textbf{
    \underline{#1}}
}

\def\calO{{\mathcal{O}}}
\def\th{{\theta}}
\def\_{{\hspace{1mm}}}
\def\<{{\langle}}
\def\>{{\rangle}}


\newcounter{problemcnt}
\setcounter{problemcnt}{0}

\newcommand{\Problem}{{
    \vspace{5mm}
    \stepcounter{problemcnt}
    \noindent
    \arabic{problemcnt}. 
}
}

\newcommand{\nProblem}[1]{
    \vspace{5mm}
    \noindent
    \setcounter{problemcnt}{#1}
    \arabic{problemcnt}. 
}


\newcommand{\Proof}{{
    \vspace{2mm}
    \noindent
    \textbf{
    \underline{Proof}}
}
}

\newcommand{\textOr}{
    {
        \hspace{5mm}
        \textrm{or}
        \hspace{5mm}
    }
}

\newcommand{\textAnd}{
    {
        \hspace{5mm}
        \textrm{and}
        \hspace{5mm}
    }
}

\def\twobytwoMat(#1, #2, #3, #4){
    {
        \begin{bmatrix}
            {#1} & {#2}\\
            {#3} & {#4}
        \end{bmatrix}
    }
}

\def\twobyoneMat(#1, #2){
    {
        \begin{bmatrix}
            {#1}\\
            {#2}
        \end{bmatrix}
    }
}

\def\twobytwoDet(#1, #2, #3, #4){
    {
        \begin{vmatrix}
            {#1} & {#2}\\
            {#3} & {#4}
        \end{vmatrix}
    }
}



\begin{document}
\begin{center}
\LARGE
PHYS 202 HW3

\Large
Daniel Son
\end{center}

\normalsize

\new{Q1}
Identify the faulty labeling from the power 
resonance curve. 

\new{Solution}

The peak of the curve occurs 
at $f = 22.5Hz$ so 
$\omega_res \approxeq 172kHz$. The FWHM of the 
curve is about $\gamma  = 2\pi \cdot 500Hz
\approxeq 3kHz $. We claim that the 
wrong label is the inductance, and hence 
the correct label is 
$(R, L, C) = (30mH, 1nF, 100 \Omega)$. 

Assuming the purported label is correct, 
we compute the theoretical resonant frequency 
and $\gamma$. 

\[
    \gamma_{th} = R/L = 3.3kHz
\]
\[
    \omega_{th} = 1/\sqrt{RC} = 180kHz
\]

Which is close enough to the experimental 
value. \hfill \qed 


\new{Q2}
A resistor, inductor, and conductor 
is connected in parallel. Assuming 
that the a varying input current is 
entering the circuit in the rate of 
$I(t)  = I_0 \cos(\omega t)$, 

a) What is the resonant frequency?

b) What is the FWHM of the power resonance curve?

c) Define $Q:= \omega_0/FWHM$. What is the 
relationship between the Q-factors 
of the parallel and series circuit?

\new{Solution}

We start off with writing the complexified 
voltage across each component. 

\[
    \tilde{I}_{in} = \sum I = 
    \tilde{V}_{in} \left( 
        \frac{1}{R} +
        \frac{1}{i\omega L} +
        i\omega C
    \right)
\]

Thus 
\[
    \tilde{V}_{in} = \tilde{I}_{in} / \left( 
        \frac{1}{R} +
        \frac{1}{i\omega L} +
        i\omega C
    \right)
\]

a)

Resonance occurs when  the complex impedences 
cancel out. That is 
\[
    \omega C = 1/(\omega L)
    \textOr 
    \omega ^2 = \frac{1}{LC}
    \textOr 
    \boxed{
    \omega_0  = \frac{1}{\sqrt{LC}
    }
    }
\]

Recovering the real value for $V_{in}$
\[
    V_{in}(t) = \frac{
        I_0 \cos(\omega t + \phi )
    }{
        \sqrt{
        1/R^2 + (\omega C - 1/(\omega L))^2
        }
    }
\]
we can write an expression for $P(t)$.
\[
    P(t) = V_{in}(t)^2/R = 
\frac{
        I_0^2 \cos^2(\omega t + \phi )
    }{
        R(
        1/R^2 + (\omega C - 1/(\omega L))^2
        )
    }
\]

The time average of the squared cosine 
function is $1/2$. Thus 

\[
     \<P(t)\> = 
\frac{
        I_0^2 
    }{
        2R(
        1/R^2 + (\omega C - 1/(\omega L))^2
        )
    }
\]

Ignoring the constants, we recognize that 
the FWHM occurs when 
\[
    1/R^2 = (\omega C - 1/(\omega L))^2
\]

Multiply both sides by $L^2 \omega^2$

\[
    \omega^2 /\gamma ^2 = 
(\omega^2 LC - 1)^2
\textAnd 
\pm \omega/\gamma = 
\omega^2 LC - 1
\]

Call the solutions to the two equations 
as $\omega_1$ and $\omega_2$.

\[
\omega_1/\gamma = 
\omega_1^2 LC - 1
\textAnd
-\omega_2/\gamma = 
\omega_2^2 LC - 1
\]

Subtracting the bottom equation from the top 
\[
    (\omega_1 +\omega_2)/\gamma = 
    (\omega_1^2-\omega_2^2)LC
    \textOr 
    (\omega_1-\omega_2) = \frac{1}{\gamma LC}
    = \boxed{\omega_0^2/\gamma}
\]

c)
We compute the Q-factor for the parallel circuit.
\[
    Q_{par} = \omega_0 / {\omega_0^2/\gamma}
     = \gamma / \omega_0
\]
We also know the Q-factor for series circuits. 
\[
    Q_{ser} = \omega_0/\gamma 
\]
Thus 
\[
    \boxed{Q_{ser} = 1/Q_{par}}
\]

\new{Q3 Reduced Mass}
Two masses are connected to each 
other with a string with constant 
$k$.

a) Write down the equation of Newton's II 
law for the two masses. Write the matrix 
form of the equation. 

\new{Solution}
We write the two equations.
\[
    m_1\ddot{x_1} = -k(x_1-x_2) 
    \textAnd 
    m_2\ddot{x_2} = -k(x_2 - x_1)
\]
We wish to obtain a homogeneous system.

\[
    m_1\ddot{x_1} +k(x_1-x_2) =0
    \textAnd 
    m_2\ddot{x_2} +k(x_2 - x_1)=0
\]

We guess the solutions to be in the form 
of $\tilde{x}_1 = Ae^{i\omega t}, \tilde{x}_2 = Be^{i\omega t}$.
Plugging in and cancelling the exponential term, we write 

\[
    -\omega^2 m_1 A + kA - kB = 0 
    \textAnd 
     -\omega^2 m_2 B + kB - kA = 0 
\]

In matrix form
\[
    \twobytwoMat(
        -\omega^2 m_1 + k, 
        -k, 
        -k, 
        -\omega ^2 m_2 + k
    )
    \twobyoneMat(A, B)
    = 
    \twobyoneMat(0, 0)
\]

b) Determine the normal modes of the system 

\new{Solution}

The normal modes occur when the 
determinant of the coefficient matrix equals zero. That is 
\[
     \twobytwoDet(
        -\omega^2 m_1 + k, 
        -k, 
        -k, 
        -\omega ^2 m_2 + k
    )
    = 
    (k - \omega^2 m_1)(k - \omega^2 m_2) - k^2 = 0
\]
\[
    \omega^4 m_1 m_2 - k(m_1 + m_2)\omega^2 = 0
    \textOr 
    \omega^2 (\omega^2m_1 m_2 - km_1 - km_2) = 0
\]

The solutions to this equation are 
\[
\boxed{
    \omega = 0 \textOr 
    \sqrt{\frac{
        k(m_1 + m_2)
    }{m_1m_2}}
    = \sqrt{k/\mu}
}
\]

c) Determine the eigenvector which corresponds to each normal mode frequency. For the vibrational mode, how does the ratio of the two masses’ ranges of motion relate to their respective
masses?

\new{Solution}
Write the coefficient matrix for the 
two possible cases of $\omega$. If 
the frequency is zero and nonzero, the 
coefficient matricies are respectively
\[
    \twobytwoMat(k, -k, -k, k)
    \textAnd 
    \twobytwoMat(k - m_1 k /\mu, -k, 
    -k, k - m_2 k / \mu)
    = 
    -\twobytwoMat(
        m_1 k /m_2, k, 
        k, m_2 k /m_1
    )
\]

Upon inspection, the eigenvectors 
that correspond to each of the matricies are 
\[
    \twobyoneMat(1, 1) \textAnd \twobyoneMat(m_2/m_1, -1)
\]

The former mode is the stationary mode 
and the latter mode is the vibrational mode. 
For the vibrational mode, if the mass of the 
two bodies are comparable to each other, the 
amplitude of their oscilations will also be 
comparable. Nonetheless, if one of the bodies 
are much heavier, the oscillation amplitude of 
the lighter mass will be larger than the 
lighter mass. Also, the phase of the two 
oscillations will differ by $\pi$. 

\hfill \qed


\end{document}