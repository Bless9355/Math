\documentclass{article}
\usepackage{amsfonts}
\usepackage{amsthm}
\usepackage{amssymb}
\usepackage{amsmath}
\usepackage{graphicx}
\usepackage{subcaption}
\usepackage{xcolor}
\usepackage{mathtools}
\usepackage{ wasysym }


\newcommand{\new}[1]{
    \vspace{2mm}
    \noindent
    \textbf{
    \underline{#1}}
}

\def\calO{{\mathcal{O}}}
\def\th{{\theta}}
\def\_{{\hspace{1mm}}}
\def\<{{\langle}}
\def\>{{\rangle}}

\DeclarePairedDelimiter\bra{\langle}{\rvert}
\DeclarePairedDelimiter\ket{\lvert}{\rangle}
\DeclarePairedDelimiterX\braket[2]{\langle}{\rangle}{#1\,\delimsize\vert\,\mathopen{}#2}



\newcounter{problemcnt}
\setcounter{problemcnt}{0}

\newcommand{\Problem}{{
    \vspace{5mm}
    \stepcounter{problemcnt}
    \noindent
    \arabic{problemcnt}. 
}
}

\newcommand{\nProblem}[1]{
    \vspace{5mm}
    \noindent
    \setcounter{problemcnt}{#1}
    \arabic{problemcnt}. 
}


\newcommand{\Proof}{{
    \vspace{2mm}
    \noindent
    \textbf{
    \underline{Proof}}
}
}

\newcommand{\textOr}{
    {
        \hspace{5mm}
        \textrm{or}
        \hspace{5mm}
    }
}

\newcommand{\textAnd}{
    {
        \hspace{5mm}
        \textrm{and}
        \hspace{5mm}
    }
}

\newcommand{\Ixp}[1]{
    {
        e^{i{#1}}
    }
}



\newcommand{\halfFigure}[1]{
\begin{center}
\includegraphics[width = .5\linewidth]{{#1}}
\end{center}
}

\newcommand{\fullFigure}[1]{
\begin{center}
\includegraphics[width = .9\linewidth]{{#1}}
\end{center}
}

\def\twobytwoMat(#1, #2, #3, #4){
    {
        \begin{bmatrix}
            {#1} & {#2}\\
            {#3} & {#4}
        \end{bmatrix}
    }
}

\def\twobyoneMat(#1, #2){
    {
        \begin{bmatrix}
            {#1}\\
            {#2}
        \end{bmatrix}
    }
}

\def\twobytwoDet(#1, #2, #3, #4){
    {
        \begin{vmatrix}
            {#1} & {#2}\\
            {#3} & {#4}
        \end{vmatrix}
    }
}



\begin{document}
\begin{center}
\LARGE
PHYS 202 HW6

\Large
Daniel Son
\end{center}

\normalsize 

\new{Q1 Matching initial conditions for the wave equation} 

We know that the 1D wave equation 
\[
    y'' = v_p^2\ddot{y}
\]
is satisfied by the solution 
\[
    y(x, t) = f(x - v_p t) + g(x + v_p t)
\]
for arbitrary functions $f, g$. Note that $f$ is the forward 
moving wave and $g$ is the backward moving wave. 

A wave is known to have an initial displacement 
\[
    y(x, 0) = \frac {.001} {1 + x^2}
\]
Compute the solution for the wave moving in $10m/s$ given that:

a) the wave is moving forwards 

b) the wave is moving backwrads

c) the wave is traveling in opposite directions 

\new{Solution}
The case a, b can be dispatched with ease. We know that $g = 0$ and 
$f = 0$ for case a, b, respectively. Thus, for a, 
\[
    f(x - v_p t) \bigg|_{t = 0} =  \frac {.001} {1 + x^2}
    \textOr 
    f(x) = \frac {.001} {1 + x^2}
\].
Likewise, for b, 
\[
    g(x + v_p t) \bigg|_{t = 0} =  \frac {.001} {1 + x^2}
    \textOr 
    g(x) = \frac {.001} {1 + x^2}
\].
Now we write the solution for the wave equation. 

\[
    y_a(x, t) = f(x - v_p t) =  \frac {.001} {1 + (x - 10m/s \cdot t)^2}
\]

\[
    y_b(x, t) = g(x + v_p t) =  \frac {.001} {1 + (x + 10m/s \cdot t)^2}
\]

Case 3 can be resolved as considering the wave as a superposition 
of the two waves $y_a, y_b$ with half amplitudes. In symbols, 

\[
    y_c = \frac 1 2 (y_a + y_b)
    = .0005 \left(
        \frac 1 {1 + (x - 10m/s \cdot t)^2}
        + 
        \frac 1 {1 + (x + 10m/s \cdot t)^2}
    \right)
\]

\fullFigure{Q1_animation.png}


\new{Q2 Schrodinger's Equation}

Recall the general Schrodinger's Equation. 
\[
    i\hbar \dot{\ket{\psi}} = \hat H \ket{\psi}
\]

The Hamiltonian of the 1D particle can be 
written as the sum of the kinetic and potential energy. 

\[
    \hat H = \hat P + \hat V = 
    \frac {\hat p ^2} {2m} + \hat V
\]

The operator $\hat p$ denotes the momentum of the particle. 
\[
    \hat p = -i \hbar \triangledown 
\]

For a free particle, there is no potential energy. Rewrite the Hamiltonian. 

\[
    \hat H_{free} = - \frac {\hbar^2}{2m} \triangledown^2
    = -\frac{\hbar^2}{2m} \Delta
\]
where $\Delta$ denotes the Laplacian. 

Now, consider a free particle in 1 dimension. The equation simplifies to: 
\[
    i \hbar \frac{\partial\psi}{\partial t} = -\frac{\hbar^2}{2m} \frac{\partial^2 \psi}{\partial x^2}
\]

\new{a)} Can $\psi(x, t)$ be in the form of $\Ixp{(kx - \omega t)}$? If so, 
what is the dispersion relation?

For the complex exponential form of $\psi$, the time derivative 
acts as multiplying $-i\omega$ and the space derivative $ik$. Hence, 
the 1D Schrodinger equation can be rewritten as 

\[
    i \hbar (-i\omega) = -\frac {\hbar^2}{2m} (ik)^2
    \textOr 
    \boxed{
    \frac \omega {k^2} = \frac \hbar {2m}
    }
\]

We also take note of the phase velocity 
\[
    v_p = \frac \omega k = \frac{\hbar k} {2m}
\]

\new{b)} Show that for an arbitrary function $f$, the Schrodinger's 
equation cannot be satisfied by $\psi(x, t) = f(x - vt)$ for any velocity v. 

Plug the candidate solution into the Schrodinger's equation. We get 
\[
    i\hbar (-v)f'(x - vt) = -\frac{\hbar^2}{2m} f''(x - vt)
\]

This implies that the ratio $f''/f'$ is a constant. This is not true 
in general. Hence, $f(x - vt)$ is not a solution. 

\new{c)} Let $\tilde \psi(x, t)$ be a solution to TDSE. Is the 
realified version, $\Re{\tilde \psi} := \psi$ always a solution 
for TDSE?

Unfortunately, no. Let $\tilde \psi := A\Ixp{{kx - \omega t}}$
be a solution to the TDSE. We take the real part and plug it 
into the TDSE. 

\[
    \psi = \Re ( A\Ixp{{kx - \omega t}})= A\cos(kx - \omega t)
\]

Try $\psi$ as a candidate solution for TDSE. 

\[
    i \hbar \frac{\partial (A\cos(kx - \omega t))}{\partial t}
    = -\frac{\hbar^2}{2m} A\cos(kx - \omega t)''
\]

The LHS of the equation is imaginary while the RHS is real. 
Thus, both sides must identically equal zero, implying $A = 0$. 
This is not true for a general $\tilde\psi$. \lightning

\new{d)} In quantum mechanics, energy and momentum can be interpreted 
as a multiple of frequency and wavenumber. We know that the 
energy of a photon is determined by $E = hf$. Thus 
\[
    E = \hbar \omega
\]
By the DeBroglie equation, we have $p = h/\lambda$. $k = 2\pi / \lambda$ 
so 
\[
    p = \hbar k
    \] 

Recall 
\[
    \frac \omega {k^2} = \frac \hbar {2m}
\]
. Replace $\omega, k$ with energy and momentum.
\[
    \frac E {p^2} = \frac 1 {2m} 
    \textOr 
    \boxed{
        E = \frac{p^2}{2m}
    }
\]

\new{e)} The Klein-Gordon Equation reads 
\[
    -\ddot\phi + c^2 \phi '' =
    \left(
        \frac {mc^2} \hbar
    \right)^2 
    \phi
\]

Let $\phi = A\Ixp{(kx - \omega t)}$. Again, the 
time derivative operator and the space derivative operator 
corresponds to $-i\omega$, $ik$ respectively. Thus, the equation 
simplifies to 
\[
    -(i\omega)^2 \phi + c^2 (ik)^2 \phi = \left(
        \frac {mc^2} \hbar
    \right)^2 
    \phi
\]. 
Divide by $\phi$ and simplify to obtain the dispersion relation. 
\[
    \omega^2 - (kc)^2 = \left(
        \frac {mc^2} \hbar
    \right)^2 
\]
Now utilize $E = \hbar \omega$ and $p = \hbar k$. Multiply both sides by $\hbar^2 $
\[
    E^2 = (pc)^2 + (mc^2)^2
\]
We have obtained the equation for relativistic energy. 

\new{f)} Consider the 1D diffusion equation, 
\[
    \dot \phi = \gamma^2 \phi ''
\]
Does $\phi(x, t) = A\Ixp{(kx - \omega t)}$ satisfy this equation 
if $k, \omega$ are both real? What if $\omega$ is not necessarily real? 
How does the temperature distribution change over time?

Again, using differential operators, we rewrite the equation as 
\[
   (-i\omega) \phi = \gamma^2 (ik)^2   \phi
   \textOr 
   \omega = -i\gamma^2k^2
\]

If $k\in \mathbb{R}$, necessarily $\omega$ must be purely imaginary. So 
assume this to be the case. Rewrite the temperature distribution $\tilde \phi$. 

\[
    \tilde \phi = A \Ixp{kx} e^{-i(-i\gamma^2k^2t)} = Ae^{-\gamma^2k^2t} e^{ikx} 
\]

Take the real part to retrieve $\phi$
\[
    \phi =  Ae^{-\gamma^2k^2t} \cos(kx)
\]

At any time, the temperature is distributed sinusoidially along 
the x-axis. As time goes on, the amplitude of the distribution 
decreases exponentially. Eventually, the temperature is distributed 
equally along the whole plane. 


\new{Q3 Multiple interferences}

An infinite string has a wave number of $k_1$. A short segment 
of length $L$ has a wave number of $k_2$. We distinguish the 
string into three regions, region I, II, III. A wave travels 
through this string from region I. Some reflection/transmission 
occurs at two boundaries. Let $A_1, A_2, A_3$ be the 
three waves traveling in the forward direction for each region. 
Likewise, let $B_1, B_2$ be the backward traveling wave. 

a) Compute $|A_{III}|/ |A_{I}|$ and $|B_I| / |A_I|$ given 
$k_2L = 2\pi$. 

b) Compute the two qualities when $k_2L = \pi$

\new{Solution}
The wave must be continuous and smooth at the boundaries, i.e. 
$x = 0, L$. Thus, we write 

\begin{align*}
    A_1(0, t) + B_1(0, t) = A_2(0, t) + B_2(0, t)\\
    (A_1(0, t) + B_1(0, t))' = (A_2(0, t) + B_2(0, t))'\\
    A_2(L, t) + B_2(L, t) = A_3(L, t)\\
    (A_2(L, t) + B_2(L, t))' = (A_3(L, t))'
\end{align*}

We also know that each complexified $A, B$ must be in the follwing form.

\[
    \tilde A_1(x, t) = a_1 \Ixp{(\omega t - k_1 x)}
    \textAnd 
    \tilde A_2(x, t) = a_2 \Ixp{(\omega t- k_2 x)}
    \textAnd
    \tilde A_3(x, t) = a_3 \Ixp{(\omega t - k_1 x)}
\]

\[
    \tilde B_1(x, t) = b_1 \Ixp{(\omega t + k_1 x)}
    \textAnd 
    \tilde B_2(x, t) = b_2 \Ixp{(\omega t + k_2 x)}
\]
\color{red}
Note that it is convinient to use $\omega t - kx$ in the exponent, 
for we wish to extract the $\Ixp{\omega t}$ for any time $t$!
\color{black}

Realifying the forms and plugging in yields the following four formulas. 
\begin{align}
    a_1 \Ixp{\omega t} + b_1 \Ixp{\omega t}
     = a_2 \Ixp {\omega t} + b_2 \Ixp{\omega t}
    \nonumber \\
    -k_1a_1 \Ixp{\omega t} + k_1b_1 \Ixp{\omega t}
     = -k_2 a_2 \Ixp{\omega t} + k_2 b_2 \Ixp{\omega t}
    \nonumber  \\
     a_2 \Ixp{(\omega t - k_2 L)} + b_2 \Ixp{(\omega t + k_2 L)}
     = 
     a_3 \Ixp{(\omega t - k_1 L)}
    \nonumber \\
     -k_2a_2 \Ixp {(\omega t - k_2 L)}+ k_2b_2 \Ixp{(\omega t + k_2 L)}
     = 
     -k_1 a_3 \Ixp{(\omega t - k_1 x)}
\end{align}

Let $k_2 L = 2\pi$. Equations in (1) reduce to 
\begin{align}
    a_1 \Ixp{\omega t} + b_1 \Ixp{\omega t}
     = a_2 \Ixp {\omega t} + b_2 \Ixp{\omega t}
    \nonumber \\
    -k_1a_1 \Ixp{\omega t} + k_1b_1 \Ixp{\omega t}
     = -k_2 a_2 \Ixp{\omega t} + k_2 b_2 \Ixp{\omega t}
    \nonumber  \\
     a_2 \Ixp{(\omega t)} + b_2 \Ixp{(\omega t)}
     = 
     a_3 \Ixp{(\omega t - k_1 L)}
    \nonumber \\
     -k_2a_2 \Ixp {\omega t }+ k_2b_2 \Ixp{\omega t }
     = 
     -k_1 a_3 \Ixp{(\omega t - k_1 x)}
\end{align}

Let $t = 0$. 
Comparing the equations in pairs as of the 1st 3rd, and 2nd 4th, we deduce 
\[
    a_1 + b_1 = a_3 \Ixp{k_1 L}
\]
\[
    -a_1 + b_1 = -a_3 \Ixp{k_1 L}
\]

Thus, $b_1 = 0$ and $|a_3|/|a_a| = |\Ixp{k_1L}| = 1$

Let $k_2 L = \pi$. Equations in (1) reduce to
\begin{align}
    a_1 \Ixp{\omega t} + b_1 \Ixp{\omega t}
     = a_2 \Ixp {\omega t} + b_2 \Ixp{\omega t}
    \nonumber \\
    -k_1a_1 \Ixp{\omega t} + k_1b_1 \Ixp{\omega t}
     = -k_2 a_2 \Ixp{\omega t} + k_2 b_2 \Ixp{\omega t}
    \nonumber  \\
     a_2 \Ixp{(\omega t)} + b_2 \Ixp{(\omega t)}
     = 
     -a_3 \Ixp{(\omega t - k_1 L)}
    \nonumber \\
     -k_2a_2 \Ixp {\omega t }+ k_2b_2 \Ixp{\omega t }
     = 
     k_1 a_3 \Ixp{(\omega t - k_1 x)}
\end{align}



Again, compare the equation in pairs as we did for the previous case. 
\[
    a_1 + b_1 = -a_3 \cos(k_1 L)
\]
\[
    -a_1 + b_1 = a_3 \cos(k_1 L)
\]

Thus, $b_1 = 0$ and $|a_3/a_1| = 1$

To sum up, the wave is amplified by a factor of $\sec(k_1 L) > 1$ 
after passing the middle segment. If $k_2L = 2\pi$, the phase 
of the wave is preserved. If $k_2 L = \pi$, the phase of the wave 
is shifted by $\pi$. In both cases, there is no reflection in the 
original string. Regadless of $k_2 L$, 

\[
    \boxed{
        |B_I|/|A_I| = 0
        \textAnd 
        |A_{III}|/|A_I| = 1
    }
\]

\new{Q4} BNC line 

a) Describe the reflected pulse on a BNC cable where one 
of the ends are open or shorted. 

\halfFigure{Q4_bnc.png}

When the end is shorted, the voltage of the end must be zero, 
but the current is allowed to change. The current is preserved 
and the voltage is reversed. When the end is open, the voltage 
difference can change but the current between the two wires are 
fixed to be zero. Hence the voltage is preserved and the current 
is reversed. 

b) Use the notion of Impedence to explain your results above 

The voltage wave of the shorted end 
or the current wave of the open end
 can be considered as a string connected 
to another string with infinite impedence. 
\[
    Z_2 = \infty
\]
\[
    R = \frac{Z_1 - Z_2} {Z_1 + Z_2} = -1
\]
So the wave is reversed. 

The current wave of the shorted end 
or the voltage wave of the open end
can be considered as a string connected 
to another string with zero impedence. 
\[
    Z_2 = 0
\]
\[
    R = \frac{Z_1 - Z_2} {Z_1 + Z_2} = 1
\]
So the amplitude is preserved. 

If we fix $Z_{open} = \infty$ and $Z_{short} = 0$ 
for both current and voltage, 
we see from the results above that the 
\color{red}
reflection 
formula works only for the current wave but not 
the voltage wave. 
\color{black}

\new{Remark} Image waves 
To remember fixed and free ends, consider image waves 
traveling on the other side of the string. 

c) Use the differential equations that relate the voltage 
and the current wave to find an impedence value that works 
for the voltage current. 

From our results above, for the current wave, 
\[
    R_I = \frac{Z_1 - Z_2}{Z_1 + Z_2}
\]

We also know that the following differential equation 
holds by charge conservation. 
\[
    - \frac {\partial I } {\partial x} 
    = C_0 \frac {\partial V} {\partial t}
\]
The incident and reflect current wave is known to us. 
\[
    \tilde I_i = I_0 \Ixp{(kx - \omega t)} 
    \textAnd 
    \tilde I_r = R_I I_0 \Ixp{(kx + \omega t)}
\]
Likewise, we can express the incident and 
reflect voltage wave. 
\[
    \tilde V_i = V_0 \Ixp{(kx - \omega t)} 
    \textAnd 
    \tilde V_r = R_V V_0 \Ixp{(kx + \omega t)}
\]
Relate $I_i, V_i$ and $I_r, V_r$ using the 
differential equation. 
\[
    -kI_0 = -\omega C_0 V_0 
    \textAnd 
   - kR_II_0 = \omega C_0 R_V V_0
\]
Divide RHS by LHS and multiply by $-1$. 
\[
    \boxed{
    R_V = -R_I = -\frac{Z_1 - Z_2}{Z_1 + Z_2}
    }
    \]

\new{Q5 Dispersion relation and LC transmission line}

Consider a LC transmission line comprised of capacitors 
connected in parallel. Each of the capacitors are connected with 
an inductor. Let the inductance and the capacitance 
be $L$ and $C$. Remember the mechanical analog of the circuit. 

\[
  \begin{split}
    q \mapsto x \\
    L \mapsto m \\
    C \mapsto 1/k\\ \textrm{k string constant}
  \end{split}  
\]

We can compute the angular frequency of the circuit using the 
dispersion relation. Beware that $k$ is the wavenumber, not 
the spring constant. 

\[
    \omega = 2\omega_0 \sin (ka / 2 )
\]

The natural frequency of the circuit is 
\[
    \omega_0 = \sqrt{
        \frac k m
    } \mapsto \sqrt{\frac 1 {LC}}
\]

At continum, $a \rightarrow 0$. We approximate the sin as 
an identity function. 

\[
    \omega \approxeq 2\omega_0 \frac {ka} 2
    = \sqrt{\frac 1{LC}}  a k 
    = \sqrt{\frac {a^2}{LC}}  k 
    =  \sqrt{\frac 1{L_0C_0}}  k 
\]

The phase velocity is 
\[
    \boxed{
    v_p = \omega / k = 
\sqrt{\frac 1{L_0C_0}}
    }
    \]
\new{Q6 Alternate derivattion of the impedence of BNC cable}
\halfFigure{Q6_circuit.png}
Let the impedence of the upper and the lower circuits be $Z_1$ and $Z_2$ 
respectively. We know that $Z_1 = Z_2$. Also, 
\[
    Z_1 = i\omega L + \frac1 {i\omega C + \frac 1 {Z_\infty}}
    \textAnd 
    Z_2 =  Z_\infty
\]
The equality of the two impedences imply 
\[
    i\omega L + \frac1 {i\omega C + \frac 1 {Z_\infty}}
    =
     Z_\infty
\]
And with some algebra, we deduce 
\[
    Z_\infty ^2 - i\omega L Z_\infty - \frac L C = 0
\]
Apply the quadratic formula to compute $Z_\infty$. 
\[
    \boxed{
    Z_\infty = \frac {i\omega L} 2
    + \sqrt{
        \frac L C - \left(
            \frac {\omega L} {2}
        \right)^2
    }
    }
\]

Now, we wish to take the continum limit of this impedence to find 
the impedence of the BNC cable. Let $L_0, C_0$ be the impedence 
and capacitence per unit length of the cable. Consider a 1m segment of the 
cable. The total inductance and the capacitance is 
\[
    L_{tot} = (1m) L_0 \textAnd C_{tot} = (1m) C_0
\]
The continum case can be thought as disecting this 1m cable into 
n segments with inductance and capacitance $L_{tot}/n$ and $C_{tot}/n$. 
Plug this into our result from (a). 
\[
    Z_\infty = \frac {i\omega L_{tot}} {2n}
    + \sqrt{
        \frac {L_{tot}/n} {C_{tot}/n} - \left(
            \frac {\omega L_{tot}} {2n}
        \right)^2
    }
\]
Send $n \rightarrow \infty$. Observe that $L_{tot}/C_{tot} = L_0/C_0$. 
\[
    \boxed{
    Z_\infty = \sqrt{
        \frac {L_0}{C_0}
    }
    }
\]

\new{Q7} Malus' Law

a) A light passes through three polarizers. 
The pass axis of the polarizers are arranged 
0, 45, 90 degrees with respect to the electric field of 
the light. Compute the intensity of the light after 
it passes the polarizers. 

The first polarizer does not affect the light. Apply 
Malus's Law twice. 

\[
    I'/I = (\cos(\pi / 4))^ 2 = \boxed{1/ 4}   
\]

b) Consider $N+1$ polarizers that rotate 
$\pi / N$ each. 

Let $I_n$ be the intensity of the light after 
passing the nth polarizer. By Malus' Law, 
\[
   I_n = (\cos(\pi / 2N)^2) I _{n - 1} 
\]

The first polarizer does not affect intensity. After 
passing $N$ polarizers, the intensity is multiplied 
by the cos square term $N$ times. 

\[
    \boxed{
    I_{N + 1}/ I_0 = 
    \cos(\pi / 2N)^{2N}
    }
\]

c) 
\halfFigure{Q7_Malus.png}


d) Approximate the intensity at the limit $N \rightarrow \infty$
\[
    \begin{split}
    I_{N + 1}/ I_0 = 
    \cos(\pi / 2N)^{2N}
    =
    (1 - \sin(\pi / 2N)^2)^{2N}
    \\
    \approxeq 
    (1 - (\pi / 2N)^2)^{2N}
    \approxeq 
    1 + (\pi / 2N)^2 \cdot 2N 
    \approxeq  1
    \end{split}
\]

So the intensity does not change. 

\end{document}