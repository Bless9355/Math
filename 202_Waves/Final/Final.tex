\documentclass{article}
\usepackage{amsfonts}
\usepackage{amsthm}
\usepackage{amssymb}
\usepackage{amsmath}
\usepackage{graphicx}
\usepackage{subcaption}
\usepackage{xcolor}


\usepackage{mathtools}



\DeclarePairedDelimiter\bra{\langle}{\rvert}
\DeclarePairedDelimiter\ket{\lvert}{\rangle}
\DeclarePairedDelimiterX\braket[2]{\langle}{\rangle}{#1\,\delimsize\vert\,\mathopen{}#2}



\newcommand{\new}[1]{
    \vspace{2mm}
    \noindent
    \textbf{
    \underline{#1}}
}

\def\calO{{\mathcal{O}}}
\def\th{{\theta}}
\def\_{{\hspace{1mm}}}
\def\<{{\langle}}
\def\>{{\rangle}}


\DeclareMathOperator{\sinc}{sinc}


\newcounter{problemcnt}
\setcounter{problemcnt}{0}

\newcommand{\Problem}{{
    \vspace{5mm}
    \stepcounter{problemcnt}
    \noindent
    \arabic{problemcnt}. 
}
}

\newcommand{\nProblem}[1]{
    \vspace{5mm}
    \noindent
    \setcounter{problemcnt}{#1}
    \arabic{problemcnt}. 
}


\newcommand{\Proof}{{
    \vspace{2mm}
    \noindent
    \textbf{
    \underline{Proof}}
}
}

\newcommand{\textOr}{
    {
        \hspace{5mm}
        \textrm{or}
        \hspace{5mm}
    }
}

\newcommand{\textAnd}{
    {
        \hspace{5mm}
        \textrm{and}
        \hspace{5mm}
    }
}

\newcommand{\textThen}{
    {
        \hspace{5mm}
        \textrm{then}
        \hspace{5mm}
    }
}


\newcommand{\Ixp}[1]{
    {
        e^{i#1}
    }
}

\newcommand{\partialderiv}[2]{
    {
        \frac{\partial #1}{\partial #2}
    }
}



\newcommand{\halfFigure}[1]{
\begin{center}
\includegraphics[width = .5\linewidth]{{#1}}
\end{center}
}

\newcommand{\fullFigure}[1]{
\begin{center}
\includegraphics[width = .9\linewidth]{{#1}}
\end{center}
}

\def\twobytwoMat(#1, #2, #3, #4){
    {
        \begin{bmatrix}
            {#1} & {#2}\\
            {#3} & {#4}
        \end{bmatrix}
    }
}

\def\twobyoneMat(#1, #2){
    {
        \begin{bmatrix}
            {#1}\\
            {#2}
        \end{bmatrix}
    }
}

\def\twobytwoDet(#1, #2, #3, #4){
    {
        \begin{vmatrix}
            {#1} & {#2}\\
            {#3} & {#4}
        \end{vmatrix}
    }
}



\begin{document}
\begin{center}
\LARGE
PHYS 202 Final Exam


\Large
Daniel Son
\end{center}

\newpage

\fullFigure{A1.png}

\new{Solution}

We know the following Taylor expansions. 
\[
    e^x \approxeq 1 + x + \frac {x^2} 2
    \textAnd 
    \ln(1 - x) \approxeq -\left(
        x + \frac {x^2} 2
    \right)
\]
Hence, 
\[
    f(x) = e^x \left(
        1 - \ln(1 - x)
    \right)
    \approxeq 
    \left(
        1 + x + \frac {x^2} 2 
    \right)
    \left(
        1 + x + \frac {x^2} 2
    \right)
\]
\[
    \approxeq 1 + 2x + x^2 \left(
        \frac 1 2 + 1 + \frac 1 2
    \right) + \cdots 
    = \boxed{
        1 + 2x + 2x^2
    }
\]
Just up to a leading order, we can also write $1 + 2x$. 
Plugging in $x = .1$ yields $f(x) = 1.22$ by computing 
$f$ through a calculator. By our approximation, we also 
get $f(x) \approxeq 1.22$. 
\hfill \qed

\fullFigure{A2.png}

\new{Solution for a, b}

We first write out the Newton's 2nd law for oscillators 
connected to one side of the wall. 

\[
    m \ddot{x} = -k_{eff} x
\]

We wish to derive an equation of this form for the two-wall oscillator. 
Here is a free body diagram for the two-wall oscillators. 


\newpage
\fullFigure{A2_diagrams.png}

Write out Newton's second law. Let $x$ be the 
raw position away from zero. 

\[
    m \ddot x = -k (x - l_0) + k (2l - x)
\]

We know that at $x = l$, the force exerted on the oscillator 
is zero. 

\[
    0 = -k(l - l_0) + k(2l - l)
\]

Thus, by subracting zero from the first equation, 

\[
    m \ddot x - 0 = 
    -k(x - l_0 - l + l_0) + k (2l - x - 2l + l)
\]
\[
    m \ddot x = -k(x - l) - k (x - l) = -2k (x - l)
\]

Relabel $x$ to be the displacement. The old $x - l$ 
becomes the new $x$. 

\[
    \boxed{
    m\ddot x = -2k} x \textOr 
    m \ddot x -k_{eff} x
\]

where $\boxed{k_{eff} = 2k}$. 
\hfill \qed

\new{Solution for c} We repeat a similar procedure 
for the vertical oscilator. Apply Newton's second law. 
Start with $x$ denoting the raw position of the mass. 
Suppose, at the equilibrium state, the top spring has length 
$l_1$ and the bottom string has length $l_2$. 
\[
    m \ddot x = m g  -k (-l_1 - l_2 + l_0 + x) - k (x - l_0) 
\]
At equilibrium, no force acts on the mass. 
\[
    0 = m g  -k (-l_1 - l_2 + l_0 + l_1) - k (l_1 - l_0) 
\]

Subtracting the second equation from the first, we again obtain, 
\[
    m \ddot x = -2k (x - l_1) 
\]
and relabeling $x$ to be the displacement from the equilibrium 
position $l_1$, 
\[
    m \ddot x = -2k x 
\]
Which is exactly the equation that we derived in the previous part. 
The fundamental frequency of the oscillator is 
\[
    \omega_0 = \sqrt {\frac {2k} m}
\]
and this value does not change when the oscillator is tilted so 
that the gravity affects the motion. 


\newpage
\fullFigure{A3.png}

\new{Solution for a)}
A wave passing through a birefringent material experiences 
different speed depending on the axis of measurement. 
Assume that the light passing through the medium is linearly polarized, 
entering the birefringent material in a manner where 
the magnitude of the slow component is equal to the fast component. 
\halfFigure{A3_pol.png}
The wave component that is passing through 
the slow axis will experience more phase shift compared to the 
component passing through the fast axis. Thus, if the phase 
difference experienced by the slow component is $\pi / 2$ 
more than the phase difference experienced by the fast component, 
the material acts like a quarter waveplate, circularly polarizing the light. 

\new{Solution for b)}
Let the width of the material be $w$. We compute the 
phase difference experienced by the fast and the slow component. 

\[
    \phi _{slow} = k_{slow} w = \frac{2\pi w} {\lambda_{slow}} 
    = \frac {2\pi w n_{slow}} {\lambda}
\]

By analogy, 
\[
    \phi_{fast} = \frac {2\pi w n_{fast}} {\lambda}
\]

Write out the phase difference. We wish the value to equal $\pi / 2$. 
\[
    \phi_{slow} - \phi_{fast} = \frac \pi 2 = 
     \frac {2\pi w (n_{slow} - n_{fast})} {\lambda}
\]

Fiddling over the constants, we deduce the following. 
\[
    \boxed{
    w = \frac {\lambda} {4 (n_{slow} - n_{fast})} \approxeq 16 \mu m
    }
\]

\new{Solutions for c)}
A circularly polarized light will yield half of the original intensity 
after passing a linear polarizer. This can be easily demonstrated\footnote{
    We can indeed show this for an arbitrary linear polarizer. 
    Let $M$ denote a matrix that passively changes the bases to 
    a desired linear polarizer. $M$ must be real and orthogonal. The 
    first entry of $M \twobyoneMat(1, i)$ describes the magnitude 
    of the electric field. This value equals to $M_{11} + iM_{22}$. 
    The magnitude of this component is $M_{11}^2 + M_{22}^2$, which 
    must equal to 1 because $M$ is known to be orthogonal. 
} through 
a simple Jones Calculus. Write out a Jones vector 
for some circularly polarized light with intensity 2. 
\footnote{
The intensity can be computed by multiplying the complex conjugate 
of the vector.} 
\[
    \twobyoneMat(1, i)
\]
Now apply a linear polarizer that has a pass axis of $\hat x$. 
\[
    \twobytwoMat(1, 0, 0, 0) 
\twobyoneMat(1, i)
= \twobyoneMat (1, 0)
\]
This light has an intensity $1$. Indeed this result generalizes for 
any axis. So to check circular polarization, place the linear polarizer 
in any orientation, and check if the intensity is halved. 

\hfill \qed

\newpage 
\fullFigure{A4.png}
\halfFigure{A4_diagram.png}
\new{Solution}
We wish to achieve TM polarizaton.\footnote{In fact, 
the magnitude for this specific case is not very different 
between TE and TM polarization}
This happens when the magnetic vector of the incident 
wave is parallel to the boundary plane, that is, 
perpendicular to the plane of incidence.\footnote{refer to Fowles p41-42} Here is a nice figure 
from Fowles p48. 

\halfFigure{Fowles_48.png}

Using our imaginary powers, we orient the dipoles of the ice 
and the glass to create this configuration. 

Send in the light through the Brewster's angle for perfect 
transmission. 
We measure the brewster's angle by the following formula. 
\[
    \boxed{
    \theta_{b} = \arctan\left(\frac {n_{glass}} {n_{ice}}\right) 
    \approxeq
    49^{\circ}
    }
\]

Also, we know that at the Brewster's angle, 
the reflected ray is perpendicular to the refracted ray.\footnote{Or, one can use Snell's law to find the angle of 
refraction}
Thus, $\phi = 41 ^ {\circ}$. 
Fowles also provides a transmission coefficient 
for TM polarization. 

\[
    t_s = \frac{2\cos(\theta)\sin{\phi}} {\sin (\theta + \phi) 
    \cos(\theta - \phi)} 
    \approxeq .87
\]

Squaring the transmission coefficient provides the power ratio 
that is transmitted into the glass. 

\[
    \boxed{
    P_t/P_0 = t_s^2 \approxeq .76
    }
\]

\newpage
\fullFigure{A5.png}
\halfFigure{A5_wingnut.png}
\new{Solution}
The word is Wingnut and it was mentioned in Mar 15, 2024. 

\newpage
\fullFigure{B1.png}

\new{Solution for a, b}
\halfFigure{B1_FBD.png}
Let $x_1, x_2$ be 
the horizontal displacement of the two masses from the equilibrium positon. 
Assuming the displacements are small, we approximate that all movements of the 
masses are entirely horizontal.
By this approximation, the vertical force component must equal zero. 
This way, we can measure the sum of the tension 
and gravity in simple terms. 

\[
    \vec T + \vec F_g = - mg \frac {x_1} l \hat{x}
\]
\footnote{$\hat{x}$ is a convention used instead of $\hat{i}$. It is 
a unit vector pointing in the direction of x. }

\newpage
Write out Newton's 2nd law for the two masses in the $\hat x$ direction. 
\[
    -k (x_1 - x_2) - \frac {x_1} l mg = m \ddot x_1
\]
\[
    -k (x_2 - x_1) - \frac {x_2} l mg = m \ddot x_2
\]

Assume the solution to be some complex exponential both for $x_1, x_2$. 
\[
    x_1 = Ae^{i \omega t} \textAnd x_2 = Be^{i \omega t}
\]

Plug in the solutions. After some lines of algebra, we obtain 
a system of linear equations for $A, B$. 

\[
    \left(-m\omega^2 + \frac{mg} l + k  \right) A - kB = 0
\]
\[
    -kA +\left(-m\omega^2 + \frac{mg} l + k\right) B = 0
\]

In matrix form, the equation nicely reduces to the following. 
\[
    \twobytwoMat(
 {\left(-m\omega^2 + \frac{mg} l + k \right)}, -k, 
 -k,  {\left(-m\omega^2 + \frac{mg} l + k \right)}
    )
    \twobyoneMat(A, B) = \twobyoneMat(0, 0)
\]

For the amplitudes to be nonzero, the 2x2 matrix must not be 
invertible. This means that the determinant of the matrix must be zero, 
which leads to the following identity. 

\[
     \left(-m\omega^2 + \frac{mg} l + k \right)^2 - k^2 = 0
\]
\[
     \left(-m\omega^2 + \frac{mg} l \right)\left(-m\omega^2 + \frac{mg} l + 2k \right) = 0
\]

For the identity to hold, positive $\omega$ must take one 
of the two values. 

\[
    \omega = \sqrt{\frac g l} \textOr \sqrt{\frac g l + \frac {2k} m}
\]

The first frequency, which has a lower value, corresponds to the 
symmetric mode. The two masses move parallel to each other at the 
same magnitude. The second frequency, with a higher value, corresponds 
to the antisymmetric mode. The two masses move opposite to each other 
at the same magnitude. 

\hfill \qed

\new{Solution for c}
Any general movement of the oscillator can be described 
by a linear combination of the two normal modes. Let the 
complexified $x_1, x_2$ to be written in complex amplitudes $A, B$. 
For simplicity, omit the tilde.\footnote{
    In fact, $A, B$ must be real. We know that the 
    velocity of the mass at $t = 0$ must be zero for $x_1, x_2$. 
    This means $\dot x_1 = \dot x_2 = 0$. With some algebra, 
    one can deduce $Im (\omega_s A) = Im (\omega_f B) = 0$. 
    All frequencies are real, and thus $Im(A) = Im(B) = 0$. 
}

\[
    \ddot x_1 = Ae^{i\omega_s} + Be^{i\omega_f}
    \textAnd
    \ddot x_2 = Ae^{i\omega_s} - B e^{i\omega_f}
\]

At time $t = 0$, the displacement of the two masses are 
measured as $-.05d$ and $.1d$. Write down the system in matrix form. 

\[
    \twobytwoMat(1, 1, 1, -1) \twobyoneMat(A, B) = 
    \twobyoneMat(-.05d, .1d)
\]

Multiply the inverse of the 2x2 matrix to compute $A, B$. 
\[
    \twobyoneMat(A, B) = \twobytwoMat(1, 1, 1, -1) ^{-1} \twobyoneMat(-.05d, .1d)
    = \frac d {40} \twobyoneMat(1, -3)
\]

Finally, plug in the values and take the real part to obtain the 
displacements $x_1, x_2$. 

\[
    x_1 = \frac d {40} (\cos(\omega_s t) - 3 \cos (\omega_f t))
\]
\[
    x_2 = \frac d {40} (\cos(\omega_s t) + 3 \cos (\omega_f t))
\]

where \[(\omega_s, \omega_f) = \left(\sqrt{\frac g l} , \sqrt{\frac g l + \frac {2k} m}\right)\]

As a sanity check, we verify that the velocity of both masses 
at $t = 0$ must vanish, for the time derivative would be 
some combination of $\sin$ waves. It is trivial to see that 
the displacement at $t = 0$ is also satisfied in our solution. 
\hfill \qed

\newpage 
\fullFigure{B2.png}

\new{Solution for a}

We ignore damping for our calculations. We deduce the following. 
\[
    m \ddot x = -kx + F(t)
\]

\new{Solution for b}

We first observe that the function $F$ cannot be constantly nonzero. 
If that is the case, the oscilaltor will not reach a stable state, but 
will continuously shift to one direction. 

Complexify the equation, and suppose the force takes the form of 
\[
    F(t) = c_n e^{i2\pi n t/T}
\]
Assume the displacement to take a similar form. 
\[
    x(t) = x_n e^{i2\pi n t/T}
\]
The differential equation simplifies to a relation between the constants, 
independant of time. 
\[
    -m\left(
        \frac {2\pi n} T
    \right)^2 + k = c_n / x _n
\]
\[
     x_n = \frac{c_n } {
        k - m\left(
            \frac {2\pi n} T
        \right)^2
    }
\]

By the fourier series expansion, it is possible to write out 
any $F(t)$ that is physical (which means continuous, and smooth) 
as a sum of complex exponentials. 

Using Fourier decomposition, decompose $F(t)$ into the following. 
\[
    F(t) = \sum_{n \in \mathbb Z} c_n e^{i2\pi n t/T}
\]\footnote{To compute $\tilde c_n$, refer to Boaz}

Then, the displacement function is given as 
\[
    \boxed{
    x(t) = \sum_{n \in \mathbb Z}  \frac{ c_n } {
        k - m\left(
            \frac {2\pi n} T
        \right)^2
    } e^{i2\pi n t/T}
    }
\]

\new{Solution for c}
\newcommand{\mydenom}{
{
        k - m\left(
            \frac {2\pi n} T
        \right)^2
    }
}
We know one thing about $F(t)$-it is a real valued function. 
For real valued function, we know that the fourier coefficient 
$ c_n$ satisfies the following relation. 
\[
     c_{-n} = ( c_{n})^*
\]

To see why this is true, consider the equations for $c_n$. 
Up to constants, 
\[
    c_n \sim \int_{-T/2} ^{T/2} e^{-2\pi i n t / T} F(t) dt
\]
\[
    c_{-n} \sim \int_{-T/2} ^{T/2} e^{2\pi i n t / T} F(t) dt
\]
Since $F$ is real, it is easy to see that the two coefficients are 
complex conjugates of each other. 

Now, write out the equation for $x(t)$. 

\[
    x(t) = \sum_{n = -\infty} ^ \infty \frac{c_n e^{2\pi i n t /T}} \mydenom
    = \frac {c_0} k + \sum_{n = 1}^\infty \frac {c_n e^{2\pi i n t/T}} \mydenom
    + \sum_{n = 1}^\infty c_{-n} \frac {e^{-2\pi i n t/T}} \mydenom
\]
\[
    = \frac{c_0} k + \sum_{n = 1}^\infty \frac{c_n e^{2\pi i n t/T}} \mydenom + 
    \sum_{n = 1}^\infty (c_{n})^* \left(\frac{e^{2\pi i n t/T}} \mydenom\right)^*
    = \frac{c_0} k + 2 \Re \{\sum_{n = 1}^\infty \frac{c_n e^{2\pi i n t/T}} \mydenom\}
\]\[
    =
    \frac {c_0} k + \sum_{n = 1}^\infty 2\Re\{\frac{c_n e^{2\pi i n t / T}}\mydenom\}
    =
    \boxed{
        \frac {c_0} k + \sum_{n = 1}^\infty 2|c_n| \frac{\cos(2\pi n t / T + \phi_n)} \mydenom
    }
\]\footnote{
    Though this result seems more complicated than that of part b, 
    it reduces the number of computations by a factor of 1/2. 
    Suppose we are approximating $x(t)$ up to the first n terms. Each summand 
    of this result corresponds to two summands of the result in b. 
}

Where $\phi_n = \arg(c_n)$.  \hfill \qed

\newpage 
\fullFigure{B3.png}

\new{Solution for a}
Considering the three slits as a bundle, there must be a large envelope 
function that encloses the intensity projected on the wall. It is likely 
that some sinusoid that decays at the edges is involved. The actual intensity 
function will be a collection of fringes under the envelope. The 
fringes are formed by interference. 

\new{Solution for b}. 
\fullFigure{B3_def.png}
Also, define $I_0$ to be the intensity when $\theta = 0$.
This will be the point on the center of the wall. That is, 
if we draw a normal line from the slit, the intersection between 
the line and the screen will be the point where $\theta = 0$. 
The interferecne 
is entirely constructive here. Let $k$ be the wavenumber and define 
\[
    \delta = kd\sin(\theta)
\]

For convinience, set the top ray to be the reference ray. 
We will compute the relative phase and magnitude changes 
assuming that the top ray to have phase zero at the screen. 
The two bottom rays will have additional phase difference incurred 
by extra distance traveled, assuming $\theta > 0$. 
At the point of the screen corresponding to angle $\theta$, 
we can write the intensity as follows. 
\[
    \left|
        \frac I {I_0}
    \right|
    = 
    \left|\frac {1 + e^{i\delta} + e^{3i\delta}} 3 \right|^2
    = \left(
        \frac {1 + e^{i\delta} + e^{3i\delta}} 3
    \right)\left(
        \frac {1 + e^{-i\delta} + e^{-3i\delta}} 3
    \right)
\]
With some lines of algebra, we graciously derive the following 
\[
    \left|
        \frac I {I_0}
    \right|
    = \frac 1 3 + \frac 2 9 (
        \cos(\delta) + \cos(2\delta) + \cos(3\delta)
    )
\]

\new{Solutiion for c}

Here are some phasor diagrams. The blue arrows are component 
vectors, and the red vector is the sum corresponding to 
the square root of the intensity. 
\fullFigure{B3_phasor.png}

From these phasor diagrams, we deduce the following intensity plot. 
\fullFigure{B3_plot.png}

The red points mark the maximum, and the blue points mark the minimum. 
With some algebra, we can deduce that in the window of $\delta \in 
[0, 2\pi]$, the intensity ratio takes exactly seven extremas.\footnote{
    Please see B3c-continued in scratchwork.
}

\newpage
\fullFigure{B4.png}

\new{Solution for a}
\fullFigure{B4_laser.png}

Also, note that 
\[
    FSR_{lazer} = 4FSR_{cavity}
\]

\new{Solution for b}
The light traverses the cavity four times. Also, the spikes 
occur when the rays interfere constructively. Thus, for 
one free spectral range movement, the cavity lens either 
comes close or goes further away by $\lambda/4$. 

Over one cycle of the oscilloscope, the PZT voltage is applied 
for twelve FSR's. $\lambda/4 \cdot 12 = 3\lambda$. We conclude 
that the PZD moved by $3\lambda$. 

\hfill \qed

\new{Solution for c}
The laser cavity is made out of a tube surrounded by two high 
reflectivity mirrors on each end with a gain medium at the center. 

Let $d$ be the length of the laser cavity and $R$ the 
distance between the mirrors for the Faby-Perot cavity. 
As stated in part a, the FSR of the laser is four times 
that of the confocal Faby Perot Interferometer\footnote{We 
assume confocal, because the question said "just as we did in lab"}. 
Thus, we derive the following identity. 
\[
    \frac c {2d} = 4\cdot\frac c {4R} 
    \textAnd 
    \boxed{R = 2d}
\]

\new{Solution for d}
We wish to derive an equation for the wavelength of the laser. 
The laser cavity can be considered as a fixed end string. 
\fullFigure{B4_n.png}

From the boundary condition, we write 
\[
    \frac {n\lambda} 2 = d = \frac R 2
    \textOr 
    \lambda = \frac R n
\]

Here is a diagram of the Michelson Interferometer. 
\halfFigure{B4_Mich.png}

The horizontal wave travels by an extra length of 
$\Delta x$, which is displacement of the moving translation 
stage. The extra phase difference of this ray compared 
to the vertical ray is 
\[
    \Delta \phi  = 2\Delta x k = \frac {4\pi \Delta x} \lambda
    = \frac {4\pi n \Delta x} R
\]\footnote{
    An attentative reader might ask why we are ignoring 
    the phase retardation incurred by the reflection and 
    the transmission of the beam splitter. Note that the 
    vertical ray is once reflected and transmitted, and 
    the horizontal ray is once transmitted and reflected, 
    in that order. This means that the reflections 
    and transmissions do not affect the relative phase difference. 
}

Let $I_0$ be the intensity measured at the bullseye pattern 
when the length of the two legs are equal, i.e. $\Delta x = 0$. 
Setting the electric field magnitude of one of the rays as $E_0$, 
we deduce that $I_0 \sim (2E_0)^2 = 4E_0^2$. 
When the translation 
stage has moved by $x$, we can compute the intensity ration. 
For simplicity, write $x, \phi$ instead of $\Delta x, \Delta \phi$

\[
    I_{x} /I = \left|\frac {E_0(1 + e^{i\phi})} {2E_0}\right|^2
    = \left|\frac {1 + e^{i\phi}} 2\right|^2
    = \left(\frac {1 + e^{i\phi}} 2\right)\left(\frac {1 + e^{i\phi}} 2\right)^*
\]
\[=
    \left(\frac {1 + e^{i\phi}} 2\right) \left(\frac {1 + e^{-i\phi}} 2\right)
    = \frac {2 + e^{i\phi} + e^{-i\phi}} 4 
    = \frac 1 2 + \frac 1 2 \cos(\phi)
\]

Replace $\phi$ with the equation that we derived previously. 

\[
    \boxed{\frac {I_x} {I_0} = \frac 1 2 + \frac 1 2 \cos\left(\frac{4\pi n x} R\right)}
\]

Qualitatively, the ratio above describes the sharpness of the fringes in 
the bullseye pattern. When the Intensity value reaches its peak, 
for example when $x = 0$, then the distinction between the 
fringes would be most sharp. However, as $x$ slowly increases, 
there will be a point where the interference between the two rays 
would be completely destructive. In that case, the bullseye pattern would 
be gone, and it would be hard to distinguish between the fringes 
around this region. 

\newpage

\fullFigure{B5.png}


\new{Solution for a}

This might sound like a boring solution, but simply take the 
fourier transform of $E(t)$. 
\newcommand{\fourier}[2]{
\int_{#1}^{#2} e^{-ikt} E(t) dt
}

\[
    \hat E(k) = \frac 1 {\sqrt{2\pi}}
    \fourier{-\infty}{\infty}
    = \frac 1 {\sqrt{2\pi}}\left(
    \fourier{0}{\tau} + \fourier {T}{T + \tau}\right)
\]


\newcommand{\expterms}[3] {
    %exponent, from, to
\int_{#2}^{#3} \frac {e^{#1}} 2 dt
}


\newcommand{\dblexpterms}[3] {
    %exponent, from, to
\int_{#2}^{#3} {e^{#1}} dt
}



Using Euler's formula, expand the $\cos$ function into 
complex exponentials. We write the following. 

\[
    =
    \frac 1 {\sqrt {2\pi}}
    \left(
    \expterms{i(\omega_m - k) t}{0} {\tau} +
    \expterms{-i(\omega_m + k) t}{0} {\tau} +
    \expterms{i(\omega_m - k) t}{T} {T + \tau} +
    \expterms{-i(\omega_m + k) t}{T} {T +\tau} +
    \right)
\]

We wish to unify all the ranges of integration. Focus on 
the third summand. Apply a simple u-substitution, $u = t - T$. 
\[
    \expterms{i(\omega_m - k) t}{T}{T + \tau} = 
    \int_{u = 0}^{\tau} \frac {e^{i(\omega_m - k) (u + T)}} 2 du
    = e^{i(\omega_m - k)T} \expterms{i(\omega_m - k) t}{0}{\tau}
\]
We can simplify the last summand in a similar manner. We proceed with 
\[
    \hat E(k) = 
    \frac 1 {\sqrt{2\pi}} \left(
        (1 + e^{i(\omega_m - k)T})
         \expterms{i(\omega_m - k) t}{0} {\tau}
         + 
          (1 + e^{-i(\omega_m + k)T})
\expterms{-i(\omega_m + k) t}{0} {\tau} 
    \right)
\]
\begin{equation*}
    \begin{split}
    =\frac 1 {\sqrt{2\pi}}
    \bigg(
    \cos((\omega_m - k)T/2) e^{i(\omega_m - k)T/2}
    \dblexpterms{i(\omega_m - k)}{0}{\tau}
    +\\
    \cos((\omega_m + k)T/2) e^{-i(\omega_m + k)T/2}
    \dblexpterms{-i(\omega_m + k)}{0}{\tau}
    \bigg)
    \end{split}
\end{equation*}

We now interpret this result. Square 
the fourier transform. 
We end up with three terms, which is the square of 
the first and the second summand, and a cross term 
which is twice the multiple of the two summands. 

Pay attention to the two integrals. 
The particle experiences 
enough oscillations within the two short regions. This means 
that the complex integral 
\[
 \dblexpterms{i(\omega_m - k)}{0}{\tau}
 \textAnd 
\dblexpterms{-i(\omega_m + k)}{0}{\tau}
\]
is nonzero only if the exponents are close enough to zero. 
This implies that the cross term can be ignored. This is because 
$\omega_m$ is large enough that the neighborhood of 
$+\omega_m$ does not intersect with the neighborhood of $-\omega_m$. 
If the frequency lied outside the neighborhood of one region, one 
of the two integrals will vanish. 

Focus on the two square terms. These two integrals will define an envelope, 
each located 
around $k = +\omega_m, -\omega_m$. Inside the two envelopes, 
the E field magnitude is determined by the two cosine functions. 
Nonetheless, we want the intensity, so square the two cosine functions. 
The intensity fringes are described by 
\[
    \cos^2((\omega_m - k)T/2) \textAnd \cos^2((\omega_m + k)T/2)
\]\footnote{We can ignore the exponential coefficients, for we are taking 
the square magnitude. Also, we are not interested in the magnitude 
of these fringes, but just the frequencies. }
Using trig identity, rewrite the two terms. 
\[
    \frac{\cos((\omega_m - k)T) + 1} 2 \textAnd \frac{\cos((\omega_m + k)T) + 1} 2
\]

Both of the functions have a width of $\frac {2\pi} T$ in frequency 
space. This means that the peak-to-peak distance of the two 
frequencies that spike are$\frac {2\pi} T$ away from each other.  



In light of our descoveries, we plot the power specturm. 
\fullFigure{B5_PS.png}

\new{Solution for b}
The asked quantities were all derived in part a. 
The width of the entire spectrum is about $\boxed{2\omega_m}$, 
so it is determined by the microwave frequency. 
The peak-to-peak distance of the fringe is $\boxed{2\pi / T}$, so 
it is determined by the time the atom is airborne. 
Also, we note that as the timeslot $\tau$ decreases, the peaks 
of the envelope function becomes more wider. If $\tau$ increases, 
lesser and lesser fringes will fit into the envelope. 

\newpage
\fullFigure{B6.png}
\new{Solution for a}
\fullFigure{B6_lens.png}

\new{Solution for b}
The diagram suggests that the image of the two lens 
occur at the same position. We justify this rigorously 
by ray matricies. The ray matrix for a lens of focal length $f$ is 

\[
    M_f = \twobytwoMat(1, 0, -1/f, 1)
\]

Thus, 
\[
    M_{-2f} = \twobytwoMat(1, 0, 1/(2f), 1)
    \textAnd 
    M_{f} = \twobytwoMat(1, 0, -1/f, 1)
\]

Label the two different lens configuration as top and bottom, 
arranged in part a. Compute the corresponding ray matrix for 
each configuration. 

\[
    M_t = M_{-2f}M_{f} = \twobytwoMat(1, 0, -1/(2f), 1)
\]
\[
    M_b = M_{f}M_{-2f} = \twobytwoMat(1, 0, -1/(2f), 1)
\]

So both configurations act like a converging lens that has 
a focal length of $2f$. Recall the formula 
\[
    \frac 1 s + \frac 1 {s'} = \frac 1 f
\]
We know the focal length $f$ and the location of the image $s$. Thus, 
\[
    \frac 1 {3f} + \frac 1 {s'} = \frac 1 {2f}
    \textOr 
    2s' + 6f = 3s' 
    \textOr 
    s' = 6f
\]
So, for both matricies, the image is created at a distance $6f$ from 
the lens, opposite to the original image. 
\hfill \qed

\new{Solution for c}
We know that the magnifying factor $m$ is $-s'/s$. By the formula, 
our image has a magnifying factor of $-6f/f = -6$. So the image 
for both orientations would be six times the original height, and 
the image would be inverted. 
\hfill \qed


\end{document}