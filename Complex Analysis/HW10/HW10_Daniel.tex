\documentclass{article}
\usepackage{amsfonts}
\usepackage{amsthm}
\usepackage{amsmath}
\usepackage{amssymb}
\usepackage{wasysym}


\newcommand{\new}[1]{
    \vspace{2mm}
    \noindent
    \textbf{
    \underline{#1}}
}

\def\calO{{\mathcal{O}}}
\def\ZZ{{\mathbb{Z}}}
\def\RR{{\mathbb{R}}}
\def\_{{\hspace{1mm}}}

\def\contradiction{{\lightning}}



\newcounter{problemcnt}
\setcounter{problemcnt}{0}

\newcommand{\Problem}{{
    \vspace{5mm}
    \stepcounter{problemcnt}
    \noindent
    \arabic{problemcnt}. 
}
}

\newcommand{\nProblem}[1]{{
    \noindent
    \setcounter{problemcnt}{#1}
    \arabic{problemcnt}. 
}
}

\newcommand{\Proof}{{
    \vspace{2mm}
    \noindent
    \textbf{
    \underline{Proof}}
}
}

\newcommand{\textOr}{
    {
        \hspace{5mm}
        \textrm{or}
        \hspace{5mm}
    }
}

\newcommand{\textAnd}{
    {
        \hspace{5mm}
        \textrm{and}
        \hspace{5mm}
    }
}

\newcommand{\Exp}{
    {
        \textrm{Exp}
    }
}

\begin{document}

\Problem
Find a functional equation for the following function:

\[
    G(s) := \int_{0}^{\infty}\Exp[-x^2]x^{s-1}dx
\]

\new{Solution}
We first observe the following derivative:

\[
    \frac{d}{dx}e^{-x^2} = -2xe^{-x^2}
\]

Hence:

\[
    \int xe^{-x^2}dx = -\frac{e^{-x^2}}{2} + C
\]

With this indefinite integral in mind, we integrate G 
by parts. Apply the following substitutions:

\[
    u = x^{s-2} \textAnd du = (s-2)x^{s-3}
\]
\[
    dv = x\Exp[-x^2]dx \textAnd 
    v = -\frac{\Exp[-x^2]}{2}
\]

Integrate $G$:
\[
    G(s) = uv \bigg|_{x = 0}^\infty - \int_{x = 0}^{\infty}vdu
\]
\[
    = \left[
        -\frac{x^{s-2}\Exp[-x^2]}{2}
    \right]_0^\infty
    -\int_0^\infty \left(
        -\frac{(s-2)\Exp[-x^2]x^{s-3}}{2}
    \right)dx
\]
\[
    = \left[
        -\frac{x^{s-2}\Exp[-x^2]}{2}
    \right]_0^\infty
    +
    \frac{(s-2)}{2}
    \int_0^\infty 
    \left(
        \Exp[-x^2]x^{s-3}
    \right)dx
\]

For the sake of convergence of the first summand,
we assume $s\geq2$. As for $s = 2$, we evaluate:

\[
    G(2) = \int_0^{\infty}xe^{-x^2}dx = -\frac{e^{-x^2}}{2}\bigg|_0^\infty
    = 1
\]

As for $s > 2$, we compute the equality obtained by the by parts
integration. Notice that the first summand converges to zero. 
Hence:

\[
    G(s) = \frac{s-2}{2}G(s-2)
\]
\qed

\newpage

\Problem
Find an analytic continuation of the function:

\[
    H(x) = 1 + z^2 + z^4 + z^6 + \cdots+ z^{2n} + \cdots
\]

For which value is the continuation undefined? 
What is the value of the continuation at $z = 2$?

\new{Solution}
The natural response after observing the function is 
to consider the geometric series. We define the 
continuation to be $\bar{H}(z)$. Write:

\[
    \bar{H}(z) = \frac{1}{1-z^2}
\]

Clearly, for values where $|z^2| < 1$, $H$ and $\bar{H}$ agrees. 
It is possible to obtain a sequence that accumulate on some point 
in the unit circle that is not the center. For example, consider:

\[
    z_n := \frac{1}{2}e^{\pi i / n}
\]

The series of points converge to $1/2$, and $H, \bar{H}$ agrees 
for any point $z_n$.

Upon inspection, we notice that $\bar{H}(z)$ is defined everywhere 
other than $z = \pm 1$. Other then at these two poles, 
$\bar{H}$ is holomorphic, for $z^2-1$ is holomorphic and the 
reciprocal of a holomorphic function must be holomorphic as long 
as the function is defined. 
\qed


\newpage
\Problem
For which value of $s$ is the following 
integral defined?

\[
    L(s):= 
    \int_{0}^\infty \frac{x^s }{x^2+1}dx
\]

\new{Claim}
$L(s)$ is defined for $s \in (0, 1)$

\Proof
It is trivial to notice that the integrand blows up at $x = 0$ 
when $s$ is negative. We only consider positive $s$. Divide 
the integral into two intervals. 

\[
    \int_{0}^\infty \frac{x^s }{x^2+1}dx
    = \int_{0}^1+\int_1^\infty
\]

In the first interval, the integrand takes some finite 
value, regardless of the value of $s$, as long as it is positive. 
Thus, we focus on the latter summand. Notice:

\[
    0<
    \int_{1}^\infty \frac{x^s }{x^2+1}dx \leq
    \int_{1}^\infty \frac{x^s }{x^2}dx
    =
    \int_{1}^\infty  x^{s-2}dx
\]

For $s < 1$, $s-2 < -1$ hence:

\[
    \int_{1}^\infty  x^{s-2}dx
    = \frac{x^{s-1}}{s-1}\bigg|_{x = 1}^\infty = \frac{1}{1-s}
\]

Thus, the integral is bounded between zero and some positive value. 
For the integrand is always positive in the interval $(1, \infty)$,
the integral monotonically increases as the upper bound is sent 
to infinity. Bounded monotone sequences must converge, and hence 
we conclude the the integral converges for $s<1$. 

It remains to show that the integral diverges for $s\geq 1$. 
Consider the following inequality:

\[
    0<
    \int_{1}^\infty \frac{x^s }{2x^2}dx \leq
    \int_{1}^\infty \frac{x^s }{x^2+1}dx
\]

Assume $s \geq 1$. We evaluate the left integral:

\[
    \int_{1}^\infty \frac{x^s }{2x^2}dx =
    \int_{1}^\infty \frac{x^{s-2}}{2}dx
\]

If $s = 1$, the following integral equals to:

\[
    ln(x)/2\bigg|_1^\infty
\]

which diverges. If $s > 1$, then:

\[
    \frac{x^{s-1}}{2(s-1)}\bigg|_1^\infty
\]

which also diverges. We have verified that 
the integral diverges to infinity for $s \geq 1$. \qed


\newpage

\Problem
i) Define:
\[
    \zeta_{alt}(s) = \sum_{n = 1}^\infty \frac{(-1)^{n-1}}{n^s}
\]
Prove that for $Re(s) > 1$, this series converges

\new{Solution}
It suffices to show the absolute convergence 
of the sequence. We show the convergence of the 
following sequence.

\[
    \sum_{n = 1}^\infty \bigg|
        \frac{1}{n^s}
    \bigg|
    \textOr
    \sum_{n = 1}^\infty \bigg|
        {n^{-s}}
    \bigg|
    =
    \sum_{n = 1}^\infty \bigg|
        {e^{-sln(n)}}
    \bigg|
    =
    \sum_{n = 1}^\infty 
        {e^{-Re(s)ln(n)}}
    =
    \sum_{n = 1}^\infty
        n^{-Re(s)}
\]
Which, in fact equals to:

\[
    \sum_{n = 1}^\infty 
        \frac{1}{n^{Re(s)}}
\]

This series is known to converge 
by the p-series test, for 
it is given that $Re(s) > 1$. \qed

\vspace{5mm}

ii) Prove:
\[
    \zeta_{alt}(s) = \zeta(s) - (2/2^s)\zeta(s)
\]

\new{Solution}

Denote the nth partial sum of 
$\zeta(s), \zeta_{alt}(s)$ as 
$P_n(s), P'_n(s)$. That is:

\[
    P_n(s) := 
    \sum_{k = 1}^{n} \frac{1}{k^s}
    \textAnd
    P'_n(s) := 
    \sum_{k = 1}^{n} \frac{(-1)^k}{k^s}
\]

$\zeta(s)$ converges absolutely for 
$Re(s) > 1$. Thus, as $n \rightarrow \infty$,
$P_n(s) \rightarrow \zeta(s)$ and 
$P'_n(s) \rightarrow \zeta_{alt}(s)$. 

For finite $n$, notice the following equality:

\[
    P_{2n}(s) - \frac{2}{2^s}P_n(s) = 
    1 + \frac{1}{2^s} + \frac{1}{3^s} + \dots + \frac{1}{(2n)^s}
\]

\[
    -\frac{2}{2^s} - \frac{2}{4^s} - \cdots - \frac{2}{(2n)^s}
\]

\[
    = 1 - \frac{1}{2^s} + \frac{1}{3^s}
    \cdots -\frac{1}{(2n)^s}
    =
    P'_{2n}(s)
\]

Set $n \rightarrow \infty$. The equality 
converts to:
\[
    \zeta(s)-\frac{2}{2^s}\zeta(s)= \zeta_{alt}(s)
\]
\qed 

\newpage
Evaluate the integral:

\[
    I := \int_0^\infty \frac{x^4}{x^8+1}
\]

\new{Solution}
We take a semicircular toy contour centered 
at the origin with an angle of $\pi/4$, 
radius of $R \rightarrow \infty$. 
Call the contour $\gamma$. 

First show that the circular region of 
the contour converges to zero as 
$R \rightarrow \infty$. Call 
the circular region $\gamma_c$. Notice:

\[
    \bigg|
        \int_{\gamma_c}
        \frac{z^4}{z^8+1} dz
    \bigg|
    \leq
    \int_{\gamma_c}
    \frac{|z^4|}{R^8|(z/R)^8+1/R^8|}R\frac{dz}{R}
\]
\[
    =\frac{1}{R^3}
    \int_{\gamma_c} \frac{|dz/R|}{|(z/R)^8+1/R^8|}
\]

Now substitute:

\[
    z/R = e^{i\theta}
    \textAnd 
    dz/dR = ie^{i\theta}d\theta
\]

The circular contour is bounded by:

\[
    \frac{1}{R^3}
    \int_{\theta = 0}^{\pi}
    \frac{d\theta}{|e^{8i\theta}+1/R^8|}
\]

And for sufficiently large $R$, we can bound 
this integral by:

\[
    \frac{\pi}{R^3(1-1/R^8)}
\]

As $R\rightarrow \infty$, the bound converges 
to zero. By the squeeze limit theorem, we conclude 
that the circular region of the integral converges to zero. 

By the residue theorem:

\[
    \int_{-R}^{R}\frac{x^4}{1+x^8}dx + \int_{\gamma_c}\frac{z^4}{1+z^8}dz = 2\pi i \sum_{k = 0}^n Res(z_k)
\]
where $z_k$ denotes the poles. 

Notice that the integrand is even 
in the reals. As $R\rightarrow \infty$, 
write:

\[
    \int_0^\infty \frac{x^4}{1+x^8}dx = \pi i \sum_{k = 0}^n Res(z_k)
\]

It remains to compute the residues in 
the selected toy contour. The poles 
are the 8th roots of unity that lies 
on the first and the first and 
second quadrant. They all have a 
modulus of 1, and an argument 
of $\frac{1+2k}{8}\pi$ for $k = 0, 1, 2, 3$. 

The order of each pole is one. denote the pole 
as $z_k$. Compute the residue as follows:

\[
    Res(z_k) = 
    \lim_{z\rightarrow z_k}
    \frac{z^4(z-z_k)}{1+z^8}
    =^H
\lim_{z\rightarrow z_k}
    \frac{5z^4-4z^3z_k}{8z^7}
    = \frac{z_k^4}{8z_k^7}
    = \frac{1}{8z_k^3}
\]

Add up the residues:
\[
    \sum_{k = 0}^{3}Res(z_k)
     =
     \frac{1}{8}
     [z_0^{-3}+z_1^{-3}+z_2^{-3}+z_3^{-3}]
\]

\[
    = \frac{1}{8}
    [e^{-3\pi i/8}+ e^{-9\pi i/8}+ e^{-15\pi i/8}+ e^{-21\pi i/8}]
\]
And through some geometry:
\[
    = \frac{1}{4}(sin(\pi/8)-cos(\pi/8))i
\]

We must compute $(sin(\pi/8)-cos(\pi/8))$. 
Square it:

\[
    (sin(\pi/8)-cos(\pi/8))^2 = 
    1-2sin(\pi/8)cos(\pi/8)
\]
\[
    = 1- sin(\pi/4) = 1- 1/\sqrt{2}
\]

The value of $cos$ is greater than $sin$. Thus:
\[
    (sin(\pi/8)-cos(\pi/8)) = -\sqrt{1- 1/\sqrt{2}}
\]

Finally, we write:

\[
    I = \pi i \sum_{k = 0}^{3}Res(z_k)
    =\frac{ \pi i^2}{4} (-\sqrt{1- 1/\sqrt{2}})
    = \frac{\pi}{4} \sqrt{1- 1/\sqrt{2}}
\]

\qed

\newpage

\Problem
Prove:

\[
    |\Gamma(1/2+it)| = \sqrt{\frac{2\pi}
    {e^{\pi t}+e^{-\pi t}}
    }
\]
\new{Proposition}
The complex conjugate of the gamma function 
is the value of the gamma function evaluated 
at the conjugate of the argument. That is:
\[
   \overline{\Gamma(z)} = \Gamma(\bar{z})  
\]
\Proof
We expand the gamma function back to its definiton. 

\[
    \overline{
        \Gamma(z)
    }
    = \overline{
        \int_{x = 0}^\infty
        x^{z-1}e^xdx
    }
    = \int_{x = 0}^\infty
    \overline{x^{z-1}}e^x dx
\]
Rewrite the base of $x$ as 
a base of $e$. 
\[
   = \int_0^\infty 
   \overline{
    e^{ln(x)(z-1)}
   } 
   e^x dx
\]
Observe that $ln(x)$ is always a 
real value. The conjugate of an exponent
is the exponent of the conjugate of 
its argument. Hence:

\[
    = 
    \int_0^\infty
    e^{ln(x)(\bar{z}-1)}e^x dx
    = 
    \int_0^\infty
    x^{\bar{z}-1}e^x dx
    =
    \Gamma(\bar{z})
\]
\qed

\new{Solution}
We are given the reflection formula:
\[
    \Gamma(s)\Gamma(1-s)
    =\frac{\pi}{sin(\pi s)}
\]
Plug in $s = 1/2 + it$. Write:

\[
    \Gamma(1/2+it)\Gamma(1/2-it)
    =\frac{\pi}{sin(\pi (1/2 + it))}
\]

With a litte manipulation, we rewrite 
the LHS. 

\[
    LHS = \Gamma(1/2+it)\Gamma(
        \overline{
            1/2+it
        }
    )
    =\Gamma(1/2+it)\overline{
        \Gamma(1/2+it)
    }
\]
\[
    = |\Gamma(1/2+it)|^2
\]

Rewrite the RHS using the basic 
trig identity and Euler's formula:

\[
    RHS = \frac{\pi}
    {cos(\pi it)}
    = \pi 
    \frac{2}{e^{i^2\pi t}+e^{-i^2\pi t}}
    = \frac{2\pi}{e^{\pi t}+e^{-\pi t}}
\]
\newpage

$LHS = RHS$ converts to:

\[
    |\Gamma(1/2+it)|^2 
    =
\frac{2\pi}{e^{\pi t}+e^{-\pi t}}
\]

Finally:

\[
    |\Gamma(1/2+it)|
    =
\sqrt{
    \frac{2\pi}{e^{\pi t}+e^{-\pi t}}
}
\]
\qed



\end{document}
