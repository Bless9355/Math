\documentclass{article}
\usepackage{amsfonts}
\usepackage{amsthm}
\usepackage{amsmath}
\usepackage{amssymb}
\usepackage{wasysym}


\newcommand{\new}[1]{
    \vspace{2mm}
    \noindent
    \textbf{
    \underline{#1}}
}

\def\calO{{\mathcal{O}}}
\def\ZZ{{\mathbb{Z}}}
\def\RR{{\mathbb{R}}}
\def\_{{\hspace{1mm}}}

\def\contradiction{{\lightning}}



\newcounter{problemcnt}
\setcounter{problemcnt}{0}

\newcommand{\Problem}{{
    \vspace{5mm}
    \stepcounter{problemcnt}
    \noindent
    \arabic{problemcnt}. 
}
}

\newcommand{\nProblem}[1]{{
    \noindent
    \setcounter{problemcnt}{#1}
    \arabic{problemcnt}. 
}
}

\newcommand{\Proof}{{
    \vspace{2mm}
    \noindent
    \textbf{
    \underline{Proof}}
}
}

\newcommand{\textOr}{
    {
        \hspace{5mm}
        \textrm{or}
        \hspace{5mm}
    }
}

\newcommand{\textAnd}{
    {
        \hspace{5mm}
        \textrm{and}
        \hspace{5mm}
    }
}

\newcommand{\Exp}{
    {
        \textrm{Exp}
    }
}

\begin{document}

\Problem
Find a functional equation for the following function:

\[
    G(s) := \int_{0}^{\infty}\Exp[-x^2]x^{s-1}dx
\]

\new{Solution}
We first observe the following derivative:

\[
    \frac{d}{dx}e^{-x^2} = -2xe^{-x^2}
\]

Hence:

\[
    \int xe^{-x^2}dx = -\frac{e^{-x^2}}{2} + C
\]

With this indefinite integral in mind, we integrate G 
by parts. Apply the following substitutions:

\[
    u = x^{s-2} \textAnd du = (s-2)x^{s-3}
\]
\[
    dv = x\Exp[-x^2]dx \textAnd 
    v = -\frac{\Exp[-x^2]}{2}
\]

Integrate $G$:
\[
    G(s) = uv \bigg|_{x = 0}^\infty - \int_{x = 0}^{\infty}vdu
\]
\[
    = \left[
        -\frac{x^{s-2}\Exp[-x^2]}{2}
    \right]_0^\infty
    -\int_0^\infty \left(
        -\frac{(s-2)\Exp[-x^2]x^{s-3}}{2}
    \right)dx
\]
\[
    = \left[
        -\frac{x^{s-2}\Exp[-x^2]}{2}
    \right]_0^\infty
    +
    \frac{(s-2)}{2}
    \int_0^\infty 
    \left(
        \Exp[-x^2]x^{s-3}
    \right)dx
\]

For the sake of convergence of the first summand,
we assume $s\geq2$. As for $s = 2$, we evaluate:

\[
    G(2) = \int_0^{\infty}xe^{-x^2}dx = -\frac{e^{-x^2}}{2}\bigg|_0^\infty
    = 1
\]

As for $s > 2$, we compute the equality obtained by the by parts
integration. Notice that the first summand converges to zero. 
Hence:

\[
    G(s) = \frac{s-2}{2}G(s-2)
\]
\qed

\newpage

\Problem
Find an analytic continuation of the function:

\[
    H(x) = 1 + z^2 + z^4 + z^6 + \cdots+ z^{2n} + \cdots
\]

For which value is the continuation undefined? 
What is the value of the continuation at $z = 2$?

\new{Solution}
The natural response after observing the function is 
to consider the geometric series. We define the 
continuation to be $\bar{H}(z)$. Write:

\[
    \bar{H}(z) = \frac{1}{1-z^2}
\]

Clearly, for values where $|z^2| < 1$, $H$ and $\bar{H}$ agrees. 
It is possible to obtain a sequence that accumulate on some point 
in the unit circle that is not the center. For example, consider:

\[
    z_n := \frac{1}{2}e^{\pi i / n}
\]

The series of points converge to $1/2$, and $H, \bar{H}$ agrees 
for any point $z_n$.

Upon inspection, we notice that $\bar{H}(z)$ is defined everywhere 
other than $z = \pm 1$. Other then at these two poles, 
$\bar{H}$ is holomorphic, for $z^2-1$ is holomorphic and the 
reciprocal of a holomorphic function must be holomorphic as long 
as the function is defined. 
\qed


\newpage
\Problem
For which value of $s$ is the following 
integral defined?

\[
    L(s):= 
    \int_{0}^\infty \frac{x^s }{x^2+1}dx
\]

\new{Claim}
$L(s)$ is defined for $s \in (0, 1)$

\Proof
It is trivial to notice that the integrand blows up at $x = 0$ 
when $s$ is negative. We only consider positive $s$. Divide 
the integral into two intervals. 

\[
    \int_{0}^\infty \frac{x^s }{x^2+1}dx
    = \int_{0}^1+\int_1^\infty
\]

In the first interval, the integrand takes some finite 
value, regardless of the value of $s$, as long as it is positive. 
Thus, we focus on the latter summand. Notice:

\[
    0<
    \int_{1}^\infty \frac{x^s }{x^2+1}dx \leq
    \int_{1}^\infty \frac{x^s }{x^2}dx
    =
    \int_{1}^\infty  x^{s-2}dx
\]

For $s < 1$, $s-2 < -1$ hence:

\[
    \int_{1}^\infty  x^{s-2}dx
    = \frac{x^{s-1}}{s-1}\bigg|_{x = 1}^\infty = \frac{1}{1-s}
\]

Thus, the integral is bounded between zero and some positive value. 
For the integrand is always positive in the interval $(1, \infty)$,
the integral monotonically increases as the upper bound is sent 
to infinity. Bounded monotone sequences must converge, and hence 
we conclude the the integral converges for $s<1$. 

It remains to show that the integral diverges for $s\geq 1$. 
Consider the following inequality:

\[
    0<
    \int_{1}^\infty \frac{x^s }{2x^2}dx \leq
    \int_{1}^\infty \frac{x^s }{x^2+1}dx
\]

Assume $s \geq 1$. We evaluate the left integral:

\[
    \int_{1}^\infty \frac{x^s }{2x^2}dx =
    \int_{1}^\infty \frac{x^{s-2}}{2}dx
\]

If $s = 1$, the following integral equals to:

\[
    ln(x)/2\bigg|_1^\infty
\]

which diverges. If $s > 1$, then:

\[
    \frac{x^{s-1}}{2(s-1)}\bigg|_1^\infty
\]

which also diverges. We have verified that 
the integral diverges to infinity for $s \geq 1$. \qed







\end{document}