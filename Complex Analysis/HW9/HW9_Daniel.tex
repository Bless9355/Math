\documentclass{article}
\usepackage{amsfonts}
\usepackage{amsthm}
\usepackage{amsmath}
\usepackage{amssymb}
\usepackage{wasysym}


\newcommand{\new}[1]{
    \vspace{2mm}
    \noindent
    \textbf{
    \underline{#1}}
}

\def\calO{{\mathcal{O}}}
\def\ZZ{{\mathbb{Z}}}
\def\RR{{\mathbb{R}}}
\def\_{{\hspace{1mm}}}

\def\contradiction{{\lightning}}



\newcounter{problemcnt}
\setcounter{problemcnt}{0}

\newcommand{\Problem}{{
    \vspace{5mm}
    \stepcounter{problemcnt}
    \noindent
    \arabic{problemcnt}. 
}
}

\newcommand{\nProblem}[1]{{
    \noindent
    \setcounter{problemcnt}{#1}
    \arabic{problemcnt}. 
}
}

\newcommand{\Proof}{{
    \vspace{2mm}
    \noindent
    \textbf{
    \underline{Proof}}
}
}

\newcommand{\textOr}{
    {
        \hspace{5mm}
        \textrm{or}
        \hspace{5mm}
    }
}

\newcommand{\textAnd}{
    {
        \hspace{5mm}
        \textrm{and}
        \hspace{5mm}
    }
}

\begin{document}


%Uniform Continuity
\Problem
Define:
\[
    f_n(x) := 
    \frac{n}{1+nx^2}
\]

Prove that $f_n$ is uniformly continuous. 

Is the family of function
$\mathcal{F}:=\{f_n(x)|n\in \ZZ_{pos}\}$ equicontinuous?

\new{Proposition} $\frac{x}{1+nx^2}$ is bounded. 

\Proof
First, consider the function in the range 
$x \in \RR \setminus [-1, 1]$. Within the range, 
we have $|x| > 1$ and hence $x^2 > 1$. Write:

\[
    |\frac{x}{1+nx^2}| < |\frac{x}{nx^2}| = |\frac{1}{nx}| < 1
\]
Now, consider the range $x \in [-1, 1]$. Write:

\[
    |\frac{x}{1+nx^2}| < |\frac{x}{1}| < 1
\]

This shows that our function is bounded for $x\in \RR$
\qed

\new{Claim} $f_n(x)$ is uniformly continuous. 
Given any $\epsilon > 0$, we wish to obtain a $\delta_{max}$ where
for any $\delta$ such that $|\delta| < \delta_{max}$ satisfies:

\[
    |f_n(x)-f_n(x+\delta)| < \epsilon
\]
Or equivalently
\[
  |\frac{n}{1+nx^2} -\frac{n}{1+n(x+\delta)^2}| <\epsilon 
\]
Notice:
\[
  |\frac{n}{1+nx^2} -\frac{n}{1+n(x+\delta)^2}| <
|\frac{n}{1+nx^2} -\frac{n}{1+n(x+\delta_{max})^2}| 
\]
It suffices to construct $\delta_{max}$ that satisfies:

\[
    |\frac{n}{1+nx^2} -\frac{n}{1+n(x+\delta_{max})^2}| 
    < \epsilon
\]

Through some algebra:
\[
    |\frac{
    n 
    \left[
    1+n(x+\delta_{max})^2 - (1+nx^2) 
    \right]
    }
    {(1+nx^2)(1+n(x+\delta_{max})^2)}|
     < \epsilon
\]
\[
    |\frac{
    n 
    \left[
    2nx\delta_{max} + n\delta_{max}^2
    \right]
    }
    {(1+nx^2)(1+n(x+\delta_{max})^2)}|
     < \epsilon
\]
\[
    |\frac{
    n^2\delta_{max}
    \left[
    2x + \delta_{max}
    \right]
    }
    {(1+nx^2)(1+n(x+\delta_{max})^2)}|
     < \epsilon
\]

Set $\delta_{max} < 1$. Also, notice that the 
terms in the denominators are both greater than 1. 
We construct a $\delta_{max}$ that satisfies a stronger 
condition:

\[
    |\frac{
        n^2\delta_{max}(2x+1)
    }
    {1+nx^2}|
    <\epsilon
    \textOr
    |\delta_{max}||n^2||
    \frac{2x}{1+nx^2}+
    \frac{1}{1+nx^2}|
    < \epsilon
\]

It is easy to see that the function $\frac{1}{1+nx^2}$
is bounded. The denominator is always greater than 1, 
so the function is bounded by 1. We have shown that the second
summand is bounded for any real $x$. Again, we construct 
a stronger $\delta_{max}$ that satisfies:

\[
    |\delta_{max}|B<\epsilon
\]

Where $B$ is the maximum bound of the other terms. 
If $B < 0$, the statement is a tauology. Otherwise, 
set $\delta = \epsilon /(2B)$. This concludes the proof. 
\qed 

\new{Claim} The family $\mathcal{F}$ is not equicontinuous

\Proof  We claim that equicontinuity is 
violated at $x = 0$. Notice that $f_n(0) = n$.
Assume for a contradiction, that $\mathcal{F}$
is equicontinuous at $x = 0$. For $\epsilon = 1$, 
it must be possible to obtain a $\delta_{max}$ 
where for all $\delta$ such that
$|\delta| < \delta_{max}$, $\delta$ satisfies:
\[
    |f_n(0) - f_n(\delta)| < 1 
    \textOr
    |n - f_n(\delta)| < 1
\]
So 
\[
    |f_n(\delta)| > n-1
\]
$\delta \neq 0$ by assumption, so as $n \rightarrow \infty$, 
$|f_n(\delta)| \rightarrow 1$. This is a contradiction. 
\qed

\newpage
%PW convergence of spike
\Problem
Define:
\[
    f_n(x) := 
    \begin{cases}
        1-nx & (x \in (0, 1/n)) \\ 
        0 & (\textrm{Otherwise})
    \end{cases}
\]
Show that the limit of this function exists and 
find the limit. 

\new{Solution}
Consider the function: 
\[
    f(x) := 
    \begin{cases}
        1 & (x = 0) \\ 
        0 & (x \neq 0)
    \end{cases}
\]
We claim that $f_n$ converges to $f$ pointwise. We must show
that for any $x_0 \in \RR$, the following equality holds:

\[
    \lim_{n \rightarrow \infty}
    f_n(x_0) = f(x_0)
\]

If $x_0 \leq 0$, the problem becomes trivial. If the inequality
is strict, $f_n(x_0) = 0$
regardless of the value of $n$. Also computing the value 
of $f_n(0)$, we notice that the value is identically $1$, 
regardless of the value of $n$. 

It remains to demonstrate the equality for $x_0 > 0$. 
Recall that $\lim_{n\rightarrow \infty} 1/n = 0$. Hence, it 
is possible to obtain a sufficiently large integer $N$ such that 
for any $n > N$, we have $1/n < x_0$. By the construction of 
$f_n(x)$, $f_n(x_0) = 0$ for any $n > N$. This concludes the proof. 
\qed 


\newpage
%PW convergence of spike
\Problem
Define:
\[
    f_n(x) := 
    \begin{cases}
        1-nx & (x \in (0, 1/n)) \\ 
        0 & (\textrm{Otherwise})
    \end{cases}
\]

and,
\[
    \mathcal{F}:= \{f_n(x)|n \in \ZZ_{pos}\}
\]

Is the family $\mathcal{F}$ normal?

\new{Claim}
No, $\mathcal{F}$ is not normal. 

\Proof
Assume for a contradiction, that indeed the family is normal. 
Then, the entire family $\mathcal{F}$ must have some subsequence
of functions that converge uniformly. Let the sequence of functions 
$\{f_{m_1}, f_{m_2}, f_{m_3}, \dots \}$ be such a sequence of functions.

For the value $\epsilon = 1/4$, we extract some integer $N$ such that 
for any $n > N$, the function achieves:

\[
    |f_{m_n}(x)-f(x)| < 1/4
\]

For any real value $x$. $f(x)$ is some imaginary function 
that the subsequence uniformly converges to. Extract another 
arbitrary integer $k > N$ that satisfies the same condition. 
Adding the two inequalities, we obtain:

\[
    |f_{m_n}(x)-f(x)| + |f_{m_k}(x)-f(x)| < 1/2
\]

Which implies, by the triangle inequality:

\[
    |f_{m_n}(x) - f_{m_k}(x)| < 1/2
\]

And this is for any values of $n, k > N$. 
We explicitly construct a value $x_0$ that violates 
this inequality. 

Take any integer $n$ greater than $N$. Obtain 
$k$ such that $m_k > 2m_n$. This is possible because 
$m$ is a strictly increasing sequence of integers. 
Set $x_0 = 1/m_k$. Write:

\[
|f_{m_n}(x_0) - f_{m_k}(x_0)| = |f_{m_n}(1/m_k)-f_{m_k}(1/m_k)|
\]
Notice that the latter summand vanishes. Also the fraction 
$1/m_k$ is between zero and $1/m_n$. We proceed to write:
\[
    |f_{m_n}(x_0) - f_{m_k}(x_0)| = |1-m_n/m_k| > 1/2
\]

by construction. But then again, this whole absolute value 
must be less than $1/2$, which is a contradiction. \qed

\newpage

%The first integral problem
\Problem
Solve the integral:
\[
    \int_{-\infty}^{\infty}\frac{x^2}{x^4+x^2+1}dx
\]

\new{Solution}
Define the integand as a complex function. That is:

\[
    f(z):=\frac{z^2}{z^4+z^2+1}
\]

We look at the semicircular contour with radius $R$
that is centered at the origin. The contour occupies the 
first and the second quadrant. Call the contour $\gamma$. 

The circular part of the contour vanishes as $R \rightarrow \infty$. 
Write:

\[
    \left|\oint_{\gamma_c}f\right|
    =\left|\int_{\theta = 0}^{\pi}
    \frac{R^2e^{2i\theta}}{R^4e^{4i\theta}+R^2e^{2i\theta}+1}
    Rie^{i\theta}d\theta
    \right|
    <2\pi/R
\]

and as $R\rightarrow \infty$, clearly the integral converges to 
zero. 

The problem reduces down to identifying the poles of the function 
$f$. We must identify all the zeros of the denominator. Notice:

\[
    (x^2-1)(x^4+x^2+1) = (x^6-1)
\]

Ergo, the zeros of the denominators are the four complex roots
of $z^6 - 1$. The two real roots $z = \pm 1$ can be excluded 
by computing $f(1) = 3, f(-1) = 3$. In the contour $\gamma$, 
the two poles are:

\[
    z_0 = e^{i\pi/3} \textAnd z_1 = e^{2i\pi/3}
\]

Both poles are of order 1. To compute the residue, apply 
L'Hopital's rule. For any of the poles $p \in {z_0, z_1}$, 

\[
    Res_f(p) = \lim_{z\rightarrow p} 
    \frac{(z-p)z^2}{z^4+z^2+1}
\]
By taking derivatives in both the numerator and the denominator:

\[
    \lim_{z\rightarrow p} 
    \frac{(z-p)z^2}{z^4+z^2+1}
=
\lim_{z\rightarrow p} 
    \frac{3z^2-2zp}{4z^3+2z}
=
\frac{3p-2p}{4p^2+2}
=\frac{1}{4p+2p^{-1}}
\]

Plugging in the appropriate values of $p$, we write:

\[
    Res_f(e^{i\pi/3}) = 
    \frac{1}
    {
       4e^{i\pi/3}+2e^{-i\pi/3} 
    }
    =
    \frac{1}
    {
        2+2\sqrt{3}i
        +1-\sqrt{3}i
    }
    =\frac{1}
    {
        3+\sqrt{3}i
    }
    =\frac{3-\sqrt{3}i}{12}
    \]\[
  Res_f(e^{2i\pi/3}) = 
    \frac{1}
    {
       4e^{2i\pi/3}+2e^{-2i\pi/3} 
    }
    =
    \frac{1}
    {
        -2+2\sqrt{3}i
        -1-\sqrt{3}i
    }
    =
    \frac{1}
    {
        -3+\sqrt{3}i
    }
    =
    \frac{-3-\sqrt{3}i}
    {
        12
    }
\]

By the residue theorem, we evaluate the contour integral:

\[
    \oint_{\gamma}f = 2\pi i [Res_f(z_0)+Res_f(z_1)]
     = 2\pi i \frac{-2\sqrt{3}i}{12} = 
     \frac{\pi}{\sqrt{3}}
\]
\newpage

Finally we conclude 
\[
    \boxed{
        \int_{-\infty}^{\infty}
        \frac{x^2}{x^4+x^2+1} = 
        \frac{\pi} {\sqrt{3}}
    }
\]

\newpage
\Problem
Solve the integral:
\[
    I:=
    \int_{0}^{2\pi}
    \frac{1}{a+bsin(\theta)} d\theta
\]

\new{Solution}
First, consider when the integral is valid. The integrand 
must be finite, that is, the numerator must be nonzero. 
The magnitude of the function $bsin(\theta)$ must not be greater 
than $a$. Otherwise, the numerator will hit zero at some point, 
and the integral will be invalid. From now on, assume $|b|<|a|$. 

Also, if $b = 0$, the integral becomes trivial. $I = 2\pi/a$ 
given that $a$ is nonzero. 

Recall Euler's formula:
\[
    sin(\theta) = \frac{e^{i\theta}+e^{-i\theta}}{2i}
\]

Now manipuate the integral:

\[
    I = 
 \int_{0}^{2\pi}
 \frac{2i}{2ia+2ibsin(\theta)} d\theta
=
 \int_{0}^{2\pi}
 \frac{2i}{2ia+b(e^{i\theta}+e^{-i\theta})} d\theta
\]
\[
    I = \frac{2}{b}
 \int_{0}^{2\pi}
 \frac{ie^{i\theta}}
 {e^{2i\theta}+2iae^{i\theta}/b+1}d\theta
\]

Let $\zeta := e^{i\theta}$. $\zeta$ forms 
a unit circle in the range $\theta\in[0, 2\pi]$. 
Rewrite the integral as:

\[
    I = \frac{2}{b}
    \oint_{\zeta \in C}
    \frac{d\zeta}
    {\zeta^2+2ik\zeta+1}
\]

Where $k:=a/b$ and $C$ is the unit circle. 
By the quadratic formula, the integrand has poles at:

\[
    \zeta = -ik\pm \sqrt{-k^2+1}
\]

$k^2 > 1$ since $|a/b| >1$. $\zeta$ is purely imaginary. 
Note that one of the poles necessarily fall into the circle 
and that the other does not. The pole $p$ that falls into 
the unit circle is:

\[
    p= -ik+\sqrt{-k^2+1}
\]

Denote the integrand of the contour function f. 
\[
    f(z) = 
    \frac{1}{z^2+2ikz+1}
\]

Compute the residue at $p$. The pole is a simple pole. 

\[
    Res_f(p) = 
    \lim_{z\rightarrow p}
    \frac{z-p}
    {z^2+2ikz+1}
    =
      \lim_{z\rightarrow p}
    \frac{1}{2z+2ik}
\]
\[
    =
    \frac{1}{2}
    \frac{1}{-ik+\sqrt{-k^2+1}+ik}
    =
    \frac{1}{2}
    \frac{1}{\sqrt{1-k^2}}
\]

By the residue theorem:

\[
    I = \frac{2}{b}2\pi i [Res_f(p)] = 
    \frac{2\pi}{b\sqrt{k^2-1}}
    =\pm\frac{2\pi}{\sqrt{a^2-b^2}}
\]

The $\pm$ sign depends on the sign of $b$. As 
inserting $b$ into the square root, we must seperate 
its sign. 

We conclude, for $|b|> |a|$

\[
    \boxed{
        I = 
        \begin{cases}
            \frac{2\pi}{\sqrt{a^2-b^2}} & (b>0)\\
            -\frac{2\pi}{\sqrt{a^2-b^2}} & (b<0)\\
        \end{cases}
    }
\]  



\end{document}