\documentclass{article}
\usepackage{amsfonts}
\usepackage{amsthm}
\usepackage{amsmath}
\usepackage{amssymb}
\usepackage{wasysym}


\newcommand{\new}[1]{
    \vspace{2mm}
    \noindent
    \textbf{
    \underline{#1}}
}

\newcommand{\deriv}[1]{{
    \frac{d}{d{#1}}
}
}

\newcommand{\parderiv}[1]{{
    \frac{\partial}{\partial{#1}}
}
}

\def\calO{{\mathcal{O}}}
\def\ZZ{{\mathbb{Z}}}
\def\RR{{\mathbb{R}}}
\def\_{{\hspace{1mm}}}

\def\contradiction{{\lightning}}



\newcounter{problemcnt}
\setcounter{problemcnt}{0}

\newcommand{\Problem}{{
    \vspace{5mm}
    \stepcounter{problemcnt}
    \noindent
    \arabic{problemcnt}. 
}
}

\newcommand{\nProblem}[1]{{
    \noindent
    \setcounter{problemcnt}{#1}
    \arabic{problemcnt}. 
}
}

\newcommand{\Proof}{{
    \vspace{2mm}
    \noindent
    \textbf{
    \underline{Proof}}
}
}

\newcommand{\textOr}{
    {
        \hspace{5mm}
        \textrm{or}
        \hspace{5mm}
    }
}

\newcommand{\textAnd}{
    {
        \hspace{5mm}
        \textrm{and}
        \hspace{5mm}
    }
}

\newcommand{\Exp}{
    {
        \textrm{Exp}
    }
}

\begin{document}

\begin{center}
\LARGE
Midterm II-part I

\Large
Daniel Son
\end{center}

\normalsize

\Problem
Prove or disprove: there is an entire analytic function with real part 
$x-xy$. If there is such an analytic function, find all such functions. 
Also, find the series expansion of the function of $z$ around the origin. 

\new{Solution}
Let function $f(z) = f(x + iy)$ be such a function that satisfies 
the condition. 
Analytic functions are necessarily holomorphic and vice versa. Hence, 
it is possible to apply the Cauchy-Riemann Equations in this context. 
Define:

\[
    u:=Re(f(x+iy))
    \textAnd 
    v := Im(f(x+iy))
\]

It is given that $u = x-xy$. We compute:

\[
    u_x = 1-y \textAnd u_y = -x 
\]

By the Cauchy-Riemann Equations, we deduce:

\[
    u_x = v_y \textAnd u_y = -v_x 
\]
\[
    v_x = -u_y = x \textAnd v_y = u_x = 1-y
\]

The function $v(x, y)$ must be expressed as the following:

\[
    v(x, y) = x^2/2 + C(y) = y-y^2/2 + D(x)
\]

Where $C, D$ are functions that map real values to real values that 
depend solely on $y$ and $x$ respectively. The two expressions of 
$v(x, y)$ must equate each other. Write:

\[
    C(y) - y + y^2/2 = D(x)-x^2/2
\]

Recognize that the LHS is independent of $x$ and the RHS independent of $y$. 
Thus, we conclude that both expressions equal to a constant, say $C$. 

\[
    D(x) = x^2/2 + C \textAnd 
    v(x, y) = x^2/2 + y - y^2/2 + C
\]

Compute the complex derivative of $f$ by differentiating it over 
the real axis. The holomorphicity of $f$ guarantees that the derivative 
is unique. Write:

\[
    \deriv{z}f(z) = 
    \parderiv{x}u(x, y) + \parderiv{x}v(x, y) i
\]

\[
    = (1-y)+xi = 1-iz
\]

Taking the antiderivative, we conclude, for some complex constant $C'$, 

\[
    f(z) = z-iz^2/2+C'
\]

The real part of $f$ does not contain a constant. Hence, we narrow 
down $C' = Ci$ where $C$ is a real value. 

We have shown that a function 
$f$ that satisfies $Re(f) = x-xy$ must be in the form of:

\[
    f(x) = Ci + z-iz^2/2\hspace{5mm} (C \in \mathbb{R})
\]

Indeed all such functions must be holomorphic, for $f$ is 
a complex polynomial of order two. Moreover, by some algebra, 
we notice that such functions always have a real part $x-xy$. 
We conclude that the functions of the form above are all the 
analytic entire 
functions that have a real part of $x - xy$. The function 
is already written as its series expansion about the origin. 

\qed

\newpage
\Problem Compute four integrals. 


i) Compute:

\[
    I := \int_{-\infty}^\infty 
    \frac{dx}{cosh(x)}
     = \int_{-\infty}^{\infty}
    \frac{2dx}{e^x+e^{-x}}
\]

\new{Soultion}
The integrand is an even function. Hence we write:
\[
    I = 4\int_0^\infty \frac{dx}{e^x+e^{-x}}
    \textAnd 
    I/4 = \int_{-\infty}^\infty \frac{e^xdx}{e^{2x}+1}
\]
Apply the u-subsitution, $u = e^x$:

\[
    I/4 = \int^\infty_{-\infty}
    \frac{du}{u^2+1}
    =arctan(u)\bigg|^\infty_{-\infty}
    =\pi
\]
Hence:

\[
    I = 4\pi
\]

\qed
\newpage
ii) Let $\zeta$ be any real number and $a > 0$. Evaluate:

\[
    I:=
    \int_{-\infty}^\infty 
    \frac{e^{-2\pi \zeta x}}
    {x^2 + a^2}
    dx
\]

\new{Solution}
Define a holomorphic function $f(z)$ as follows:

\[
    f(z) = \frac{e^{-2\pi \zeta z}}{z^2 - a^2}
\]

The numerator and the denominator are known to be holomorphic. Thus 
the function is holomorphic everywhere other than the poles which are 
located at $z = \pm a$. Draw a semicircular contour centered 
at the origin that occupies quadrant I and IV. Call this 
contour $\gamma$, and denote the radius as $R$. 

Take the contour integral of $f(z)$ over $\gamma$. Let the 
straight segment of the contour be called $S$, and the circular 
region $C$. 

We claim that the integral over the circular region vanishes. That is, a
as $R \rightarrow \infty$, $
    \oint_C f = 0
    $

    Notice:
\[
    \bigg|
    \oint_C f
    \bigg|
    = \bigg|
    \int_{z \in C} 
    \frac{e^{-2\pi \zeta z}}{z^2+a^2}
    dz
    \bigg|
    \leq 
    \int_{z \in C} 
    \frac{max|e^{-2\pi \zeta z}|}
    {R^2 - a^2}dz
\]

Note that the modulus of an exponent is the exponent 
of the modulus of the argument. That is:

\[
    |e^{-z}| = e^{Re(-2\pi \zeta z)}
\]

And for $z \in C$, the quality is bounded under 1. 
Thus:
\[
 \bigg|
    \oint_C f
    \bigg|
    \leq 
    \frac{
    2\pi R
    }
    {R^2 - a^2}
\]
And the upper bound converges to zero as $R$ approaches infinity. 
This shows that the circular region converges to zero. 
\checkmark 

By the residue theorem:

\[
    \oint_C f + \oint_S f = 2\pi i Res_f(a)
\]

The first summand of the LHS vanishes. The second summand can 
be computed with some algebra. We write:

\[
\oint_S f = 
    \int_{x = -\infty}^{\infty}
    \frac{
        e^{-2\pi \zeta i x} \cdot (-i)dx
    }{ 
        (xi)^2 - a^2
    }
    = 
    i \int_{x = -\infty}^{\infty}
    \frac{e^{-2\pi \zeta i x} dx} 
    {x^2 + a^2}dx
    = iI
\]

The residue can be computed with ease:

\[
    Res_f(a) = 
    \lim_{z\rightarrow a} 
    \frac{e^{-2\pi \zeta z}(z-a)}{z^2-a^2}
    =
    \lim_{z\rightarrow a} 
    \frac{e^{-2 \pi \zeta z}}{z+a}
    =\frac{e^{-2 \pi \zeta a}}{2a}
\]

Combining the results, we write:

\[
    iI = 2\pi i \frac{e^{-2 \pi \zeta a}}{2a}
    \textOr
    \boxed{
        I = \frac{\pi e^{-2 \pi \zeta a}}{a}
    }
\]
\newpage

iii) Compute:

\[
    \frac{I}{2\pi i} = \frac{1}{2\pi i}
    \oint_{|z| = 2} \frac{zdz}{z^2 - 1}
\]

\new{Solution} The function 
\[
    f(z) = \frac{z}{z^2-1}
\]
is holomorphic outisde the two poles $z = \pm 1$. By the residue 
theorem, the integral $I$ equals to the sum of the residues 
multiplied by $2\pi i$. Our answer is the following sum:

\[
    Res_f(1)+ Res_f(-1)
\]

Write:
\[
    Res_f(1) = \lim_{z \rightarrow 1} 
    \frac{z(z - 1)}{z^2 - 1}
    = z/(z+1)\bigg|_{z = 1} = 1/2
\]
\[
    Res_f(-1) = \lim_{z \rightarrow -1} 
    \frac{z(z + 1)}{z^2 - 1}
    = z/(z-1)\bigg|_{z = -1} = 1/2
\]

Thus:
\[
    \boxed{
    \frac{I}{2\pi i} = 1
    }
\]




\newpage

iv) Compute:
\[
    I:=
    \int_0^\infty {
        \frac{x^{-1/2}}
        {x+1}
        dx
    }
\]
\new{Solution}
We use two identities about the beta function. Recall
the definiton:
\[
    B(n, m) := \int_0^1 x^n(1-x)^mdx
\]

And the two identities:

\[
    B(n, m) = \frac{\Gamma(n)\Gamma(m)}{\Gamma(n+m)}
    \textAnd 
    B(n, m) = \int_0^\infty 
    \frac{u^{\alpha - 1}}{(1+u)^{\alpha + \beta}} du
\]

By the seond condition, the integral simplies to:

\[
    I = B(1/2, 1/2)
\]

And by the first identity:

\[
    B(1/2, 1/2) = \Gamma(1/2)^2/\Gamma(1) = \pi
\]

We conclude:
\[
    I = \pi
\]

\qed

\newpage

\Problem Consider the following infinite products:

\[
    I_1(a) := \prod_{n = 1}^\infty (1+a_n)
    \textAnd 
    I_2(b) := \prod_{m = 1}^\infty 
    \prod_{n = 1}^\infty(1 + b_{mn})
\]

a) State the definition of convergence of $I_1(a)$. 
Give an example of a product that converges to 
a finite, nonzero number, and an example that diverges. 

\new{Definition} 
We define the partial product $S_N$ as follows:

\[
    S_N := \sum_{n = 1}^N (1+a_n)
\]
If the partial product converges as $N \rightarrow \infty$, 
then the infinite product $I_1(a)$ is defined to converge. 

Consider the case where $a_n = 0$ identically. Trivially, 
$S_N = 1$ regardless of $N$. The infinite series converges 
to 1. 

Now, let $a_n = 1/n$. By induction, it is possible to 
show $S_N = N + 1$. For the base case, $S_1 = 1 + a_1 = 2$. 
For the inductive case:

\[
    S_{N+1} = \prod^{N+1}_{n=1}(1+\frac{1}{n})
    =\frac{N+2}{N+1} S_{N} = N+2
\]
which proves the claim. Ergo, $S_{N+1}$ diverges to infinity. 

\vspace{5mm}
b) State the definition for the convergence of the infinite product 
$I_2(b)$.

\new{Definition}
It would be nice if the nested products all converge. That is:
$I_1(b_k)$ converges for any $k$. The sequence $b_k$ denotes the 
sequence:

\[
    b_{k1}, b_{k2}, b_{k3}, \dots, b_{kn}, \dots
\]

Given that $I_1(b_k)$ converges for any $k$, we define $I_2(b)$ 
to converge if $I_1(I_1(b_k) - 1)$ converges. In words, if each 
column of $b$ converges, we write out all the products associated 
with column. Subtract 1 from each of the results, and take 
the infinite product of the results. If any of the columns 
has a divergent infinite product, we define the doubly product 
to also diverge. 

\newpage
c) Does there exist a bounded sequence 
$\{b_{mn}\}$ such that each $b_{mn} \neq -1$ 
and:
\[
    \prod_{m = 1}^\infty 
    \left[
        \prod_{n = 1}^\infty 
        (1+ b_{mn})
    \right]
    \neq 
    \prod_{n = 1}^\infty 
    \left[
        \prod_{m = 1}^\infty 
        (1+b_{mn})
    \right]
\]

Either prove no such sequence exists, or find one where 
the two products are not equal. 

\new{Solution} 
There exists a sequence where the left product is undefined 
but the right product converges to zero. Consider the following 
sequence:

\[
    b_{mn} = \frac{(-1)^m}{(m + 1)^2}
\]

Note that $b_{mn}$ is independent with regards to $n$. Hence 
the product 

\[
    \prod_{n = 1}^\infty (1+b_{mn})
\]

diverges for even $m$. Each term is constantly greater than 1. 
By the definition of nested infinite products in part b), we conclude 
that the left product is undefined. 

However, the right product converges to zero. It suffices to show:

\[
    0 < \prod_{m = 1}^\infty (1 + b_{mn}) < 1
\]

Before proving the inequality, we first demonstrate that the 
product indeed converges. Recall(from the textbook) that 
if the infinite sum $\sum^\infty a_k$ converges absolutely, then so does 
$\prod^\infty (1+a_k)$. To show convergence of the infinite product, 
we show the convergence of the following sum:

\[
    \sum_{m = 1}^\infty \frac{1}{(m+1)^2}
\]

This sum converges by the p-series test. $p = 2 > 1$ so the 
sequence converges, and hence the product converges too. 
Evidently, the product is greater than zero. 

It remains to indeed show that the infinite product converges to 
a value less than 1. Now that the convergence of the product is 
guaranteed, we group the product into pairs. That is:


\newcommand{\term}[1]{{
    \left(
        1 + \frac{1}{#1}
    \right)
}
}

\newcommand{\nterm}[1]{{
    \left(
        1 - \frac{1}{#1}
    \right)
}
}
\[
    \prod^\infty_{m = 1}\left(1 + \frac{(-1)^m}{m + 1}\right)
    =
    \nterm{2} \term{3} \nterm{4} \term{5} \cdots
\]
\[
    = 
    \prod^\infty_{m = 1}\nterm{2m - 1}\term{2m}
    \leq 
    \prod^\infty_{m = 1}\nterm{2m - 1}\term{2m - 1}
    = 
    \prod^\infty_{m = 1}\nterm{(2m-1)^2} < 1
\]

We conclude that the nested product on the right converges to zero. 
\qed



\end{document}