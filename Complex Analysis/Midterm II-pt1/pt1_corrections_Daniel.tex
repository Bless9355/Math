\documentclass{article}
\usepackage{amsfonts}
\usepackage{amsthm}
\usepackage{amsmath}
\usepackage{amssymb}
\usepackage{wasysym}
\usepackage{xcolor}


\newcommand{\new}[1]{
    \vspace{2mm}
    \noindent
    \textbf{
    \underline{#1}}
}

\newcommand{\deriv}[1]{{
    \frac{d}{d{#1}}
}
}

\newcommand{\parderiv}[1]{{
    \frac{\partial}{\partial{#1}}
}
}

\def\calO{{\mathcal{O}}}
\def\ZZ{{\mathbb{Z}}}
\def\RR{{\mathbb{R}}}
\def\_{{\hspace{1mm}}}

\def\contradiction{{\lightning}}



\newcounter{problemcnt}
\setcounter{problemcnt}{0}

\newcommand{\Problem}{{
    \vspace{5mm}
    \stepcounter{problemcnt}
    \noindent
    \arabic{problemcnt}. 
}
}

\newcommand{\nProblem}[1]{{
    \noindent
    \setcounter{problemcnt}{#1}
    \arabic{problemcnt}. 
}
}

\newcommand{\Proof}{{
    \vspace{2mm}
    \noindent
    \textbf{
    \underline{Proof}}
}
}

\newcommand{\textOr}{
    {
        \hspace{5mm}
        \textrm{or}
        \hspace{5mm}
    }
}

\newcommand{\textAnd}{
    {
        \hspace{5mm}
        \textrm{and}
        \hspace{5mm}
    }
}

\newcommand{\Exp}{
    {
        \textrm{Exp}
    }
}

\begin{document}

\begin{center}
\LARGE
Part I Corrections

\Large
Daniel Son
\end{center}

\normalsize

ii) Let $\zeta$ be any real number and $a > 0$. Evaluate:

\[
    I(\zeta):=
    \int_{-\infty}^\infty 
    \frac{e^{-2\pi \zeta x}}
    {x^2 + a^2}
    dx
\]

\new{Solution}

If $\zeta < 0$, then apply the 
substitution $x \rightarrow -x$. 

The integral converts to:

\[
    I(\zeta)=
    \int_{\infty}^{-\infty} 
    \frac{e^{-2\pi (-\zeta) x}}
    {(-x)^2 + a^2}
    (-dx)
    =
    \int_{-\infty}^{\infty} 
    \frac{e^{-2\pi (-\zeta) x}}
    {x^2 + a^2}
    dx
    =I(-\zeta)
\]
So the kernel is an even function 
with respect to $\zeta$. 

Define a holomorphic function $f(z)$ as follows:

\[
    f(z) = \frac{e^{-2\pi \zeta z}}{z^2 - a^2}
\]

The numerator and the denominator are known to be holomorphic. Thus 
the function is holomorphic everywhere other than the poles which are 
located at $z = \pm a$. Draw a semicircular contour centered 
at the origin that occupies quadrant I and IV. Call this 
contour $\gamma$, and denote the radius as $R$. 

Take the contour integral of $f(z)$ over $\gamma$. Let the 
straight segment of the contour be called $S$, and the circular 
region $C$. 

We claim that the integral over the circular region vanishes. That is, a
as $R \rightarrow \infty$, $
    \oint_C f = 0
    $

    Notice:
\[
    \bigg|
    \oint_C f
    \bigg|
    = \bigg|
    \int_{z \in C} 
    \frac{e^{-2\pi \zeta z}}{z^2+a^2}
    dz
    \bigg|
    \leq 
    \int_{z \in C} 
    \frac{max|e^{-2\pi \zeta z}|}
    {R^2 - a^2}dz
\]

Note that the modulus of an exponent is the exponent 
of the modulus of the argument. That is:

\[
    |e^{-z}| = e^{Re(-2\pi \zeta z)}
\]

And for $z \in C$, the quality is bounded under 1. 
Thus:
\[
 \bigg|
    \oint_C f
    \bigg|
    \leq 
    \frac{
    2\pi R
    }
    {R^2 - a^2}
\]
And the upper bound converges to zero as $R$ approaches infinity. 
This shows that the circular region converges to zero. 
\checkmark 

By the residue theorem:

\[
    \oint_C f + \oint_S f = 2\pi i Res_f(a)
\]

The first summand of the LHS vanishes. The second summand can 
be computed with some algebra. We write:

\[
\oint_S f = 
    \int_{x = -\infty}^{\infty}
    \frac{
        e^{-2\pi \zeta i x} \cdot (-i)dx
    }{ 
        (xi)^2 - a^2
    }
    = 
    i \int_{x = -\infty}^{\infty}
    \frac{e^{-2\pi \zeta i x} dx} 
    {x^2 + a^2}dx
    = iI
\]

The residue can be computed with ease:

\[
    Res_f(a) = 
    \lim_{z\rightarrow a} 
    \frac{e^{-2\pi \zeta z}(z-a)}{z^2-a^2}
    =
    \lim_{z\rightarrow a} 
    \frac{e^{-2 \pi \zeta z}}{z+a}
    =\frac{e^{-2 \pi \zeta a}}{2a}
\]

Combining the results, we write, for $\zeta > 0$:

\[
    iI(\zeta) = 2\pi i \frac{e^{-2 \pi \zeta a}}{2a}
    \textOr        I(\zeta) = \frac{\pi e^{-2 \pi \zeta a}}{a}
\]

Finally, for any real value $\zeta$, we 
conclude:

\[
    \boxed{
   I(\zeta) = \frac{\pi e^{-2 \pi |\zeta| a}}{a}  
    }
\]

\newpage
\vspace{5mm}
\new{Q3}
Consider the following infinite products:

\[
    I_1(a) := \prod_{n = 1}^\infty (1+a_n)
    \textAnd 
    I_2(b) := \prod_{m = 1}^\infty 
    \prod_{n = 1}^\infty(1 + b_{mn})
\]
b) State the definition for the convergence of the infinite product 
$I_2(b)$.

\new{Definiton} 
Define a double partial product $S_{ab}$ 
as follows:

\[
    S_{ab} = \prod_{m = 1}^a 
    \prod_{n = 1}^b(1 + b_{mn})
\]

We claim that if the limit of 
the partial product converges regardless 
of the relative growth of a, b, then 
the infinite product converges. That is:

\[
    \lim_{a, b\rightarrow \infty} 
    S_{ab} = L
\]

for some $L \in \mathbb{R}$. \qed 


\newpage 



\end{document}