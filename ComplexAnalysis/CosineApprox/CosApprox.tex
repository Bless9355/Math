\documentclass{article}
\usepackage{amsfonts}
\usepackage{amsthm}
\usepackage{amssymb}

\newcommand{\new}[1]{
    \vspace{2mm}
    \noindent
    \textbf{
    \underline{#1}}
}

\def\calO{{\mathcal{O}}}
\def\th{{\theta}}
\def\_{{\hspace{1mm}}}


\newcounter{problemcnt}
\setcounter{problemcnt}{0}

\newcommand{\Problem}{{
    \vspace{5mm}
    \stepcounter{problemcnt}
    \noindent
    \arabic{problemcnt}. 
}
}

\newcommand{\nProblem}[1]{{
    \noindent
    \setcounter{problemcnt}{{#1}}
    \arabic{problemcnt}. 
}
}

\newcommand{\Proof}{{
    \vspace{2mm}
    \noindent
    \textbf{
    \underline{Proof}}
}
}

\begin{document}

\begin{center}
    \LARGE
    Approximating the cosine function

    \normalsize
    Daniel Son
\end{center}

\new{Problem}
Find the average error of the following approximation
in the range $x \in [0, 2\pi]$:

Given an integer $N$, theta $n < N$ is defined as:

\[
    \th_n := 2\pi\frac{n}{N}
\]

We approximate the function $cos(x)$ by taking 
the largest $\th_n$ that is less than $x$. 

\new{Lemma}
\[
    \sum_{n=0}^{N-1}cos(2\th_n) = 0
    \hspace{5mm}
    \sum_{n=0}^{N-1}sin(2\th_n) = 0
\]
\Proof
The idea is to apply the Euler's formula 
and consider the sum as a geometric series. 
Write:

\[
    \sum_{n=0}^{N-1}cos(2\th_n)
    =
    \sum_{n=0}^{N-1}
    \frac
    {
        e^{2i\th_n}+
        e^{-2i\th_n}
    }
    {2}
     =
    \sum_{n=0}^{N-1}
    \frac
    {
        e^{n(2i\th_1)}+
        e^{n (-2i\th_1)}
    }
    {2}
\]

The sum can be computed as two seperate 
geometric series. We obtain:

\[
     \sum_{n=0}^{N-1}cos(2\th_n)
    =
    \frac{1}{2}
    \left[
        \sum_{n = 0}^{N}e^{n(2i\th_1)}
        +
        \sum_{n = 0}^{N}e^{n(-2i\th_1)}
    \right]
\]
\[
    = 
    \frac{1}{2}
    \left[
        \frac{1-e^{N2i\th_1}}{1-e^{2i\th_1}}
        +
        \frac{1-e^{-N2i\th_1}}{1-e^{-2i\th_1}}
    \right]
\]

Note that $N2i\th_1 = 4i\pi$. Thus, 
$e^{N2i\th_1} = 1$
Furthermore, the numerators of the fractions
reduce to zero. This concludes the proof of the first 
equation. Repeat a similar process for the 
second equation.  


\[
    \sum_{n=0}^{N-1}sin(2\th_n)
    =
    \sum_{n=0}^{N-1}
    \frac
    {
        e^{2i\th_n}-
        e^{-2i\th_n}
    }
    {2i}
     =
    \sum_{n=0}^{N-1}
    \frac
    {
        e^{n(2i\th_1)}-
        e^{n (-2i\th_1)}
    }
    {2i}
\]

The sum can be computed as two seperate 
geometric series. We obtain:

\[
     \sum_{n=0}^{N-1}sin(2\th_n)
    =
    \frac{1}{2i}
    \left[
        \sum_{n = 0}^{N}e^{n(2i\th_1)}
        -
        \sum_{n = 0}^{N}e^{n(-2i\th_1)}
    \right]
\]
\[
    = 
    \frac{1}{2i}
    \left[
        \frac{1-e^{N2i\th_1}}{1-e^{2i\th_1}}
        -
        \frac{1-e^{-N2i\th_1}}{1-e^{-2i\th_1}}
    \right]
\]

And again, both fractions are zero. 
\qed

\new{Solution}
The idea to compute the approximation is to 
cut the function into $N$ regions, each of length
$\th_1$. 

Define the following integral:

\[
    I_n := \int_{\th_n}^{\th_{n+1}}[cos(x)-cos(\th_n)]^2dx
\]

Through some algebra, we reduce the integrand as:

\[
    cos^2(x)-2cos(\th_n)cos(x)+cos^2(\th_n)
    \] \[
    =
    \frac{1+cos(2x)}{2}
    -2cos(\th_n)cos(x)
    +\frac{1+cos(2\th_n)}{2}
\]

Integrate over the range 
$x \in [\th_n, \th]$. This results 
in three integrals. 

\[
    I_n = 
    \int_{\th_n}^{\th_{n+1}}
    \frac{1+cos(2x)}{2}dx
    -2cos(\th_n)
    \int_{\th_n}^{\th_{n+1}}
    cos(x)dx
    +
    \int_{\th_n}^{\th_{n+1}}
    \frac{1+cos(2\th_n)}{2}dx
\]

Label them as:
\[
    I_n := A_n-2B_n+C_n
\]

Adding up all the $I_n$'s, we get:
\[
     I = \sum_{n = 0}^{N-1} (A_n-2B_n+C_n)
     = \sum_{n = 0}^{N-1} A_n 
     -2\sum_{n = 0}^{N-1} B_n 
     +\sum_{n = 0}^{N-1} C_n 
\]

Evaluate each integrals. Start with $A$:
\[
    A_n = 
    \frac{1}{2}
    [x+sin(2x)/2]_{\th_n}^{\th_{n+1}}
    =
    \frac{1}{2}
    [\th_1+sin(2\th_{n+1})/2-sin(2\th_{n})/2]
\]
\[
    \sum_{n = 0}^{N-1} A_n = 
    \frac{N\th_1}{2}+sin(2\th_{N})/2-sin(0)/2
    =\pi
\]

For $C_n$, notice that the integrand is constant:
\[
    C_n = \th_1\left[
    \frac{1+cos(2\th_n)}{2}
    \right]
\]
\[
    \sum_{n = 0}^{N-1}C_n = 
    \frac{\th_1}{2}\sum_{n = 0}^{N-1}
    \left[
        1+cos(2\th_n)
    \right]
\]
And notice that the latter term vanishes 
by the lemma. Thus:
\[
    \sum_{n = 0}^{N-1}C_n =  N\th_1/2 = \pi
\]

B is tricky. Before we proceed, compute the 
following sum:

\newpage

\[
    cos(\th_n)sin(\th_{n+1}) = 
    \frac{e^{\th_1ni}+e^{-\th_1ni}}{2}
    \frac{e^{\th_1(n+1)i}-e^{-\th_1(n+1)i}}{2i}
\] \[
    =
    \frac{e^{\th_1(2n+1)i}
        +e^{\th_1i}
        -e^{-\th_1i}
        -e^{-\th_1(2n+1)i}
    }
    {4i}
\]
\[
    =
    \frac{e^{\th_1(2n+1)i}
        -e^{-\th_1(2n+1)i}
    }
    {4i}
    +
    \frac{e^{\th_1i}
        -e^{-\th_1i}}
        {2\cdot 2i}
\]
\[
    =
    \frac{e^{\th_1(2n+1)i}
        -e^{-\th_1(2n+1)i}
    }
    {4i}
    +
    \frac{sin(\th_1)}
        {2}
\]

Applying sums:
\[
    \sum_{n = 0}^{N-1}
 cos(\th_n)sin(\th_{n+1})
  = \sum_{n = 0}^{N-1}
\left(
\frac{e^{\th_1(2n+1)i}
        -e^{-\th_1(2n+1)i}
    }
    {4i}
\right)
+\frac{Nsin(\th_1)}{2}
\]

\[
    =
    \frac{1}{4i}
    \left[
    e^{\th_1i}
    \sum_{n = 0}^{N-1}
    e^{2\th_1ni}
    -
    e^{-\th_1i}
    \sum_{n = 0}^{N-1}
    e^{-2\th_1ni}
    \right]
+\frac{Nsin(\th_1)}{2}
\]

\[
    =
    \frac{1}{4i}
    \left[
    e^{\th_1i}
    \frac{1-e^{2\th_1Ni}}
    {1-e^{2\th_1i}}
    -
    e^{-\th_1i}
    \frac{1-e^{-2\th_1Ni}}
    {1-e^{-2\th_1i}}
    \right]
+\frac{Nsin(\th_1)}{2}
\]

\[
    =\frac{Nsin(\th_1)}{2}
\]

Now, compute $B_n$ and its sum:
\[
    B_n = 
    cos(\th_n)
    \int_{\th_n}^{\th_{n+1}}
    cos(x)dx
    =cos(\th_n)\left[
        sin(x)
    \right]_{\th_n}^{\th_{n+1}}
\]
\[
    =cos(\th_n)sin(\th_{n+1})
    -cos(\th_n)sin(\th_{n})
    \]\[
    =cos(\th_n)sin(\th_{n+1})
    -sin(2\th_n)/2
\]

\[
    \sum_{n = 0}^{N-1}B_n =
     \sum_{n = 0}^{N-1}cos(\th_n)sin(\th_{n+1})
    -\sum_{n = 0}^{N-1}sin(2\th_n)/2
\]

By the lemma and our established result above:
\[
    \sum_{n = 0}^{N-1}B_n = 
    \frac{Nsin(\th_1)}{2}
\]

Finally, compute $I$:
\[
     I 
     = \sum_{n = 0}^{N-1} A_n 
     -2\sum_{n = 0}^{N-1} B_n 
     +\sum_{n = 0}^{N-1} C_n 
\]
\[
    =\pi-2\frac{Nsin(\th_1)}{2}+\pi = 2\pi - Nsin(\th_1)
\]
To compute the average error over the 
region $x \in [0, 2\pi]$, divide $I$ by 
$2\pi$. The average error is thus:

\[
    1-\frac{Nsin(\th_1)}{2\pi} = 1-\frac{sin(\th_1)}{\th_1}
\]

As a sanity check, we observe 
that as $N\rightarrow\infty$, 
$\th_1\rightarrow 0$ and the error 
also goes to zero, for $sin(\th_1)\rightarrow \th_1$
\qed
\end{document}

