\documentclass{article}
\usepackage{amsfonts}
\usepackage{amsthm}
\usepackage{amssymb}
\usepackage{amsmath}
\usepackage{graphicx}
\usepackage{subcaption}

\newcommand{\new}[1]{
    \vspace{2mm}
    \noindent
    \textbf{
    \underline{#1}}
}

\def\calO{{\mathcal{O}}}
\def\th{{\theta}}
\def\_{{\hspace{1mm}}}
\def\<{{\langle}}
\def\>{{\rangle}}


\newcounter{problemcnt}
\setcounter{problemcnt}{0}

\newcommand{\Problem}{{
    \vspace{5mm}
    \stepcounter{problemcnt}
    \noindent
    \arabic{problemcnt}. 
}
}

\newcommand{\nProblem}[1]{
    \vspace{5mm}
    \noindent
    \setcounter{problemcnt}{#1}
    \arabic{problemcnt}. 
}


\newcommand{\Proof}{{
    \vspace{2mm}
    \noindent
    \textbf{
    \underline{Proof}}
}
}

\newcommand{\textOr}{
    {
        \hspace{5mm}
        \textrm{or}
        \hspace{5mm}
    }
}

\newcommand{\textAnd}{
    {
        \hspace{5mm}
        \textrm{and}
        \hspace{5mm}
    }
}

\def\Ohm{{\Omega}}


\newcommand{\hx}{\hat{x}}


\newcommand{\hy}{\hat{y}}


\newcommand{\hz}{\hat{z}}


\newcommand{\deriv}[1]{{
    \frac{d}{d{#1}}
}
}

\newcommand{\parderiv}[1]{{
    \frac{\partial}{\partial{#1}}
}
}




\begin{document}

\new{P\&M 9.26}
An electromagnetic field in free space is described 
by the following two equations:

\[
    \vec{E} = \hy E_0 \sin(kx + \omega t)
\]
\[
    \vec{B} = -\hz (E_0/c) \sin(kx + \omega t)
\]

i) Show that the field satisfies the Maxwell's equations 

\new{Solution}
It is easy to observe that the divergence of both field 
equals to zero. The $\hy$ component of the electric field 
has no $y$ dependance and the $\hz$ component of the 
magnetic field has no $z$ dependance. 

It suffices to verify the relationship regarding the divergence. 

With some computation, we deduce:

\[
    \Delta \times \vec{E} 
    = 
    \hz E_0 k \cos(kx+ \omega t)
\]
\[
    \Delta \times \vec{B}
    =
    \hy E_0(k/c) \cos(kx + \omega t)
\]

\[
    \parderiv{t} \vec{E}
    = 
    \omega E_0 \cos(kx + \omega t)
\]

\[
    \parderiv{t} \vec{B}
    =
    -\omega (E_0/c) \cos(kx + \omega t)
\]

Recall the Faraday's Law and Ampere's Law in differential form:

\[
    \Delta \times \vec{E} 
    = 
    - \parderiv{t} \vec{B}
    \textAnd
    \Delta \times \vec{B}
    =
    \frac{1}{c^2}\parderiv{t} \vec{E}
\]

With cancellation of the cosine terms, the two equations convert to:

\[
    E_0 k = \frac{E_0 \omega}{c}
    \textAnd
    \frac{E_0k}{c} = 
    \frac{\omega E_0}{c^2}
\]

With more cancellation, both equations imply:

\[
    \boxed{
    c = \frac{\omega}{k}
    }
\]


\vspace{5mm}
ii) Suppose $\omega = 10^{10}s^{-1}$ and $E_0 = 1kV/m$
. Compute the energy density of the field per cubic meter 
and the rate of energy transfer per square meter. 

\new{Solution}
The energy density of an electromagnetic field is 
given as:

\[
    \frac{E^2\epsilon_0}{2} + \frac{ B^2}{2\mu_0}
\]

Where $E$ and $B$ denotes the magnitude of the electric 
and magnetic field. 

The field is given as:

\[
    \vec{E} = \hy E_0 \sin(kx + \omega t)
\]
\[
    \vec{B} = -\hz (E_0/c) \sin(kx + \omega t)
\]

The energy density at a point can thus be computed by:

\[
    \frac{
\epsilon_0
E_0^2 \sin^2(kx + \omega t)
    }
    {2}
    + 
    \frac{ 
(E_0/c)^2 \sin^2(kx + \omega t)
    }
    {2\mu_0}
\]

For sufficiently large enough range of time and space, 
the square of the sine term averges out to $1/2$. 
The following relationship between $\mu_0, \epsilon_0, c$ 
comes handy:

\[
    \mu_0\epsilon_0 = 1/c^2
\]

The average energy density is:

\[
    \frac{E_0^2\epsilon_0}{4}
    +
    \frac{1}{c^2\mu_0}
    \frac{E_0^2}{4}
    =
    \boxed{
    \frac{E_0^2 \epsilon_0} {2}
    \approxeq 
    4.4 \cdot 10^{-6} J/m^3
    }
\]

To compute the power flow, it is useful to 
use the method of Poynting vectors. The 
Poynting vector is a vector field defined as:

\[
    \vec{S} = 
    \frac{\vec{E} \times \vec{B}}
    {\mu_0}
\]

Thus:
\[
    \vec{S} = \hy E_0 \sin(kx + \omega t) \times 
    -\hz (E_0/c) \sin(kx + \omega t)
    /\mu_0
\]
\[
    = -\hx E_0^2/(c^2\mu_0) \sin^2(kx + \omega t)
\]

Take a large surface perpendicular to $\hx$. 
The power transfer along a surface can be computing 
by computing the surface integral of the Poyinting 
vector. 
Given that the area is sufficiently large enough, 
the average charge density will equal to the average 
of the magnitude of the Poynting vectors. 

We write the average power density as:

\[
    \boxed{
    \bigg|\parderiv{t}U\bigg| = \frac{\epsilon_0 E_0^2}{2c}
\approxeq 1300 J/(m^2s)
    }
\]

\newpage

\new{P\&M 9.27}
A sinusodial wave reflects at the surface of a medium which 
absorbs the half of the energy of the incident wave. 
Compute VSWR(Voltage Standing Wave Ratio) of the standing 
EM wave created by the reflection. 

\new{Solution}
From the Poynting vector equation, we observe that the magnitude 
of energy is proportional to the product of the magnitude of 
the electric and magnetic field. The magnetic field is a constant 
multiple of the electric field for traveling waves in free space. 
Hence, we conclude that the energy of a traveling wave is proportional 
to the square amplitude of the electric field. 

If the medium absorbs half the energy of the incident wave, the 
reflection will have an amplitude reduced by a factor of $1/\sqrt{2}$. 
VSWR, by definition, is the maximum amplitude of the standing wave 
divided by the minimum amplitude. The standing wave reaches its 
maximum amplitude when the incident and the reflection interacts 
constructively. Minimum amplitude is achieved when the two waves 
cancel each other out. Thus, we write:

\[
    \boxed{
    VSWR = \frac{(1+1/\sqrt{2})E}{(1-1/\sqrt{2})E}
    = \frac{(\sqrt{2}+1)^2}{2-1} \approxeq 5.83
    }
\]



\end{document}
