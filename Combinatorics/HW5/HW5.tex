\documentclass{article}
\usepackage{amsfonts}
\usepackage{amsthm}
\usepackage{amssymb}
\usepackage{amsmath}
\usepackage{graphicx}
\usepackage{subcaption}
\usepackage[shortlabels]{enumitem}
\usepackage{xcolor}
\usepackage{tikz}

\newcommand{\new}[1]{
    \vspace{2mm}
    \noindent
    \textbf{
    \underline{#1}}
}

\def\calO{{\mathcal{O}}}
\def\th{{\theta}}
\def\_{{\hspace{1mm}}}
\def\<{{\langle}}
\def\>{{\rangle}}


\newcounter{problemcnt}
\setcounter{problemcnt}{0}

\newcommand{\Problem}{{
    \vspace{5mm}
    \stepcounter{problemcnt}
    \noindent
    \arabic{problemcnt}. 
}
}

\newcommand{\nProblem}[1]{
    \vspace{5mm}
    \noindent
    \setcounter{problemcnt}{#1}
    \arabic{problemcnt}. 
}


\newcommand{\Proof}{{
    \vspace{2mm}
    \noindent
    \textbf{
    \underline{Proof}}
}
}

\newcommand{\textOr}{
    {
        \hspace{5mm}
        \textrm{or}
        \hspace{5mm}
    }
}

\newcommand{\textAnd}{
    {
        \hspace{5mm}
        \textrm{and}
        \hspace{5mm}
    }
}

\newcommand{\m}{
    \cdot
}

\newcommand{\Pt}[1]{
    $P_{#1}$
}

\newcommand{\Szpt}[1]{
    $|P_{#1}|$
}

\newcommand{\bbinom}[2]{
    \bigg(\binom{#1}{#2}\bigg)
}



\begin{document}
\begin{center}
\LARGE
Combinatorics HW5

\Large
Daniel Son
\end{center}

Section 2.7: 42, 50, 54

Section 5.7: 27, 29, 36


\new{Sec2.7Q42} 
Determine the number of ways to distribute 10 orange drinks, 1 lemon drink, 
and 1 lime drink to four thirsty students so that each student gets at least one 
drink, and the lemon and lime drinks go to different students.

\new{Solution}
Apply the principle of multiplication. Give out the lemon 
and the lime juice to two distinct students. Then, give out 
two orange juices to the two student who has not yet received 
any juice. Then, distribute the eight orange juices by random. 

\[
    P(4, 2) \bbinom{4}{8} = 4\m3 \binom{11}{3} = \boxed{1980}
\]

\new{Sec2.7Q50}
In how many ways can five identical rooks be placed on the squares of an 8-by-8 
board so that four of them form the corners of a rectangle with sides parallel to 
the sides of the board? 

\new{Solution}
Choosing two rows and two columns determine the position 
of the four rooks that form a rectangle. Choose the last 
rook from any of the $64-4 = 60$ positions. We count the answer 
as follows. 
\[
    \binom{8}{2} ^2\m60 = \boxed{47040}
\]

\new{Sec2.7Q54}
Determine the number of towers of the form $\emptyset \subseteq A \subseteq B \subseteq [n]$. 

\new{Solution} 
Notice that each tower correspond to a unique partition of the 
cannonical set $[n]$, that is $A, B\setminus A, [n] \setminus B$. 
Calling these parts $X, Y, Z$ respectively, we notice that 
$(A, B) = (X, X\cup Y)$. We have demonstrated an injective mapping 
from the tower to the partition, and from partition to the tower. Hence, 
the mapping is bijective. 

The number of ways to partition $[n]$ to three parts is $3^n$. Thus, 
there are $\boxed{3^n}$ towers. We list out all the towers for $n = 2$. 

\[
    (A, B) = 
    (\emptyset, \emptyset),
    (\emptyset, \{1\}),
    (\emptyset, \{2\}),
(\emptyset, \{1, 2\}),
    (\{1\} ,  \{1, 2\}),
    (\{1\},  \{1\} ),
(\{2\},  \{2\} ),
(\{2\},  \{1, 2\} ),
(\{1, 2\},  \{1, 2\} )
\]

\new{Sec5.7Q27} 
Give a combinatorial proof of the identity 
\[
  n(n+1) 2^{n - 2} = \sum_{k = 1}^{n}k^2\binom{n}{k}
\]

\new{Solution}
Consider the following scenario. A toy shop has $n$ different 
toys. John plans to buy some set of the toys for his children 
Amy and Bob. After John has bought $k \leq n$ toys from the shop, 
he takes it home where Amy and Bob choose to play with one of the toys 
each. We assume Amy and Bob to behave nicely, meaning that even if they 
choose the same toy, they will not fight each other. 

If John brings $k$ toys from home, there are $k^2$ scenarios 
where Amy and Bob chooses a single toy. John brings $k$ toys from 
a total of $n$ toys from the shop. Hence, the total configuration 
of the toys at the end of the day is 
\[
    \sum_{k = 1}^n k^2 \binom{n}{k}
\]

Suppose Charles, the owner of the toy shop, computes the total possible 
arrangement of the toys. The first possible scenario is that Amy and 
Bob both choose the same toy. The chosen toy must be taken by John, 
and all the remaining $n - 1$ toys can be taken home or stay in the shop. 
There are 
\[
    n\m2^{n - 1}
\]
such cases. 

The alternate scenario is when the two children choose different 
toys. By a similar argument there are 
\[
    P(n, 2) 2^{n - 2}
\]
such cases. 

Adding up the size of the two parts, we obtain 
\[
      n\m2^{n - 1} +   P(n, 2) 2^{n - 2}
      = (2n + n(n - 1))2^{n - 2} = n(n + 1)2^{n-2}
\]
For we have double counted all the possible toy arrangements, 
the combinatorial sum must equal to the term that we have obtained. 
\[
  n(n+1) 2^{n - 2} = \sum_{k = 1}^{n}k^2\binom{n}{k}
\]
\hfill \qed

\new{Sec5.7Q29} 
Find and prove a formula for 
\[
    \sum_{\substack{r, s, t \geq 0\\r + s + t = n}}
    \binom{m_1}{r} \binom{m_2}{s} \binom{m_3}{t}
\]

\new{Solution}
Use a similar argument used for Vandermonde Convolutions. From 
a set of 
$m_1 + m_2 + m_3$ elements, imagine choosing a total of $n$ 
elements. Partitioning the orignal set into three sets of 
size $m_1, m_2, m_3$, we observe that the total ways 
to choose $n$ element from the original set is to count 
the number of ways to choose $r, s, t$ elements from the 
respective parts, such that the scripts range under 
the condition $r + s + t = n$. 

On the other hand, it is trivial that there are $\binom{m_1+m_2+m_3}{n}$
ways to choose a subset of size $n$ from the original set. For 
we have counted the same quantity twice, we establish the 
following identity. 

\[
    \sum_{\substack{r, s, t \geq 0\\r + s + t = n}}
    \binom{m_1}{r} \binom{m_2}{s} \binom{m_3}{t}
    =
\binom{m_1+m_2+m_3}{n}
\]
\hfill \qed

\new{Sec5.7Q36} 
Prove the Vandermonde Convolutions using the Binomail Theorem. 

\new{Solution}
We start with the identity 
\[
    (1+x)^{m_1}(1+x)^{m_2} = (1+x)^{m_1+m_2}
\]
Foil both sides using the Binomial Theorem. 
\[
    \bigg(\sum_{k = 0}^{m_1} \binom{m_1}{k} x^k\bigg)
    \bigg(\sum_{j = 0}^{m_2} \binom{m_2}{j} x^j\bigg)
    = 
    \sum_{l = 0}^{m_1 + m_2} \binom{m_1 + m_2}{l} x^l
\]
Focus on the coefficient of $x^l$ for some fixed $l$. 
For the term $x^k$, the corresponding multiple must be $x^{l - k}$. 
The coefficients of $x^l$ must be equal for both sides. Hence, 
\[
    \sum_{
        \substack{
            k + j = l \\
            0 \leq k \leq m_1 \\ 
            0 \leq j \leq m_2
        }
    }
        \binom{m_1}{k}
        \binom{m_2}{j}
    = \binom{m_1 + m_2}{l}
\]

Note that the binomial term vanishes if the bottom term is less 
than equal to zero. That is $\binom{a}{b} = 0$ for any $b \leq 0$. 
Rewrite the LHS. 

\[
    \sum_{
        k = 0
    }
    ^{l}
        \binom{m_1}{k}
        \binom{m_2}{l - k}
    = 
\binom{m_1 + m_2}{l}
\]
\hfill \qed

\end{document}