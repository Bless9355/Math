\documentclass{article}
\usepackage{amsfonts}
\usepackage{amsthm}
\usepackage{amssymb}
\usepackage{amsmath}
\usepackage{graphicx}
\usepackage{subcaption}
\usepackage[shortlabels]{enumitem}
\usepackage{xcolor}
\usepackage{tikz}

\newcommand{\new}[1]{
    \vspace{2mm}
    \noindent
    \textbf{
    \underline{#1}}
}

\def\calO{{\mathcal{O}}}
\def\th{{\theta}}
\def\_{{\hspace{1mm}}}
\def\<{{\langle}}
\def\>{{\rangle}}


\newcounter{problemcnt}
\setcounter{problemcnt}{0}

\newcommand{\Problem}{{
    \vspace{5mm}
    \stepcounter{problemcnt}
    \noindent
    \arabic{problemcnt}. 
}
}

\newcommand{\nProblem}[1]{
    \vspace{5mm}
    \noindent
    \setcounter{problemcnt}{#1}
    \arabic{problemcnt}. 
}


\newcommand{\Proof}{{
    \vspace{2mm}
    \noindent
    \textbf{
    \underline{Proof}}
}
}

\newcommand{\textOr}{
    {
        \hspace{5mm}
        \textrm{or}
        \hspace{5mm}
    }
}

\newcommand{\textAnd}{
    {
        \hspace{5mm}
        \textrm{and}
        \hspace{5mm}
    }
}

\newcommand{\m}{
    \cdot
}

\newcommand{\Pt}[1]{
    $P_{#1}$
}

\newcommand{\Szpt}[1]{
    $|P_{#1}|$
}

\newcommand{\bbinom}[2]{
    \bigg(\binom{#1}{#2}\bigg)
}



\begin{document}
\begin{center}
\LARGE
Combinatorics HW5

\Large
Daniel Son
\end{center}

\new{Counting Derangements}
A derangement is a permutation of the set 
where no elements are fixed. 
We define $D_n$ to be the number of 
derangements of the cannonical set $[n]$.
By the inclusion-exclusion 
principle, we derive 
\[
    D_n = n!\left(1 - \frac{1}1 + \frac1{2!} - \frac1{3!} + \cdots +(-1)^n\frac1{n!}\right)
\]
By the alternating series test, we conclude 
\[
    D_n = \left\{
        \frac{n!}e
    \right\}
\]

\end{document}