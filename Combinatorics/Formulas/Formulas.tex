\documentclass{article}
\usepackage{amsfonts}
\usepackage{amsthm}
\usepackage{amssymb}
\usepackage{amsmath}
\usepackage{graphicx}
\usepackage{subcaption}
\usepackage[shortlabels]{enumitem}
\usepackage{xcolor}
\usepackage{tikz}

\newcommand{\new}[1]{
    \vspace{2mm}
    \noindent
    \textbf{
    \underline{#1}}
}

\def\calO{{\mathcal{O}}}
\def\th{{\theta}}
\def\_{{\hspace{1mm}}}
\def\<{{\langle}}
\def\>{{\rangle}}


\newcounter{problemcnt}
\setcounter{problemcnt}{0}

\newcommand{\Problem}{{
    \vspace{5mm}
    \stepcounter{problemcnt}
    \noindent
    \arabic{problemcnt}. 
}
}

\newcommand{\nProblem}[1]{
    \vspace{5mm}
    \noindent
    \setcounter{problemcnt}{#1}
    \arabic{problemcnt}. 
}


\newcommand{\Proof}{{
    \vspace{2mm}
    \noindent
    \textbf{
    \underline{Proof}}
}
}

\newcommand{\textOr}{
    {
        \hspace{5mm}
        \textrm{or}
        \hspace{5mm}
    }
}

\newcommand{\textAnd}{
    {
        \hspace{5mm}
        \textrm{and}
        \hspace{5mm}
    }
}

\newcommand{\textThen}{
    {
        \hspace{5mm}
        \textrm{then}
        \hspace{5mm}
    }
}




\newcommand{\halfFigure}[1]{
\begin{center}
\includegraphics[width = .5\linewidth]{{#1}}
\end{center}
}

\newcommand{\fullFigure}[2]{
\begin{center}
\includegraphics[width = .9\linewidth]{{#1}}
\end{center}
}



\newcommand{\m}{
    \cdot
}

\newcommand{\Pt}[1]{
    $P_{#1}$
}

\newcommand{\Szpt}[1]{
    $|P_{#1}|$
}

\newcommand{\bbinom}[2]{
    \bigg(\binom{#1}{#2}\bigg)
}



\begin{document}
\begin{center}
\LARGE
Combinatorics HW5

\Large
Daniel Son
\end{center}

\new{Counting Derangements}
A derangement is a permutation of the set 
where no elements are fixed. 
We define $D_n$ to be the number of 
derangements of the cannonical set $[n]$.
By the inclusion-exclusion 
principle, we derive 
\[
    D_n = n!\left(1 - \frac{1}1 + \frac1{2!} - \frac1{3!} + \cdots +(-1)^n\frac1{n!}\right)
\]
By the alternating series test, we conclude 
\[
    D_n = \left\{
        \frac{n!}e
    \right\}
\]

\new{Posets and Convolutions}

Let $(X, \leq)$ be a finite poset. We consider 
a class of functions that map pairs of the poset 
$X$ to the reals. Let $f, g: X \times X \rightarrow \mathbb{R}$. 
Define a discrete convolution of the two posets as follows. 

\[
    f * g(x, y) = \sum_{x \leq z \leq y} f(x, z) \m g(z, y)
\]

We define three important functions, each corresponding 
to the identity, the ordering, and the inverse of the ordering. 
They are called the Kronecker Delta, Zeta, and the Mobius Function. 

\[
\delta(x, y) =
\begin{cases}
     1 &\text{if } x = y \\ 
    0 &\text{otherwise} \\
\end{cases}
\textAnd
\zeta(x, y) =
\begin{cases}
     1 &\text{if } x \leq y \\ 
    0 &\text{otherwise} \\
\end{cases}
\]

It is trivial to find out that the delta function is 
the convolutional identity. 

Before writing out the Mobius function, we introduce a 
constructive method to obtain the convolutional inverse 
of an arbitrary function $f$. We requre $f(y, y)$ to be 
nonzero. 

Let $g$ be the left inverse of $f$. We easily observe 
that for nondistinct paris, $g$ must be the reciprocal of 
$f$. 
\[
    g(y, y) = \frac 1{f(y, y)} \hspace{5mm} \forall y \in X
\]

For distinct pairs, the convolution of $f, g$ must yield zero. 
If $x > y$, then the convolution is automatically zero. 
That is, 
assuming $x < y$, 
\[
    f*g(x, y) = 
    \sum_{x \leq z \leq y} f(x, z) \m g(z, y) = 0
\]
Break down the sum. 
\[
    f(x, x) \m g(x, y) + \sum_{x < z \leq y} f(x, z) \m g(z, y)
     = 0
\]
Sove for $g(x, y)$. 
\[
    g(x, y) = -
    \frac{1}{f(x, x)} \sum_{x < z \leq y} f(x, z) \m g(z, y)
\]

It is not hard to see that convolutions are associative. 
This leads us to conclude that the left inverse equals to 
right inverse. 

\[
    f_l * f * f_r = \delta *f_r = \delta * f_l 
    \textOr 
    f_r = f_l
\]

Finally, we present the Mobius Function. The mobius 
function is defined as the inverse of the zeta function. 
plug in $f \mapsto \zeta$. 

\[
    \mu(x, y) = 
    \begin{cases}
        1 & \text{if x = y}\\
    - \sum_{\substack{x < z \leq y}} \mu(z, y) & \text{otherwise}
    \end{cases}
    \textThen 
    \mu * \zeta = \delta
\]

\new{Proof of Mobius Inversion}
\fullFigure{Mobius_proof.png}

\new{Tips for Mobius Inversion}

It is necessary that the cumulative function $G$ 
is of simple form. If is is not clear what $G$ is, 
then take the compliment of $G$'s argument with 
respect to the universal set. 

For example, it is horrendous to compute:
\[
    G(n) = \sum_{i|n} \phi(i)
\]
However, consider
\[
    G(n) = \sum_{i|n} \phi(n/i)
\]
Each divisor $i$ uniquely maps to another divisor $n/i$. 
If a number $\xi$ is coprime with $n/i$, $gcd(\xi \m i, n) = i$. 
More precisely, $(\xi, n/i) = 1$ iff $(\xi \m i, n) = i$. $\phi(n/i)$
counts the number of such $\xi$, and this corresponds to the numbers 
that have a gcd $i$ with $n$. Each number in $[n]$ must have 
some gcd that divides n. Thus, $G(n)$ counts all numbers between $1, n$. 

\new{Classic Mobius Inversion}

Memorize this sum:
\[
    \sum_{i | n} \mu(n/i) i = \phi(i)
\]


\end{document}