\documentclass{article}
\usepackage{amsfonts}
\usepackage{amsthm}
\usepackage{amssymb}
\usepackage{amsmath}
\usepackage{graphicx}
\usepackage{subcaption}
\usepackage{xcolor}

\newcommand{\new}[1]{
    \vspace{2mm}
    \noindent
    \textbf{
    \underline{#1}}
}

\def\calO{{\mathcal{O}}}
\def\th{{\theta}}
\def\_{{\hspace{1mm}}}
\def\<{{\langle}}
\def\>{{\rangle}}


\newcounter{problemcnt}
\setcounter{problemcnt}{0}

\newcommand{\Problem}{{
    \vspace{5mm}
    \stepcounter{problemcnt}
    \noindent
    \arabic{problemcnt}. 
}
}

\newcommand{\nProblem}[1]{
    \vspace{5mm}
    \noindent
    \setcounter{problemcnt}{#1}
    \arabic{problemcnt}. 
}


\newcommand{\Proof}{{
    \vspace{2mm}
    \noindent
    \textbf{
    \underline{Proof}}
}
}

\newcommand{\textOr}{
    {
        \hspace{5mm}
        \textrm{or}
        \hspace{5mm}
    }
}

\newcommand{\textAnd}{
    {
        \hspace{5mm}
        \textrm{and}
        \hspace{5mm}
    }
}

\newcommand{\m}{
    \cdot
}

\newcommand{\Ixp}[1]{
    {
        e^{i{#1}}
    }
}



\newcommand{\halfFigure}[1]{
\begin{center}
\includegraphics[width = .5\linewidth]{{#1}}
\end{center}
}

\newcommand{\fullFigure}[1]{
\begin{center}
\includegraphics[width = .9\linewidth]{{#1}}
\end{center}
}

\def\twobytwoMat(#1, #2, #3, #4){
    {
        \begin{bmatrix}
            {#1} & {#2}\\
            {#3} & {#4}
        \end{bmatrix}
    }
}

\def\twobyoneMat(#1, #2){
    {
        \begin{bmatrix}
            {#1}\\
            {#2}
        \end{bmatrix}
    }
}

\def\twobytwoDet(#1, #2, #3, #4){
    {
        \begin{vmatrix}
            {#1} & {#2}\\
            {#3} & {#4}
        \end{vmatrix}
    }
}



\begin{document}
\begin{center}
\LARGE
Combinatorics HW7

\Large
Daniel Son
\end{center}

\normalsize 

\new{Q1 sums of fibbonacci numbers}

Find a closed form solution for all the partial sums defined below. 

a) \[
s_n := \sum_{i = 1}^n f_{2i - 1}    
\]

b) \[
s_n := \sum_{i = 0}^n f_{2i}    
\]

c) \[
s_n := \sum_{i = 0}^n (-1)^if_{i}    
\]

d) \[
s_n := \sum_{i = 0}^n f_{i}^2    
\]



\new{Solution}

1. 
\[
    s_1 = 1 \textAnd 
    s_n = f_{2n}
\]

2. \[
    s_0 = 0 \textAnd 
    s_n = f_{2n + 1} - 1 
\]

3. \[
    s_n = (-1)^{n} f_{n - 1} - 1  
    \textAnd f_{-1} := 1
\]

4. \[
    s_n = f_{n + 1}^2 - f_{n}^2 - (-1)^n
\]

\new{Proof a}
The base case holds trivially. $s_1 = f_1 = 1$. 
For $n > 1$, 
\[
    s_n  = f_{2n -1} + s_{n -1} = f_{2n - 1} + f_{2n - 2}
\]
by the inductive hypothesis. Use the recursive relation for $f_n$. 
\[
    \boxed{
    s_n = f_{2n}
    }
\]

\new{Proof b}
The base case holds trivially. $s_0 = f_1 - 1 = 0$. 
For $n > 0$, 
\[
    s_n  = f_{2n} + s_{n -1} = f_{2n} + f_{2n - 1} - 1
\]
by the inductive hypothesis. Use the recursive relation for $f_n$. 
\[
    \boxed{
    s_n = f_{2n + 1} - 1
    }
\]

\new{Proof c}
Apply a similar argument for a, b. $s_0 = (-1)^0f_{-1} - 1 =1-1= 0$ 
for the base case. For the inductive case, 
\[
    s_{n + 1}  =  (-1)^{n + 1}f_{n +1} + s_{n} = (-1)^{n + 1}f_{n+1} + (-1)^{n}f_{n - 1} - 1
\]
\[
    = (-1)^{n+1} (f_{n+1} - f_{n - 1}) - 1 = (-1)^{n+1} f_{n}  - 1
\]
which concludes the proof. 
\hfill \qed

\new{Proof d}
We present an arbitrary polynomial division that 
will come in handy later. 
\[
    x^2 - x - 1 
    \parallel
x^4 - 3x^2 + 1
\]
and this is over the field of real numbers. 
This implies that the roots of $x^2 - x -1$ are also 
roots for  $x^4 - 3x^2 + 1$. 

Now we prove the identity 
\begin{equation}
    3 f_{n+1}^2 - f_{n+2}^2 - f_n^2 - 2(-1)^n = 0
\end{equation}

Recall that the nth fibbonacci number can be represented 
as follows.
\[
    f_n = 
    \frac{p^n - q^n} {\sqrt{5}}
\]
p, q are $(1 + \sqrt{5})/2, (1 - \sqrt{5}) /2$ respectively. 
They are roots of the polynomial $ x^2 - x - 1 $.
To prove the desired identiy in (1), we evaluate the 
left hand side directly and show that it is identically zero
. We obtain 

\begin{equation}
    \begin{split}
    \frac 3 {{5}} 
    \left[
        p^{2n + 2} + q^{2n + 2} - 2(pq)^{n + 1}
    \right]
    -\frac 1 {{5}}
    \left[
        p^{2n + 4} + q^{2n + 4} - 2(pq)^{n + 2}
    \right] 
    \\
-\frac 1 {{5}}
    \left[
        p^{2n} + q^{2n} - 2(pq)^{n}
    \right] 
    - 2(-1)^n
    \end{split}
\end{equation}

\begin{equation}
    \begin{split} = 
        -\frac 1{{5}} \left[
                (p^4 - 3p^2 + 1)p^{2n}
+ (q^4 - 3q^2 + 1)q^{2n}
        \right] 
        \\ 
        -\frac 6 5 (pq)^{n + 1} 
        + \frac 2 5 (pq)^{n + 2} 
        + \frac 2 5 (pq)^{n} - 2(-1)^n
    \end{split}
\end{equation}

We know that $p, q$ are roots to the quintic polynomial. 
Also, we know $pq = 1$. 

\[
    = 0 - \frac 6 5 (-1)^{n + 1} 
    + \frac 2 5 (-1)^{n + 2}
    + \frac 2 5 (-1)^{n} - 2(-1)^n
\]
\[=
    \frac{6 + 2 + 2}{5} (-1)^n - 2(-1)^n = 0
\]

Identity (1) can be slightly manipulated as 
\begin{equation}
    f_{n + 2}^2 - f_{n+1}^2 - (-1)^{n+1} 
    = 
    2f_{n + 1}^2 - f_n^2 - (-1)^n
\end{equation}

Now, we purport 
\[
    \boxed{
    \sum_{i = 0}^{n} f_i ^2= f_{n + 1}^2 - f_n^2 - (-1)^n
    }
\]

We prove this formula by induction on the integer $n$. 
For the base case $n=1$, 
\[
0 = f_1^2 - f_0^2 - (-1)^0
\textOr 
0 = 1 - 0 - 1 
\textOr 
0 = 0
\]
Which is a tautology. 

Assume the formula holds for $n$. We prove the formula 
for $n +1$. That is, assuming (4), we prove 
\[
    \sum_{i = 0}^{n + 1} f_i ^2= f_{n + 2}^2 - f_{n + 1}^2 - (-1)^{n + 1}
\]

We evaluate the left sum to show equality. 
\[
    \begin{split}
    \sum_{i = 0}^{n + 1} f_i^2 = 
 f_{n + 1}^2  + 
    \sum_{i = 0}^{n} f_i^2 
    = 
 f_{n + 1}^2  + 
 f_{n + 1}^2 - f_n^2 - (-1)^n
    \\=    
 2f_{n + 1}^2 - f_n^2 - (-1)^n
    \end{split}
\]

Apply equation (4). 
\[
    \sum_{i = 0}^{n + 1} f_i ^2= f_{n + 2}^2 - f_{n + 1}^2 - (-1)^{n + 1}
\]

\hfill \qed

\new{Q8} 
A 1xn chessboard is given. Each block can 
be colored either red or blue. No two consecutive 
squares can be colored red. Compute $h_n$, the number 
of distinct colorings of the chessboard. 

\new{Solution} 
We define two auxillary sequances $r_n, b_n$ 
which count the number of colorings that satisfy the 
coloring rule and has either a red or a blue coloring 
on the nth block. We notice $r_1 = b_1 = 1$. 
Also, by the nature of the coloring 
\[
    r_{n+1} = b_n \textAnd b_{n + 1} = r_n + b_n
\]
To color the last square red, the previous square 
must be colored blue. To complete the board with blue, 
the previous coloring does not matter. Also, we can write 
\[
    b_{n + 1} = b_n + b_{n - 1}
\]
Which is the linear recurrence for the fibbonacci sequence. 
Also, $b_1 = 1, b_2 = 2$ so we conclude $b_n = F_n$. It 
follows that $r_n = F_{n - 1}$. 
\[
    \boxed{
    h_n = b_n + r_n = F_n + F_{n - 1} = F_{n + 1}
    }
\]

\new{Q10} 
Resolve Fibbonacci's rabbit problem assuming that the 
cage has 2 pairs of rabbits in the beginning. 

\new{Solution}
A simple solution is to consider the new cage 
as two cages that each include one rabbit pair. It follows 
that the number of pairs in the nth month is $2F_n$. 

In mathematical language, the linear recurrence relation 
is linear. Let $G_n$ be the number of rabbits in the nth 
month in the new cage. We know $G_n = 2F_n$ satisfies the 
recurrence relation $G_{n + 2} = G_{n + 1} + G_n$ for 
any $n \geq 0$. Also, $G_0 = 2\m F_0 = 2$ satisfies 
the initial condition. Hence, $G_n = 2F_n$. 

So the answer is 
$\boxed{2F_{12} = 466}$

\new{Q12}
Define a sequence $h_n := n^3$. Verify the recurrence 
relation 
\[
    h_n =  h_{n - 1} + 3n^2 - 3n + 1
\]

\new{Proof}
Prove the relation directly by subsituting $h_n$ with 
its definition. We evaluate the RHS. 

\[
    h_{n - 1} + 3n^2 - 3n + 1
    = 
    (n - 1)^3  + 3n^2 - 3n + 1
\]

Apply the binomial theorem to expand the cubic. 

\[
    = (n^3 - 3n^2 + 3n - 1 )+ 3n^2 - 3n + 1 
    = n^3 = h_n
\]
\hfill \qed

\new{Additional Problem 1}
Let sequence $g_n$ count the number of coverings 
of a 2xn chessboard with dominos. Prove that $g_n$ 
satisfies the fibbonacci recurrence relation 
by a bijective proof. 

\new{Solution}
We know that the number of ways to disect 
a word of length n into bead of length 1 and 2 
satisfied the fibbonacci recurrence relation. That is, 
if $f_n$ counts the number of ways to break down 
a word of length n into beads smaller than 2, 
\[
    f_{n + 2} = f_{n + 1} + f_{n}
\]

Consider a domino tiling of a 2xn board. Either 
the domino must be arraged horizontal or vertical. 
If a domino is arranged horizontally, a parallel domino 
must be placed on top of it or under it
\fullFigure{DominoCovers.png}

The slanted orientation would have to repeat itself 
to produce a perfect cover. As the pattern reaches 
the boundary of the board, it is evident that 
a perfect cover is not attainable. 

Map the two horizontal dominos as a long bead and 
a single vertical domino as a short bead. Below is an example. 
\fullFigure{mapping.png}

From any covering, it is possible to obtain 
a unique arrangement of beads. Also, from any bead 
arrangement, it is possible to create a covering. 
Hence, this mapping is bijective. The number of 
bead arrangements satisfy the fibbonacci recurrence 
relation, and so does the number of domino coverings. 

\[
    g_{n + 2} = g_{n + 1}+ g_n 
\]
\hfill \qed

\new{Additional Problem 2} 
Find the generating functions for the given 
sequences. 

\new{Solution}
The general strategy is to use derivatives and series 
sums in a clever way to obtain a nice closed form for 
the generating function 
\[
    \sum_{n = 0}^\infty a_n x^n
\]

The algebra behind deriving the sums are mostly trivial. 
We demonstrate the process for one sample sequence 
\[
    a_n = n^2
\]

From the geometric sequence formula, 
\[
    1 + x + x^2 + \cdots + x^n \cdots = \frac 1 {1 - x}
\]
Or, in sigma notation, 
\[
    \sum_{x = 0}^\infty x^n = \frac 1 {1 - x}
\]
Take the derivative once to obtain 
\begin{equation}
    \sum_{x = 0}^\infty nx^{n - 1} = \frac 1 {(1 - x)^2}
\end{equation}
Multiply both sides by x. 
\begin{equation}
    \sum_{x = 0}^\infty nx^{n} = \frac x {(1 - x)^2}
\end{equation}
From (5), take the derivative again. 
\[
    \sum_{x = 0}^\infty n(n - 1)x^{n - 2} = \frac 2 {(1 - x)^3}
\]
Multiply both sides by $x^2$
\begin{equation}
    \sum_{x = 0}^\infty n(n - 1)x^{n} = \frac {2x^2} {(1 - x)^3}
\end{equation}
Add (6), (7) and write 
\[
   \sum_{n = 0}^\infty 
        (n(n - 1) + n)x^n
        = 
        \frac {2x^2} {(1-x)^3} + 
        \frac x {(1- x)^2}
\]
Which simplifies to
\[
   \sum_{n = 0}^\infty n^2 x^n = 
   \frac{x^2 + x} {(1- x)^3} 
\]

Here is a table of sequences and their 
corresponding generating functions. 
\begin{center}
\begin{tabular}{|c|c|} 
 \hline
 $a_n$ & GF\\
 \hline 
 2 & \parbox{3cm}{\[
 \frac 2 {1 - x}   
 \]}\\
 \hline

 $2^n$ & \parbox{3cm}{
    \[
        \frac 1 {1 - 2x}
    \]
 }
\\ \hline 
$n - 1$ & \parbox{3cm}{
    \[
        \frac{2x - 1}{1-x^2}
    \]
}
\\ \hline 
$(n\%3 = 0)$ & \parbox{3cm}{
    \[
        \frac 1 {1-x^3}
    \]
    }
    \\ \hline 
$\binom{n}{2}$ & \parbox{3cm}{
    \[
        \frac {x^2} {(1-x)^3}
    \]}
    \\ \hline 
    $n^2$ & \parbox{3cm} {
        \[
            \frac{x^2 + x}{(1 - x)^3}
        \]
    }
    \\ \hline

\end{tabular}
\end{center}



Mathematica confirms some of these results 
\fullFigure{GFs.png}

\new{Additonal Problem 3}
Find the generating function for the number
of ways to make change equal to $n $cents out of nickels, dimes, and silver
dollars.

\new{Solution}
Provided an infinite number of nickels, the 
generating function to make $n$ cents is as follows. 
\[
    1 + x^5 + x^{10} + \cdots = \frac 1 {1 - x^5} 
\]
and similarly, we write the generating functions for 
only using the dimes and dollars. 
\[
    \frac 1{1 - x^{10}} \textAnd 
    \frac 1{1 - x^{100}}
\]
To find the generating function of using all three 
kinds of coins, we multiply the generating functions. 
\[
    \boxed{
        G(x) = 
        \frac 1 {
            (1 - x^5)
            (1 - x^{10}) 
            (1- x^{100})
        }
    }
\]

\new{Additional Problem 4} 
Find a GF for the number of solutions to the equation 
\[
    e_1 + e_2 + e_3 + e_4 = n
\]
given the restrictions on the variable $e_i$. 

\new{Solution}
Adding a little rigor to our notion of generating 
functions would help. We prove the following claim. 
For the rest of this problem, let $e_i$ be some 
variable that takes integer values. 

\new{Notation}
We intorduce a notation to express the number 
of solutions to a particular equation. Given 
variables $e_i$ for some integers $i$, the number 
of solutions of the equation 
\[
    \sum e_i = n
\]
is denoted by 
\[
    \langle
        e_1, e_2, \dots, e_i
    \rangle_n
\]
Depending on the value of $n$, this value can be 
interpreted as a sequence. 
Also, we provide a notation for the generating function 
of this sequence. 
\[
    G\langle
        e_1, e_2, \dots, e_i
    \rangle(x)
\]

The restrictions on $e_i$ determine the generating function 
$G\langle e_i \rangle(x)$. For example, if $e_i$ is 
restricted to be a positive integer, 
\[
    G\langle e_i\rangle(x) = x + x^2 + \cdots = 
    \frac x {1 - x}
\]
This fact trivially follows from the definition. $e_i = n$ 
has a solution if and only if $e_i$ is allowed to be the 
value $n$. 

\new{Proposition} 
Adding two variables in the equations results 
in multiplication of the generating functions. 
\[
    G\langle
        e_1, e_2, \dots, e_n
    \rangle(x)
    = 
    \prod_{i = 1}^n 
    G\langle e_i \rangle(x)
\]

\new{Proof}
We prove by induction on $n$. The base case is trivial. 
By the inductive hypothesis, we assume
\[
    G\langle
        e_1, e_2, \dots, e_n
    \rangle(x)
    = 
    \prod_{i = 1}^n 
    G\langle e_i \rangle(x)
\]

Now, we wish to find the generating function 
\[
    G\langle
        e_1, e_2, \dots, e_{n+ 1}
    \rangle(x)
\]
In order to find the generating function, we 
must find the value of the sequence 
\[
    \langle
        e_1, e_2, \dots, e_{n+ 1}
    \rangle_n
\]
Which is the number of solutions to the equation 
\[
     e_1 + e_2+ \dots+ e_{n+ 1} = N
\]
We partition all the solution based on the 
possible values of $e_{n + 1}$
Fix the value of $e_{n + 1} = l$. The size of the 
corresponding part will be the number of solutions 
to 
\[
     e_1 + e_2+ \dots+ e_n = N - l
\]
Which is in fact, the value 
$
    \langle 
        e_1, e_2 , \dots, e_n
    \rangle _{N - l}
$. This value is given my the coefficient of 
$x^{N - l}$ of the polynomial 
$G\langle e_1 , \dots, e_n \rangle(x)$

Consider the poynomial 
\[
    G\langle e_1 , \dots, e_n \rangle(x)
    G\langle e_{n + 1} \rangle
    = 
    \prod_{i = 1}^{n + 1} 
    G\langle e_i \rangle(x)
\]where the equality follows by the inductive hypothesis. 
The coeffieient of $x^{N}$ of this polynomial 
will be the sum of the coefficients of $x^{N - l}$
in the polynomial $G\langle e_1 , \dots, e_n \rangle(x)$ 
for all values of $l$ which $G\langle e_{n+1}\rangle$
is nonzero. In symbols, the $x^N$ coefficient is 
\[
   \sum_{l \textrm{ valid}}
    \langle 
        e_1, e_2 , \dots, e_n
    \rangle _{N - l}
    = \langle 
        e_1, e_2 , \dots, e_{n + 1}
    \rangle _{N}
\]

We have directly shown that 
\[
    \prod_{i = 1}^{n + 1} 
    G\langle e_i \rangle(x)
\]
Is a generating function of $\langle 
        e_1, e_2 , \dots, e_{n + 1}
    \rangle _{N}$. 
    
    \hfill \qed

In light of this powerful machinery, we can find the 
GFs for variables that are independant. 

\new{Question}
Find $G\langle e_1, e_2, e_3, e_4\rangle (x)$ for 

a) $e_i \in \mathbb{Z}^+$ for all $i \leq 4$

b) $e_i \in \mathbb{Z}^+$ for all $i \leq 4$
and $e_i$ is a multiple of i

c) $e_i \in \mathbb{N}$ for all $i \leq 4$
and $e_2 = 2e_1$, $e_3 + e_4 = 4$

\new{Answer}

a) 
\[G\langle e_1, e_2, e_3, e_4\rangle (x) 
= 
\left(
\frac x {1- x}
\right)^4
\]

b) 
\[
    \begin{split}G\langle e_1, e_2, e_3, e_4\rangle (x) 
= 
\frac x {1- x}
\frac {x^2} {1- x^2}
\frac {x^3} {1- x^3}
\frac {x^4} {1- x^4}
\\= 
\frac {x^{10} }
{
    (1 - x)(1 - x^2)(1- x^3) (1 - x^4)
}
\end{split}
\]

c)
This is trickier since the variables 
are interrelated. Write the equation
\[
        e_1+ e_2+ e_3+ e_4  = N
\]
Applying the substitution, this condition simplifies to 
\begin{equation}
    3e_1 = N - 4
\end{equation}
$e_2$ is entirely determined by $e_1$. Also, there 
are five solutions to the eqation $e_3 + e_4 = 4$
regardless of the value $N$. 

We directly compute the generating function by counting 
the values of $N$m which yields a solution for (8). 
$e_1 \in \mathbb{N}$, so allowed values of $N$ are 
$N = 4, 7, 10, 13, \dots $. Finally, we write the 
generating function. 

\[
    G\langle e_1, e_2, e_3, e_4\rangle (x) 
    = 
    5(x^4 + x^7 + x^{10} + \cdots) = \frac{5x^4}{1-x^3}
\]

\new{Additional Problem 5} A certain flower shop has 2 black roses, 2 white
roses, and an unlimited number of red roses. Let $b_n$ be the number of
ways to form a bouquet of n roses, where $n \in N$.

a) Find a formula for $b_n$

b) Find the GF of $b_n$

\new{Solution}
In fact, it is easy to write out a GF for $b_n$ using 
our Proposition proved in the previous problem. 

\[
    \boxed{
    G(x) = \frac {(1 + x + x^2)^2} {1 - x}    
    }
    = 
    \frac 9 {1-x} - (8 + 6x + 3x^2 + x^3)
\]
\[
    = 1 + 3x + 6x^2 + 8x^3 + 9x^4 + \cdots
\]

Note that the coefficient for $x^i$ for all $i \geq 4$ 
is $9$. Thus, we write the formula for $b_n$. 

\[
    \boxed{
    (b_0, b_1, b_2, b_3) = (1, 3, 6, 8)
    \textAnd 
    b_i = 9 \textrm{  }^\forall i \geq 4
    }
\]


\end{document}