\documentclass{article}
\usepackage{amsfonts}
\usepackage{amsthm}
\usepackage{amssymb}
\usepackage{amsmath}
\usepackage{graphicx}
\usepackage{subcaption}
\usepackage{xcolor}

\newcommand{\new}[1]{
    \vspace{2mm}
    \noindent
    \textbf{
    \underline{#1}}
}

\def\calO{{\mathcal{O}}}
\def\th{{\theta}}
\def\_{{\hspace{1mm}}}
\def\<{{\langle}}
\def\>{{\rangle}}


\newcounter{problemcnt}
\setcounter{problemcnt}{0}

\newcommand{\Problem}{{
    \vspace{5mm}
    \stepcounter{problemcnt}
    \noindent
    \arabic{problemcnt}. 
}
}

\newcommand{\nProblem}[1]{
    \vspace{5mm}
    \noindent
    \setcounter{problemcnt}{#1}
    \arabic{problemcnt}. 
}


\newcommand{\Proof}{{
    \vspace{2mm}
    \noindent
    \textbf{
    \underline{Proof}}
}
}

\newcommand{\textOr}{
    {
        \hspace{5mm}
        \textrm{or}
        \hspace{5mm}
    }
}

\newcommand{\textAnd}{
    {
        \hspace{5mm}
        \textrm{and}
        \hspace{5mm}
    }
}

\newcommand{\m}{
    \cdot
}

\newcommand{\Ixp}[1]{
    {
        e^{i{#1}}
    }
}



\newcommand{\halfFigure}[1]{
\begin{center}
\includegraphics[width = .5\linewidth]{{#1}}
\end{center}
}

\newcommand{\fullFigure}[1]{
\begin{center}
\includegraphics[width = .9\linewidth]{{#1}}
\end{center}
}

\def\twobytwoMat(#1, #2, #3, #4){
    {
        \begin{bmatrix}
            {#1} & {#2}\\
            {#3} & {#4}
        \end{bmatrix}
    }
}

\def\twobyoneMat(#1, #2){
    {
        \begin{bmatrix}
            {#1}\\
            {#2}
        \end{bmatrix}
    }
}

\def\twobytwoDet(#1, #2, #3, #4){
    {
        \begin{vmatrix}
            {#1} & {#2}\\
            {#3} & {#4}
        \end{vmatrix}
    }
}



\begin{document}
\begin{center}
\LARGE
Combinatorics HW6

\Large
Daniel Son
\end{center}

\normalsize 

\new{Sec5.7q48} Use Theorem 5.6.1 to show that, if m and n are positive integers, then a partially 
ordered set of mn + 1 elements has a chain of size m + 1 or an antichain of size 
n+1. 

\new{Solution} 
We proceed with a constructive proof. 
Consider poset $S$ and an ordering $(S, \leq)$. 
Among all the antichain covers of $S$, let $P$ be the minimal 
antichain cover. That is, any antichain cover of $S$ involves 
at least $|P|$ antichains. By Mirski's theorem, we know 
that the largest chain in the poset $|S|$ must have a length 
of $|P|$. 

If $|P| > m$, $|P| \geq m + 1$ and the longest chain has 
more than $m + 1$ elements. We consider the case where 
$|P| \leq m$. In the cover $P$, there must exist a antichain 
larger than
\[
    \bigg\lceil 
    \frac{mn + 1}{|P|}
    \bigg\rceil
    \geq 
    \bigg\lceil 
    \frac{mn + 1}{m}
    \bigg\rceil
    = n + 1
\]
. This is by the pigeonhole principle. 
So there exists some antichain that has a size greater than 
$n + 1$. 

We have shown that the longest chain must be longer than $m + 1$
or the largest antichain must be larger than $n + 1$. From the largest 
chain/antichain, exclude elements to obtain the desired size $n+1$
or $m +1$.  \hfill \qed


\new{Sec5.7q49} 
Use the result of the previous exercise to show that a sequence of mn + 1 real 
numbers either contains an increasing subsequence of m + 1 numbers or a decreasing subsequence of n + 1 number. 

\new{Solution}
Let the sequence of real numbers be deonted as 
\[
    a_1, a_2, \dots, a_{nm + 1}
\]
We define a set of tuples as follows. 
\[
    S := \{(a_i, i)| 0 \leq i \leq nm + 1\}
\]
Also, define a partial ordering on $S$. 
For $s_i = (a_i, i), s_j = (a_j, j) \in S$, $s_i \leq s_j$ if and only if 
\[
    a_i \leq a_j \textAnd i \leq j
\]

Invoke the result from problem 48. There must either be 
a chain in $S$ that has length of $m + 1$ or antichain of 
length $n + 1$. We consider each case. Write out the chain 
as 
\[
    (a_{p1}, p1) \leq (a_{p2}, p2) \leq \cdots \leq (a_{p(m+1)}, p(m+1))
\]

By the definition of the partial order, we deduce 
\[
    a_{p1} \leq a_{p2} \leq \cdots \leq a_{p(m+1)}
\]
and we have obtained an increasing subsequence of length $m + 1$. 

If such a chain does not exist, there must exist an antichain 
of length $n + 1$. Write out the elements in increasing index

\[
    (a_{q1}, q1) , (a_{q2}, q2) , \cdots , (a_{q(n+1)}, q(n+1))
\]
\[
    \textAnd 
    q1 \leq q2 \leq \cdots \leq q(n+1)
\]
If $a_{qi} \leq a_{q(i + 1)}$, then $(a_{qi}, qi) \leq (a_{q(i + 1)}, q(i + 1))$. 
Such result will contradict that all antichain elements are incomparable. 
For all $i < n + 1$,  $a_{qi} > a_{q(i + 1)}$. We have constructed
a decreasing subsequence. 

\[
    a_{q1} > a_{q2} > \cdots > a_{q(n + 1)}
\]
\hfill \qed


\new{Sec5.7q50} 
Consider the poset $([12], |)$. 

a) Determine the longest chain 
and the partion of $[12]$ into the smallest number of antichains. 

Motivated by the proof of Mirski's theorem, we proceed by eliminating 
the minimal elements. Denote $min(X)$ to be the minimal elements 
of the set $X$. Also, define $S_0 = min([12]), P_0 = [12] \setminus S_0$. 
Recursively define:
\[
    S_i = min(S_{i - 1}) 
    \textAnd 
    P_i = P_{i - 1} \setminus S_i 
\]

\begin{align*}
    S_0 = \{1\} & & P_0 = \{2, 3, 4, 5, 6, 7, 8, 9, 10, 11, 12\}\\
    S_1 = \{2, 3, 5, 7, 11\} & & P_1 = \{4, 6, 8, 9, 10, 12\} \\ 
    S_2 = \{4, 6, 9, 10\} & & P_2 = \{8, 12\}\\
    S_3 = \{8, 12\}
\end{align*}

So the longest chain is $\{1, 2, 4, 8\}$ and the minimal antichain 
cover is $\{S_i\}_{i \leq 3}$ \hfil\qed

b) Determine the longest antichain 
and the partion of $[12]$ into the smallest number of chains. 

Upon inspection, it seems like that the longest antichain is 
$\{4,6,7,9,10,11\}$. We can also find a minimal chain cover 
$\{S_i\}_{i \leq 6}$. $S_i$ is defined as follows. 
\[
\begin{split}
    S_1 := \{1, 2, 4, 8\}\\
    S_2 := \{3, 6, 12\}, S_3 := \{5, 10\}\\
    S_4 := \{7\}, S_5 := \{9\}, S_6 := \{11\}
\end{split}
\]

By the generalized version of the pigeonhole principle, a 
subset of $X$ that has a size greater than $6$ must include 
two elements in the same chain. Thus, indeed the assumed 
maximum antichain is verifed to be maximum. By Dilworth's 
theorem, the cover we have obtained is the minimum chain cover. 


\new{Sec6.7Q11}
Determine the number of permutations of $[8]$ in which no even integer 
is in its natural position. 

\new{Solution}
Define a universial set $S$ to be all the permutations 
of $[8]$. A permutation satisfies property $P_i$ if the 
number $i$ does not show up in position $i$. Let $A_i$ 
be the set of permutations that satisfy $P_i$. 
We wish to count 
\[
    |\cap_{i \in \{2, 4, 6, 8\}} \bar{A}_i|
\]
. 
Apply the inclusion exclusion principle. 
\[
    |\cap_{i \in \{2, 4, 6, 8\}} \bar{A}_i|
    = 
    |S| - \sum_i |A_i| + \sum_{i, j} |A_i \cap A_j|
    - \sum_{i, j, k} |A_i \cap A_j \cap A_k|
    + |A_1\cap A_2 \cap A_3 \cap A_4|
\]
\[
    =
    8! - 4\m 7! + 6\m 6! - 4\m 5! + 4!
    = \boxed{24204} 
\]

\new{Sec6.7Q12}
Determine the number of permutations of $[8]$ in which 
exactly four numbers are in their natural position. 

\new{Solution}
Construct all such permutations sequentially and apply the 
principle of multiplication. Choose four elements and fix 
them in their original position. Then, derange 
the other four. We count the answer to be 
\[
    \binom{8}{4} \m D_4 = \boxed{1660} 
\]

\new{Sec6.7Q14}
Determine the number of permutations of $[n]$ in which exactly 
k numbers are in their natural position. 

\new{Solution}
Apply the same logic from problem 12. 
\[
    \boxed{
    \binom{n}{k} \m D_{n-k}
    }
\]




\new{Sec5.7q24}
\fullFigure{Fig24.png}

\new{a}
Let condition $P_i$ be the condition that the rook in column 
$i$ to be in the forbidden position. There is only one forbidden 
position for each column. Define $A_i$ be the set of arrangements 
that satisfy condition $P_i$ and where the six rooks don't attack 
each other. With ease, for all $i \leq 6$
\[
    |A_i| = 5!
\] 

To compute the intersect of the two sets, we start with some 
specific cases. 
\[
    |A_1\cap A_2| = 0 \textAnd 
    |A_1 \cap A_3| = 4!
\]
For the three pairs, $A_1\cap A_2$, $A_3\cap A_4$, $A_5\cap A_6$, 
the intersects are empty. For all the other twelve pairs, the 
size of the set is $4!$. 

For the intersect of three parts, the only nonzero cases 
are when each rook is chosen for every column. For each set 
$\{P_1, P_2\}, \{P_3, P_4\}, \{P_5, P_6\}$, make a choice between the two. 
There are a total 8 choices, and the size of each intersect is 
$3!$. In specific, 
\[
    |A_1 \cap A_3 \cap A_5| = 3!
\]

The intersect of more than four parts will always be zero. 
If more than four conditions are met, two of the rooks will be 
next to each other. 

Apply the inclusion-exclusion principle to compute the number 
of allowed permutations. 

\[
    |\cap_{n \leq 6} \bar{A}_n|
    = 
    6! - 
    \sum_{n\leq 6}|A_n|
    + \sum_{n, m \leq 6} |A_n \cap A_m|
    - \sum_{n, m, k \leq 6} |A_n \cap A_m \cap A_k| 
\]
    \[=
    6! - 6\m5! + 12\m4! - 8 \m 3! 
    = 288 - 48 = \boxed{240}
\]

\new{b} 
Define $P_i, A_i$ as we did in part a. 
Also, group rooks $(1, 2), (3, 4), (4, 6)$ into three groups. 
To apply the inclusion-exclusion principle, the 
greatest challenge will be to count the number 
of multiplicites involved in each intersect. 

We refine our notation for writing down intersects. 
Let $\{s_i\}$ be a subset of $[6]$. All intersects of $A_i$'s 
can be written as $
    \cap_{i} A_{s_i}
$. From now on, we refer to each intersect by the 
corresponding subset which includes all the subscripts 
ranging from 1 to 6. 
For example, $A_1\cap A_2$ is referred as $\{1, 2\}$. 

Define a grouping function $\phi:[6]\mapsto \{1, 2, 3\}$
defined as $\phi(n) = \lceil n/2\rceil$. The grouping function 
will connect rook $i$ to its corresponding group number. 
We say that an intersect $\{s_i\}$ has $m$ groups involved if 
the image of the subset under the function $\phi$ has 
a size of $m$.  For example, $A_1\cap A_2$, or $\{1, 2\}$
has an image $\{\phi(1), \phi(2)\} = \{1\}$ under the grouping function 
$\phi$. Thus, the intersect has only one group involved. 

Let intersect $\{s_i\}$ to be an intersect with size $n$. 
Also, suppose the intersect involves $m$ groups. We claim 
the size of the intersect equals 
\[
    2^m (6 - n)!
\]
To justify this counting, we construct all valid arrangements 
of nonattacking rooks for intersect $\{s_i\}$. We first arrange 
all the rooks that are to be in the forbidden position, that is 
all the rook $s_i$'s. Observe that each group, regardless if 
there are one or two rooks in the group, has two degrees of freedom. 
For the free rooks not included in the intersect, arrange 
them as free permutations. The principle of multiplication 
yields the result above. 

We consider all the possible intersects. 
However, we are only concerned with the group number 
of each intersect. 
We wish to count the number of intersects of size $n$
involving $m$ groups. Say there are $x$ such intersects. 
We write $x|mg$. Here is an exhaustive list of 
all the intersects. 
\newcommand{\size}[1]{
    {
        \textrm{size }{#1}:
        \hspace{3mm}
    }
}

\begin{align*}
    &\size1 6|1g\\
&\size 2 3|1g, 12|2g\\
&\size3 12|2g, 8|3g\\
&\size4 3|2g, 12|3g\\
&\size5 6|3g\\
&\size6 1|3g
\end{align*}

Applying our observation on the sizes of the intersects, 
we list out the number of permutations corresponding 
to the size of each intersect. 

\begin{align*}
    &\size1 6\m2\m5! = 1440\\
&\size2 3\m2\m4! + 12\m2^2\m4! = 1296  \\
&\size3 12\m2^2\m3!+ 8\m2^3\m3! = 672\\
&\size4 3\m2^2\m2!+ 12\m2^3\m2! = 216\\
&\size5 6\m2^3\m1! = 48\\
&\size6 1\m2^3\m0! = 8
\end{align*}

By the inclusion-exclusion principle, we count the answer. 

\[
    |S| + \sum_{i = 1}^{6} (-1)^i|[
        \textrm{all intersects of size i}
    ]|
\]
\[
    = 720 - 1440 + 1296-672+216 -48 + 8
    = \boxed{80}
\]

\end{document}