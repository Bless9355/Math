\documentclass{article}
\usepackage{amsfonts}
\usepackage{amsthm}
\usepackage{amssymb}
\usepackage{amsmath}
\usepackage{graphicx}
\usepackage{subcaption}
\usepackage[shortlabels]{enumitem}
\usepackage{xcolor}

\newcommand{\new}[1]{
    \vspace{2mm}
    \noindent
    \textbf{
    \underline{#1}}
}

\def\calO{{\mathcal{O}}}
\def\th{{\theta}}
\def\_{{\hspace{1mm}}}
\def\<{{\langle}}
\def\>{{\rangle}}


\newcounter{problemcnt}
\setcounter{problemcnt}{0}

\newcommand{\Problem}{{
    \vspace{5mm}
    \stepcounter{problemcnt}
    \noindent
    \arabic{problemcnt}. 
}
}

\newcommand{\nProblem}[1]{
    \vspace{5mm}
    \noindent
    \setcounter{problemcnt}{#1}
    \arabic{problemcnt}. 
}


\newcommand{\Proof}{{
    \vspace{2mm}
    \noindent
    \textbf{
    \underline{Proof}}
}
}

\newcommand{\textOr}{
    {
        \hspace{5mm}
        \textrm{or}
        \hspace{5mm}
    }
}

\newcommand{\textAnd}{
    {
        \hspace{5mm}
        \textrm{and}
        \hspace{5mm}
    }
}

\newcommand{\m}{
    \cdot
}

\begin{document}
\begin{center}
\LARGE
Combinatorics HW2

\Large
Daniel Son
\end{center}


%do 1, 2, 4, 7, 8, 9

\new{Question 1}
For each of the subsets of property a, b, count the number of four
 digit numbers whose digits are either 1, 2, 3, 4, 5. 

 \begin{enumerate}[a)]
    \item The digits are distinct 
    \item The number is even
\end {enumerate}

\begin{proof}
    Start off with counting the digits that need not satisfy both 
    a or b. By principle of multiplication, we count $5^4 = \boxed{625}$. 
    
    Now we count the digits that satisfy only condition a. Again, 
    by principle of multiplication, $5\cdot4\cdot3\cdot2 = \boxed{120}$. 

    For the digits that only satisfy condition b, we start with choosing 
    the last digit. If the last digit is even, the whole number is even. 
    There are two even numbers within the set $\{1, 2, 3, 4, 5\}$. So, 
    by principle of multiplication, we count $2\cdot5\cdot5\cdot5 = \boxed{250}$.
    
    Finally, for the digits that satisfy both of the conditions, 
    we again count from the last digit. By principle of multiplication, 
    the answer is $2\cdot4\cdot3\cdot2 = \boxed{48}$. 
    
\end{proof}

\new{Question 2} 
How many orderings are there for a deck of cards if all the cards 
in the same suite are together?

\new{Solution}
 First, ignore the order of the cards 
but only consider the ordering of the suites. There are four possible suites, 
so there are $4! = 24$ ways of ordering. For each suite, there are 
13 cards. The 13 cards can be arranged each in $13!$ ways. Thus, by 
principle of multiplication, there are a total of $\boxed{4!(13!)^4}$ ways of 
ordering. \hfill \qed


\new{Question 4} 
How many distinct positive divisors does the each of the following numbers have:
$3^4\cdot5^2\cdot7^6\cdot11, 620, 10^{10}$

\new{Proposition} 
Let n be a positive integer. By the Fundamental Theorem of Arithmetic, 
it is possible to write $n$ as:

\[
    n = \prod_{i = 0}^{k} {p_i}^{a_i}
\]
where $p_i$'s are prime and $a_i$'s are positive integers. The number 
of positive divisiors of $n$ is

\begin{equation} \label{eq:numdiv}
    \prod_{i = 0}^{k} (a_i + 1)
\end{equation}

\begin{proof}
    The divisors of $n$ must involve only the prime numbers 
    $\{p_1, ..., p_n\}$. The power of the prime $p_i$ can range 
    from $a_i + 1$. Then, we proceed with the principle of multiplication 
    to obtain the answer.
\end{proof}

\new{Solution} 
In light of the proposition, the question boils down to finding 
the prime factorization of the three numbers, which is given as follows. 

\[
    3^4\cdot5^2\cdot7^6\cdot11,\hspace{3mm} 2^2\cdot5\cdot31, \hspace{3mm} 2^{10}\cdot5^{10}
\]
The number of their divisors are $5\m3\m7\m2 = \boxed{210}$,  $3\m2\m2 = \boxed{12}$,  $11\m11 = \boxed{121}$. \qed

\new{Question 7} In how many ways can four men and eight women be 
seated around a round table if there are two women between consecutive 
men around the table?

\new{Solution} 
Take any three consecutive seats of the table. We observe that there 
must be a men sitting in one of the three tables. Otherwise, there 
will be a pair of two consecutive men where there is only one women 
in between the men. Moreover, there can be exactly one men sitting 
on one of these three seats. 

Designate one of the three seats to be a seat for men. Such a choice 
will fix the seats where the men must sit. Afterwards, we count 
the number of ways to permute the four men and eight women, which 
equals $4!\m8!$. 

By principle of multiplication, we deduce that the total possible 
seatings are $3\m4!\m8!$.

\color{red}
We have overcounted each arrangements 12 times, for there are 12 possible rotations for each sitting. 
Thus, the answer is $\boxed{3\m4!\m8!/12 = 3!\m8!}$
\color{black}

\hfill \qed

\new{Question 8} How many ways can six women and six men could 
be seated if men and women are to sit in alternate seats?

\new{Solution} 
Apply the reasoning that we used for the previous problem. 
From two consecutive seats, choose one seat for the men, which 
decides all the seats where the men should sit. Multiply by the 
permutations for men and women. We conclude that the number of 
total possible sittings are $2\m (6!)^2$. 

\color{red}
We have overcounted each arrangements 12 times, for there are 12 possible rotations for each sitting. 
Thus, the answer is $\boxed{2\m (6!)^2/12 = 5!\m6!}$
\color{black}

\hfill \qed

\new{Question 9} 
In how many ways can 15 people be seated at a round table 
if B refuses to sit next to A? What if B only refuses to sit 
on A's right?

\new{Solution} 
We partion all possible sittings $P$ into three parts $P_1, P_2, P_3$. 
Let $P_1$ refer to all the sittings where A and B do 
not sit next to each other. Let $P_2$ be the sittings 
where A sits to the left of B and $P_3$ be the sittings 
where B sits to the left of A. The question boils down 
to computing the size of $P_1$ and $P_1 \cup P_3$. 

First observe that there is a one-to-one correspondence 
between $P_3$ and $P_1$. By switching the seats of A and 
B, it is possble to obtain a sitting in $P_3$ from a sitting 
in $P_1$ and vice versa. It is trivial to see that $|P| = 14!$. Divide the linear 
permutation of the 15 people by 15 to account for rotations. 

Compute the size of the part $P_3$. Fix the position of A
to ignore rotations. 
B must sit on the left of A, hence the position of B is fixed. 
The size of $P_3$ is the number of ways to permute the remaining 
13 people, and hence $|P_3| = 13!$. Also, by our previous observation, 
$|P_2| = |P_3| = 13!$. 

By the addtion principle, 
\[
    |P| = |P_1| + |P_2| + |P_3| \textOr 
    14! = |P_1| + 2\m 13!
\]
and thus 
\color{red}
\[
    |P_1| = 12\m13!
\]
\color{black}

so we have computed the sizes of all partitions. 

Finally, we write the sizes of the two partitions of our interest.
\color{red} 
\[
    \boxed{
    |P_1| = 12\m 13!
    }
    \textAnd
    \boxed{
    |P_1 \cup P_3| = |P_1| + |P_3| = 13\m 13!
    }
\]
\color{black}

\hfill \qed


\new{Additional Problem 1}
I computed ten permutations and got nine correct. I would give 
myself a 90/100. I confused the permutation with a combination for 
some reason and divided by an extra factor of two. Other than that, 
the math was correct. 

Also the following code will compute $P(n, k)$ in python. 
\begin{verbatim}
def P(n, k):
     ans = 1
     mtp = n
     for i in range(k):
             ans = ans * mtp;
             mtp = mtp - 1;
     return ans
\end{verbatim}


\new{Additional Problem 2}
How many positive divisors of 2310 have themselves a maximum of 
four positive divisors?

\new{Solution}
A key observation is that 2310 factors into a product of five distint 
prime numbers. 
\[
    2310 = 2\m3\m5\m7\m11 
\]
Any divisor of 2310 will have some subset of these five numbers 
without multiplicity. Also, notice that if the number has less than 
two prime numbers in its factorization, it will have less than 
four divisors. This is by the proposition presented in Question 4
\eqref{eq:numdiv}. So we count the number of ways to choose zero, one, 
or two prime numbers among the five prime numbers. 

\[
    \binom{5}{0} + \binom{5}{1}
    + \binom{5}{2} = 1 + 5 + 10 = \boxed{16}
\]
So there are 16 such divisors. \hfill \qed

\new{Additional Problem 3}
Mother Bear’s pizzeria offers 3 sizes of pizza,
5 protein toppings, and 5 vegetable toppings. In how many ways can I
order a pizza with up to 3 protein toppings and any number of vegetable
toppings?

\new{Solution}
Apply the principle of multiplication. Choose the size, 
then choose the protein toppings, then the vegetable toppings. 
There are three ways to choose the size of the pizza. As for 
the protein toppings we want to compute the ways to choose three 
or less toppings out of five. This can be computed by adding some 
binomial coefficients. 
\[
    \binom{5}{0} + \binom{5}{1} + \binom{5}{2} + \binom{5}{3} = 
    1 + 5 + 10 + 10 = 26
\]
As for the vegetable toppings, encode the choice of toppings to 
a binary string. For example, if we choose to add the first and 
the third topping, the corresponding string would be 10100. It is 
not hard to see a one-to-one correspondence between the binary 
string and the choice of toppings. There are $2^5 = 32$ binary 
strings of length five. 

So by the principle of multiplication, we write our solution. 
\[
    3\m26\m32 = \boxed{2496}
\]
And there are a total 2496 ways to choose the pizza. 

\new{Additional Problem 4}
I got everything correct! 100/100. 
Also the following code will compute $\binom{n}{k}$ in python. 
\begin{verbatim}
def C(n, k):
    return P(n, k) / P(k, k)
\end{verbatim}

\new{Question 10}
A committee of five people is to be chosen from a club that boasts a membership 
of 10 men and 12 women. How many ways can the committee be formed if it is 
to contain at least two women? How many ways if, in addition, one particular 
man and one particular woman who are members of the club refuse to serve 
together on the committee?

\new{Solution}
\color{red}
From all possible choice of a 5 person comittee, we will 
subtract the partion with zero or one women. This will 
yield all ways to choose a five person comittee including at 
least two women. 

\[
    \binom{22}{5} - \binom{10}{5} - \binom{10}{4}\binom{12}{1}
    = \boxed{23562}
\]


To solve the second problem, we partition the choice of all five 
person committees into the choice that includes both the women and men 
of the problematic pair, and the ones that does not include both 
of the two. For the second part, the problematic women or the men 
can be included, but not both. In fact, it is easy to compute the size 
of the first part. We induct the two people into the comittee, then 
choose additional three people. 

The problem converts to counting the number of 3-person 
comittees chosen from 11 women and 9 men, where the comittee 
must include at least one women. Again, we count by the principle 
of subtraction. 
\[
    \binom{20}{3} - \binom{9}{3} = 1056
\]

To obtain the size of the second part, subtract the first part 
from the whole set. 

\[
    23562 - 1056 = \boxed{22506}
\]

So there are 23562 ways to choose the committee without the restriction 
of the problematic pair, and 22506 when the pair of man and women refuses 
to sit in the committee together. 
\color{black}
\hfill \qed

\new{Question 14}
A classroom has two rows of eight seats each. There are 14 students, 5 of whom 
always sit in the front row and 4 of whom always sit in the back row. In how 
many ways can the students be seated? 

\new{Solution} 
Seat the students sequentially in the following order. First, seat 
the five students who must sit in the front row. Then, seat the 
four students who must sit in the back row. Finally, seat the rest 
of the five students in any of the remaining seven seats. By the 
principle of multiplication, we count the number of possible seatings. 
\color{red}
Note that we must count using permutations for each table!
\color{black}

\color{red}
\[
    P(8, 5) \m P(8, 4) \m P(7, 5) = \boxed{8449792000}
\]
\color{black}
So there are 8449792000 possible seatings. 
\hfill \qed

\new{Question 15}
At a party there are 15 men and 20 women. 
(a) How many ways are there to form 15 couples consisting of one man and 
one woman? 
(b) How many ways are there to form 10 couples consisting of one man and 
one woman? 

\new{Solution} For part (a), we notice that all men must necessarily 
have a partner. We iterate starting from the first man and let 
each man to decide his partner. There are $\boxed{P(20, 15) = 20274183401472000}$ matchings possible. 

For part (b), start with obtaining the list of 10 men who have a partner. 
There are $\binom{15}{10}$ ways to make such a choice. Using the same reasoning 
for part (a), there must be $P(20, 10)$ ways to assign a partner to the 
selected 10 men. By the principle of multiplication, the total number 
of couplings are as follows.

\[
    \boxed{\binom{15}{10} \m P(20, 10) = 2013339046118400}
\]
\hfill \qed

\new{Question 21}
How many permutations are there of the letters of the word ADDRESSES? How 
many 8-permutations are there of these nine letters?

\new{Solution}
Decompose the string ADDRESSES into a multiset. 
\[
    \{A\m 1, D\m 2, E\m 2, R\m 1, S \m 3\}
\]
The permutation of this multiset is given by the following multinomial. 
\[
    \boxed{\binom{9}{3, 2, 2, 1, 1} = 
    \binom{9}{3} \binom{6}{2} \binom{4}{2} \binom{2}{1}=
    15120}
\]

Consider a 8-permutation of this multiset. Notice that the one 
element that is not used in the permutation is uniquely determined 
by teh permutation. Indeed, the number of 8-permutation equals 
to the number of permutations of the set. There are $\boxed{15120}$ 8-permutations. 

\hfill \qed

\end{document}
