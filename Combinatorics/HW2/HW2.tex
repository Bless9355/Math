\documentclass{article}
\usepackage{amsfonts}
\usepackage{amsthm}
\usepackage{amssymb}
\usepackage{amsmath}
\usepackage{graphicx}
\usepackage{subcaption}
\usepackage[shortlabels]{enumitem}

\newcommand{\new}[1]{
    \vspace{2mm}
    \noindent
    \textbf{
    \underline{#1}}
}

\def\calO{{\mathcal{O}}}
\def\th{{\theta}}
\def\_{{\hspace{1mm}}}
\def\<{{\langle}}
\def\>{{\rangle}}


\newcounter{problemcnt}
\setcounter{problemcnt}{0}

\newcommand{\Problem}{{
    \vspace{5mm}
    \stepcounter{problemcnt}
    \noindent
    \arabic{problemcnt}. 
}
}

\newcommand{\nProblem}[1]{
    \vspace{5mm}
    \noindent
    \setcounter{problemcnt}{#1}
    \arabic{problemcnt}. 
}


\newcommand{\Proof}{{
    \vspace{2mm}
    \noindent
    \textbf{
    \underline{Proof}}
}
}

\newcommand{\textOr}{
    {
        \hspace{5mm}
        \textrm{or}
        \hspace{5mm}
    }
}

\newcommand{\textAnd}{
    {
        \hspace{5mm}
        \textrm{and}
        \hspace{5mm}
    }
}

\newcommand{\m}{
    \cdot
}

\begin{document}
\begin{center}
\LARGE
Combinatorics HW2

\Large
Daniel Son
\end{center}


%do 1, 2, 4, 7, 8, 9

\new{Question 1}
For each of the subsets of property a, b, count the number of four
 digit numbers whose digits are either 1, 2, 3, 4, 5. 

 \begin{enumerate}[a)]
    \item The digits are distinct 
    \item The number is even
\end {enumerate}

\begin{proof}
    Start off with counting the digits that need not satisfy both 
    a or b. By principle of multiplication, we count $5^4 = 625$. 
    
    Now we count the digits that satisfy only condition a. Again, 
    by principle of multiplication, $5\cdot4\cdot3\cdot2 = 120$. 

    For the digits that only satisfy condition b, we start with choosing 
    the last digit. If the last digit is even, the whole number is even. 
    There are two even numbers within the set $\{1, 2, 3, 4, 5\}$. So, 
    by principle of multiplication, we count $2\cdot5\cdot5\cdot5 = 250$.
    
    Finally, for the digits that satisfy both of the conditions, 
    we again count from the last digit. By principle of multiplication, 
    the answer is $2\cdot4\cdot3\cdot2 = 48$. 
    
\end{proof}

\new{Question 2} 
How many orderings are there for a deck of cards if all the cards 
in the same suite are together?

\new{Solution}
 First, ignore the order of the cards 
but only consider the ordering of the suites. There are four possible suites, 
so there are $4! = 24$ ways of ordering. For each suite, there are 
13 cards. The 13 cards can be arranged each in $13!$ ways. Thus, by 
principle of multiplication, there are a total of $\boxed{4!(13!)^4}$ ways of 
ordering. \hfill \qed


\new{Question 4} 
How many distinct positive divisors does the each of the following numbers have:
$3^4\cdot5^2\cdot7^6\cdot11, 620, 10^{10}$

\new{Proposition} 
Let n be a positive integer. By the Fundamental Theorem of Arithmetic, 
it is possible to write $n$ as:

\[
    n = \prod_{i = 0}^{k} {p_i}^{a_i}
\]
where $p_i$'s are prime and $a_i$'s are positive integers. The number 
of positive divisiors of $n$ is

\[
    \prod_{i = 0}^{k} (a_i + 1)
\]

\begin{proof}
    The divisors of $n$ must involve only the prime numbers 
    $\{p_1, ..., p_n\}$. The power of the prime $p_i$ can range 
    from $a_i + 1$. Then, we proceed with the principle of multiplication 
    to obtain the answer.
\end{proof}

\new{Solution} 
In light of the proposition, the question boils down to finding 
the prime factorization of the three numbers, which is given as follows. 

\[
    3^4\cdot5^2\cdot7^6\cdot11,\hspace{3mm} 2^2\cdot5\cdot31, \hspace{3mm} 2^{10}\cdot5^{10}
\]
The number of their divisors are $5\m3\m7\m2 = 210$,  $3\m2\m2 = 12$,  $11\m11 = 121$. \qed

\new{Question 7} In how many ways can four men and eight women be 
seated around a round table if there are two women between consecutive 
men around the table?

\new{Solution} 
Take any three consecutive seats of the table. We observe that there 
must be a men sitting in one of the three tables. Otherwise, there 
will be a pair of two consecutive men where there is only one women 
in between the men. Moreover, there can be exactly one men sitting 
on one of these three seats. 

Designate one of the three seats to be a seat for men. Such a choice 
will fix the seats where the men must sit. Afterwards, we count 
the number of ways to permute the four men and eight women, which 
equals $4!\m8!$. 

By principle of multiplication, we deduce that the total possible 
seatings are $\boxed{3\m4!\m8!}$.

\hfill \qed

\new{Question 8} How many ways can six women and six men could 
be seated if men and women are to sit in alternate seats?

\new{Solution} 
Apply the reasoning that we used for the previous problem. 
From two consecutive seats, choose one seat for the men, which 
decides all the seats where the men should sit. Multiply by the 
permutations for men and women. We conclude that the number of 
total possible sittings are $\boxed{2\m (6!)^2}$. \hfill \qed

\normalsize
\end{document}
