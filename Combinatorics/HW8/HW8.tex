\documentclass{article}
\usepackage{amsfonts}
\usepackage{amsthm}
\usepackage{amssymb}
\usepackage{amsmath}
\usepackage{graphicx}
\usepackage{subcaption}
\usepackage{xcolor}

\newcommand{\new}[1]{
    \vspace{2mm}
    \noindent
    \textbf{
    \underline{#1}}
}

\def\calO{{\mathcal{O}}}
\def\th{{\theta}}
\def\_{{\hspace{1mm}}}
\def\<{{\langle}}
\def\>{{\rangle}}


\newcounter{problemcnt}
\setcounter{problemcnt}{0}

\newcommand{\Problem}{{
    \vspace{5mm}
    \stepcounter{problemcnt}
    \noindent
    \arabic{problemcnt}. 
}
}

\newcommand{\nProblem}[1]{
    \vspace{5mm}
    \noindent
    \setcounter{problemcnt}{#1}
    \arabic{problemcnt}. 
}


\newcommand{\Proof}{{
    \vspace{2mm}
    \noindent
    \textbf{
    \underline{Proof}}
}
}

\newcommand{\textOr}{
    {
        \hspace{5mm}
        \textrm{or}
        \hspace{5mm}
    }
}

\newcommand{\textAnd}{
    {
        \hspace{5mm}
        \textrm{and}
        \hspace{5mm}
    }
}

\newcommand{\m}{
    \cdot
}

\newcommand{\Ixp}[1]{
    {
        e^{i{#1}}
    }
}



\newcommand{\halfFigure}[1]{
\begin{center}
\includegraphics[width = .5\linewidth]{{#1}}
\end{center}
}

\newcommand{\fullFigure}[1]{
\begin{center}
\includegraphics[width = .9\linewidth]{{#1}}
\end{center}
}

\def\twobytwoMat(#1, #2, #3, #4){
    {
        \begin{bmatrix}
            {#1} & {#2}\\
            {#3} & {#4}
        \end{bmatrix}
    }
}

\def\twobyoneMat(#1, #2){
    {
        \begin{bmatrix}
            {#1}\\
            {#2}
        \end{bmatrix}
    }
}

\def\twobytwoDet(#1, #2, #3, #4){
    {
        \begin{vmatrix}
            {#1} & {#2}\\
            {#3} & {#4}
        \end{vmatrix}
    }
}



\begin{document}
\begin{center}
\LARGE
Combinatorics HW8

\Large
Daniel Son
\end{center}

\normalsize 

\new{Q16}
Formulate a combinatorial problem for which the generating function is 
\[
    (1 + x + x^2)(1 + x^2 + x^4 + x^6) 
    (1 + x^2 + x^4 + \cdots) (x + x^3 + x^5 + \cdots)
\]
\new{Solution} 
Alice decided to spend the weekend in an arcade game plaza "Wonderland."
Wonderland has four games. To enter the center, Alice has to pay 
a quarter. The first game is Dune, and Alice 
has to pay a quarter to play a round of the game. Nonetheless, 
due to the popularity of the game, a player is allowed to play 
a maximum number of two rounds of Dune for each visit. Wonderland 
also has a Karaoke machine, and Alice has to pay two quarters 
for each song. A player is allowed to sing maximum three songs 
every visit. The other two games are pacman and tetris. These classic 
games cost two quarters for each round. They are less popular than 
the other two machines, so Alice can play as many games as she want. 

The number of ways Alice can spend her time with n quarters in 
Wonderland generates the generating function in question. We assume 
that the order in which Alice plays each game does not affect 
how she spends her time. That is, if Alice plays tetris then pacman, 
the arrangement is considered equivalent if she plays pacman first 
then tetris.

\hfill \qed

\new{Q17}
Determine the generating function for the number $h_n$ of bags of fruit of apples, 
oranges, bananas, and pears in which there are an even number of apples, at 
most two oranges, a multiple of three number of bananas, and at most one pear. 
Then find a formula for hn from the generating function. 

\new{Solution}
Upon inspection, we write the generationg function for $h_n$. 
\[
    \begin{split}
    G(x) = (1 + x^2 + x^4 + \cdots) (1 + x + x^2) 
    (1 + x^3 + x^6 + \cdots) (1 + x)
    \\
    = 
    \frac 1 {1- x^2 }
    \frac {1- x^3} {1 -x}
    \frac 1 {1 - x^3}
    (1 + x) 
    = 
    \frac 1 {(1-x)^2}
    =
    \sum_{n = 0}^\infty (n + 1) x^n
    \end{split}
\]

\new{Q22}
Determine the exponential generating function for the sequence of factorials. 

\new{Solution}

Recall the definition of exponential generating functions. 
For a sequence $\{a_n\}_{n \in \mathbb{N}}$, the exponential 
generating function is 
\[
    E(x) := \sum_{n = 0}^\infty \frac {a_nx^n} {n!}
\]
We know $a_n = n!$. Plug in the definition. 
\[
    \boxed{
    E(x) = \sum_{n = 0}^\infty x^n = \frac 1 {1 - x}
    }
\]

\hfill \qed

\new{Preliminary for Q22}
To better understand how EGFs can be used, we present 
the following theorem, which is a slight generalization 
of Thm 7.3.1 of the textbook. 

\new{Theorem} Multiplying to EGFs generates the EGF 
of a sequence that accounts for partitions. 

Let $f_i(x)$ be the EGF of the sequence $\{a^i_n\}_{n \in \mathbb{N}}$. 
The function 
\[
    \prod_{i \leq N} f_i(x)
\]
is an EGF of the sequence 
\[
    h_n := \sum_{m_1 + \dots m_N = n}\binom{n}{m_1, m_2, \dots, m_N}
    \prod_{i \leq N} a_i
\]

A short proof can be written similarly to that of Thm 7.3.1. 

\new{Q24}
\fullFigure{Q24.png}

\new{Solution}

Use the theorem presented in the preliminary. For part a, the EGF 
of $h_n$ is the product of the EGF for each of the objects. 
We add the constant 1 to account for $h_0 = 1$. 
\[
    E_a(x) = 1 + \left(x + \frac {x^3}{3!} + \frac {x^5}{5!} + \cdots\right)^k
    = 1 +\sinh^k(x)
\]

Similarly for part b, 
\[
    E_b(x) = 1 + \left(\frac{x^4}{4!} + \frac {x^5}{5!} + \frac {x^6}{6!} + \cdots\right)^k
    = 1 + \left(e^x - 1 - x - \frac {x^2} {2!} - \frac{x^3}{3!}\right)^k
\]

For part c, 
\[
    E_c(x) = 
    1 + \left(
        x + \frac {x^2}{2!} + \frac {x^3}{3!} + \cdots
    \right)
\left(
    \frac {x^2}{2!} + \frac {x^3}{3!} + \frac{x^4}{4!} + \cdots
    \right)
    \cdots
\left(
        \frac {x^k}{k!} + \frac {x^{k + 1}}{(k + 1)!} + \cdots
    \right)
\]

For part d, do not add the constant 1. 
\[
    E_c(x) = 
    \left(
        1 + x
    \right)
\left(
    1 + x + \frac {x^2}{2!} 
    \right)
    \cdots
\left(
     1 + x + \frac {x^2}{2!} + \frac{x^3}{3!} + \cdots + \frac{x^k}{k!}
    \right)
\]

\hfill 
\qed

\new{Q25} 
Let $h_n$ denote the number of ways to color the squares of a I-by-n board with 
the colors red, white, blue, and green in such a way that the number of squares 
colored red is even and the number of squares colored white is odd. Determine 
the exponential generating function for the sequence $h_0, h_1, . .. , h_n , ... ,$ and then 
find a simple formula for $h_n$. 

\new{Solution} The generating function for $h_n$ can be written as 
\[
    \left(
        1 + \frac {x^2} {2!} + \frac {x^4} {4!} + \cdots
    \right)
\left(
        x + \frac {x^3} {3!} + \frac {x^5} {5!} + \cdots
    \right)
     \left(
        1 + \frac {x^2} {2!} + \frac {x^3} {3!} + \cdots
    \right)^2
\]
\[
    = \left(
        \frac {e^x + e^{-x}} 2
    \right)
\left(
        \frac {e^x - e^{-x}} 2
    \right)
    e^{2x}
    = 
    \frac 1 4 (e^{4x} - 1)
\]
\[
    = 
    \sum_{k = 0} ^ \infty 
    4^{k - 1} x^{k}
     - \frac 1 4
     =
     \sum_{k = 1} ^ \infty 
     4^{k - 1} x^k
\]
\[
   = x + 4x^2 + 16 x^3 \cdots 
\]
Thus, we conclude 
\[
    h_n = 
    \begin{cases} 
        0 & (n = 0)
        \\
        4^{n - 1} & (n \geq 1)
    \end{cases}
\]

\new{Q26} Determine the number of ways to color the squares of a I-by-n chessboard, 
using the colors red, blue, green, and orange if an even number of squares is to 
be colored red and an even number is to be colored green. 

\new{Solution} Same as Q25, we write the generating function. 
\[
    \left(
        1 + \frac {x^2} {2!} + \frac {x^4} {4!} + \cdots
    \right)
\left(
        1 + \frac {x^2} {2!} + \frac {x^4} {4!} + \cdots
    \right)
     \left(
        1 + \frac {x^2} {2!} + \frac {x^3} {3!} + \cdots
    \right)^2
\]

\[
    = \left(
        \frac {e^x - e^{-x}} 2
    \right)
\left(
        \frac {e^x - e^{-x}} 2
    \right)
    e^{2x}
    = 
    \frac 1 4 (e^{4x} + 2e^{2x} +1)
\]
\[
    \frac 1 4 
    \sum_{n = 0}^\infty 
    \left(
        4^n + 2^{n+1}
    \right)x^n
    +1
\]
Thus, \[
 h_n = 
 \begin{cases}
    1 &(n = 0)\\
    4^{n - 1} + 2^{n - 1} & ( n \geq 1)
 \end{cases}
\]

\hfill \qed


\end{document}