\documentclass{article}
\usepackage{amsfonts}
\usepackage{amsthm}
\usepackage{amssymb}
\usepackage{amsmath}
\usepackage{graphicx}
\usepackage{subcaption}
\usepackage{xcolor}

\newcommand{\new}[1]{
    \vspace{2mm}
    \noindent
    \textbf{
    \underline{#1}}
}

\def\calO{{\mathcal{O}}}
\def\th{{\theta}}
\def\_{{\hspace{1mm}}}
\def\<{{\langle}}
\def\>{{\rangle}}


\newcounter{problemcnt}
\setcounter{problemcnt}{0}

\newcommand{\Problem}{{
    \vspace{5mm}
    \stepcounter{problemcnt}
    \noindent
    \arabic{problemcnt}. 
}
}

\newcommand{\nProblem}[1]{
    \vspace{5mm}
    \noindent
    \setcounter{problemcnt}{#1}
    \arabic{problemcnt}. 
}


\newcommand{\Proof}{{
    \vspace{2mm}
    \noindent
    \textbf{
    \underline{Proof}}
}
}

\newcommand{\textOr}{
    {
        \hspace{5mm}
        \textrm{or}
        \hspace{5mm}
    }
}

\newcommand{\textAnd}{
    {
        \hspace{5mm}
        \textrm{and}
        \hspace{5mm}
    }
}

\newcommand{\m}{
    \cdot
}

\newcommand{\Ixp}[1]{
    {
        e^{i{#1}}
    }
}



\newcommand{\halfFigure}[1]{
\begin{center}
\includegraphics[width = .5\linewidth]{{#1}}
\end{center}
}

\newcommand{\fullFigure}[1]{
\begin{center}
\includegraphics[width = .9\linewidth]{{#1}}
\end{center}
}

\def\twobytwoMat(#1, #2, #3, #4){
    {
        \begin{bmatrix}
            {#1} & {#2}\\
            {#3} & {#4}
        \end{bmatrix}
    }
}

\def\twobyoneMat(#1, #2){
    {
        \begin{bmatrix}
            {#1}\\
            {#2}
        \end{bmatrix}
    }
}

\def\twobytwoDet(#1, #2, #3, #4){
    {
        \begin{vmatrix}
            {#1} & {#2}\\
            {#3} & {#4}
        \end{vmatrix}
    }
}



\begin{document}
\begin{center}
\LARGE
Combinatorics HW9

\Large
Daniel Son
\end{center}

\normalsize 
\new{Exercise 1}
\fullFigure{Q1.png}

\new{Exercise 3}
Find a value of $n$ that satisfies 
\[
    \Delta n^x = n^x
\]
\new{Solution}
Upon inspection, $n = 2$ works. 
\[
    \Delta 2^x = 2^{x + 1} - 2^x = 2 \cdot 2^x - 2^x = 2^x 
\]

\hfill \qed

\new{Exercise 5a}
Prove the linearity of the difference operator. That is, 
for all real sequences $f, g$ and real numbers $c_1, c_2$, 
\[
    \Delta (c_1 f + c_2 g) = c_1 \Delta f + c_2 \Delta g
\]

\new{Solution }

We prove the preservation of addition and scalar multiplication 
separately. Let $\{f_n\}_{n \in \mathbb{N}}$, $\{g_n\}_{n \in \mathbb{N}}$
be real sequences. We wish to show, for any natural number n, 
\[
    \Delta(f_n + g_n) = \Delta f_n + \Delta g_n
\]. Directly evaluate the LHS. 
\[
    \Delta(f_n + g_n) = 
    (f_{n + 1} + g_{n + 1} - f_n - g_n )
=
(f_{n + 1} - f_{n} + g_{n + 1} - g_n ) 
= 
\Delta f_n + \Delta g_n 
\]

Now, move on to show the preservation of scalar 
multiplication. Let $c \in \mathbb{R}$ and $n$ be any natural number. 
We wish to show
\[
    \Delta(cf_n) = c\Delta f_n 
\]
Again, by directly evaluating the LHS, 
\[
    \Delta(c f_n) = (cf_{n + 1} - cf_n) = c(f_{n + 1} - f_n) = c\Delta f_n
\]
Finally, consider the following line of algebra. 
\[
    \Delta (c_1 f + c_2 g) = 
    \Delta(c_1 f) + \Delta (c_2 g) =
    c_1 \Delta f + c_2 \Delta g
\]

\hfill \qed


\new{Exercise 5b}
Prove 
\[
\Delta (f g) = \Delta f\Delta g - f\Delta g - g\Delta f    
\]

\new{Solution}

 $\{f_n\}_{n \in \mathbb{N}}$, $\{g_n\}_{n \in \mathbb{N}}$
be real sequences. We wish to show, for any $n \in \mathbb{N}$, 
\[
\Delta (f g)_n = \Delta f_n\Delta g_n + f_n\Delta g_n + g_n\Delta f_n    
\]

We prove this by directly evaluating the RHS. 
\[
    RHS = (f_{n + 1} - f_n )(g_{n + 1} - g_n ) 
    +f_n (g_{n + 1} - g_n )  +g_n(f_{n + 1} - f_n ) 
\]
\[
    = f_{n + 1} g_{n + 1} - f_ng_n = \Delta (fg)_n
\]

\hfill \qed



\end{document}

