\documentclass{article}
\usepackage{amsmath, amsthm, amssymb}

\theoremstyle{plain}
\newtheorem{theorem}{Theorem}

\begin{document}
\begin{center}
    \Large
    \textbf{Strong Induction for AM-GM}

    \large
    Benevolent Tomato, ChatGPT
\end{center}
\begin{theorem}[AM-GM Inequality]
Let $x_1, x_2, \dots, x_N$ be non-negative real numbers. Then the following inequality holds:
\[
\frac{x_1 + x_2 + \cdots + x_N}{N} \geq \sqrt[N]{x_1 x_2 \cdots x_N}
\]
with equality if and only if $x_1 = x_2 = \cdots = x_N$.
\end{theorem}

\begin{proof}
We will use strong induction on $N$ to prove the AM-GM inequality.

\textbf{Base Case:} For $N = 2$, the inequality is the classical AM-GM inequality for two numbers:
\[
\frac{x_1 + x_2}{2} \geq \sqrt{x_1 x_2}
\]
This holds by the trivial identity derived from squaring both sides:
\[
(x_1 - x_2)^2 \geq 0
\]
Thus, the base case holds.

\textbf{Inductive Step:} Suppose the AM-GM inequality holds for $n < N$. We will prove it for $N$.

For $N$ even, let $x_1, \dots, x_N$ be non-negative real numbers. Then,
\[
\frac{x_1 + \cdots + x_{N/2}}{N/2} \geq \sqrt[N/2]{x_1 \cdots x_{N/2}} \quad \text{and} \quad \frac{x_{N/2+1} + \cdots + x_N}{N/2} \geq \sqrt[N/2]{x_{N/2+1} \cdots x_N}
\]
Taking the arithmetic mean of these inequalities, we obtain:
\[
\frac{x_1 + \cdots + x_N}{N} \geq \sqrt[N]{x_1 \cdots x_N}
\]
Now, if $N$ is odd, we can show that for any natural number $N$, adding an additional variable $x_{N+1}$, we have:
\[
\frac{x_1 + \cdots + x_N + x_{N+1}}{N+1} \geq \sqrt[N+1]{x_1 \cdots x_N x_{N+1}}
\]
(denote this as $(*)$).

If we set $x_{N+1} = k(x_1 + \cdots + x_N)$, then:
\[
\sum_{i \leq N} x_i + x_{N+1} = (k+1) \sum_{i \leq N} x_i
\]
and $(*)$ simplifies to:
\[
\frac{k+1}{N+1} \left(\sum_{i \leq N} x_i \right) \geq \sqrt[N+1]{x_1 \cdots x_N \cdot k \sum_{i \leq N} x_i}
\]
Taking the logarithm of both sides, we obtain:
\[
\frac{1}{k}\left(\frac{k+1}{N+1}\right)^{N+1} \left(\sum_{i \leq N} x_i \right) \geq \sqrt[N]{x_1 \cdots x_N}
\]
Thus, by the intermediate value theorem, there exists a $k > 0$ such that:
\[
\frac{(k+1)^{N+1}}{k(N+1)^{N+1}} = 1
\]
Therefore, the inequality holds.

If $k \to 0$, the left-hand side blows up and hence guarantees the inequality. For $k = N$, the left-hand side is $\frac{N}{N+1} < 1$, so by the intermediate value theorem, there exists a $k$ satisfying the condition.

Thus, the AM-GM inequality is proved by induction.
\end{proof}

\end{document}
