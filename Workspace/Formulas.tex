\documentclass{article}
\usepackage{amsfonts}
\usepackage{amsthm}
\usepackage{amssymb}
\usepackage{amsmath}
\usepackage{graphicx}
\usepackage{subcaption}
\usepackage{xcolor}
\usepackage{mathtools}
\usepackage{ wasysym }


\newcommand{\new}[1]{
    \vspace{2mm}
    \noindent
    \textbf{
    \underline{#1}}
}

\def\calO{{\mathcal{O}}}
\def\th{{\theta}}
\def\_{{\hspace{1mm}}}
\def\<{{\langle}}
\def\>{{\rangle}}

\DeclarePairedDelimiter\bra{\langle}{\rvert}
\DeclarePairedDelimiter\ket{\lvert}{\rangle}
\DeclarePairedDelimiterX\braket[2]{\langle}{\rangle}{#1\,\delimsize\vert\,\mathopen{}#2}



\newcounter{problemcnt}
\setcounter{problemcnt}{0}

\newcommand{\Problem}{{
    \vspace{5mm}
    \stepcounter{problemcnt}
    \noindent
    \arabic{problemcnt}. 
}
}

\newcommand{\nProblem}[1]{
    \vspace{5mm}
    \noindent
    \setcounter{problemcnt}{#1}
    \arabic{problemcnt}. 
}


\newcommand{\Proof}{{
    \vspace{2mm}
    \noindent
    \textbf{
    \underline{Proof}}
}
}

\newcommand{\textOr}{
    {
        \hspace{5mm}
        \textrm{or}
        \hspace{5mm}
    }
}

\newcommand{\textAnd}{
    {
        \hspace{5mm}
        \textrm{and}
        \hspace{5mm}
    }
}

\newcommand{\Ixp}[1]{
    {
        e^{i{#1}}
    }
}



\newcommand{\halfFigure}[1]{
\begin{center}
\includegraphics[width = .5\linewidth]{{#1}}
\end{center}
}

\newcommand{\fullFigure}[1]{
\begin{center}
\includegraphics[width = .9\linewidth]{{#1}}
\end{center}
}

\def\twobytwoMat(#1, #2, #3, #4){
    {
        \begin{bmatrix}
            {#1} & {#2}\\
            {#3} & {#4}
        \end{bmatrix}
    }
}

\def\twobyoneMat(#1, #2){
    {
        \begin{bmatrix}
            {#1}\\
            {#2}
        \end{bmatrix}
    }
}

\def\twobytwoDet(#1, #2, #3, #4){
    {
        \begin{vmatrix}
            {#1} & {#2}\\
            {#3} & {#4}
        \end{vmatrix}
    }
}



\begin{document}
\begin{center}
\LARGE
Formula Workspace

\normalsize
\end{center}

Let $X, A$ be elements of a lie algebra $\mathfrak g$. 
Recall that the elements of the algebra generates elements 
in the lie group. This means, 
\[
    \exp(Xt) \in G
\]
For any small enough t. We know that conjugation is a group 
action on $G$. Take some element $A \in G$. 

\[
    A \exp(Xt) A^{-1} \in G
\]

A nice property of this element is that the element 
evaluates to the identity around $t = 0$. Oh, taking 
the differential at $t=0$ must yield a lie algebra. Hence, 
\[
    A X A^{-1} \in \mathfrak g
\]

All the elements of the algebra and the group are matricies. 
The operation between the elements are matrix multiplication, 
which is known to be linear. Hence, this action is indeed 
a representation. We call this representation as \textbf{
    the adjoint representation
}. 

By the symmetry of the adjoint, we denote that 
$\mathfrak g$ is stable under the adjoint representation. 


Some comment on Operators 

\begin{align}
    [a_n, a_n^\dag] \ = \ a_n a_n ^\dag \varphi_0 - a_n^\dag a_n \varphi_0  \ = \ \varphi_0
\end{align}

\end{document}