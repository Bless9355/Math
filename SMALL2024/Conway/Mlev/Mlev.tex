\documentclass{article}
\usepackage{amsfonts}
\usepackage{amsthm}
\usepackage{amssymb}
\usepackage{amsmath}
\usepackage{graphicx}
\usepackage{subcaption}
\usepackage{xcolor}
\usepackage{mathtools}
\usepackage{ wasysym }
\usepackage{enumerate}


\newcommand{\new}[1]{
    \vspace{2mm}
    \noindent
    \textbf{
    \underline{#1}}
}

\def\calO{{\mathcal{O}}}
\def\th{{\theta}}
\def\_{{\hspace{1mm}}}
\def\<{{\langle}}
\def\>{{\rangle}}

\DeclarePairedDelimiter\bra{\langle}{\rvert}
\DeclarePairedDelimiter\ket{\lvert}{\rangle}
\DeclarePairedDelimiterX\braket[2]{\langle}{\rangle}{#1\,\delimsize\vert\,\mathopen{}#2}



\newcounter{problemcnt}
\setcounter{problemcnt}{0}

\newcommand{\Problem}{{
    \vspace{5mm}
    \stepcounter{problemcnt}
    \noindent
    \arabic{problemcnt}. 
}
}

\newcommand{\nProblem}[1]{
    \vspace{5mm}
    \noindent
    \setcounter{problemcnt}{#1}
    \arabic{problemcnt}. 
}


\newcommand{\Proof}{{
    \vspace{2mm}
    \noindent
    \textbf{
    \underline{Proof}}
}
}

\newcommand{\textOr}{
    {
        \hspace{5mm}
        \textrm{or}
        \hspace{5mm}
    }
}

\newcommand{\textAnd}{
    {
        \hspace{5mm}
        \textrm{and}
        \hspace{5mm}
    }
}


\newcommand{\textWhere}{
    {
        \hspace{5mm}
        \textrm{where}
        \hspace{5mm}
    }
}



\newcommand{\Ixp}[1]{
    {
        e^{i{#1}}
    }
}



\newcommand{\halfFigure}[1]{
\begin{center}
\includegraphics[width = .5\linewidth]{{#1}}
\end{center}
}

\newcommand{\fullFigure}[1]{
\begin{center}
\includegraphics[width = .9\linewidth]{{#1}}
\end{center}
}

\def\twobytwoMat(#1, #2, #3, #4){
    {
        \begin{bmatrix}
            {#1} & {#2}\
            {#3} & {#4}
        \end{bmatrix}
    }
}

\def\twobyoneMat(#1, #2){
    {
        \begin{bmatrix}
            {#1}\
            {#2}
        \end{bmatrix}
    }
}

\def\twobytwoDet(#1, #2, #3, #4){
    {
        \begin{vmatrix}
            {#1} & {#2}\
            {#3} & {#4}
        \end{vmatrix}
    }
}


\newcommand{\RR}{\mathbb{R}}
\newcommand{\CC}{\mathbb{C}}

\begin{document}
\begin{center}
    \Large
    \textbf{Variants of Conway's Soldiers: Monovariant 
    methods and Fibonacci jumping}

    \large
    Conway group
\end{center}

%0. Establishing conventions Inverse fibbonacci
%1. Two Problems of interest
%2. Monovariants and upper bounds
%3. Useful arrangements
%4. Strategy for the M game
%5. Additional problems. 


\section{Review of notations and preliminaries}

We define our fibbonacci sequence with the following initial 
conditions and recurrence relation. 
\[
    (F_0, F_1) = (0, 1)
    \textAnd 
    F_n = F_{n - 1} + F_{n - 2} \hspace{5mm}\forall n \geq 2
\]

By Binet's Formula, it is possible to write out an explicit equation 
for the $n$th Fibonacci number. 
\[
    F_n =
    \frac 1 {\sqrt 5} 
    \left(
        \varphi^n - (1 - \varphi)^n 
    \right)
    \textWhere 
    \varphi = \frac {1 + \sqrt 5} 2
\]  
Assuming $n \geq 1$, it is possible to obtain the following inequality. 
\[
    \bigg|F_n  - \frac {\varphi^n} {\sqrt 5}\bigg| \leq \frac 1 {\sqrt 5}(\varphi - 1)^n 
    < .277
\]
We are interested in computing the fibbonacci inverse function. 
Introduce the notation 
\[
    \mathcal{F}^{-1} (m) = n
\]. 

\new{Proposition 1} Inverse Fibbonacci Inequality 

\[
    \log_{\varphi}(m - .277) +.167 <  \mathcal{F}^{-1} (m) < \log_{\varphi}(m + .277) +.168
\]
\proof
Given an integer $m$, we wish to compute $n$ that satisfies the following 
inequality. 
\begin{equation}
    \label{invfib}
    F_{n - 1} < m \leq F_{n}
\end{equation}
Using our estimation, we can write 
\begin{equation*}
 \frac {\varphi^{n - 1}} {\sqrt 5} - .277
    <
    m \leq  \frac {\varphi^n} {\sqrt 5} + .277
\end{equation*}.
With some algebra, it is not hard to derive
\begin{equation}
    \log_{\varphi}(m - .277) +.167 < n < \log_{\varphi}(m + .277) +.168
\end{equation}

\hfill \qed

\new{Remark} For large enough $m$, it is reasonable to make the estimation 
\[
    m \approx \log_{\varphi}(m) +.167
\]



\section{Conway Soldiers and two related problems}
\textit{figure to be added...}
Consider an infinite chessboard. Draw a horizontal line 
across the board, and fill all the grids under the line 
with one checkers. \footnote{In some formulations, 
the checkers are formulated as soldiers, hence the name Conway 
Soldiers}. In each turn, we are allowed to take a soldier and 
jump across a neighboring checker, moving the checker two grids. 
If a jump occurs, the neighboring checker which is jumped upon is taken 
away. We continue this game for a finite number of moves. 

We are interested in two main problems. 

\new{Problem 1}
\textit{
 The M-level Problem
}

What is the highest level $M$ which 
can be reached by the rules of Conway Checkers?

\new{Problem 2 }
\textit{
 The Rallying Problem
}

Select any grid that is adjacent and below 
the borderline. What is the maximum number of soldiers we 
can focus on the target grid? 

We reserve the variables $M, R$ for the solution for the 
two problems. That is, given multiplicity $m$, $M$ 
will refer to the optimal solution for the M-level problem, 
and $R$ will refer to the maximum number of soldiers that 
can be rallied at the target square. 

\section{Monovariant Methods to compute upper bounds}
\textit{More figure, and reference}

We start with computing the bound of $M$ using the 
monovariant method. Assume, after some sequence of moves, 
it is possible to send a single checker to level $M$. 
By the symmetry of the infinite board, we can choose 
any square that is $M$ grids north from the boundary. 
At the end of the game, the score of the board must be 
strictly greater than 1. After a sequence of moves, there 
must be one piece in the target grid. Also, there must be 
some other grids in the board, since we are only allowed 
to conduct finite number of jumps. \footnote{The board 
must have a infinite number of checkers after a finite number 
of moves. }
 
\newpage

\new{Theorem} \textit{Upper bound for M}
\[
    \log_\varphi(m) + 5 > M
\]
\proof
Compute the inital score of the board which we will call $T$. For simplicity, 
consider the score of one horizontal strip, and call it $s$. 
Remember that there are $m$ checkers in each board initially. 
Using the geometric series formula, it is possible to derive the following. 
\[
T = (\varphi - 1)^M \frac \varphi {5 - 3\varphi} 
\]
$T$ must be greater than $1$. Thus, we can write 
\[
m(\varphi - 1)^M \frac \varphi {5 - 3\varphi} > 1
\]
Which leads to 
\[
m \frac \varphi {5 - 3\phi} > \left(
    \frac 1 {\varphi - 1}
\right)^M = \varphi^M
\]
Taking the log base golden ratio gives us the desired result. 
\[
    \log_\varphi(m) + 5 > M
\]
We note that equality can be achieved when we allow infinite number of 
steps. 
\hfill \qed

\new{Proposition} \textit{Fibonacci Climb}
Given two grids that each have $F_{n+1}, F_n$ checkers, 
it is possible to send a single checker $n$ grids away from 
the beginning location. 

\new

\end{document}