\documentclass{article}
\usepackage{amsfonts}
\usepackage{amsthm}
\usepackage{amssymb}
\usepackage{amsmath}
\usepackage{graphicx}
\usepackage{subcaption}
\usepackage{xcolor}
\usepackage{mathtools}
\usepackage{ wasysym }
\usepackage{enumerate}


\newcommand{\new}[1]{
    \vspace{2mm}
    \noindent
    \textbf{
    \underline{#1}}
}

\def\calO{{\mathcal{O}}}
\def\th{{\theta}}
\def\_{{\hspace{1mm}}}
\def\<{{\langle}}
\def\>{{\rangle}}

\DeclarePairedDelimiter\bra{\langle}{\rvert}
\DeclarePairedDelimiter\ket{\lvert}{\rangle}
\DeclarePairedDelimiterX\braket[2]{\langle}{\rangle}{#1\,\delimsize\vert\,\mathopen{}#2}



\newcounter{problemcnt}
\setcounter{problemcnt}{0}

\newcommand{\Problem}{{
    \vspace{5mm}
    \stepcounter{problemcnt}
    \noindent
    \arabic{problemcnt}. 
}
}

\newcommand{\nProblem}[1]{
    \vspace{5mm}
    \noindent
    \setcounter{problemcnt}{#1}
    \arabic{problemcnt}. 
}


\newcommand{\Proof}{{
    \vspace{2mm}
    \noindent
    \textbf{
    \underline{Proof}}
}
}

\newcommand{\textOr}{
    {
        \hspace{5mm}
        \textrm{or}
        \hspace{5mm}
    }
}

\newcommand{\textAnd}{
    {
        \hspace{5mm}
        \textrm{and}
        \hspace{5mm}
    }
}


\newcommand{\textWhere}{
    {
        \hspace{5mm}
        \textrm{where}
        \hspace{5mm}
    }
}



\newcommand{\Ixp}[1]{
    {
        e^{i{#1}}
    }
}



\newcommand{\halfFigure}[1]{
\begin{center}
\includegraphics[width = .5\linewidth]{{#1}}
\end{center}
}

\newcommand{\fullFigure}[1]{
\begin{center}
\includegraphics[width = .9\linewidth]{{#1}}
\end{center}
}

\def\twobytwoMat(#1, #2, #3, #4){
    {
        \begin{bmatrix}
            {#1} & {#2}\\
            {#3} & {#4}
        \end{bmatrix}
    }
}

\def\twobyoneMat(#1, #2){
    {
        \begin{bmatrix}
            {#1}\\
            {#2}
        \end{bmatrix}
    }
}

\def\twobytwoDet(#1, #2, #3, #4){
    {
        \begin{vmatrix}
            {#1} & {#2}\\
            {#3} & {#4}
        \end{vmatrix}
    }
}


\newcommand{\RR}{\mathbb{R}}
\newcommand{\CC}{\mathbb{C}}

\begin{document}
\begin{center}
    \Large
    \textbf{Rallying Soldiers in a Conway Game}

    \large
    Conway group
\end{center}

So previously, we have studied on how to send a single soldier to the 
highest level possible, given a board multiplicity m. In this paper, 
we focus on a similar but different problem: given 
a board multiplicty m, what is the maximum amount of soldiers we can 
rally on a single grid?

\new{Problem 1}
\textit{
 The M-level problem
}

Assume we have a two dimensional checkerboard that 
extends infinitely both in the horizontal and the vertical direction. 
A borderline is drawn, and we are provided with $m $soldiers on 
each grid of the checkerboard under 
the reference line. 
What is the highest level $M$ which 
can be reached by the rules of Conway Checkers?

\new{Problem 2 }
\textit{
 The Rallying Problem
}

Consider the same board from Problem 1. Select any grid that is adjacent and below 
the borderline. What is the maximum number of soldiers we 
can focus on the target grid? 

\halfFigure{mtp1.png}
\begin{center}
    \textbf{Figure 1. The rallying problem for m = 1}
\end{center}

It is easier to study this problem on a one dimensional strip. 
From the 2D board, focus on one vertical strip and ignore the rest of 
the board. Assuming $x = \psi - 1$ \footnote{$\psi$ is the golden ratio 
$\frac {\sqrt 5 - 1} 2$}, we compute that the monovariant of the single strip is 
\[
    \frac m {1 - x} = \frac m {2 - \psi} = (1 + \psi) m 
\]
Where the last equality follows from the formula $\psi^2 + \psi - 1 = 0$. 

Assume, after a series of manipulations, we end up with $r$ soldiers 
in the target square. The score of the board will be $r$. The monovariant
 is invariant under a Conway move. Hence, we obtain the following inequality. 
 \[
    r < (1 + \psi) m
 \]\footnote{equality holds assuming we allow infinite number of moves } 

We have shown the following. 

\new{Proposition 1} The theoretical upper bound of $r$ in a 
1D strip is 
\[
    \lceil (1 + \psi)\rceil - 1
\]. 

In a 1D strip, it is possible to come up with an optimal strategy 
that asymptotically approaches this limit. For utility, define 
the following. 

\new{Definiton} \textit{Fibonacci inversion function}

Let 
\[
    \mathcal{F}^{-1} (N) = n   
\]
where $n$ is the index of the largest fibonacci number less than 
or equal to  $N$. 
That is, $F_{n} \leq N < F_{n + 1}$. 

\new{Proposition 2} \textit{The heavy-headed stick}
Suppose $n = \mathcal{F} (m)$. In a 1D strip of the Conway game 
with multiplicity $m$, it is possible to rally $F_{n + 1}$ 
Soldiers to one target grid. 

\proof Here is a visual demonstration. 
\fullFigure{HeavyHead.png}

\new{Proposition 3} \textit{Multiple Heavy Heads}

Suppose $n - 1 = \mathcal{F}^{-1} (m/2)$. Consider a 1D strip of the Conway game 
with multiplicity $m$. 
Suppose that the strip has a fixed head(or the top grid) 
and extends infintely below. 
It is possible to rally $m + F_{n + 1}$ 
Soldiers to the head/top grid.


\proof Here is a visual demonstration. 
\fullFigure{Strategy1.png}

The construction of $n$ 
follows from the fact that the superposed board needs the 
following multiplicity $m$. 
\[
    m = \max_{0 \leq j < N}\{(j + 1)F_{N - j}\}
\]
\hfill \qed

\new{Corollary} 
We can approximate the number of added soldiers, $F_{n+1}$. 
\[
    F_{n + 1} \approx \varphi^{n + 1}  
    \approx \varphi^{\log_\varphi (m/2) + 2}
    = \frac {m \varphi^2} 2
\]
\hfill \qed





\end{document}