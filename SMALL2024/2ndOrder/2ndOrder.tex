\documentclass{article}
\usepackage{amsfonts}
\usepackage{amsthm}
\usepackage{amssymb}
\usepackage{amsmath}
\usepackage{graphicx}
\usepackage{subcaption}
\usepackage{xcolor}
\usepackage{mathtools}
\usepackage{ wasysym }
\usepackage{enumerate}


\newcommand{\new}[1]{
    \vspace{2mm}
    \noindent
    \textbf{
    \underline{#1}}
}

\def\calO{{\mathcal{O}}}
\def\th{{\theta}}
\def\_{{\hspace{1mm}}}
\def\<{{\langle}}
\def\>{{\rangle}}

\DeclarePairedDelimiter\bra{\langle}{\rvert}
\DeclarePairedDelimiter\ket{\lvert}{\rangle}
\DeclarePairedDelimiterX\braket[2]{\langle}{\rangle}{#1\,\delimsize\vert\,\mathopen{}#2}



\newcounter{problemcnt}
\setcounter{problemcnt}{0}

\newcommand{\Problem}{{
    \vspace{5mm}
    \stepcounter{problemcnt}
    \noindent
    \arabic{problemcnt}. 
}
}

\newcommand{\nProblem}[1]{
    \vspace{5mm}
    \noindent
    \setcounter{problemcnt}{#1}
    \arabic{problemcnt}. 
}


\newcommand{\Proof}{{
    \vspace{2mm}
    \noindent
    \textbf{
    \underline{Proof}}
}
}

\newcommand{\textOr}{
    {
        \hspace{5mm}
        \textrm{or}
        \hspace{5mm}
    }
}

\newcommand{\textAnd}{
    {
        \hspace{5mm}
        \textrm{and}
        \hspace{5mm}
    }
}


\newcommand{\textWhere}{
    {
        \hspace{5mm}
        \textrm{where}
        \hspace{5mm}
    }
}



\newcommand{\Ixp}[1]{
    {
        e^{i{#1}}
    }
}



\newcommand{\halfFigure}[1]{
\begin{center}
\includegraphics[width = .5\linewidth]{{#1}}
\end{center}
}

\newcommand{\fullFigure}[1]{
\begin{center}
\includegraphics[width = .9\linewidth]{{#1}}
\end{center}
}

\def\twobytwoMat(#1, #2, #3, #4){
    {
        \begin{bmatrix}
            {#1} & {#2}\\
            {#3} & {#4}
        \end{bmatrix}
    }
}

\def\twobyoneMat(#1, #2){
    {
        \begin{bmatrix}
            {#1}\\
            {#2}
        \end{bmatrix}
    }
}

\def\twobytwoDet(#1, #2, #3, #4){
    {
        \begin{vmatrix}
            {#1} & {#2}\\
            {#3} & {#4}
        \end{vmatrix}
    }
}


\newcommand{\RR}{\mathbb{R}}
\newcommand{\CC}{\mathbb{C}}

\begin{document}
\begin{center}
    \Large
    \textbf{Recurrence relations with bounded behavior}
    \large
    Daniel Son
\end{center}

Let $x, y \in [0, 1]$ be a real value that describes 
the population proportion of the prey and predator respectively. 
The Lokta-Volterra relation that describes the predator-prey 
model is a 2nd order differential equation as the following. 

\[
\frac d {dx} x' := \alpha x + \beta xy
\]
\[
\frac d {dy} y' := \gamma y - \delta xy
\]

It would be nice to solve the difference equation 
related to the differential equation. Given some 
small discrete timestep, we can use Euler's method to 
write a system of difference equations that approximates the predator-prey model.  

\[
x' := x + t\alpha x + t\beta xy
\]
\[
y' := y + t\gamma y - t\delta xy
\]

We do not know how to solve this equation directly. Instead, 
we wish to solve the a related difference equation of the second order. 
We hope that our solution for this related system potrays similar 
behavior as the Lokta-Volterra system. 

\begin{eqnarray}
    (ax' + by' + c) & = &  M (ax + by + c)  (\bar ax + \bar by + \bar c) \nonumber\\
    \label{eqn:wierdSys}
    (\bar ax' + \bar by' + \bar c) & = &  \bar M (\bar ax + \bar by + \bar c)
\end{eqnarray}

At equilibrium, $(x', y') = (x, y)$. 
Denote the value at equilibrium as $(x, y) = (x_{eq}, y_{eq})$. 
Plugging this condition into 
($\ref{eqn:wierdSys}$), we deduce 
\begin{eqnarray*}
    (ax_{eq} + by_{eq} + c) & = & 0 \\ 
(\bar ax_{eq} +\bar by_{eq} + \bar c) & = & 0 
\end{eqnarray*}
which determines the value of $c, \bar c $ as 
\begin{eqnarray*}
    c & = &-( ax_{eq} + by_{eq}) \\ 
 \bar c & = & -(\bar ax_{eq} +\bar by_{eq} )
\end{eqnarray*}
In light these relations, it is natural to look at a 
normalized quantity of the variables. Subtract 
a constant factor from both $x, y$ and define the following 
quantities. 
\[
    \Delta  x := (x - x_{eq}) 
    \textAnd 
    \Delta y := (y - y_{eq})
\]

The system in (\ref{eqn:wierdSys}) is converted 
into a purely second order system. 
\begin{eqnarray}
    (a\Delta x' + b\Delta y') & = &  M (a\Delta x + b\Delta y)  (\bar a\Delta x + \bar b\Delta y ) \nonumber\\
    \label{eqn:wierdSys2}
    (\bar a\Delta x' + \bar b\Delta y') & = &  \bar M (\bar a\Delta x + \bar b\Delta y)
\end{eqnarray}

The values of $x, y$ over time create a real value sequence 
$\{x_n\}, \{y_n\}$ where 
\[
x_n := x_0 ^{(n)} \textAnd y_n := y_0 ^{(n)}
\]
\footnote{The power of (n) means that we operate the time 
difference operation n times. e.g. $x^(2) = x''$. }
where $x_1, y_1$ are the initial values of the population ratio. 
Define an auxillary sequence as the following. 
\begin{eqnarray*}
    U_n :=a\Delta x + b \Delta y =  a(x_n - x_{eq}) + b (y_n - y_{eq})\\
    V_n :=\bar a\Delta x + \bar b \Delta y =  a(x_n - x_{eq}) + b (y_n - y_{eq})
\end{eqnarray*}

The system (\ref{eqn:wierdSys2}) simplifies even more in terms of 
$U_n$ and $V_n$. 
\begin{eqnarray*}
    U_{n + 1} = M U_n V_n \textAnd
    V_{n + 1} = \bar M V_n
\end{eqnarray*}
Taking the natural logarithm both sides reduces the system to 
\begin{eqnarray*}
    \ln (U_{n + 1}) &=& \ln(M) +\ln(U_n) +\ln(V_n) \nonumber\\
    \ln (V_{n + 1}) &=& \ln(\bar M)+\ln( V_n)
\end{eqnarray*}
Adopt more shorthands. 
\[
    \ln(U_n) := u_n \textAnd \ln(V_n) := v_n 
    \]\[ \ln(M) := m 
    \textAnd \ln(\bar M) := \bar m
\]
Graciously, we arrive at the following equation. 
\begin{eqnarray*}
    u_{n + 1} &=& m +u_n +v_n \nonumber\\
    v_{n + 1} &=& \bar m+v_n
\end{eqnarray*}
In matrix form, this equation can be written as 
\begin{equation}
    \label{eqn:matrixRe}
\twobyoneMat(u_{n + 1}, v_{n + 1})
= 
F 
\twobyoneMat(u_n , v_n ) + \twobyoneMat(m, \bar m)
\end{equation}
where $F := \twobytwoMat(1, 1, 1, 0)$ is the fibbonacci matrix. 

\pagebreak
\new{Proposition}
The solution for (\ref{eqn:matrixRe}) is 
\[
\twobyoneMat(u_n, v_n) = \twobyoneMat(
    F_{n + 1} (u_0 + \bar m + m) + F_n (v_0 + m) - \bar m - m, 
    {F_{n} \left(u_0 + \bar m + m\right) + F_{n - 1} \left(v_0 + m\right) - m})
\]
where $F_n$ is the nth fibbonacci number with the initial conditions 
$F_0 = F_1 = 1$. 

\proof It is easy to verify the equation for the $n$th power of $F$
for $n > 0$. 
\[
    F^n := \twobytwoMat(F_{n + 1}, F_n, F_n , F_{n - 1})
\]
Adopt the shorthand $\vec v_n := \twobyoneMat(u_n, v_n)$ and 
$\vec m :=  \twobyoneMat(m, \bar m)$. 

\noindent
We have the recurrence 
\[
\vec u_{n + 1} = F \vec u_n + \vec m
\]
which indicates the formula 
\[
    \vec u^{(n)} = 
    F^{n}\vec u +
    \sum_{k = 0}^{n - 1} F^{k} \vec m
\]. 
Using the geometric series formula, we deduce the follows. 
Note that \linebreak $(F - I)^{-1} = F$ where $I$ is the identity matrix. 
\[
\sum_{k = 0}^{n - 1} F^{k} = F(F^n - I)
\]
Thus, 
\[
    \vec u^{(n)} = 
    F^{n}\vec u +
    F^n F\vec m - F\vec m
\]
which indicates the result 
\[
\twobyoneMat(u_n, v_n) = \twobyoneMat(
    F_{n + 1} (u_0 + \bar m + m) + F_n (v_0 + m) - \bar m - m, 
    {F_{n} \left(u_0 + \bar m + m\right) + F_{n - 1} \left(v_0 + m\right) - m})
\]
\hfill \qed

\new{Corollary 1}
Taking the exponential of $u_n, v_n$, it is possible to obtain the \linebreak following. 
\[
\twobyoneMat(U_n, V_n) = \twobyoneMat(
     (U_0 M \bar M)^ {F_{n + 1}} (V_0M)^{F_n} /(M\bar M), 
    {(U_0 M \bar M)^{F_n} \left(V_0 M\right)^{F_{n - 1}} /M})
\]

\new{Remark}
Eventually, by taking linear combinations, we can recover a closed 
form formula for $x, y$ . 


\end{document}