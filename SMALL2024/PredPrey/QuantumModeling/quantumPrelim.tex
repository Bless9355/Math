\documentclass{article}
\usepackage{amsfonts}
\usepackage{amsthm}
\usepackage{amssymb}
\usepackage{amsmath}
\usepackage{graphicx}
\usepackage{subcaption}
\usepackage{xcolor}
\usepackage{mathtools}
\usepackage{ wasysym }
\usepackage{enumerate}
\usepackage{verbatim}


\newcommand{\new}[2]{
    \vspace{2mm}
    \noindent
    \textbf{
    \underline{#1}}
    \textit{{#2}}
    \
    \newline
}

\def\<{{\langle}}
\def\>{{\rangle}}

\DeclarePairedDelimiter\bra{\langle}{\rvert}
\DeclarePairedDelimiter\ket{\lvert}{\rangle}
\DeclarePairedDelimiterX\braket[2]{\langle}{\rangle}{#1\,\delimsize\vert\,\mathopen{}#2}


\newcommand{\textOr}{
    {
        \hspace{5mm}
        \textrm{or}
        \hspace{5mm}
    }
}

\newcommand{\textAnd}{
    {
        \hspace{5mm}
        \textrm{and}
        \hspace{5mm}
    }
}


\newcommand{\textWhere}{
    {
        \hspace{5mm}
        \textrm{where}
        \hspace{5mm}
    }
}



\newcommand{\Ixp}[1]{
    {
        e^{i{#1}}
    }
}



\newcommand{\halfFigure}[1]{
\begin{center}
\includegraphics[width = .5\linewidth]{{#1}}
\end{center}
}

\newcommand{\fullFigure}[1]{
\begin{center}
\includegraphics[width = .9\linewidth]{{#1}}
\end{center}
}

\def\twobytwoMat(#1, #2, #3, #4){
    {
        \begin{bmatrix}
            {#1} & {#2}\\
            {#3} & {#4}
        \end{bmatrix}
    }
}

\def\twobyoneMat(#1, #2){
    {
        \begin{bmatrix}
            {#1}\\
            {#2}
        \end{bmatrix}
    }
}

\def\twobytwoDet(#1, #2, #3, #4){
    {
        \begin{vmatrix}
            {#1} & {#2}\\
            {#3} & {#4}
        \end{vmatrix}
    }
}


\newcommand{\RR}{\mathbb{R}}
\newcommand{\CC}{\mathbb{C}}
\newcommand{\ZZ}{\mathbb{Z}}
\newcommand{\Zpos}{\mathbb{Z}_{pos}}
\newcommand{\NN}{\mathbb{N}}
\newcommand{\HH}{\mathcal{H}}

\newtheorem{theorem}{Theorem}
\newtheorem{prop}{Proposition}
\newtheorem{lemma}{Lemma}
\newtheorem{cor}{Corollary}
\newtheorem{remark}{Remark}
\newtheorem{definition}{Definition}
\newtheorem{ex}{Example}
\newtheorem{conj}{Conjecture}
\newtheorem{question}{Question}

\newcommand{\ch}{\text{ch}}

\numberwithin{equation}{section}

\begin{document}
\begin{center}
    \Large
    \textbf{Quantum Ladder Operators for Predator Prey Model}

    \large
    Benevolent Tomato
\end{center}

\setcounter{section}{-1}

\section{Preliminary}

\subsection {Setting up the space}

$B(\HH)$ is defined as the space of bounded operators in the Hilbert 
space $\HH$. $B(\HH)$ can be considered as a group representation of 
an abstract $C*$-algebra. A $C*$-algebra is a algebra that satisfies
\begin{equation}
    \|A\|^2 \ = \  \|A*A\| \hspace{5mm} \forall a \in B(\HH)
\end{equation}
Note that unbounded operators can be bounded by the exponential map. 
For example, suppose $X$ is an operator with unbounded operator norm. 
The following function maps $X$ to a bounded operator. 
\begin{equation}
    X \mapsto e^{iX}
\end{equation}

\subsection{Canonical Commutation Relation(CCR)}
We choose $2L$ operators from the space $B(\HH)$. 
\begin{equation}
    \{ \hat a_l, \hat a^\dag_l| \ l \in [L] \ 
    \} 
\end{equation}
Also, set this set of operators to satisfy CCR. 
\begin{definition}[CCR]
    The set of operators satisfy CCR if $\forall l, m \in [N]$
    \begin{enumerate}
        \item $[a_l, a^\dag_m] = \delta_{l, m} I$
        \item $[a_l, a_m] = [a^\dag_l, a^\dag_m] = 0$
    \end{enumerate}
    . This means the operators $a_l$ commute with each other and so does 
    $a^\dag_l$. Also, 
    \begin{equation}
    a_l a^\dag_l = a_l^\dag a_l + I 
    \end{equation}
    so pushing a $a_l$ to the right costs an additional identity matirx. 
    Moreover, if the indicies of the operators do not match, the just 
    commpute. 
\end{definition}
We also define two operators, $\hat n_l, \hat N$
\begin{eqnarray}
    \hat n_l \ = \ a^\dag_l a_l \nonumber\\ 
    \hat N \ = \ \sum_{l \in [N]} \hat n_l
\end{eqnarray}
Here is a motivating example. Suppose $\varphi_0$ is the vaccum whcih 
gets anhilated by any of the operator $a$. e.g. $a_1\varphi_0 = 0$. 
\begin{eqnarray}
    \hat n_1 (a^\dag_1)^3 \varphi_0\ = \ 
    (a^\dag_1 a_1) (a^\dag_1)^3\varphi_0 \ =\ 
    (a^\dag_1)(a^\dag_1 a_1 + I)(a^\dag _1)^2\varphi_0 \nonumber\\ 
    = \cdots = \ 3 (a_1^\dag)^3
\end{eqnarray}

We call the $a_l$ operators as the anhilation operator, and 
$a_l^\dag$ as the creation operator.\footnote{anhilator starts with 
an $a$ so the anti-anhilator is the creator}

It is possible to create an orthonormal set of basis in $\HH$ by the 
vaccum $\varphi_0$. 
\begin{equation}
    \varphi_{n_1, \dots, n_L} \ = \ 
    \frac {1} {\sqrt{
            n_1! \cdots n_L!
        }}
    (a_1^\dag)^{n_1}  \cdots
    (a_L^\dag)^{n_L} 
    \varphi_0 
\end{equation}
where $n_1, \dots, n_L \in \NN$. 


\end{document}