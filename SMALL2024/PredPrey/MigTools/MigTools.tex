\documentclass{article}
\usepackage{amsfonts}
\usepackage{amsthm}
\usepackage{amssymb}
\usepackage{amsmath}
\usepackage{graphicx}
\usepackage{subcaption}
\usepackage{xcolor}
\usepackage{mathtools}
\usepackage{ wasysym }
\usepackage{enumerate}
\usepackage{verbatim}


\newcommand{\new}[2]{
    \vspace{2mm}
    \noindent
    \textbf{
    \underline{#1}}
    \textit{{#2}}
    \
    \newline
}

\def\<{{\langle}}
\def\>{{\rangle}}

\DeclarePairedDelimiter\bra{\langle}{\rvert}
\DeclarePairedDelimiter\ket{\lvert}{\rangle}
\DeclarePairedDelimiterX\braket[2]{\langle}{\rangle}{#1\,\delimsize\vert\,\mathopen{}#2}


\newcommand{\textOr}{
    {
        \hspace{5mm}
        \textrm{or}
        \hspace{5mm}
    }
}

\newcommand{\textAnd}{
    {
        \hspace{5mm}
        \textrm{and}
        \hspace{5mm}
    }
}


\newcommand{\textWhere}{
    {
        \hspace{5mm}
        \textrm{where}
        \hspace{5mm}
    }
}



\newcommand{\Ixp}[1]{
    {
        e^{i{#1}}
    }
}



\newcommand{\halfFigure}[1]{
\begin{center}
\includegraphics[width = .5\linewidth]{{#1}}
\end{center}
}

\newcommand{\fullFigure}[1]{
\begin{center}
\includegraphics[width = .9\linewidth]{{#1}}
\end{center}
}

\def\twobytwoMat(#1, #2, #3, #4){
    {
        \begin{bmatrix}
            {#1} & {#2}\\
            {#3} & {#4}
        \end{bmatrix}
    }
}

\def\twobyoneMat(#1, #2){
    {
        \begin{bmatrix}
            {#1}\\
            {#2}
        \end{bmatrix}
    }
}

\def\twobytwoDet(#1, #2, #3, #4){
    {
        \begin{vmatrix}
            {#1} & {#2}\\
            {#3} & {#4}
        \end{vmatrix}
    }
}


\newcommand{\RR}{\mathbb{R}}
\newcommand{\CC}{\mathbb{C}}
\newcommand{\ZZ}{\mathbb{Z}}
\newcommand{\Zpos}{\mathbb{Z}_{pos}}
\newcommand{\NN}{\mathbb{N}}

\newtheorem{theorem}{Theorem}
\newtheorem{prop}{Proposition}
\newtheorem{lemma}{Lemma}
\newtheorem{cor}{Corollary}
\newtheorem{remark}{Remark}
\newtheorem{definition}{Definition}
\newtheorem{ex}{Example}
\newtheorem{conj}{Conjecture}
\newtheorem{question}{Question}

\newcommand{\ch}{\text{ch}}

\begin{document}
\begin{center}
    \Large
    \textbf{Title}

    \large
    Benevolent Tomato
\end{center}

\section{Abstract}
This paper presents some useful linear algebra tools 
when analyzing migration in terms of matricies. 
For simplicity, we assume the Leslie matricies 
of each population are either identical or a constant 
multiple of each other. Given a positive fertility rate, 
the leslie matrix is always invertible. All terms that 
show up in the difference equations are a memeber of the 
algebra formed by the leslie matrix, and they commute. 

We first present a cyclic migration model, then 
present some lemmas that simplify the computation. 

\section{The N-Migration Model}
We consider $N$ populations of the same species, 
each denoted by $p_1, p_2, \dots, p_N$. Each 
population $p_i$ has an influx from $p_{i - 1}$ 
and an outflux to $p_{i + 1}$. For convinience, we denote 
$p_0 = p_N$. Each population grows by a Leslie growth, and 
there is a constant infulx of population $M$ coming from some 
unknown source. 

\begin{definition}
    Define a square matrix $J$ of 
    order $N$ as follows. 
    \[
    (J)_{ij} \ := \ 
    \begin{cases}
        1 & i = j \\ 
        -1 & j - i = 1 \mod n
    \end{cases}
    \]
    For example, if the degree of the matrix is 4, 
    \[
    J = \begin{bmatrix}
        1 & -1 & 0 & 0\\
        0 & 1 & -1 & 0 \\
        0 & 0 & 1 & -1 \\ 
        -1 & 0 & 0 & 1
    \end{bmatrix}
    \]
\end{definition}


The population obeys the following difference equation. 

\[
\vec P^{(t)} = 
k^t J^t \vec P + \frac 
{(kJ)^t - I}{kJ - I} \vec M 
\]

\section{Linear Algebra Tools}
\begin{theorem}[Power of matrix $J$]
    Given $p < N $, we can explicitly compute each entry of $J^p$. 
    \[
    (J^p)_{ij} \ = \ (-1)^t \binom p t 
    \]
    The value $t$ is defined as 
        \[
        t \ = \ j - i \mod n
    \]
\end{theorem}

\begin{theorem}{Inverse of $J - kI$}
    Each entry of $(J - kI)^{-1}$ can be computed as a fraction 
    of binomial sums. 
    \[
    (J - kI)^{-1}_{ij} \ = \ 
    -\frac{
        \sum_{i = 0}^{t} \binom{t}{i}(-k)^i
    }{
        \sum_{i = 1}^{n} \binom{n}{i}(-k)^i
    } \ = \
    \frac {-(1 - k)^t} {1 - (1 - k)^n }
    \] 
\end{theorem}

\end{document}