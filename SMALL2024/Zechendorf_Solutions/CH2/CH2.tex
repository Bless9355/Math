\documentclass{article}
\usepackage{amsfonts}
\usepackage{amsthm}
\usepackage{amssymb}
\usepackage{amsmath}
\usepackage{graphicx}
\usepackage{subcaption}
\usepackage{xcolor}
\usepackage{mathtools}
\usepackage{ wasysym }
\usepackage{enumerate}
\usepackage{hyperref}


\newcommand{\new}[1]{
    \vspace{2mm}
    \noindent
    \textbf{
    \underline{#1}}
}

\def\calO{{\mathcal{O}}}
\def\th{{\theta}}
\def\_{{\hspace{1mm}}}
\def\<{{\langle}}
\def\>{{\rangle}}

\DeclarePairedDelimiter\bra{\langle}{\rvert}
\DeclarePairedDelimiter\ket{\lvert}{\rangle}
\DeclarePairedDelimiterX\braket[2]{\langle}{\rangle}{#1\,\delimsize\vert\,\mathopen{}#2}



\newcounter{problemcnt}
\setcounter{problemcnt}{0}

\newcommand{\Problem}{{
    \vspace{5mm}
    \stepcounter{problemcnt}
    \noindent
    \arabic{problemcnt}. 
}
}

\newcommand{\nProblem}[1]{
    \vspace{5mm}
    \noindent
    \setcounter{problemcnt}{#1}
    \arabic{problemcnt}. 
}


\newcommand{\Proof}{{
    \vspace{2mm}
    \noindent
    \textbf{
    \underline{Proof}}
}
}

\newcommand{\textOr}{
    {
        \hspace{5mm}
        \textrm{or}
        \hspace{5mm}
    }
}

\newcommand{\textAnd}{
    {
        \hspace{5mm}
        \textrm{and}
        \hspace{5mm}
    }
}


\newcommand{\textWhere}{
    {
        \hspace{5mm}
        \textrm{where}
        \hspace{5mm}
    }
}



\newcommand{\Ixp}[1]{
    {
        e^{i{#1}}
    }
}



\newcommand{\halfFigure}[1]{
\begin{center}
\includegraphics[width = .5\linewidth]{{#1}}
\end{center}
}

\newcommand{\fullFigure}[1]{
\begin{center}
\includegraphics[width = .9\linewidth]{{#1}}
\end{center}
}

\def\twobytwoMat(#1, #2, #3, #4){
    {
        \begin{bmatrix}
            {#1} & {#2}\\
            {#3} & {#4}
        \end{bmatrix}
    }
}

\def\twobyoneMat(#1, #2){
    {
        \begin{bmatrix}
            {#1}\\
            {#2}
        \end{bmatrix}
    }
}

\def\twobytwoDet(#1, #2, #3, #4){
    {
        \begin{vmatrix}
            {#1} & {#2}\\
            {#3} & {#4}
        \end{vmatrix}
    }
}


\newcommand{\RR}{\mathbb{R}}
\newcommand{\CC}{\mathbb{C}}

\begin{document}
\begin{center}
    \LARGE
    \textbf{Solutions for CH2}

    \large
    Daniel Son
\end{center}

\new{Ex 2.1}

For convergence, we present the \textbf{Hadamard's formula} 
that provides the radius of convergence for any 
generating function. Let $g_a(x)$ be the generating 
function of the sequence $a_n$. We define 
\[
\frac 1 R := \limsup_{n \rightarrow \infty} |a_n|^{1/n}
\]
. For $x < R$, the series converges, and for $x>R$ the 
series diverges. The behavior of the function at $x=R$ 
depends on the sequence. The convergence/divergence 
can be proved by comparing the partial sums to geometric 
series.
\footnote{To learn more, visit 
\url{https://en.wikipedia.org/wiki/Cauchy-Hadamard_theorem}}

Using the formula, we write out the radius of convergence 
for both $a_n, b_n$. 

\[
\frac 1 {R_a} = \limsup_{n \rightarrow \infty} |a^n|^{1/n} 
\textAnd 
\frac 1 {R_b} = \limsup_{n \rightarrow \infty} |n!|^{1/n} 
\]

To compute $R_b$, recall the Stirling's formula which 
provides an asymptotic estimation
\footnote{Asymptotic estimation means that the ratio 
of the estimate to the value converges to zero at the $n \rightarrow 
\infty$ limit. }
of $n!$. 

\[
    n! \sim \left(
        \frac n e
    \right)^n \sqrt{2\pi n}
\]

We finally present the values of $R_a, R_b$. 

\[
    R_a = 1/|a| \textAnd R_b = \lim_{n \rightarrow \infty}e/n = 0
\]

$R_b = 0$ means that $g_b(x)$ diverges for all values of nonzero $x$. 
\hfill \qed

\new{Ex2.2} 

Make use of Hadamard's formula. We wish to compute the following. 
\begin{equation} \label{22r}
\frac 1 R = \limsup_{n\rightarrow \infty} \binom{2n}{n} ^{1/n}
= 
\limsup_{n\rightarrow \infty}
\left(
    \frac {(2n)!}{(n!)^2}
\right)^{1/n}
\end{equation}

Using Stirling's formula, it is possible to write the multiplicative 
identity in two different ways. 


\[
    \lim_{n \rightarrow \infty} 
    \left(\frac {n!}{(n/e)^n \sqrt{2\pi n}}
    \right)^2
    = 
    \lim_{n \rightarrow \infty} \frac {(2n)!}{(2n/e)^{2n} \sqrt{4\pi n}}
    = 1
\]
\pagebreak

Multiply and divide by 1 in (\ref{22r}). We obtain the following. 

\[
    \frac 1 R 
    = 
    \limsup_{n \rightarrow \infty} 
    \left(
    \frac{
        (2n/e)^{2n} \sqrt{4\pi n}
    }{
        (n/e)^{2n} 2\pi n
    }
    \right)^{1/n}
    = 4
\]
Thus, the radius of convergence is 
\[
\boxed{
    R = \frac 1 4
}
\]. 

\new{Ex2.5}

Let $g_k(x)$ be the generating function of $a_n = n^k$
Take the derivative of $g_k(x)$ to derive the recurrence
\[
x(G_k(x))' = G_{k+1}(x)
\]

Notice that $g_0$ is the geometric series. Thus 
\[
    g_0(x) = \frac 1 {1 - x}
\]
Some computation shows that:
\begin{eqnarray*}
g_1(x) & = & \frac x {(1-x)^2}
\\
g_2(x) & = & \frac {(x +x^2)}{(1-x)^3}
\\
g_3(x) & = & \frac{(x + 4x^2 + x^3)}{(1-x)^4}
\end{eqnarray*}
\hfill \qed

\new{Ex2.10}

We will show the first identity and leave the proof 
of the second and third as an exercise to the reader. 
So the decimal expansion is created so that each digit 
is a fibbonacci number. Call the desired fraction $d$. 
We write the following. 

\[
    d = .\overline{F_1F_2F_3\cdots} = F_1 (.1) + F_2 (.1)^2 + F_3 (.1)^3 + \cdots 
\]

Write $d$ in sigma notation. 
\[
    d = \sum_{n = 0}^\infty F_n (.1)^n
\]

In fact, $d$ is the generating function of the fibbonacci sequence evaluated 
\linebreak at $x = .1$. 
\[
    d = g_f(.1) = \frac x {1 - x - x^2} \bigg|_{x = .1}
    = \frac {10} {89}
\]
\hfill \qed

\end{document}