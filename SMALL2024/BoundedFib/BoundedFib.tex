\documentclass{article}
\usepackage{amsfonts}
\usepackage{amsthm}
\usepackage{amssymb}
\usepackage{amsmath}
\usepackage{graphicx}
\usepackage{subcaption}
\usepackage{xcolor}
\usepackage{mathtools}
\usepackage{ wasysym }
\usepackage{enumerate}


\newcommand{\new}[1]{
    \vspace{2mm}
    \noindent
    \textbf{
    \underline{#1}}
}

\def\calO{{\mathcal{O}}}
\def\th{{\theta}}
\def\_{{\hspace{1mm}}}
\def\<{{\langle}}
\def\>{{\rangle}}

\DeclarePairedDelimiter\bra{\langle}{\rvert}
\DeclarePairedDelimiter\ket{\lvert}{\rangle}
\DeclarePairedDelimiterX\braket[2]{\langle}{\rangle}{#1\,\delimsize\vert\,\mathopen{}#2}



\newcounter{problemcnt}
\setcounter{problemcnt}{0}

\newcommand{\Problem}{{
    \vspace{5mm}
    \stepcounter{problemcnt}
    \noindent
    \arabic{problemcnt}. 
}
}

\newcommand{\nProblem}[1]{
    \vspace{5mm}
    \noindent
    \setcounter{problemcnt}{#1}
    \arabic{problemcnt}. 
}


\newcommand{\Proof}{{
    \vspace{2mm}
    \noindent
    \textbf{
    \underline{Proof}}
}
}

\newcommand{\textOr}{
    {
        \hspace{5mm}
        \textrm{or}
        \hspace{5mm}
    }
}

\newcommand{\textAnd}{
    {
        \hspace{5mm}
        \textrm{and}
        \hspace{5mm}
    }
}


\newcommand{\textWhere}{
    {
        \hspace{5mm}
        \textrm{where}
        \hspace{5mm}
    }
}



\newcommand{\Ixp}[1]{
    {
        e^{i{#1}}
    }
}



\newcommand{\halfFigure}[1]{
\begin{center}
\includegraphics[width = .5\linewidth]{{#1}}
\end{center}
}

\newcommand{\fullFigure}[1]{
\begin{center}
\includegraphics[width = .9\linewidth]{{#1}}
\end{center}
}

\def\twobytwoMat(#1, #2, #3, #4){
    {
        \begin{bmatrix}
            {#1} & {#2}\\
            {#3} & {#4}
        \end{bmatrix}
    }
}

\def\twobyoneMat(#1, #2){
    {
        \begin{bmatrix}
            {#1}\\
            {#2}
        \end{bmatrix}
    }
}

\def\twobytwoDet(#1, #2, #3, #4){
    {
        \begin{vmatrix}
            {#1} & {#2}\\
            {#3} & {#4}
        \end{vmatrix}
    }
}


\newcommand{\RR}{\mathbb{R}}
\newcommand{\CC}{\mathbb{C}}

\begin{document}
\begin{center}
    \Large
    \textbf{Recurrence relations with bounded behavior}
    \large
    Daniel Son
\end{center}

Let $x_n$ be a sequence of real numbers. 
The goal of this paper is to solve the following recurrence. 
\[
    x_n = x_{n - 1} + x_{n - 2} - x_{n -1} x_{n - 2} /M 
\]

The motivation for this recurrence is a model of a population with 
one species which asymptotically grows to a population cap $M$. 
It is natural to assume that $x_0, x_1 \ll M$. 
\footnote{
    To derive this recurrence, start from the fibbonacci sequence, 
    and multiply the growth term by an adjustment factor, 
    $\frac {M - x} M$
}

Upon massaging the equation, we can write the following. 
\begin{equation}
     \label{eqn:x_recurrence}
     M - x_n = 
     \frac{
        (M - x_{n - 1})(M - x_{n - 2})
     } M
\end{equation}

We define an auxillary sequence, $y_n$ as follows. 
\[
    y_n := \ln(M - x_n)
\]

The recurrence relation of $y$ can be written easily by 
taking the natural log of (\ref{eqn:x_recurrence}). Also for convinience, let 
$m := \ln M$

\begin{equation}
    \label{y_recurrence}
    y_n = y_{n - 1} + y_{n - 2} - m
\end{equation}

Which is similar to the fibbonacci recurrence. 
We transcribe this into a matrix relation, presenting the 
following proposition. 

\new{Proposition 1} 
The recurrence relation of (\ref{y_recurrence}) can be solved by the 
following matrix recurrence. 

\begin{equation}
    \label{matrix_recurrence}
    \twobyoneMat(y_n, y_{n - 1})
     = 
     \twobytwoMat(1, 1, 1, 0) \twobyoneMat(y_{n - 1}, y_{n - 2}) - 
     \twobyoneMat(m , 0)
\end{equation}
We further simplify the recurrence by introducing following notations. 
\[
    \vec y_n := \twobyoneMat(y_n , y_{n - 1}) 
    \textAnd 
    F := \twobytwoMat(1, 1, 1, 0)
\]
Then, more succintly, 
\begin{equation}
    \label{simple_matrix_recursion}
    \vec y_n = F \vec y_{n - 1} - m \twobyoneMat(1, 0)
\end{equation}

\new{Proposition 2}
The solution for (\ref{simple_matrix_recursion}) is 
\[
    \vec y_n := F^{n - 1} \vec y_1 - m \twobyoneMat(F_n - 1, F_{n - 1} - 1)
\]
for $n \geq 2$. 

\proof 
Trivially, the equation holds for the case when $n = 2$. 
Use induction to proceed. Using the inductive hypothesis, 
write out a solution for $y_{n}$ where $n\geq 2$. 

\[
    \vec y_{n} = F^{n - 1} \vec y_1 - m \twobyoneMat(F_n - 1, F_{n - 1} - 1)
\]

We wish to explicitly compute $y_{n + 1}$ using the recurrence relation. 
Write the following. 
\[
    \vec y_{n + 1} = F \vec y_n - m \twobyoneMat(1, 0) = 
    F^n \vec y_1 - m F \twobyoneMat(F_n - 1, F_{n - 1} - 1) - m \twobyoneMat(1, 0)
\]
\[
    = F^n \vec y_1 - m \twobyoneMat(
        F_n - 1 + F_{n - 1} - 1, F_n - 1
    ) - m \twobyoneMat(1, 0)
    = F^{n} \vec y_1 - m \twobyoneMat(F_{n + 1} - 1, F_{n} - 1)
\]
\hfill \qed

The matrix power $F^n$ can be expressed in terms of fibbonacci numbers. 
\begin{equation}
    \label{eqn:FibPower}
    F^n = \twobytwoMat(F_n, F_{n - 1}, F_{n - 1}, F_{n - 2})
\end{equation}

With some algebra, we present the following result. 

\new{Theorem 1} The closed form solution of $\vec y_n$ is 
\[
    \vec y_n = 
    \twobyoneMat(
        y_1F_{n - 1} + y_0F_{n - 2} - m F_n + m 
        ,
        y_1F_{n - 2} + y_0F_{n - 3} - m F_{n - 1} + m 
    )
\]
and this implies 
\begin{equation}
    \label{eqn: y_solution}
    y_n =  y_1F_{n - 1} + y_0F_{n - 2} - m F_n + m 
\end{equation}
\footnote{
    Proof is some tedious lines of algebra. I can email the solution 
    upon request. 
}

\new{Corollary} 
By taking the exponential of (\ref{eqn: y_solution}), it is 
possible to deduce the following. 
\[
    e^{y_n} = e^{y_1 F_{n - 1}} e^{y_0 F_{n - 2}} / e^{mF_n - m}
\]
Remember that 
\[
    e^{y_n} = M - x_n \textAnd e^m = M
\]
to conclude 
\[
    M - x_n = \frac {(M - x_1)^{F_{n - 1}} (M - x_0)^{F_{n - 2}} } 
    {M^{F_n - 1}}
    = M\frac {(M - x_1)^{F_{n - 1}} (M - x_2)^{F_{n - 2}} } 
    {M^{F_{n - 1}}M^{F_{n - 2}}}
\]
and graciously, 
\[
        M - x_n  = M\left(
            1 - \frac {x_1} M
        \right)^{F_{n - 1}}
        \left(
            1 - \frac {x_0} M 
        \right)^{F_{n - 2}}
\]. 
We present a closed form solution for $x_n$. 
\begin{equation}
    \label{eqn:x_solution}
    \boxed{
        x_n = M - M\left(
            1 - \frac {x_1} M
        \right)^{F_{n - 1}}
        \left(
            1 - \frac {x_0} M 
        \right)^{F_{n - 2}}
    }
\end{equation}

Here is a mathematica plot that 
shows that the recurrence is indeed bounded, and 
that the presented solution of the recurrence 
matches the computational result. 

\fullFigure{Plot.png}

\new{Numerical Approximations}

Recall from the beginning of the paper that 
it is natural to assume 
\linebreak
$x_0, x_1 \ll M$. In other words, 
$x_0/M, x_1/M \approx 0$. We can
rewrite our solution in (\ref{eqn:x_solution}) 
using the Taylor approximation. 
\[
    x_n 
    \approx 
    M - M \left(1 - F_{n - 1} \frac {x_1} M \right)\left(1 - F_{n - 2} \frac {x_0} M 
    \right)
    = 
    F_{n - 1} x_1 + F_{n - 2} x_0 - \frac{F_{n - 1} F_{n - 2} x_0 x_1} M 
\]

The equation can be even more cleared out under the initial condition 
\linebreak
$x_0 = x_1$. 
\begin{equation}
    \label{eqn:approx1}
  x_n \approx F_n x_0 - \frac{F_{n - 1}F_{n - 2} {x_0}^2} M
  = x_0\left(
    F_n - \frac{F_{n - 1} F_{n -2}} M x_0
  \right)
\end{equation}

From binet's formula, we can approximate the fibbonacci sequence 
to a decent accuracy. 
\[
    F_n \approx \frac {\varphi^n} {\sqrt 5}
    \textWhere \varphi := \frac {1 + \sqrt 5} 2
    \]

Back to our approximation in (\ref{eqn:approx1}), we write:
\[
    x_n \approx x_0 \left(
        \frac {\varphi^n} {\sqrt 5} - 
        \frac {\varphi^{2n - 3}} {5M} x_0
    \right)
    = \boxed{
        x_0 \frac {\varphi^n} {\sqrt 5} \left(
            1 - \frac {\varphi^{n - 3}} {\sqrt 5 M} x_0
        \right)
    }
\]

Here is a mathematica plot that compares the approximation 
with the exact values. The model performs decently until 
it reaches the maximum populatiion. After equilibrium, the model 
fails. 

\fullFigure{Approx.png}

\end{document}