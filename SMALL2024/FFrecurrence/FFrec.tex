\documentclass{article}
\usepackage{amsfonts}
\usepackage{amsthm}
\usepackage{amssymb}
\usepackage{amsmath}
\usepackage{graphicx}
\usepackage{subcaption}
\usepackage{xcolor}
\usepackage{mathtools}
\usepackage{ wasysym }
\usepackage{enumerate}


\newcommand{\new}[1]{
    \vspace{2mm}
    \noindent
    \textbf{
    \underline{#1}}
}

\def\calO{{\mathcal{O}}}
\def\th{{\theta}}
\def\_{{\hspace{1mm}}}
\def\<{{\langle}}
\def\>{{\rangle}}

\DeclarePairedDelimiter\bra{\langle}{\rvert}
\DeclarePairedDelimiter\ket{\lvert}{\rangle}
\DeclarePairedDelimiterX\braket[2]{\langle}{\rangle}{#1\,\delimsize\vert\,\mathopen{}#2}



\newcounter{problemcnt}
\setcounter{problemcnt}{0}

\newcommand{\Problem}{{
    \vspace{5mm}
    \stepcounter{problemcnt}
    \noindent
    \arabic{problemcnt}. 
}
}

\newcommand{\nProblem}[1]{
    \vspace{5mm}
    \noindent
    \setcounter{problemcnt}{#1}
    \arabic{problemcnt}. 
}


\newcommand{\Proof}{{
    \vspace{2mm}
    \noindent
    \textbf{
    \underline{Proof}}
}
}

\newcommand{\textOr}{
    {
        \hspace{5mm}
        \textrm{or}
        \hspace{5mm}
    }
}

\newcommand{\textAnd}{
    {
        \hspace{5mm}
        \textrm{and}
        \hspace{5mm}
    }
}


\newcommand{\textWhere}{
    {
        \hspace{5mm}
        \textrm{where}
        \hspace{5mm}
    }
}



\newcommand{\Ixp}[1]{
    {
        e^{i{#1}}
    }
}



\newcommand{\halfFigure}[1]{
\begin{center}
\includegraphics[width = .5\linewidth]{{#1}}
\end{center}
}

\newcommand{\fullFigure}[1]{
\begin{center}
\includegraphics[width = .9\linewidth]{{#1}}
\end{center}
}

\def\twobytwoMat(#1, #2, #3, #4){
    {
        \begin{bmatrix}
            {#1} & {#2}\\
            {#3} & {#4}
        \end{bmatrix}
    }
}

\def\twobyoneMat(#1, #2){
    {
        \begin{bmatrix}
            {#1}\\
            {#2}
        \end{bmatrix}
    }
}

\def\twobytwoDet(#1, #2, #3, #4){
    {
        \begin{vmatrix}
            {#1} & {#2}\\
            {#3} & {#4}
        \end{vmatrix}
    }
}


\newcommand{\RR}{\mathbb{R}}
\newcommand{\CC}{\mathbb{C}}

\begin{document}
\begin{center}
    \Large
    \textbf{Recursion on Finite Field Matricies}

    \large
    Daniel Son
\end{center}

From Lidi's result, we know that a recursive sequence over 
any finite field is periodic. We wish to stretch this result 
to matricies. 

\new{Proposition}
Let $F$ be a finite field. Consider a fixed depth recursive relation 
over matricies of dimension N by N. Denote the sequence as 
\[
    X_n := 
    \begin{bmatrix}
        x_n^{11}  && \cdots && x_n^{1N} \\ 
        \vdots && \ddots && \vdots \\ 
        x_n^{N1}  && \cdots && x_n^{NN} \\ 
    \end{bmatrix}
\]
and the depth of the recursion $d$. The recursive relation 
can be written as 
\[
    X_{n} = f(X_{n - 1}, X_{n - 2}, \dots , X_{n - d})
\]   
where $f$ is some polynomial involving $d$ constants. 

Any such finite recursion must be periodic. 

\proof 
We apply the pigeonhole principle. The $n$th matrix is 
entirely determined by the previous $d$ matricies, 
$X_{n -1}, \dots, X_{n - d}$. 
In other words, if the previous $d$ matricies agree, 
then the two matricies must agree.  
In symbols, if
\[
    (X_{m -1}, \dots, X_{m - d}) =  (X_{n -1}, \dots, X_{n - d})
\]
then,
\[
    X_m = X_n
\]. 
Call the tuple previous $d$ matricies that determine $X_n$ to be 
the \textbf{seed} of $X_n$. 
Since $F$ is a finite field, there are only finite number of different 
seeds. By the principle of multiplication, the number of possible seeds 
are 
\[
    |F|^{N^2 d}
\]  
because there are $|F|^{N^2}$ different N by N matricies over $F$ and 
$d$ of such matricies determine the seed. 

Though large, the number of different possible seeds are finite. 
This implies that in an infinite sequence, the series must be periodic.
\hfill \qed 

\new{Remark}
Note that this result applies only to recurrences with fixed depth. 
For example, the catalan recurrence does not fall under this umbrella, 
and its periodicity is undetermined by this result. 

\end{document}