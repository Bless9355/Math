\documentclass{article}
\usepackage{amsfonts}
\usepackage{amsthm}
\usepackage{amssymb}
\usepackage{amsmath}
\usepackage{graphicx}
\usepackage{subcaption}
\usepackage{xcolor}
\usepackage{mathtools}
\usepackage{ wasysym }
\usepackage{enumerate}
\usepackage{verbatim}


\newcommand{\new}[2]{
    \vspace{2mm}
    \noindent
    \textbf{
    \underline{#1}}
    \textit{{#2}}
    \
    \newline
}

\def\<{{\langle}}
\def\>{{\rangle}}

\DeclarePairedDelimiter\bra{\langle}{\rvert}
\DeclarePairedDelimiter\ket{\lvert}{\rangle}
\DeclarePairedDelimiterX\braket[2]{\langle}{\rangle}{#1\,\delimsize\vert\,\mathopen{}#2}


\newcommand{\textOr}{
    {
        \hspace{5mm}
        \textrm{or}
        \hspace{5mm}
    }
}

\newcommand{\textAnd}{
    {
        \hspace{5mm}
        \textrm{and}
        \hspace{5mm}
    }
}


\newcommand{\textWhere}{
    {
        \hspace{5mm}
        \textrm{where}
        \hspace{5mm}
    }
}



\newcommand{\Ixp}[1]{
    {
        e^{i{#1}}
    }
}



\newcommand{\halfFigure}[1]{
\begin{center}
\includegraphics[width = .5\linewidth]{{#1}}
\end{center}
}

\newcommand{\fullFigure}[1]{
\begin{center}
\includegraphics[width = .9\linewidth]{{#1}}
\end{center}
}

\def\twobytwoMat(#1, #2, #3, #4){
    {
        \begin{bmatrix}
            {#1} & {#2}\\
            {#3} & {#4}
        \end{bmatrix}
    }
}

\def\twobyoneMat(#1, #2){
    {
        \begin{bmatrix}
            {#1}\\
            {#2}
        \end{bmatrix}
    }
}

\def\twobytwoDet(#1, #2, #3, #4){
    {
        \begin{vmatrix}
            {#1} & {#2}\\
            {#3} & {#4}
        \end{vmatrix}
    }
}


\newcommand{\RR}{\mathbb{R}}
\newcommand{\CC}{\mathbb{C}}
\newcommand{\ZZ}{\mathbb{Z}}
\newcommand{\Zpos}{\mathbb{Z}_{pos}}
\newcommand{\NN}{\mathbb{N}}

\newcommand{\red}[1]{
    \color{red}
    {#1}
    \color{black}
}

\newtheorem{theorem}{Theorem}
\newtheorem{prop}{Proposition}
\newtheorem{lemma}{Lemma}
\newtheorem{cor}{Corollary}
\newtheorem{remark}{Remark}
\newtheorem{defn}{Definition}
\newtheorem{ex}{Example}
\newtheorem{conj}{Conjecture}
\newtheorem{question}{Question}

\newcommand{\ch}{\textnormal{ch}}
\newcommand{\Tr}{\textnormal{Tr}}
\newcommand{\PW}{\textnormal{PW}}
\newcommand{\E}{\mathbb{E}}


\begin{document}
\begin{center}
    \Large
    \textbf{Convergence of Anticommutator Spectral Densities}

    \large
    Benevolent Tomato
\end{center}

\begin{comment}
 \begin{eqnarray} \mu_{A_N}(x)dx & \ = \ & \frac{1}{N}
\sum_{i=1}^N
\delta\left(x - \frac{\lambda_i(A_N)}{\sqrt{N}}\right) \nonumber\\
M_m(A_N) & = & \mathbb{R} x^m \mu_{A_N}(x)dx \nonumber\\ M_m(N) & =
& \mathbb{E}[M_m(A_N)] \nonumber\\ M_m &\ =\ & \lim_{N \to \infty} M_m(N).
 \end{eqnarray}
\end{comment}

\begin{defn}[Dependency of Pairings]
    Let $\mathcal{P}_2[n \cdot 2k]$ to denote the set of all pairings of the cannonical set 
    $[n\cdot 2k]$. Consider $\pi \in \mathcal{P}_2[n\cdot 2k]$. Partition the set 
    $[n\cdot 2k]$ into $n$ blocks, namely 
    \begin{eqnarray*}
    B_1 \ = \ \{1, 2, \dots, 2k\} \\ 
    B_2 \ = \ \{2k +1 , 2k +2, \dots, 4k\} \\
    \vdots \\
    B_n \ = \ \{(n - 1)2k +1 , (n - 1)2k +2, \dots, n\cdot 2k\} \\ 
    \end{eqnarray*}
    A block $B_i$ is called to be dependent, if the image of $B_i$ 
    under the paring $\pi$ is not a subset of $B_i$. The dependency 
    of the pairing is the number of dependent blocks of a pairing. 

    For example, the pairing $\pi \in \mathcal{P}[8]$ 
    defined as 
    \[
    \pi \ = \ (12)(34)(58)(67)
    \]
    has two dependent blocks, namely $B_3, B_4$. Hence, 
    its dependency is 2. 
\end{defn}

From the works of \red{Hammond and Miller}, we employ the fourth moment 
method. With slight modification of the normalization coefficient, 
we establish the following. 

\begin{theorem}[Fourth Moment Method for Convergence]
    If, for any positive integer $m$
    \begin{equation}\label{eqn:ForthMom}
        \E \left[
            \frac 1 {N^{4m + 4}}
            \big|
            \Tr(AB + BA)^m - \E[\Tr(AB + BA)^m]
            \big|^4
        \right]
         \ = \ O\left(
                \frac 1 {N^2}
            \right)
    \end{equation}
    then spectral density of the anticommutator product converges 
    almost surely. 
\end{theorem}

We bound the lefthand side of (\ref{eqn:ForthMom}) appropriately. In order 
to do this, we first expand out the forth power by the binomial 
expansion. For convinience, introduce the 
following shorthand. 
\[
    M_j \ = \ \E\left[(\Tr(AB + BA)^m)^j\right]
\] 

\begin{prop}
    The lefthand side of (\ref{eqn:ForthMom}) can be rewritten as 
    \begin{equation} \label{eqn:binomial}
        \frac 1 {N^{4m + 4}}
        \left(
            M_{4} -4 M_{3} M_1 + 6 M_{2} M_1^2 -3 M_1^4
        \right)
    \end{equation}
\end{prop}
\begin{prop}
    Define $D_j$ to be the contribution from the summands of the trace 
    expansion that involves pairings of maximum dependency. In symbols, 
    \begin{equation}
        D_j \ = \
        \sum_{W \in PW(j\cdot 2m)} 
        \sum_{a_1, a_2, \dots, a_j}
        \sum_{
            \substack{
            \pi \in \mathcal{P}_2[j \cdot 2k]\\
            \pi \textnormal{ dependency j}
            }
        }
\E_\pi[W_{a_1s}W_{a_2s}\cdots W_{a_js}]
    \end{equation}
    \footnote{The subscript $s$ is an abuse of notation. See page 14 of 
    Hammond and Miller.}
    where $a_1, a_2, \dots, a_j$ are finite sequences of $2m$ integers 
    between $1$ and $N$. 
    $M_j$ can be rewritten as a sum involving pairings of different 
    dependencies. Namely, 
    \begin{eqnarray} \label{eqn:MjExpansions}
        M_2 \ = \ D_2 + M_1^2 \nonumber \\
        M_3 \ = \ D_3 + 3M_1D_2 + M_1^3 \nonumber \\
        M_4 \ = \ D_4 + 3D_2^2 + 6 D_2 M_1^2 + 4D_3M_1 + M_1^4 
    \end{eqnarray}
\end{prop}

\begin{proof}
    We show that the proposition must hold for $j = 2$, for the simplicity of 
    notation. For higher values of $j$, the proof is similar up to a slight modifictation, 
    and we provide a verbal reasoning without the symbolic manipulation.
    We start with the principal definition of $M_2$. 

    \begin{equation}
         M_2 \ = \ \E\left[(\Tr(AB + BA)^m)^2\right] 
         \ = \ 
            \sum_{W \in PW(2m)} \sum_{V \in PW(2m)}
            \E\left[\Tr(W)\Tr(V)\right]
    \end{equation}

    Introducing two finite sequences $a, b$ of $2m$ integers 
    between $1$ and $N$, we can further expand this equation. 
    For convinience, set $a_{2m + 1} = a_{2m}$ and \newline $b_{2m + 1} = b_{2m}$. 
    

    \begin{equation}
        M_2 \ =\ 
\sum_{W \in PW(2m)} \sum_{V \in PW(2m)}
        \sum_{a, b}
        \E\left[ W_{a_1a_2} W_{a_2a_3} \cdots W_{a_{2m}a_1}
            V_{b_1b_2} V_{b_2b_3} \cdots V_{b_{2m}b_1}
        \right]
    \end{equation}

For $W, V$ iterates through the set of all product words of length $2m$, 
We can consider the product $WV$ to iterate through all the product words 
of length $2 \cdot 2m$. Furthermore, all the random variables are Gaussian. 
Hence, we can use Wick's formula. We progress to the following equation. 

\begin{equation}
    M_2 \ = \ 
    \sum_{W \in \PW(4m)} \sum_{a, b} \sum_{\pi \in \mathcal{P}[4m]} 
    \E_\pi\left[
        \prod_{i = 1}^{2m} 
            W^{(i)}_{a_ia_{i + 1}}
        \prod_{i = 1}^{2m}
W^{(i + 2m)}_{b_ib_{i + 1}}
    \right] 
\end{equation}

Finally, split the sum of involving the pairings with respect to 
the dependency. For $j = 2$, $\mathcal{P}[4m]$ can either have a 
dependency of two or zero. 


\begin{eqnarray}
    \begin{split}
    M_2 \ = \ 
    \sum_{W \in \PW(4m)} \sum_{a, b} \sum_{\substack{\pi \in \mathcal{P}[4m]\\
    \pi \textnormal{ dependency 0}
    }} 
    \E_\pi\left[
        \prod_{i = 1}^{2m} 
            W^{(i)}_{a_ia_{i + 1}}
        \prod_{i = 1}^{2m}
W^{(i + 2m)}_{b_ib_{i + 1}}
    \right] \\
    + 
\sum_{W \in \PW(4m)} \sum_{a, b} \sum_{\substack{\pi \in \mathcal{P}[4m]\\
    \pi \textnormal{ dependency 2}
}} 
    \E_\pi\left[
        \prod_{i = 1}^{2m} 
            W^{(i)}_{a_ia_{i + 1}}
        \prod_{i = 1}^{2m}
W^{(i + 2m)}_{b_ib_{i + 1}}
    \right] 
    \end{split} \\ 
    \ = \ M_1^2 + D_2
\end{eqnarray}
The last equality follows from the nature of the paired expectations. 
\footnote{Refer to Mingo and Speicher CH1} If the pairing $\pi$ has zero dependency, it 
The pairied expectation of $\pi$ can be considered as products of 
the paired expectation of two independent blocks. The second summand 
is the principal definition of $D_2$

From the case of $j = 2$, we observe that the decomposition of $M_j$ 
is determined by the property of the pairings $\pi \in \mathcal P[j\cdot 2m]$. 
For $j = 3$, each pairings can be categorized as dependency 3 (completely dependent) 
dependency 2 (one independent block with two blocks depending on each other)
or dependency zero (all blocks independent). Upon inspection of the ways 
the blocks can be paired to each other, we conclude that there are 
only one configuration for complete dependence or independence. 
If the dependency is 2, choosing an independent block decides 
the other two dependent blocks, so there are 3 ways such pairings can occur. 

A similar argument can be carried out to the $j = 4$ case. 

\end{proof}

\begin{lemma}\label{thm:Dependencies}
    If 
    \begin{equation}\label{eqn:simpleForthMom}
        \frac 1 {N^{4m + 4}} (3D_2^2 + D_4) \ = \ O\left(
            \frac 1 {N^2}
        \right)
    \end{equation}
    then the spectral density of the anticommutator product converges almost surely. 
\end{lemma}

\begin{proof}
    Plug in equation (\ref{eqn:MjExpansions}) to equation (\ref{eqn:binomial}). 
    Simple algebra proves the result. 
\end{proof}

\begin{theorem}\label{thm:PTPTcvg}
    Suppose $A, B$ are random matricies drawn from Topelitz 
    ensembles. Then, equation (\ref{eqn:ForthMom}) indeed holds and 
    the spectral density converges almost surely. 
\end{theorem}

\begin{proof}
    It suffices to show 
    \begin{equation}
        D_2 \ = \ O(N^{2m + 1})
        \textAnd 
        D_4 \ = \ O(N^{4m + 1})
    \end{equation}
    Start with proving the first inequality. To compute $D_2$, we must consider 
    two finite sequences $a, b$. 
    Recall that the paired 
    expectation $\E_\pi$ is a product of the expected values in the form of 
    \begin{equation}
        \E[W_{i_si_{s + 1}}^{(s)} W_{i_{\pi(s)}i_{\pi(s) + 1}}^{(\pi(s))}] 
    \end{equation}
    where the sequence $i$ is either $a$ or $b$. 
    The modular restriction of the Topelitz ensenble dictates 
    that this term is $1$ if and only if 
    \begin{equation}
        i_s - i_{s + 1} \equiv i_{\pi(s)} - i_{\pi(s) + 1} \mod N
    \end{equation}
    and vanishes to zero otherwise. The anticommutator structure 
    imposes an additional restriction that the letter $W^{(s)}$ and 
    $W^{(\pi(s))}$ must both be $A$ or both be $B$'s in order for the expected 
    value to not vanish. 

    We wish to overcount the number of pairings that produces 
    a nonvanishing expectation. In order to do this, we choose 
    all the modular differences of $a, b$. Using the difference 
    notation, we note that there are $2m$ copies of $\Delta a$ 
    and $2m$ copies of $\Delta b$ to be chosen. By the nature 
    of the pairing, we observe that choosing one of the 
    differences decides another paired difference. We also have 
    the following restriction. 

    \begin{eqnarray} \label{eqn:ModRestr}
        \sum_{i = 1}^{2m} \Delta a_i \ = \ a_{2m + 1} - a_1 \ =\ 0 \nonumber \\
        \sum_{i = 1}^{2m} \Delta b_i \ = \ b_{2m + 1} - a_1 \ =\ 0 
    \end{eqnarray}

    The two equations takes away one degree of freedom from the original 
    $2m$ degrees of freedom from to choose the paired differences. The 
    fact that pairing $\pi$ is dependant accounts for the fact 
    that condition (\ref{eqn:ModRestr}) cannot be naturally met without losing a degree of freedom. 
    Also, we only lose one degree of freedom because the pairedness of the differences imply  
      \begin{eqnarray}
        \sum_{i = 1}^{2m} \Delta a_i+ \sum_{i = 1}^{2m} \Delta b_i \ =\ 0 
    \end{eqnarray}

    Finally, choosing $a_1, b_1$ determines both sequences $a, b$, and 
    this adds two degrees of freedom. Thus, there are $2m -1 + 2 = 2m + 1$ 
    degrees of freedom to choose $D_2$ and we conclude 
    \begin{equation}
        D_2 \ = \ O(N^{2m + 1})
    \end{equation}

    Similarly, for $D_4$, we lose three degrees of freedom by the dependence 
    for the four sequences, and recieve four degrees of freedom by the 
    choice of $a_1, b_2, c_1, d_1$, where the four sequence $a, b, c, d$ 
    show up in the sum index of $D_4$. The total degree of freedom is 
    $4m - 3 + 4 = 4m + 1$ which leads to the bound 
    \begin{equation}
        D_4 \ = \ O(N^{4m + 1})
    \end{equation}

\end{proof}

\begin{cor}
    Suppose $A, B$ are random matricies drawn either drawn from GOE, 
    Palindromic Toeplitz, or Block Circulant ensembles. Then, equation (\ref{eqn:ForthMom}) indeed holds and 
    the spectral density converges almost surely. 
\end{cor}

\begin{proof}
    The derivation of lemma \ref{thm:Dependencies} 
    does not involve the structure of individual matrix ensembles.
    All the three ensenbles listed in the corollary have either 
    equal or more strict modular restrictions for the paired expectations 
    to be nonvanishing, and therefore the value of $D_2, D_4$ are strictly 
    larger for the case of the anticommutator of different products 
    other than Topelitz anticommuted with Topelitz. Therefore the bounds in 
    theorem \ref{thm:PTPTcvg} carries out directly. 
\end{proof}

\end{document}