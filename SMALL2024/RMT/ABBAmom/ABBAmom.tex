\documentclass{article}
\usepackage{amsfonts}
\usepackage{amsthm}
\usepackage{amssymb}
\usepackage{amsmath}
\usepackage{graphicx}
\usepackage{subcaption}
\usepackage{xcolor}
\usepackage{mathtools}
\usepackage{ wasysym }
\usepackage{enumerate}
\usepackage{verbatim}


\newcommand{\new}[2]{
    \vspace{2mm}
    \noindent
    \textbf{
    \underline{#1}}
    \textit{{#2}}
    \
    \newline
}

\def\<{{\langle}}
\def\>{{\rangle}}

\DeclarePairedDelimiter\bra{\langle}{\rvert}
\DeclarePairedDelimiter\ket{\lvert}{\rangle}
\DeclarePairedDelimiterX\braket[2]{\langle}{\rangle}{#1\,\delimsize\vert\,\mathopen{}#2}


\newcommand{\textOr}{
    {
        \hspace{5mm}
        \textrm{or}
        \hspace{5mm}
    }
}

\newcommand{\textAnd}{
    {
        \hspace{5mm}
        \textrm{and}
        \hspace{5mm}
    }
}


\newcommand{\textWhere}{
    {
        \hspace{5mm}
        \textrm{where}
        \hspace{5mm}
    }
}



\newcommand{\Ixp}[1]{
    {
        e^{i{#1}}
    }
}


\newcommand{\tr}{
    {
        \textrm{tr}
    }
}





\newcommand{\halfFigure}[1]{
\begin{center}
\includegraphics[width = .5\linewidth]{{#1}}
\end{center}
}

\newcommand{\fullFigure}[1]{
\begin{center}
\includegraphics[width = .9\linewidth]{{#1}}
\end{center}
}

\def\twobytwoMat(#1, #2, #3, #4){
    {
        \begin{bmatrix}
            {#1} & {#2}\\
            {#3} & {#4}
        \end{bmatrix}
    }
}

\def\twobyoneMat(#1, #2){
    {
        \begin{bmatrix}
            {#1}\\
            {#2}
        \end{bmatrix}
    }
}

\def\twobytwoDet(#1, #2, #3, #4){
    {
        \begin{vmatrix}
            {#1} & {#2}\\
            {#3} & {#4}
        \end{vmatrix}
    }
}


\newcommand{\RR}{\mathbb{R}}
\newcommand{\CC}{\mathbb{C}}
\newcommand{\ZZ}{\mathbb{Z}}
\newcommand{\Zpos}{\mathbb{Z}_{pos}}
\newcommand{\NN}{\mathbb{N}}

\newtheorem{theorem}{Theorem}
\newtheorem{proposition}{Proposition}
\newtheorem{lemma}{Lemma}
\newtheorem{corollary}{Corollary}
\newtheorem{remark}{Remark}
\newtheorem{definition}{Definition}
\newtheorem{example}{Example}
\newtheorem{conjecture}{Conjecture}
\newtheorem{question}{Question}

\newcommand{\ch}{\text{ch}}

\begin{document}
\begin{center}
    \Large
    \textbf{Notes on higher moments of the anticommutator}

    \large
    Benevolent Tomato
\end{center}

%talk about how to compute trace of higher matrix powers
\section{Higher moments of a regular GOE}
Let $A$ be a square matrix of order $N$. 
We all know the following shorthand to compute trace. 
\[
    \tr(A^k) \ = \  \sum_{s} \prod_{i = 0}^{n - 1} a_{si, s(i+1)}
\]
Where $s_0, \cdots, s_{n-1}$ ranges over all finite sequences of length $k$ 
drawn from $[1, N]$. Also, for convinience, we let 
$s_n = s_0$.

Now, let $A$ to be drawn from a GOE. 
We also know that the following formula for the moment of a 
$N$-by-$N$ matrix ensenble. 
\[
\mu_k \ = \  \lim_{n \rightarrow \infty} \frac {
    \<\tr(A^k)\>
} {N^{k/2 + 1}}
\]
\footnote{denotes the expected value. That is, 
for a random variable $X$, $  \<X\>=\mathbb{E}[X] $ }

Consider the sequence $\{s\}$ as a set of verticies and 
the pair of indicies $s_i, s_{i+1}$ that occur in the trace 
expansion as edges. Each summand in the trace expansion 
corresponds to a closed walk. We have established the following. 
\begin{theorem} [Graph theoretical computation of moments]
    \label{thm:graph1}
    Let $S_k$ be the set of all closed walks of length $k$ over
    $N$ verticies. Then, the moment can be computed as follows. 
    \begin{equation}
\mu_k \ = \ \lim_{n \rightarrow \infty} \frac {\<
    \sum_{s\in S_k} \prod_{i = 0}^{n - 1} a_{si, s(i+1)}
    \>
} {N^{k/2 + 1}}
    \end{equation} 
\end{theorem}

By using the nature of expected values, it is not hard to derive 
the following Corollary. 

\begin{corollary} [Trees]
    Let $T_{2k}$ be the set of all traversals over a tree with
    \newline
    $k + 1$ verticies. Then, the moment of the 
    experimental density of a GOE can be computed as follows. 
    \begin{equation}
\mu_k \ = \  \begin{cases}\<
    \sum_{s\in T_k} \prod_{i = 0}^{n - 1} a_{si, s(i+1)}
    \>  & 2|k \\ 0 & 2 \nmid k
\end{cases}
    \end{equation} 
\end{corollary}

\section{Anticommutator of two GOEs}
Let $A, B$ be two matrices drawn from two GOEs that 
are asymptotically free. We wish to compute 
the moments of $AB + BA$. We wish to compute 
\begin{equation}\label{eqn:acTr}
\mu_k \ = \ \lim_{n \rightarrow \infty} \frac {
    \<\tr[(AB+BA)^k]\>
} {N^{k + 1}}
\ = \ \lim_{n \rightarrow \infty}
\sum_{P\in \mathcal{P}_k}
\frac{\<\tr(P)\>} 
{N^{k + 1}}
\end{equation}

\newcommand{\prodW}{\mathcal{P}}

The set $\prodW_k$ is the set of \textbf{Product Words} of length $2k$, that 
is, all strings of length $2k$ that are 
combinations of $AB$ and $BA$. For example, 
\[
    \mathcal{P}_2 = \{
    ABAB, ABBA, BAAB, BABA   
    \}
\]

\begin{definition} [Characteristic of product words]
Consider the product word $P \in \prodW_n$. The characteristic 
of the product word is denoted by $\chi(P)$, and it is the number 
of integers $0\leq i < 2n$ such that $P_i = P_{i + 1}$. For example, 
if $P = ABBA$
\[
    \chi(P) \ = \ 2
\]
since $P_2 = P_3$ and $P_4 = P_1$. 
\end{definition}

We wish to analyze the anticommutator trace in (\ref{eqn:acTr}) expansion using 
graph theory. In light of Theorem \ref{thm:graph1}, we construct a colored 
a graph for each product word. Before we move on, however, we provide some 
definitions to simplify our analysis. 
\begin{definition}[Cliff Edges]
    Consider $T_{2k}$, a tree traversal over a tree with $k + 1$ 
    verticies. We define a cliff edge to be an edge which the 
    traversal passes through and returns right after the passing. 
    Formally, it is the edge corresponding to $T_i T_{i+1}$ where 
    $T_{i + 2} = T_i$.
\end{definition}

%Some graphical explanation
It is possible to relate each summand in (\ref{eqn:acTr}) to 
a colored traversal of a tree. Below is an example. 
\halfFigure{ABBA.png}

We have a simple tree with three verticies, and the traversal 
consists of four directed edges. Each edge that corresponds 
to the matrix $A$ is colored blue, and the ones that correspond to 
$B$ black. We refer to the colored traversals corrresponding 
to the product words as \textbf{Matched Colored Traversals(MAT)}
For all the matrix entries have mean zero, we notice 
that each edge must be repeated twice. Using a degree 
of freedom argument, we verify that the two pair of edges 
must come from opposite directions\footnote{e.g. the summand 
can contain two occurences $A_{12}$ and $A_{21}$ but 
not $A_{12}$ and $A_{12}$. This is without applying symmetry condition 
of the GOE. }. Moreover, by the mean zero property of each matrix 
entry, each pair of directed edges must have a same color. We present 
the following observations. 


\begin{proposition}
    \label{thm:cliff}
    In a graph traversal, there necessarily exists a cliff edge. Moreover 
    in a MAT, there exists two cliff edges with different colors. 
\end{proposition}

\begin{theorem}
    Let $P\in \prodW_{k}$ be a product word that has a characteristic 
    less than $k$. Then 
    \[
    \lim_{n \rightarrow \infty} \frac{\<\tr(P)\>} 
{N^{k + 1}} \ = \  0
    \]
\end{theorem}

\begin{proof}
    We induct on $k$. For $k = 1$, the theorem follows trivially. 
    First, we write 
\[\lim_{n \rightarrow \infty} \frac{\<\tr(P)\>} 
{N^{k + 1}} \ = \  
    \sum_{s\in T_k} \<\prod_{i = 0}^{n - 1} P_i({s_i, s_{i+1}})
    \> 
    \]
    where $P_i$ is the $i$th character of the product word, and can 
    either be $A$ or $B$. The parantheses that follows denote 
    the entry of the matrix. For example, if $P_i = A$ then 
    $P_i({s_i, s_{i+1}}) = A_{s_i, s_{i+1}}$. 

    Consider each summand which correspond to a colored tree traversal. 
    In the tree traversal $T_k$, we know from proposition \ref{thm:cliff}
    that there must exist a cliff edge. If the traversal has 
    two directed edges with differing color for a cliff edge, the 
    entire term vanishes. 

    Otherwise, by the proposition, we exclude the two cliff edges 
    that have different colors to obtain a new traversal $\bar s \in T_{k - 1}$. 
    By the inductive hypothesis, the summand vanishes.  
\end{proof}

\begin{corollary}
    \label{thm:AABB}
    The odd moments of the anticommutator product of GOEs vanish. 
    Moreover, the even moments are dominated by two terms with the 
    maximum characteristics. In symbols, 
    \begin{equation}
        \mu_{2k + 1} = 0 
        \textAnd 
        \mu_{2k} = 2 \<\prod_{i = 0}^{n - 1} L^{(k)}_i({s_i, s_{i+1}})\>
    \end{equation}
    Where $L^{(k)}$ is an alternating combination of $AA$ and $BB$. For example, 
    $L^{(4)} \ = \ AABBAABB$\footnote{Note that $L^{(k)} \notin \prodW_k$. $L^{(k)}$ is derived 
    from the cyclicity of trace. }

\end{corollary}

\section{The Challenge}
We have narrowed down the moment computation from counting traces of $2^k$ product words 
to counting a single matrix product that has a simpler form. We wish to 
solve the following combinatorial problem. 

\begin{question}[The number of AABB-MATs]
    Consider $T_{2k}$ to be a MAT of any tree with $k + 1$ 
    verticies that correspond to the word $L^{(k)}$. How many traversals 
    $T_{2k}$ exist? 
\end{question}


\end{document}