\documentclass{article}
\usepackage{amsfonts}
\usepackage{amsthm}
\usepackage{amssymb}
\usepackage{amsmath}
\usepackage{graphicx}
\usepackage{subcaption}
\usepackage{xcolor}
\usepackage{mathtools}
\usepackage{ wasysym }
\usepackage{enumerate}


\newcommand{\new}[2]{
    \vspace{2mm}
    \noindent
    \textbf{
    \underline{#1}}
    \textit{{#2}}
    \vspace{1mm}
    \newline
}

\def\calO{{\mathcal{O}}}
\def\th{{\theta}}
\def\_{{\hspace{1mm}}}
\def\<{{\langle}}
\def\>{{\rangle}}

\DeclarePairedDelimiter\bra{\langle}{\rvert}
\DeclarePairedDelimiter\ket{\lvert}{\rangle}
\DeclarePairedDelimiterX\braket[2]{\langle}{\rangle}{#1\,\delimsize\vert\,\mathopen{}#2}



\newcounter{problemcnt}
\setcounter{problemcnt}{0}

\newcommand{\Problem}{{
    \vspace{5mm}
    \stepcounter{problemcnt}
    \noindent
    \arabic{problemcnt}. 
}
}

\newcommand{\nProblem}[1]{
    \vspace{5mm}
    \noindent
    \setcounter{problemcnt}{#1}
    \arabic{problemcnt}. 
}


\newcommand{\Proof}{{
    \vspace{2mm}
    \noindent
    \textbf{
    \underline{Proof}}
}
}

\newcommand{\textOr}{
    {
        \hspace{5mm}
        \textrm{or}
        \hspace{5mm}
    }
}

\newcommand{\textAnd}{
    {
        \hspace{5mm}
        \textrm{and}
        \hspace{5mm}
    }
}


\newcommand{\textWhere}{
    {
        \hspace{5mm}
        \textrm{where}
        \hspace{5mm}
    }
}



\newcommand{\Ixp}[1]{
    {
        e^{i{#1}}
    }
}



\newcommand{\halfFigure}[1]{
\begin{center}
\includegraphics[width = .5\linewidth]{{#1}}
\end{center}
}

\newcommand{\fullFigure}[1]{
\begin{center}
\includegraphics[width = .9\linewidth]{{#1}}
\end{center}
}

\def\twobytwoMat(#1, #2, #3, #4){
    {
        \begin{bmatrix}
            {#1} & {#2}\
            {#3} & {#4}
        \end{bmatrix}
    }
}

\def\twobyoneMat(#1, #2){
    {
        \begin{bmatrix}
            {#1}\
            {#2}
        \end{bmatrix}
    }
}

\def\twobytwoDet(#1, #2, #3, #4){
    {
        \begin{vmatrix}
            {#1} & {#2}\
            {#3} & {#4}
        \end{vmatrix}
    }
}


\newcommand{\RR}{\mathbb{R}}
\newcommand{\CC}{\mathbb{C}}

\begin{document}
\begin{center}
    \Large
    \textbf{RMT cheetsheet}

    \large
    Benevolent TomatoG
\end{center}

\new{Definition} {Wigner Matricies} 
\textit{These info are taken from feier's pdf}

Consider two probability distributions, $Y, Z$ with zero mean. 
Let $Z$ to have a variance of 1. Assume all the moments of 
these two distributions to be finite. 

These distributions form a random matrix ensenble. To construct 
a matrix from these two distributions, take the distribution $Y$ 
$N$ times and fill the diagonal entries. Then, fill up the 
upper diagonal by drawing from $Z$, $N(N-1)/2$ times. 

\new{Definition}{ESD and DSDs}
Consider a Wigner Matrix ensenble $\mathcal{M}_n$. The $n$ 
eigenvalues of any matrix in the ensenble form a \textbf{spectral density}. 
Let $M_n \in \mathcal M_n$ to have a ESD of $f(x)$. The ESD is written as follows. 
\[
    f(x) = \frac 1 n \sum_{\lambda \in \text{Spec}(M_n)} \delta(x - \frac \lambda {\sqrt n}) 
\]

The average of the ESDs provide a deterministic spectral density(DSD) of 
the entire ensenble. 

\new{Theorem}{Markov's Inequality}
This is a nice tool that bounds the probability of a random variable 
being "too large" by the expected value. \textbf{This only works when 
the random variable is positive!}
\[
    \frac{\<X\>} a \geq \mathcal{P}(x \geq a) 
\]

\new{Theorem}{Chebyshev's Inequality}
We use this inequality to bound the random variable to the mean. Unilike 
the Markov's inequality, the probability is bounded by the standard deviation. 
\[
    \frac 1 {k^2} \geq \mathcal{P}(|X - \mu| > k\sigma) 
\]

\new{Remark} {Strategy of using probability inequalities}
Take the ESD. Approximate it in some compact region, and compute the probability 
that the ESD is bounded to the desired distribution. Bound the probability using markov. 
Then, use a delta-epsilon argument to bound the probability. 

\newcommand{\Sp}{
    \text{Spec}
}

\newcommand{\tr}{
    \text{tr}
}

\new{Concept} {Stieltjes Transform}

Define the Stieltjes Transform by the following equation. 
\[
    s_n(z) := \int_{x \in \mathbb{R}} \frac{f(x)}{x-z}dx 
    = \int_{x \in \mathbb{R}} \frac{1}{x-z}d\mu 
\]
Where $\mu$ is the measure corresponding to the ESD of some random matrix. 

Stieltjes transforms are nice, for they can be written as an expression 
of the trace. 
\[
    s_n(z) = \tr(M_n/\sqrt{n} - zI)^{-1}
\]

\proof
We deduce some properties about traces. Consider a n-by-n matrix 
$A$ and its spectrum $\Sp(A)$. By the eigenvalue-trace lemma, we know 
\[
\sum_{\lambda \in \Sp(A)} \lambda = \tr(A)
\]
We easily deduce 
\[
\sum_{\lambda \in \Sp(A)} \frac 1 \lambda = \tr(A^{-1})
\textAnd 
\Sp(A - z I) = \{\lambda_i - z\}
\]
and infer 
\[
    \Sp[(M_n/\sqrt{n} - zI)^{-1}] = 
    \left\{
        \frac 1 {\lambda_i / \sqrt{n} - z}
    \right\}
\]
which implies 
\[
    \tr[(M_n/\sqrt{n} - zI)^{-1}] = \sum_{\lambda \in \Sp(M_n)}\frac 1 {\lambda / \sqrt{n} - z}
\]. 
Notice that the sum can be expressed as an integral using the ESD of 
$M_n$. Let $f(x)$ to be the ESD, and we can write the following. 
\[
    \tr[(M_n/\sqrt{n} - zI)^{-1}] = 
    \int_{x \in \mathbb{R}}\frac {f(x)} {x - z}dx := s_n(z)
\]. 
\hfill \qed

\newpage

\new{Theorem} {Moment of the GOE}
The odd moment of the GOE ensenble vanishes. The $2k$th moment 
of the GOE ensenble matches the kth Catalan number. 
\[
    C_k = \frac 1 {k + 1} \binom {2k} k
    \]

\proof 


The theorem holds by five observations. 
\begin{enumerate}
    \item Using the eigenvalue trace lemma, we can express 
    the trace of $A^{2k}$ as circuits/cycles of length 2k. 
    By the nature of expected values, all cycles that have 
    an unmatched edge can be ignored. This indicates that 
    all the odd moments vanish. 
    \item All cycles that have more than three kinds of the 
    same edges, regardless of direction, can be ignored. 
    Consider the expansion
    \[
        a_{i1i2}a_{i2i3} \dots a_{iki1}
    \]
    and simplify them to the form 
    \[
        a_{i1i2}^3a_{i2i3}^2 \dots a_{iki1}^2
    \]
    . These contribution dies out, since they have a total 
    of degree freedom $k$, hence a contribution of $N^{k}$. 
    \item All cycles that have arrows pointing in the same direction 
    dies out. All such cycles can be bijected to a partition of 
    $[k]$. The degree of freedom is also $k$, and this dies out. 
    \item So all circuits must be comprised of sets of edges 
    that travel in the different direction. Count this by the 
    nmber of Dych words and the Catalan numbers appear!
    \item Note that the double variance in the center entries 
    are countered in step 3. All pairs $a_{ii}^2$ can be 
    considered as two edges that connect to each other in different directions. 
\end{enumerate}
\hfill \qed

\end{document}