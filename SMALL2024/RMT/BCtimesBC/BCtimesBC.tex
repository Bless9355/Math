\documentclass{article}
\usepackage{amsfonts}
\usepackage{amsthm}
\usepackage{amssymb}
\usepackage{amsmath}
\usepackage{graphicx}
\usepackage{subcaption}
\usepackage{xcolor}
\usepackage{mathtools}
\usepackage{ wasysym }
\usepackage{enumerate}
\usepackage{verbatim}


\newcommand{\new}[2]{
    \vspace{2mm}
    \noindent
    \textbf{
    \underline{#1}}
    \textit{{#2}}
    \
    \newline
}

\def\<{{\langle}}
\def\>{{\rangle}}

\DeclarePairedDelimiter\bra{\langle}{\rvert}
\DeclarePairedDelimiter\ket{\lvert}{\rangle}
\DeclarePairedDelimiterX\braket[2]{\langle}{\rangle}{#1\,\delimsize\vert\,\mathopen{}#2}


\newcommand{\textOr}{
    {
        \hspace{5mm}
        \textrm{or}
        \hspace{5mm}
    }
}

\newcommand{\textAnd}{
    {
        \hspace{5mm}
        \textrm{and}
        \hspace{5mm}
    }
}


\newcommand{\textWhere}{
    {
        \hspace{5mm}
        \textrm{where}
        \hspace{5mm}
    }
}



\newcommand{\Ixp}[1]{
    {
        e^{i{#1}}
    }
}



\newcommand{\halfFigure}[1]{
\begin{center}
\includegraphics[width = .5\linewidth]{{#1}}
\end{center}
}

\newcommand{\fullFigure}[1]{
\begin{center}
\includegraphics[width = .9\linewidth]{{#1}}
\end{center}
}

\def\twobytwoMat(#1, #2, #3, #4){
    {
        \begin{bmatrix}
            {#1} & {#2}\\
            {#3} & {#4}
        \end{bmatrix}
    }
}

\def\twobyoneMat(#1, #2){
    {
        \begin{bmatrix}
            {#1}\\
            {#2}
        \end{bmatrix}
    }
}

\def\twobytwoDet(#1, #2, #3, #4){
    {
        \begin{vmatrix}
            {#1} & {#2}\\
            {#3} & {#4}
        \end{vmatrix}
    }
}


\newcommand{\RR}{\mathbb{R}}
\newcommand{\CC}{\mathbb{C}}
\newcommand{\ZZ}{\mathbb{Z}}
\newcommand{\Zpos}{\mathbb{Z}_{pos}}
\newcommand{\NN}{\mathbb{N}}

\newtheorem{theorem}{Theorem}
\newtheorem{prop}{Proposition}
\newtheorem{lemma}{Lemma}
\newtheorem{cor}{Corollary}
\newtheorem{remark}{Remark}
\newtheorem{definition}{Definition}
\newtheorem{ex}{Example}
\newtheorem{conj}{Conjecture}
\newtheorem{question}{Question}

\newcommand{\ch}{\text{ch}}

\begin{document}
\begin{center}
    \Large
    \textbf{Lower moments of Anticommutators with Block Circulant Ensembles}

    \large
    Benevolent Tomato
\end{center}


Before we move on to computing the lower moments, of the 
anticommutator products, we simplify the computation using the 
cyclicity of trace. 
Let \(A\) and \(B\) be square matrices. We observe the following. 

\begin{prop}[Trace Expansion for the second and the forth Moment]
The trace of the power of the anticommutator can be simplified as follows. 
\[
\text{Tr}[(AB + BA)^2]\ =\ 2(\text{Tr}(ABAB) + \text{Tr}(AABB))
\]
\[
\begin{aligned}
\text{Tr}[(AB + BA)^4] \ =\ & 2 \text{Tr}(ABABABAB) + 4 \text{Tr}(ABABABBA) + 2 \text{tr}(ABBAABBA) \\
& + 4 \text{Tr}(ABABBABA) + 4 \text{Tr}(ABBABABA)
\end{aligned}
\]
\end{prop}
\begin{proof}
We demostrate for the second power and leave the proof for the forth 
power as an exercise. 

First, we expand \((AB + BA)^2\):
\[
(AB + BA)^2 \ =\  ABAB + ABBA + BAAB + BABA
\]

Using the cyclic property of the trace, \(\text{Tr}(XY) = \text{Tr}(YX)\), we have:
\[
\text{Tr}(ABBA) \ =\ \text{Tr}(AABB) \quad \text{and} \quad \text{Tr}(BAAB) = \text{Tr}(AABB)
\]

And thus 
\[
\text{Tr}[(AB + BA)^2] \ =\ 2(\text{Tr}(ABAB) + \text{Tr}(AABB))
\]. 

\end{proof}

We proceed with computing the second moment of the 
anticommutator product of an anticommutator matrix 
and a GOE. Let $A$ be a GOE and $B$ be a $m$-circulant matrix. 
Also, set the order of both matricies to be $N$. 
Let $\mu_N$ denote the spectral density of the anticommutator product 
$AB + BA$ and $\mu_N^{(k)}$ the $k$th moment. 
Using 
the eigenvalue trace lemma, we obtain the following. 

\begin{equation}
    \mu_N^{(k)} \ = \  \frac 1 {N^{k + 1}} \mathbb{E}(Tr[(AB+BA)^k])
\end{equation}

\begin{theorem}[2nd and 4th moment of GOE times Block Circulant]
    \[
        \mu_N^{(2)} \ =\ 2 \textAnd 
        \mu_N^{(4)} \ =\ 10 + \frac 2 {m^2}
    \]
\end{theorem}

\begin{proof}
    Start with the second moment. 
    We use the eigenvalue trace lemma along with the trace expansion 
    for $k = 2$. Also note that the expected value is linear. 
\begin{equation}
    \label{eqn:secondGOEBC}
    \mu_N^{(2)} \ = \ \frac 1 {N^{3}} \mathbb{E}(Tr(ABAB)) + \frac 1 {N^3}\mathbb{E}(Tr(AABB))
\end{equation}
We compute each of the summands independantly. Focus on the first 
summand, and use Wick's formula to rewrite the summand in tractable 
form. \footnote{We adopt the notion from the free probability book}
\begin{equation}
    \frac 1 {N^{3}} \mathbb{E}(Tr(ABAB)) \
    =  \
    \frac 1 {N^3} 
    \sum_{1 \leq i_1, i_2, i_3, i_4 \leq N} 
    \sum_{\pi \in \mathcal{P}[4]}
    \mathbb{E}_\pi (
        A_{i_1i_2}B_{i_2i_3}A_{i_3i_4}B_{i_4i_1}
    )
\end{equation}
It is trivial that the pairings that match $A$'s with $B$'s 
vanish, for the two matricies $A$, $B$ are assumed to be indepent. 
Thus, the permutation $\pi$ must be 
\[
    \pi \ = \ (13)(24)
\]
and the double sum simplifies to 
\begin{equation}
    \frac 1 {N^{3}} \mathbb{E}(Tr(ABAB)) \
    =  \
    \frac 1 {N^3} 
    \sum_{1 \leq i_1, i_2, i_3, i_4 \leq N}
    \mathbb{E} (
        A_{i_1i_2}A_{i_3i_4}
    )
\mathbb{E} (
        B_{i_2i_3}B_{i_4i_1}
    )
\end{equation}
Since $A$ is a GOE and $B$ is a block circulant matrix, the indicies $i$ must 
satisfy the following condition. 

\begin{eqnarray}
    i_1 \ = \ i_4 \textAnd i_2 \ = \ i_3 \\
    i_2 - i_3 \ \equiv \ i_4 - i_1 \mod N \\
    i_2 \ \equiv \ i_1 \mod m
\end{eqnarray}

Notice that the choice of $i_1, i_2$ determines both $i_3, i_4$. Hence, 
there are a maximum $N^2$ sequences of $i$'s where the expected value 
is nonvanishing. So as $N \rightarrow \infty$, 

\begin{equation}
    \frac 1 {N^{3}} \mathbb{E}(Tr(ABAB)) \
    =  \ 0
\end{equation}

Repeat the procedure for $ABAB$. 
\begin{equation}
    \frac 1 {N^{3}} \mathbb{E}(Tr(AABB)) \
    =  \
    \frac 1 {N^3} 
    \sum_{1 \leq i_1, i_2, i_3, i_4 \leq N}
    \mathbb{E} (
        A_{i_1i_2}A_{i_2i_3}
    )
\mathbb{E} (
        B_{i_3i_4}B_{i_4i_1}
    )
\end{equation}
For the expected value to be nonvanishing, the sequence $i$ must satisfy 
\begin{eqnarray}
    i_1 \ = \ i_3 \textAnd i_2 \ \  \textrm{free} \\
    i_3 - i_4 \ \equiv \ i_1 - i_4 \mod N \\
    i_3 \ \equiv \ i_1 \mod m
\end{eqnarray}
The conditions simplify to $i_1 = i_3$ and other variables are free. 
Thus, there are $N^3$ sequences of $i$ where the expected value is 
nonvanishing. In the limit $N\rightarrow \infty$, 
\begin{equation}
    \frac 1 {N^{3}} \mathbb{E}(Tr(AABB)) \
    =  \ 1
\end{equation}

Finally, from (\ref{eqn:secondGOEBC}), 
\[
    \mu_N^{(2)} \ = \ 2 (0 + 1) \ = \ 2
\]

As for the forth moment, we notice that there are five summands in the 
trace expansion. However, by a degree of freedom argument, the pairings 
which have a crossings of $A$'s vanish. Hence, we deduce 
\begin{equation}
    \label{eqn:forthGOEBC}
    \mu_N^{(4)} \ = \ \frac 2 {N^{5}} \mathbb{E}(Tr(ABBAABBA)) + \frac 4 {N^5}\mathbb{E}(Tr(ABABBABA))
\end{equation}

Focus on the first summand. Use Wick's formula and rewrite as the following. 

\begin{equation}
    \begin{split}
\frac{2}{N^5} \sum_{1 \leq i_1, \dots,  i_8 \leq N} 
\sum_{\pi \in \mathcal{P}[8]} \mathbb{E}_\pi \left( A_{i_1i_2} B_{i_2i_3} B_{i_3i_4} A_{i_4i_5} A_{i_5i_6} B_{i_6i_7} B_{i_7i_8} A_{i_8i_1} \right)
    \end{split}
\end{equation}



With some brute-force condition checking, it possible to verify that 
any pairings that have a crossing with $A$'s do not contribute to the sum. 
So, the following two pairings have zero contribution as $N\rightarrow \infty$
\begin{eqnarray}
    (15)(23)(48)(67) \\ 
    (14)(27)(36)(58)
\end{eqnarray}
The first permutation has a crossing $(15)(48)$ where both transposition 
pair two $A$'s. For the first permutation, the crossing is $(14)(27)$ and 
the first transposition pairs two $A$'s while the second pairs two $B$'s. 

Note that the crossings between pairings of $B$'s do contribute to the sum. 
To demonstrate the fact, we compute the contribution of the pairing 
\[
    \pi \ = \ (18)(26)(37)(45)
\] which is 
\begin{equation}
    \begin{split}
\frac{2}{N^5} \sum_{1 \leq i_1, \dots,  i_8 \leq N} 
\mathbb{E} \left( A_{i_1i_2} A_{i_8i_1}\right)
\mathbb{E}
\left(B_{i_2i_3} B_{i_6i_7}\right) 
\mathbb{E}
\left(B_{i_3i_4} B_{i_7i_8}  \right)
\mathbb{E}
\left(
A_{i_4i_5} A_{i_5i_6}  
\right)
    \end{split}
\end{equation}
We wish to count the number of finite sequences $i$ of length 
8 that satisfies the conditions below. 
\begin{eqnarray}
i_2 \ = \ i_8 \\
i_2 - i_3 \ \equiv \ i_7 - i_6 \mod N\\ 
i_3 - i_4 \ \equiv \ i_8 - i_7 \mod N\\
i_4 \ = \ i_6 \\ 
i_2 \ \equiv \ i_7, i_3 \ \equiv \ i_6 \mod m \\ 
i_3 \ \equiv \ i_8, i_4 \ \equiv \ i_7 \mod m
\end{eqnarray}

Determine the residue of $i$'s by mod $m$ first. Notice that $i_1, i_5$ 
are free to be any value mod $m$, and all other values must be congruent 
to each other mod $m$. As for the value $\lfloor i/m\rfloor$, we determine that t
here are five degrees of freedom where the $i's$ split into the following 
equivalence classes. 
\[
\{
    i_2, i_8\}, \{i_4, i_6\}, \{i_3\}, \{i_5\}, \{i_1\}
\]
The index $i_7$ is determined by the conditions. 
Thus, there are 3 degrees of freedom to choose $i \mod m$ and 
5 degrees of freedom for $\lfloor i/m \rfloor$. The total contribution 
in the limit is 
\[
    \frac 1 {N^5}m^3 \left(
        \frac N m
    \right)^5 = \frac 1 {m^2}
\]  

If there are no crossings in the pairings, the contribution equals exactly one. 
Thus, by (\ref{eqn:forthGOEBC}), we write 
\[
    \mu_N^{(4)} \ = \ 2 \left(3 + \frac 1{m^2}\right) + 4(1) \ = \ 10 + \frac 2 {m^2}
\]

\end{proof}

\begin{theorem}[2nd moment of Block Circulant times Block Circulant]
    \[
        \mu_N^{(2)} \ =\ 2 + \frac 2 {m^2} 
    \]
\end{theorem}

The proof is similar to the case of GOE times Block Circulant. 

\end{document}
