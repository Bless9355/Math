\documentclass{article}
\usepackage{amsfonts}
\usepackage{amsthm}
\usepackage{amssymb}
\usepackage{amsmath}
\usepackage{graphicx}
\usepackage{subcaption}
\usepackage{xcolor}
\usepackage{mathtools}
\usepackage{ wasysym }
\usepackage{enumerate}
\usepackage{verbatim}
\usepackage{bbold}
\usepackage{gensymb}
\usepackage{array}


\newcommand{\new}[2]{
    \vspace{2mm}
    \noindent
    \textbf{
    \underline{#1}}
    \textit{{#2}}
    \
    \newline
}

\def\<{{\langle}}
\def\>{{\rangle}}

\DeclarePairedDelimiter\bra{\langle}{\rvert}
\DeclarePairedDelimiter\ket{\lvert}{\rangle}
\DeclarePairedDelimiterX\braket[2]{\langle}{\rangle}{#1\,\delimsize\vert\,\mathopen{}#2}


\newcommand{\textOr}{
    {
        \hspace{5mm}
        \textrm{or}
        \hspace{5mm}
    }
}

\newcommand{\textAnd}{
    {
        \hspace{5mm}
        \textrm{and}
        \hspace{5mm}
    }
}


\newcommand{\textWhere}{
    {
        \hspace{5mm}
        \textrm{where}
        \hspace{5mm}
    }
}



\newcommand{\Ixp}[1]{
    {
        e^{i{#1}}
    }
}



\newcommand{\halfFigure}[1]{
\begin{center}
\includegraphics[width = .5\linewidth]{{#1}}
\end{center}
}

\newcommand{\fullFigure}[1]{
\begin{center}
\includegraphics[width = .9\linewidth]{{#1}}
\end{center}
}

\def\twobytwoMat(#1, #2, #3, #4){
    {
        \begin{bmatrix}
            {#1} & {#2}\\
            {#3} & {#4}
        \end{bmatrix}
    }
}

\def\twobyoneMat(#1, #2){
    {
        \begin{bmatrix}
            {#1}\\
            {#2}
        \end{bmatrix}
    }
}

\def\twobytwoDet(#1, #2, #3, #4){
    {
        \begin{vmatrix}
            {#1} & {#2}\\
            {#3} & {#4}
        \end{vmatrix}
    }
}


\newcommand{\RR}{\mathbb{R}}
\newcommand{\CC}{\mathbb{C}}
\newcommand{\ZZ}{\mathbb{Z}}
\newcommand{\Zpos}{\mathbb{Z}_{pos}}
\newcommand{\NN}{\mathbb{N}}


\newcommand{\PW}{\text{PW}}
\newcommand{\NC}{\text{NC}}

\newtheorem{theorem}{Theorem}
\newtheorem{prop}{Proposition}
\newtheorem{lemma}{Lemma}
\newtheorem{cor}{Corollary}
\newtheorem{remark}{Remark}
\newtheorem{definition}{Definition}
\newtheorem{ex}{Example}
\newtheorem{conj}{Conjecture}
\newtheorem{question}{Question}

\newcommand{\ch}{\text{ch}}

\begin{document}
\begin{center}
    \Large
    \textbf{The General Moment of Anticommutator 
    Products involving Block Circulant Matricies}

    \large
    Benevolent Tomato
\end{center}

%====Prelims=====
%The modular restrictions

\begin{definition} [Equivalence relation $\approx$ and $\simeq$]

    $(i, j) \approx (i', j')$ if and only if 
    \begin{equation}
        i \ = \ j' \textAnd j \ = \ i'
    \end{equation}

    Also, 
$(i, j) \simeq (i', j')$ if and only if 
    \begin{eqnarray}
        i - j \ \equiv \ j' - i \mod N 
    \\
       i \ \equiv \ j' \textAnd j \ \equiv \ i' \mod m
    \end{eqnarray}
    The value of $N, m$ are implied from context. 
\end{definition}

\begin{definition}[Product Words]
A \textit{product word} of length \(2k\) is composed of \(k\) blocks, where each block is one of \(\{AB, BA\}\). 
We denote the set of all product words of length $2k$ as 
$\PW(2k)$.


For example, when \(k = 3\),
\[
W \ = \ AB \, BA \, AB \, BA \in \PW(8)
\]
is an example of a product word of length 6. To 
refer to the specific index of the word, use the superscript. 
For example, $W^3 \ =\ B$. 

\end{definition}

\begin{definition}[Combining pairings]
    Suppose we are given $W \in PW(4k)$ and two pairings 
    $\pi, \delta \in \mathcal{P}[2k]$. We denote the 
    combined pairing of $\pi, \delta$ with respect 
    to the product word $W$ as 
    \[
        \pi *_W \delta
    \]
    where the combined pairing denotes an element in $\mathcal{P}[4k]$ 
    where the composition between $A$'s are specified by $\pi$ and 
    composition between $B$'s are specified by $\delta$. 

    For example if 
    \[
    \pi \ = \ (1 2) (3 4) \textAnd 
    \delta \ = \ (12) (34)
    \]
    the combined pairing is 
    \[
    \pi *_W \delta \ = \ (1 4)(2 3)(5 8)(67)
    \]
\end{definition}
    

%====For GOE x BC====

We wish to compute $\mu_N^{(2k)}$, the $2k^{th}$ moment of 
the anticommutator product of ensemble $A$ which is a GOE 
and ensemble $B$ which is a m-block circulant matrix, 
where both $A, B$ are of order $N$. It is straightforward to 
verify the following. 

\begin{prop}[Even moment as product words]
    \label{thm:baseForm}
    \begin{equation}
        \mu_N^{(2k)} \ = \ 
        \sum_{W\in \PW(2k)}
        \sum_{1 \leq i_1 , \dots, i_{4k} \leq N}
        \sum_{\pi \in \mathcal P[2k]}  
        \sum_{\delta \in \NC(2k)}  
        \mathbb{E}_{(\pi*_W\delta)}\left(
        \prod_{l = 1}^{4k} W^{l}_{i_l i_{l + 1}}
        \right) \mathbb{1}_{(\pi*_W\delta)}
    \end{equation}
\end{prop}

\begin{theorem}[GOE times Block Circulant]
    \label{thm: GOEBC}
    \begin{equation}
        \mu_N^{(2k)} \ = \ 
        \sum_{W \in \PW(2k)}
        \sum_{\pi \in \mathcal P[2k]}  
        \sum_{\delta \in \NC(2k)}   
        m^{
            \#((\pi *_W \delta) \circ \gamma_{2k} )
        }
        \left(
            \frac 1 m
        \right)^{2k + 1}
        \mathbb{1}_{(\pi*_w\delta)}
    \end{equation}
\end{theorem}
\footnote{
    $\gamma_n$ denotes a permutation of the cannonical set $[n]$ 
    where $\gamma_n(x) = x + 1 \mod n$. 
}

\begin{theorem}[Block Circulant times Block Circulant]
    \label{thm:BCBC}
    \begin{equation}
        \mu_N^{(2k)} \ = \ 
        \sum_{W \in \PW(2k)}
        \sum_{\pi \in \mathcal P[2k]}  
        \sum_{\delta \in  \mathcal P[2k]}   
        m^{
            \#((\pi *_W \delta) \circ \gamma_{2k} )
        }
        \left(
            \frac 1 m
        \right)^{2k + 1}
    \end{equation}
\end{theorem}

To prove these two theorems, we need to establish 
the following propositions. 

\begin{prop} [Rules for pairing] \label{thm:pairingRule}
    For a pairing of each word, each of the compositions 
    must match $A$'s to $A$'s and $B$'s to $B$'s. Moreover, 
    the index configurations are confirmed once the 
    two matricies are matched. That is, 
    if $A_{i_s, i_{s + 1}}$ is matched with 
    $A_{i_t, i_{t + 1}}$, then 
    \begin{equation}
        (i_s, i_{s + 1}) \ \approx \ (i_t, i_{t + 1})
    \end{equation}. Also, 
    if $B_{i_s, i_{s + 1}}$ is matched with 
    $B_{i_t, i_{t + 1}}$, then 
    \begin{equation}
        (i_s, i_{s + 1}) \ \simeq \  (i_t, i_{t + 1})
    \end{equation}
\end{prop}

\begin{proof}
    Introduce the signed variable $\epsilon_j$. Adding 
    up the signed difference allows us to find 
    that if any one of the signs are nonzero, the 
    degree of freedom reduces. 
\end{proof}

\begin{prop} [GOE pairing rules]
    Let $A$ be the GOE ensemble in the anticommutator. 
    The compositions of $A$ must not cross with any other 
    compositions. The compositions between Block 
    Circulant matricies can match, and the crossings 
    do not reduce the degree of freedom. 
\end{prop}
\begin{proof}
    The compositions of $A$'s in each pairing slices 
    the entire word. For example, consider the product word 
    \[
        W \ = \ ABBABAAB
    \]
    where the pairing is given as 
    \[
        \pi \ = \ (1 4)(2 3)(5 8) (6 7)
    \]
    The composition $(14)$ slices the word into 
    \[
        W_1 \ = \ BB 
        \textAnd 
        W_2' \ = \ BAAB 
    \]
    where each word is extracted from between ($W_1$) and 
    outside ($W_2$) the composition $W^1 = W^4 = A$. Call 
    this composition of $A$'s as the slicing composition.
    Furthermore, the slicing composition $(58)$ slices $W_2'$ 
    into another word. 
    \[
        W_2 \ = \ BB
    \]

    The observation has two implications. The first implication 
    is that any transposition that crosses with the slicing 
    composition reduces a degree of freedom. Hence, crossings 
    with slicing composition, which can be any transposition between $A$'s, 
    result in a vanishing contribution. 

    The second implication is that any pairing where the 
    slicing compositions do not cross with other compositions 
    always have a positive nonzero contribution. After reducing the entire 
    product word according to all its slicing compositions, we 
    we are left with finite number of sub-words that are comprised solely 
    of $B$'s. For the word $W$, the remaining words are $W_1, W_2$. 

    Composition between $B$'s lose one degree of freedom, regardless of 
    crossings. So these always have a contribution.
\end{proof}

Finally, we present a proof of theorem \ref{thm: GOEBC}.
\begin{proof} 
    From proposition \ref{thm:baseForm}, we recognize that 
    it suffices to count the number of integer sequences 
    $i_1, \dots, i_{4k}$ that satisfy the pairing restrictions. 
    Fix a pairing $\pi$ that pairs all the GOE $A$'s and a
    pairing $\delta$  that pairs the Block Circulnat $B$'s. 
    From 
    We first configure the modular residue of $i$'s mod $m$. 
    Clearly, by proposition \ref{thm:pairingRule}, the number of such configurations are \footnote{More details to be added 
    from BC paper and FP paper. $\#(\pi)$ denotes the number of orbits of the permutation $\pi$} 
    \[
    m^{
        \# ((\pi *_W \delta)\circ \gamma_{2k})
    }
    \]
    Move on to 
    choose the value of $\lfloor i / m \rfloor$. 
    We know that as long as the slicing compositions of $A$ do 
    not cross with other compositions, the degree of freedom 
    is not reduced. Otherwise, the contribution is can be ignored 
    at the limit $N \rightarrow \infty$. Thus, the ways to configure 
$\lfloor i / m \rfloor$ is 
\begin{equation}
        \left(
            \frac 1 m
        \right)^{2k + 1}
        \mathbb{1}_{(\pi*_w\delta)}
\end{equation}
where $\mathbb{1}_{(\pi*_w\delta)}$ is defined to be $1$ if and only if 
the pairing $(\pi*_w\delta)$ is non-crossing in the sense of proposition 
\ref{thm:pairingRule} and zero otherwise. 

The variance of all the random variables involved in the matricies are 
fixed to be $1$. Thus, from proposition \ref{thm:baseForm}, we obtain 
\label{thm: GOEBC}
    \begin{equation}
        \mu_N^{(2k)} \ = \ 
        \sum_{W \in \PW(2k)}
        \sum_{\pi \in \mathcal P[2k]}  
        \sum_{\delta \in \NC(2k)}   
        m^{
            \#((\pi *_W \delta) \circ \gamma_{2k} )
        }
        \left(
            \frac 1 m
        \right)^{2k + 1}
        \mathbb{1}_{(\pi*_w\delta)}
    \end{equation}
\end{proof}

%A pairings must be Non-Crossings
%Pairings must match in opposite directions
%powers of m and (N/m)'s

%====For BC x BC====
%present result

\begin{remark}
    Though topology, we can rewrite the moment as
    \begin{equation}
        \sum_{W, \pi, \delta} m^{-2g} \mathbb{1}_{(\pi*_w\delta)}
    \end{equation}
    where $g$ is the minimum genus of the graph correlated to 
    $((\pi *_W \delta) \circ \gamma_{2k} )$. 
\end{remark}

\begin{theorem}[6, 8, 10th moment of GOE times Block Circulant]
    
\begin{table}[h!]
\centering
\caption{Spectral Density of GOE times Block Circulant}
\renewcommand{\arraystretch}{1.5} % Increase row height
\begin{tabular}{|>{\centering\arraybackslash}m{3cm}|>{\centering\arraybackslash}m{7cm}|}
\hline
\textbf{Moment} & \textbf{Value} \\
\hline
2nd moment & $2$ \\
\hline
4th moment & $10 + \frac{2}{m^2}$ \\
\hline
6th moment & $66 + \frac{38}{m^2}$ \\
\hline
8th moment & $498 + \frac{544}{m^2} + \frac{54}{m^4}$ \\
\hline
10th moment & $4066 + \frac{7000}{m^2} + \frac{2086}{m^4}$ \\
\hline
\end{tabular}
\end{table}


\begin{table}[h!]
\centering
\caption{Normalized Spectral Density of GOE times Block Circulant}
\renewcommand{\arraystretch}{1.5} % Increase row height
\begin{tabular}{|>{\centering\arraybackslash}m{3cm}|>{\centering\arraybackslash}m{10cm}|}
\hline
\textbf{Moment} & \textbf{Value} \\
\hline
Normalized 4th moment & $$\frac{5}{2} + \frac{1}{2m^2}$$ \\
\hline
Normalized 6th moment & $$\frac{33}{4} + \frac{19}{4m^2}$$ \\
\hline
Normalized 8th moment & $$\frac{249}{8} + \frac{34}{m^2} + \frac{27}{8m^4}$$ \\
\hline
Normalized 10th moment & $$\frac{2033}{16} + \frac{875}{4m^2} + \frac{1043}{16m^4}$$ \\
\hline
\end{tabular}
\end{table}

% Second Table: Normalized Spectral Density of Block Circulant times Block Circulant
\begin{table}[h!]
\centering
\caption{Normalized Spectral Density of Block Circulant times Block Circulant}
\renewcommand{\arraystretch}{1.5} % Increase row height
\begin{tabular}{|>{\centering\arraybackslash}m{3cm}|>{\centering\arraybackslash}m{10cm}|}
\hline
\textbf{Moment} & \textbf{Value} \\
\hline
Normalized 4th moment & $$ \frac{10m^4 + 86m^2 + 48}{4m^4 + 8m^2 + 4}$$ \\
\hline
Normalized 6th moment & $$\frac{66m^6 + 1890m^4 + 9084m^2 + 3360}{8m^6 + 24m^4 + 24m^2 + 8}$$ \\
\hline
Normalized 8th moment & $$ \frac{498m^8 + 33236m^6 + 529634m^4 + 1759064m^2 + 499968}{16m^8 + 64m^6 + 96m^4 + 64m^2 + 16}$$ \\
\hline
\end{tabular}
\end{table}


\end{theorem}




\section{Auxiliary Sequences and Recurrence Relations}

\begin{definition}[\(\sigma_{n, s, k}\)]
The auxiliary sequence \(\sigma_{n, s, k}\) is defined with the following initial conditions. 
\begin{enumerate}
    \item \(\sigma_{n, s, k} = 0\) if \(s + k > n\)
    \item \(\sigma_{n, s, 0} = (2n - 1)!!\), where \(C_n\) is the \(n\)-th Catalan number, \(C_n = \frac{1}{n + 1} \binom{2n}{n}\)
    \item \(\sigma_{n, s, 2k + 1} = 0\)
    \item \(\sigma_{n, s, -k} = 0\)
\end{enumerate}

The recurrence relation for \(\sigma_{n, s, 2k}\) is given by:
\[
\sigma_{n, s, 2k} = 
\sum_{p = s + 1}^{n} 
\sum_{q = p + 1}^{n}
\sum_{r = 0}^{2k}
\left[
\sigma_{n - q + p, p, r} \cdot \sigma_{q - p - 1, 0, 2k - 2 - r}
+ 
\sigma_{n - q + p - 1, p - 1, r} \cdot \sigma_{q - p, 1, 2k - 2 - r}
\right]
\]
\end{definition}

\begin{theorem}[Moments of GOE times PT]
Let $\mu^{(k)}_N$ be the $k$th moment of the spectral density of 
the anticommutator $AB + BA$ where $A$ is a GOE and $B$ is 
a palindromic topelitz matrix. Then, we obtain the following formula. 
\[
    \mu^{(k)}_N = \sigma_{N, 0, N}
\]
\end{theorem}

\end{document}


\end{document}