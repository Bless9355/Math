\documentclass[11pt,reqno]{amsart}

\usepackage{amsmath,amsthm,amssymb,comment,fullpage}
%\usepackage{epsf, subfigure, verbatim}
\usepackage{braket}
\usepackage{mathtools}
\usepackage{bbold}

%\usepackage{euler,palatino}
%\usepackage{srcltx}
%\usepackage{amsthm}
%\usepackage[latin1]{inputenc}
%\usepackage[T1]{fontenc}
%\usepackage{float}

%\usepackage{mathrsfs}
%\documentclass[12pt,reqno]{amsart}
\usepackage{caption}
\usepackage{times}
\usepackage[T1]{fontenc}
\usepackage{mathrsfs}
\usepackage{latexsym}
\usepackage[dvips]{graphics}
\usepackage{epsfig}
%\usepackage{hyperref, amsmath, amsthm, amsfonts, amscd, flafter,epsf}
\usepackage{amsmath,amsfonts,amsthm,amssymb,amscd}
\input amssym.def
\input amssym.tex
\usepackage{color}
\usepackage{hyperref}
\usepackage{url}
%\usepackage{breakurl}
\newcommand{\bburl}[1]{\textcolor{blue}{\url{#1}}}
\usepackage{diagbox}

\usepackage{tikz}
\usepackage{tkz-tab}
\usepackage{tkz-graph}
\usetikzlibrary{shapes.geometric,positioning}
%\newcommand{\burl}[1]{\textcolor{blue}{\url{#1}}}
\newcommand{\blue}[1]{\textcolor{blue}{(\bf{#1})}}


%\usepackage{showkeys}
%\DeclareGraphicsRule{.tif}{png}{.png}{`convert #1 `dirname #1`/`basename #1 .tif`.png}

\newcommand{\emaillink}[1]{\textcolor{blue}{\href{mailto:#1}{#1}}}

\newcommand{\burl}[1]{\textcolor{blue}{\url{#1}}}
\newcommand{\fix}[1]{\textcolor{red}{\textbf{ (#1)\normalsize}}}
\newcommand{\fixed}[1]{\textcolor{green}{~\\ \textbf{#1\normalsize}}\\}

\newcommand{\ind}{\otimes}
\newcommand{\witi}{\widetilde}
\newcommand{\ch}{{\bf 1}}
\newcommand{\dt}[1]{\witi{\witi #1}}
\newcommand{\ol}[1]{\overline{#1}}
\newcommand{\lr}[1]{\left\lfloor#1\right\rfloor}
\newcommand{\eqd}{\overset{\footnotesize{d}}{=}}
\newcommand{\calf}{{\mathcal F}}
\newcommand{\cal}{\mathcal}

\renewcommand{\theequation}{\thesection.\arabic{equation}}
\numberwithin{equation}{section}


\newtheorem{thm}{Theorem}[section]
\newtheorem{conj}[thm]{Conjecture}
\newtheorem{cor}[thm]{Corollary}
\newtheorem{lem}[thm]{Lemma}
\newtheorem{prop}[thm]{Proposition}
\newtheorem{exa}[thm]{Example}
\newtheorem{defi}[thm]{Definition}
\newtheorem{exe}[thm]{Exercise}
\newtheorem{que}[thm]{Question}
\newtheorem{prob}[thm]{Problem}
\newtheorem{cla}[thm]{Claim}
\newtheorem{proj}[thm]{Research Project}

\theoremstyle{plain}
\newtheorem{X}{X}[section]
\newtheorem{corollary}[thm]{Corollary}
\newtheorem{definition}[thm]{Definition}
\newtheorem{example}[thm]{Example}
\newtheorem{lemma}[thm]{Lemma}
\newtheorem{proposition}[thm]{Proposition}
\newtheorem{theorem}[thm]{Theorem}
\newtheorem{conjecture}[thm]{Conjecture}
\newtheorem{hypothesis}[thm]{Hypothesis}
\newtheorem{rem}[thm]{Remark}
%\newtheorem*{hypth\eta}[thm]{Hypothesis \ref{montgomery original}$_\th\eta$}
%\newtheorem*{hyplog}{Hypothesis \ref{montgomer original}}
%\newtheorem*{hypsmallo}[thm]{Hypothesis \ref{montgomery original}}
%\newtheorem*{hyp\eta}[thm]{Hypothesis 1.10$_{\\eta}$}


%\theoremstyle{definition}
\newtheorem{remark}[thm]{Remark}

\renewcommand\thesection{\arabic{section}}
\newcommand{\F}{\mathscr{F}}
\newcommand{\f}{\widehat{\\eta}}


%%%%%%%%%%%%%% Dirichlet characters
\newcommand{\Norm}[1]{\frac{#1}{\sqrt{N}}}

\newcommand{\sumii}[1]{\sum_{#1 = -\infty}^\infty}
\newcommand{\sumzi}[1]{\sum_{#1 = 0}^\infty}
\newcommand{\sumoi}[1]{\sum_{#1 = 1}^\infty}

\newcommand{\eprod}[1]{\prod_p \left(#1\right)^{-1}}
%$(s,b)$-Generacci
\newcommand{\sbg}{(s,b)\text{-Generacci}}
\newcommand{\sbs}{(s,b)\text{-Generacci\ sequence}}
\newcommand{\fqs}{\text{Fibonacci\ Quilt\ sequence}}
\newcommand{\fq}{\text{Fibonacci\ Quilt}}
\newcommand\st{\text{s.t.\ }}
\newcommand\be{\begin{equation}}
\newcommand\ee{\end{equation}}
\newcommand\bee{\begin{equation*}}
\newcommand\eee{\end{equation*}}
\newcommand\bea{\begin{eqnarray}}
\newcommand\eea{\end{eqnarray}}
\newcommand\beae{\begin{eqnarray*}}
\newcommand\eeae{\end{eqnarray*}}
\newcommand\bi{\begin{itemize}}
\newcommand\ei{\end{itemize}}
\newcommand\ben{\begin{enumerate}}
\newcommand\een{\end{enumerate}}
\newcommand\bc{\begin{center}}
\newcommand\ec{\end{center}}
\newcommand\ba{\begin{array}}
\newcommand\ea{\end{array}}
%\newcommand\mod{\text{mod\ }}
\newcommand{\mo}{\text{mod}\ }

\newcommand\ie{{i.e.,\ }}
\newcommand{\tbf}[1]{\textbf{#1}}
\newcommand\CF{{Continued Fraction}}
\newcommand\cf{{continued fraction}}
\newcommand\cfs{{continued fractions}}
\newcommand\usb[2]{\underset{#1}{\underbrace{#2}}}


%kmin and kmax
\newcommand{\kmin}[1]{k_{\min}(#1)}
\newcommand{\kmax}[1]{k_{\max}(#1)}
\newcommand{\kminm}{K_{\min}(m)}
\newcommand{\kmaxm}{K_{\max}(m)}
\newcommand{\B}{\mathcal{B}}
% General Symbols

\def\notdiv{\ \mathbin{\mkern-8mu|\!\!\!\smallsetminus}}
\newcommand{\done}{\Box} %use in linux
%\newcommand{\umess}[2]{\underset{(#1)}{\underbrace{#2}}}
\newcommand{\umessclean}[2]{\underset{=#1}{\underbrace{#2}}}

%Blackboard Letters

\newcommand{\R}{\ensuremath{\mathbb{R}}}
\newcommand{\C}{\ensuremath{\mathbb{C}}}
\newcommand{\Z}{\ensuremath{\mathbb{Z}}}
\newcommand{\Q}{\mathbb{Q}}
\newcommand{\N}{\mathbb{N}}
%\newcommand{\F}{\mathbb{F}}
\newcommand{\W}{\mathbb{W}}
\newcommand{\Qoft}{\mathbb{Q}(t)}  %use in linux

\newcommand\frakfamily{\usefont{U}{yfrak}{m}{n}}
\DeclareTextFontCommand{\textfrak}{\frakfamily}
\newcommand\G{\textfrak{G}}


% Fractions

\newcommand{\fof}{\frac{1}{4}}  %oneforth
\newcommand{\foh}{\frac{1}{2}}  %onehalf
\newcommand{\fot}{\frac{1}{3}}  %onethird
\newcommand{\fop}{\frac{1}{\pi}}    %1/pi
\newcommand{\ftp}{\frac{2}{\pi}}    %2/pi
\newcommand{\fotp}{\frac{1}{2 \pi}} %1/2pi
\newcommand{\fotpi}{\frac{1}{2 \pi i}}
\newcommand{\cm}{c_{\text{{\rm mean}}}}
\newcommand{\cv}{c_{\text{{\rm variance}}}}



% Theorem / Lemmas et cetera

%\theoremstyle{definition}
\newtheorem{rek}[thm]{Remark}

\newcommand{\vars}[2]{ #1_1, \dots, #1_{#2} }
\newcommand{\ncr}[2]{{#1 \choose #2}}
\newcommand{\twocase}[5]{#1 \begin{cases} #2 & \text{{\rm #3}}\\ #4
&\text{{\rm #5}} \end{cases}   }
\newcommand{\threecase}[7]{#1 \begin{cases} #2 & \text{{\rm #3}}\\ #4
&\text{{\rm #5}}\\ #6 & \texttt{{\rm #7}} \end{cases}   }
\newcommand{\twocaseother}[3]{#1 \begin{cases} #2 & \text{#3}\\ 0
&\text{otherwise} \end{cases}   }


%Formatting
\renewcommand{\baselinestretch}{1}
\newcommand{\murl}[1]{\href{mailto:#1}{\textcolor{blue}{#1}}}
\newcommand{\hr}[1]{\href{#1}{\url{#1}}}

%%% NEW COMMANDS FOR THIS PAPER
\newcommand{\dfq}{d_{\rm FQ}}
\newcommand{\dave}{d_{\rm FQ; ave}}
\newcommand{\daven}{d_{\rm FQ; ave}(n)}
\newcommand{\todo}[1]{\textcolor{red}{\textbf{#1}}}
\newcommand{\DN}[1]{\textcolor{blue}{\textbf{(DN:#1)}}}
\newcommand{\PF}[1]{\textcolor{cyan}{\textbf{(PF:#1)}}}
\newcommand{\PW}{\textnormal{conf}}
\newcommand{\NC}{\textnormal{NC}}


%%% TEXT COMMANDS
\newcommand{\textOr}{
    {
        \hspace{5mm}
        \textrm{or}
        \hspace{5mm}
    }
}

\newcommand{\textAnd}{
    {
        \hspace{5mm}
        \textrm{and}
        \hspace{5mm}
    }
}



%%% NEW COMMANDS FROM VARIANCE
\newcommand{\PP}[1]{\mathbb{P}[#1]}
\newcommand{\E}[1]{\mathbb{E}[#1]}
\newcommand{\V}[1]{\text{{\rm Var}}[#1]}
\newcommand{\BS}{\mathcal{S}}
\newcommand{\T}{\mathcal{T}}
\newcommand{\LT}{\mathcal{L_T}}
\newcommand{\LS}{\mathcal{L_S}}
\newcommand{\ZS}{\mathcal{Z_S}}
\newcommand{\ds}{\displaystyle}
\newcommand{\nsum}{Y_n}


\title{Paper template}
\author{Glenn Bruda}
\email{\textcolor{blue}{\href{mailto:glenn.bruda@ufl.edu}{glenn.bruda@ufl.edu}}}
\address{Department of Mathematics, University of Florida, Gainesville, FL 32611}

\author{Bruce Fang}
\email{\textcolor{blue}{\href{mailto:bf8@williams.edu}{bf8@williams.edu}}}
\address{Department of Mathematics and Statistics, Williams College, Williamstown, MA 01267}

\author{Raul Marquez}
\email{\textcolor{blue}{\href{mailto:raul.marquez02@utrgv.edu}{raul.marquez02@utrgv.edu}}}
\address{School of Mathematical and Statistical Sciences, University of Texas Rio Grande Valley, Brownsville, TX 78520}

\author{Steven J. Miller}
\email{\textcolor{blue}{\href{mailto:sjm1@williams.edu}{sjm1@williams.edu}},  \textcolor{blue}{\href{Steven.Miller.MC.96@aya.yale.edu}{Steven.Miller.MC.96@aya.yale.edu}}}
\address{Department of Mathematics and Statistics, Williams College, Williamstown, MA 01267}

\author{Beni Prapashtica}
\email{\textcolor{blue}{\href{mailto:bp492@cam.ac.uk}{bp492@cam.ac.uk}}}
\address{Department of Pure Mathematics and Mathematical Statistics, University of Cambridge, Cambridge, UK}

\author{Vismay Sharan}
\email{\textcolor{blue}{\href{mailto:vismay.sharan@yale.edu}{vismay.sharan@yale.edu}}}
\address{Department of Mathematics, Yale University, New Haven, CT 06511}

\author{Daniel Son}
\email{\textcolor{blue}{\href{mailto:ds15@williams.edu}{ds15@williams.edu}}}
\address{Department of Mathematics and Statistics, Williams College, Williamstown, MA 01267}

\author{Saad Waheed}
\email{\textcolor{blue}{\href{mailto:sw21@williams.edu}{sw21@williams.edu}}}
\address{Department of Mathematics and Statistics, Williams College, Williamstown, MA 01267}

\author{Janine Wang}
\email{\textcolor{blue}{\href{mailto:jjw3@williams.edu}{jjw3@williams.edu}}}
\address{Department of Mathematics and Statistics, Williams College, Williamstown, MA 01267}

\thanks{This research was supported by the National Science Foundation grant DMS-2241623, Churchill College Cambridge, the Finnerty fund, and Williams College.}

\subjclass[2010]{}

\keywords{}

\date{\today}

\title{The Limiting Spectral Measure of Various Matrix Ensembles Under the Anticommutator Operator}
\begin{document}


\begin{abstract}
We introduce the anticommutator operator $\{\cdot, \cdot\}$, where $\{A_N,B_N\} = A_NB_N + B_NA_N$, to various real symmetric random matrix ensembles, including the Gaussian orthogonal ensemble (GOE), the real symmetric palindromic Toeplitz ensemble (PTE), the $k$-checkerboard ensemble, and the real symmetric block $k$-circulant ensemble ($k$-BCE). By using classic combinatorial techniques related to the non-crossing and free-matching properties of the cyclic product, respectively of the GOE and PTE, we obtain recursive formulae for the moments of the limiting spectral distribution of $\{\textup{GOE, GOE}\}$, $\{\textup{PTE, PTE}\}$, $\{\textup{GOE, PTE}\}$ and the bulk moments of $\{\textup{GOE, }k\textup{-checkerboard}\}$ and $\{k\textup{-checkerboard, }j\textup{-checkerboard}\}$. For the anticommutator of the $m$-BCE matrices with other ensembles the combinatorics is more complicated so we develop a genus expansion formulae for these cases. For $\{\textup{GOE, }k\textup{-checkerboard}\}$ and $\{k\textup{-checkerboard, }j\textup{-checkerboard}\}$, we observe vastly different blip behaviors: while $\{\textup{GOE, }k\textup{-checkerboard}\}$ has two blips each containing $k$ eigenvalues near $\pm\frac{N^{3/2}}{k}$, $\{k\textup{-checkerboard, }j\textup{-checkerboard}\}$ has one largest eigenvalue near $\frac{2N^2}{jk}$, two intermediary blips each containing $k-1$ eigenvalues near $\pm \frac{1}{k}\sqrt{1-\frac{1}{j}}N^{3/2}$ and two intermediary blips each containing $j-1$ eigenvalues near $\pm\frac{1}{j}\sqrt{1-\frac{1}{k}}N^{3/2}$. We prove that both blips of $\{\textup{GOE, }k\textup{-checkerboard}\}$ converge to the $k\times k$ GOE up to a constant factor and the largest blip of $\{k\textup{-checkerboard, }j\textup{-checkerboard}\}$ converges to some distribution dependent on $k$ and $j$. After developing an appropriate weight function, we highlight the combinatorial difficulties of finding the moments of each intermediary blip due to inability to separate out the contribution from other blips. Using the moments we can use traditional methods to show almost-sure convergence for the bulk in all of these cases as well as the largest blip in all cases.
\end{abstract}

\maketitle

\tableofcontents

%%%%%%%%%%%%%%%%%%%%%%%%%%%%%%%%%%%%%%%%%%%%%%%%%%%%%%%%%%%%%%%%%%%%%%%%%%%%%%%%%%%%%%%%%%%%%%%%%%%%%%%%%%%%%%%%%%%%%%%%%%%%%%%%%%%%
%%%%%%%%%%%%%%%%%%%%%%%%%%%%%%%%%%%%%%%%%%%%%%%%%%%%%%%%%%%%%%%%%%%%%%%%%%%%%%%%%%%%%%%%%%%%%%%%%%%%%%%%%%%%%%%%%%%%%%%%%%%%%%%%%%%%
%%%%%%%%%%%%%%%%%%%%%%%%%%%%%%%%%%%%%%%%%%%%%%%%%%%%%%%%%%%%%%%%%%%%%%%%%%%%%%%%%%%%%%%%%%%%%%%%%%%%%%%%%%%%%%%%%%%%%%%%%%%%%%%%%%%%
\section{Introduction}

\subsection{Background}
Random matrix theory was first introduced by Wishart \cite{Wishart} in the 1920s. Since then, the eigenvalues of random matrices have been widely studied, with important applications to various fields including physics, number theory, and computer science. One cornerstone result in random matrix theory is the semi-circle law, which was discovered by Wigner while he was investigating nuclear resonance levels \cite{Wigner1,Wigner2}. The semi-circle law states that the normalized eigenvalue distribution of certain matrix ensembles converges to a semi-circle. The limiting spectral distribution of different types of random matrix ensembles is now extensively researched with several surveys relating to the topic \cite{BasBo1,BasBo2,BLMST,BHS1,BHS2,FM,GKMN,Toeplitz,McK,Me}.

Although many matrix ensembles follow the semi-circle law, their true distributions and strengths of convergence often depend on their symmetric properties. Imposing certain symmetries on the entries of matrices in an ensemble gives rise to different eigenvalue distributions. Examples of such ensembles with imposed symmetries include Toeplitz matrices \cite{Toeplitz}, $k$-checkerboard matrices \cite{split}, adjacency matrices of $d$-regular graphs \cite{GKMN}, and block circulant matrices \cite{Block Circulant}. The moments and formula of these distributions are not guaranteed to have a nice closed form, many methods have been developed to compute the former and approximate the closed form of the latter. Typically, these methods leverage the symmetries of these matrix ensembles and use combinatorics in their calculations.

A natural question that arises pertains to how different matrix ensembles can be combined under various operations. This paper expands upon previous research \cite{disco,Swirl} that combines different matrix ensembles under a ``disco'' or a ``swirl'' operation. We combine matrix ensembles using the anticommutator operator $\{\cdot, \cdot\}$, defined as $\{A, B\}:=AB+BA$. The anticommutator of two random matrix ensembles is a common object in random matrix theory and has previously been studied in \cite{commutator}. Employing tools from free probability, their method involves calculations of free cumulants via lattice of non-crossing partitions and a combinatorial Fourier transform that converts $R$-transform into $S$-transform, which yields the distribution of the anticommutator of two matrix ensembles given their respective distributions. This method, however, applies only for certain distributions such as semicircle, free Poisson, arsine, and Bernoulli, etc., due to the intractability of free cumulants calculations, especially combined with analytic transforms.

Unlike most paper in the literature that heavily relies on free probability, our paper employs combinatorial and topological tools such as recurrence and genus expansion to directly compute the moments of the anticommutator of various ensembles with additional symmetries: GOE, palindromic Toeplitz, block circulant, and $k$-checkerboard, and use these moments to prove convergence.


\subsection{Preliminaries}

We study the anticommutators of the matrix ensembles defined below. The distributions of all these matrices have been studied using the moment method, but not all of them have closed form expressions for their moments. For all of these ensembles, we use the empirical spectral measure $\nu_{A,N}$ for an $N\times N$ matrix $A$ and we normalize eigenvalues by $\sqrt{N}$. Formally, the spectral measure is defined as
\begin{align}\label{spectralmeasure}
\nu_{A,N} \ = \ \frac{1}{N}\sum_{i=1}^N\delta\left(x-\frac{\lambda_i}{\sqrt{N}}\right),
\end{align}
where $\{\lambda_i\}_{i=1}^N$ are the eigenvalues of $A$.

\begin{definition}[Limiting Spectral Distribution]
For the spectral distribution $\nu_{A,N}$ the limiting spectral distribution is $\lim_{N\to\infty}\nu_{A,N}$ where the convergence can be shown as weak converge, almost-sure convergence, or convergence in probability depending on the specific ensemble considered. The measure $\nu_{A,N}$ in the case of the Gaussian Orthogonal Ensemble was \eqref{spectralmeasure}, but in this paper we consider some different measures for different anticommutators.
\end{definition}

To find the limiting spectral distribution, we use the eigenvalue trace lemma to calculate the $m$\textsuperscript{th} moment of the distribution, taking the limit as $N\rightarrow\infty$.

\begin{definition}[Gaussian Orthogonal Ensemble (GOE)]
The GOE is a random matrix ensemble whose matrices have entries defined by $a_{ij}=a_{ji}\sim \mathcal{N}(0,1)$ for $i\neq j$ and $a_{ii}\sim \mathcal{N}(0,\sqrt{2})$.
\end{definition}

As discovered by Wigner in \cite{Wigner1}, 
the limiting distribution of the normalized eigenvalues of the GOE is the semi-circle. The proof of the semicircle law uses the moment method, with odd moments being $0$ and even moments being the Catalan numbers, which matches with the moments of the semi-circle density.

\begin{definition}[Palindromic Toeplitz]\cite{palindromicToeplitz}\label{Def-palindromicToeplitz}
An $N\times N$ real symmetric palindromic Toeplitz matrix (where $N$ is assumed to be even for simplicity) is a matrix $A_N$ whose entries are paramatrized by $b_0, b_1,\dots,b_{N/2-1}$, where the $b_i$'s are i.i.d. random variables with mean 0 and variance 1:
\begin{align}
a_{ij} = \begin{cases}
b_{|i-j|},& \text{if } 0\leq |i-j|\leq \frac{N}{2}-1\\
b_{N-1-|i-j|},& \text{if } \frac{N}{2}\leq |i-j|\leq N-1.
\end{cases}
\end{align}
This matrix is therefore of the form
\begin{align}
\begin{pmatrix}
b_0 & b_1 & b_2 & \cdots & b_2 & b_1 & b_0\\
b_1 & b_0 & b_1 & \cdots & b_3 & b_2 & b_1\\
b_2 & b_1 & b_0 & \cdots & b_4 & b_3 & b_2\\
\vdots & \vdots & \vdots & \ddots & \vdots & \vdots & \vdots\\
b_2 & b_3 & b_4 & \cdots & b_0 & b_1 & b_2\\
b_1 & b_2 & b_3 & \cdots & b_1 & b_0 & b_1\\
b_0 & b_1 & b_2 & \cdots & b_2 & b_1 & b_0
\end{pmatrix}.
\end{align}
\end{definition}

Random Toeplitz matrices were first studied in \cite{Toeplitz} where the authors noticed that there were some "Diophantine obsructions" which made the caulcation of moments more difficult in many cases. For this reason the palindromic Toeplitz ensemble was first introduced in \cite{palindromicToeplitz}, adding extra combinatorial structure to the original Toeplitz ensemble in order to get nicer combinatorial structure in the calculation. The distribution of the normalized eigenvalues of a palindromic Toeplitz matrix was established in \cite{palindromicToeplitz} to be Gaussian. This was done by proving that in the expansion of the moments using eigenvalue trace lemma all of the matchings are free, which gives that the $2m$\textsuperscript{th} moment is $(2m-1)!!$ and all the odd moments are $0$. These are exactly the moments of the Gaussian.



\begin{definition}[\textbf{Block Circulant}]\cite{Block Circulant}
Let $k|N$. An $N\times N$ real symmetric $k$-block circulant matrix is of the form
\begin{equation}
\left(\begin{array}{ccccc}
B_0 & B_1 & B_2 & \cdots & B_{\frac{N}{k}-1}\\
B_{-1} & B_0 & B_1 & \cdots & B_{\frac{N}{k}-2}\\
B_{-2} & B_{-1} & B_0 & \cdots & B_{\frac{N}{k}-3}\\
\vdots & \vdots & \vdots & \ddots & \vdots\\
B_{1-\frac{N}{k}} & B_{2-\frac{N}{k}} & B_{3-\frac{N}{k}} & \cdots & B_0
\end{array}\right),
\end{equation}
where each $B_i$ is an $k\times k$ real matrix, each $B_{-i} = B_i^T$, and specifically $B_0$ is symmetric.
\end{definition}

The block circulant matrix ensemble is studied in \cite{Block Circulant}, where they use the genus expansion to express the moments of the spectral distribution in terms of the number of pairings of the edges of a polygon giving rise to a genus $g$ surface. The moments are then used to find the exact limiting spectral distribution.

\begin{definition}[\textbf{$(k,w)$-checkerboard}]\cite{split}
An $N\times N$ matrix from a $(k,w)$-checkerboard ensemble over $\mathbb{R}$ for $k\in\mathbb{Z}_{>0}$ and $w\in\mathbb{R}$ is given by $M=(m_{ij})$ such that 
\begin{align}
m_{ij}=
\begin{cases}
a_{ij},& \text{if }i\not\equiv j\mod{k}\\
w,& \text{if }i\equiv j\mod{k},
\end{cases}
\end{align}
where $a_{ij}=a_{ji}$ and all of the distinct $a_{ij}$ terms are sampled from a distribution with mean $0$ and variance $1$. We refer to the $(k, 1)$-checkerboard ensemble as the $k$-checkerboard ensemble. Unless specified otherwise, we always assume the weight of the checkerboard to be 1.
\end{definition}

\begin{remark}
We say that $f(m)=O(g(m))$ if there exist a positive real number $C$ and a real number $M$ such that $f(m)\leq Cg(m)$ for all $m\geq M$. If $f(m)=O(g(m))$ and $g(m)=O(f(m))$, then we say that $f(m)=\Theta(g(m))$.
\end{remark}

The $k$-checkerboard matrix ensemble is studied in \cite{split}, where it is observed that the spectral distribution is split into a bulk of order $O(\sqrt{N})$ containing $N-k$ eigenvalues (with the largest eigenvalue of the bulk $\Theta(\sqrt{N})$) and a blip of order $\Theta(N)$ containing $k$ eigenvalues. It is shown that the bulk distribution is semi-circular while the blip distribution is that of a $k\times k$ hollow GOE.

\begin{definition}[\textbf{Anticommutator}]
The anticommutator\footnote{The commutator is alternatively defined as $AB-BA$, but we do not study the commutator as it is not necessarily real symmetric even if $A$ and $B$ are individually real symmetric.} ensemble of two matrix ensembles is defined as $\{A_N,B_N\}=A_NB_N+B_NA_N$, where $A_N$ is an $N\times N$ matrix sampled from an ensemble $A$ and $B_N$ an $N\times N$ matrix sampled from an ensemble $B$.
\end{definition}

\subsection{Results}
In this paper we study the spectral distribution of the anticommutators of the different ensembles above. The most natural example is the anticommutator of the GOE with another GOE, or $\{\textup{GOE, GOE}\}$. This was previously studied using free probability in \cite{commutator} without providing a moment formula. By exploiting the non-crossing properties of the cyclic product of GOE, we are able to obtain the even moments of $\{\textup{GOE, GOE}\}$ (the odd moments are 0, similar to other ensembles), as shown below.

\begin{lemma}\label{GOE-GOE moment recurrence intro}
The $2m$\textsuperscript{th} moment $M_{2m}$ of $\{\textup{GOE, GOE}\}$ is given by $M_{2m}=2f(m)$, where $f(0)=f(1)=1$, $g(1)=1$, and
\begin{align}
f(m) &\ = \ 2\sum_{j=1}^{m-1}g(j)f(m-j) + g(m), \\
g(m) &\ = \ 2f(m-1) + \sum_{\substack{0\leq x_1,x_2\leq m-2\\ x_1+x_2\leq m-2}}(1+\mathbb{1}_{x_1>0})(1+\mathbb{1}_{x_2>0})f(x_1)f(x_2)g(m-1-x_1-x_2).
\end{align}
\end{lemma}

Lemma \ref{GOE-GOE moment recurrence intro} can be naturally extended to finding the moments of the bulk distribution of $\{\textup{GOE, }\\k\textup{-checkerboard}\}$ and $\{k\textup{-checkerboard, }j\textup{-checkerboard}\}$. In general, the method for the moments of the anticommutator of GOE and a different ensemble immediately gives us the bulk moments of the anticommutator of k-checkerboard and that same ensemble, since we can always extract the constant matrix of finite rank and apply the following result from \cite{Tao1}. This reduces to the case of a $(k,0)$-checkerboard, with a factor of $\left(1-\frac{1}{k}\right)$ to some power.

\begin{lemma}{\cite{Tao1}}
Let $\{\mathcal{A}_N\}_{N\in\mathbb{N}}$ be a sequence of random Hermitian matrix ensembles such that $\\ \{\nu_{\mathcal{A}_N,N}\}_{N\in\mathbb{N}}$ converges weakly almost surely to a limit $\nu$. Let $\{\Tilde{\mathcal{A}}_N\}_{N\in\mathbb{N}}$ be another sequence of random matrix ensembles such that $\frac{1}{N}\text{rank}(\Tilde{\mathcal{A}}_N)$ converges almost surely to zero. Then $\{\nu_{\mathcal{A}_N+\Tilde{\mathcal{A}}_N,N}\}_{N\in\mathbb{N}}$ converges weakly to $\nu$.
\end{lemma}


%\newcommand{\PWD}{\text{PW}}
%\newcommand{\NC}{\text{NC}}

%\begin{theorem}[GOE times Block Circulant]
    %\label{thm: GOEBC}
    %\begin{equation}
        %\mu_N^{(2k)} \ = \ 
        %\sum_{W \in \PWD(2k)}
        %\sum_{\pi \in \mathcal P[2k]}  
        %\sum_{\delta \in \NC(2k)}   
        %m^{
            %\#((\pi *_W \delta) \circ \gamma_{2k} )
        %}
        %\left(
        %    \frac 1 m
        %\right)^{2k + 1}
        %\mathbb{1}_{(\pi*_w\delta)}
    %\end{equation}
%\end{theorem}
%\footnote{
    %$\gamma_n$ denotes a permutation of the canonical set $[n]$ 
    %where $\gamma_n(x) = x + 1 \mod n$. 
%}\footnote{
    %$\NC(2k)$ denotes the set of all pairings of the canonical set $[2k]$ which has no crossings. 
%}

%\begin{theorem}[Block Circulant times Block Circulant]
    %\label{thm:BCBC}
    %\begin{equation}
        %\mu_N^{(2k)} \ = \ 
        %\sum_{W \in \PWD(2k)}
        %\sum_{\pi \in \mathcal P[2k]}  
        %\sum_{\delta \in  \mathcal P[2k]}   
        %m^{
            %\#((\pi *_W \delta) \circ \gamma_{2k} )
        %}
        %\left(
            %\frac 1 m
        %\right)^{2k + 1}
    %\end{equation}
%\end{theorem}

Similarly, by using the non-crossng properties and the free-matching properties of the cyclic product of respectively of GOE and palindrmoic Toeplitz, we obtain the even moments of $\{\textup{GOE, PTE}\}$:

\begin{theorem}\label{sigmarecurrence intro}
The $2m$\textsuperscript{th} moment of $\{\textup{GOE, PTE}\}$ is given by $\sigma_{2m, 0, m}$, where $\sigma_{n, s, k}$ is given by the conditions:
\begin{enumerate}
\item $\sigma_{n, s, k}=0$ if $k<0$,
\item $\sigma_{n, s, k}=0$ if $s+k>n$,
\item $\sigma_{n, s, 2k+1}=0$,
\item $\sigma_{n, s, 0}=(2n-1)!!$,
\end{enumerate}
and the recurrence relation
\begin{align}
\sigma_{n, s, 2k} \ = \ \sum_{p=s+1}^n \sum_{q=p+1}^n \sum_{r=0}^{2k}\left[
\sigma_{n-q+p, p, r}\cdot\sigma_{q-p-1, 0, 2k-2-r}+\sigma_{n-q+p-1, p-1, r}\cdot\sigma_{q- p, 1, 2k-2-r}
\right].
\end{align}
\end{theorem}

For the block circulant and GOE we use the genus expansion to compute the moments of their anticommutator. For the anticommutator of the palindromic Toeplitz and block circulant ensembles the genus expansion cannot handle the complexity so we are unable to calculate the moments in this case.

For anticommutator ensembles involving block circulant ensembles, due to different weights associated with different types of matchings in the cyclic product, we are unable to provide recurrence relations for the moments. Instead, we give the genus expansion formulae for these moments. We also give the genus expansion formulae for the other ensembles:

%\begin{table}[h!]
%\begin{center}
%\begin{tabular}{|c||c|c|c|c|} 
%\hline
%\diagbox[innerwidth = 3cm, height = 4ex]{}{} & GOE & Palindromic Toeplitz & k-checkerboard & $m$-Block Circulant \\ 
%\hline
%\hline
%GOE & 
%$\sum_{C, \pi_C\in NC(2m)}\mathbb{1}$
%& $\sum_{W, \pi, \delta} \mathbb{1}_{(\pi*_w\delta)}$
%& & $\sum_{W, \pi, \delta} m^{-2g} \mathbb{1}_{(\pi*_w\delta)}$ \\ 
%\hline
%\hline
%Palindromic Toeplitz & \diagbox[innerwidth = 3cm, height = 4ex]{}{} &  $\sum_{W, \pi, \delta} 1$
%& & 
%$\sum_{W, \pi, \delta} m^{\#((\pi *_W \delta) \circ \gamma_{2k}) - k - 1}$
%\\ 
%\hline
%j-checkerboard & \diagbox[innerwidth = 3cm, height = 4ex]{}{} & \diagbox[innerwidth = 3cm, height = 4ex]{}{} & & \\
%\hline
%$m$-Block Circulant & \diagbox[innerwidth = 3cm, height = 4ex]{}{} &\diagbox[innerwidth = 3cm, height = 4ex]{}{}  & \diagbox[innerwidth = 3cm, height = 4ex]{}{} &
%$\sum_{W, \pi, \delta} m^{-2g}$  \\
%\hline
%\end{tabular}
%\caption{Moments of the bulk distribution of the anticommutator of various matrix ensembles.}
%\label{table:1}
%\end{center}
%\end{table}

\begin{table}[h!]
\begin{center}
\begin{tabular}{|c|c|} 
\hline
\diagbox[innerwidth = 3cm, height = 4ex]{}{} & $k$-Block Circulant \\
\hline
GOE & $\sum_{C\in\mathcal{C}_{2,4m}}\sum_{\pi_C\in NCF_{2,C}(4m)}k^{\#(\gamma_{4m}\pi)-(2m+1)}$ \\
\hline
$k$-Block Circulant & $\sum_{C\in \mathcal{C}_{2,4m}}\sum_{\pi_C\in\mathcal{P}_{2,C}(4m)}k^{\#(\gamma_{4m}\pi)-(2m+1)}$ \\
\hline
\end{tabular}
\caption{Moments of $\{\textup{GOE}, k\textup{-BCE}\}$ and $\{k\textup{-BCE}, k\textup{-BCE}\}$.}
\label{table:1}
\end{center}
\end{table}

All the anticommutator ensembles mentioned above have eigenvalues on the order $O(N)$. However, as we take the anticommutator of an ensemble with the $k$-checkerboard ensemble, we begin to see different regimes of eigenvalues. In the case of $\{\textup{GOE, }k\textup{-checkerboard}\}$, there are $N-2k$ eigenvalues of order $O(N)$, $k$ blip eigenvalues at $\frac{N^{3/2}}{k}+O(N)$ and $k$ eigenvalues at $-\frac{N^{3/2}}{k}+O(N)$. In the case of $\{k\textup{-checkerboard, }j\textup{-checkerboard}\}$ where $\textup{gcd}(k, j)=1$ and $jk\mid N$, there are $N-2k-2j+3$ eigenvalues of order $O(N)$, two intermediary blips each containing $k-1$ eigenvalues at $\pm \frac{1}{k}\sqrt{1-\frac{1}{j}}N^{3/2}+O(N)$, two intermediary blips each containing $j-1$ eigenvalues at $\pm \frac{1}{j}\sqrt{1-\frac{1}{k}}N^{3/2}+O(N)$, and one blip eigenvalue at $\frac{2}{jk}N^2+O(N)$. These can be proven using Weyl's inequality.

%We see that in the case where we take the anticommutator of a k-checkerboard matrix with a GOE then there are $2k$ eigenvalues on the order of $N^{3/2}$, with $k$ of these centered at $\frac{1}{k}N^{3/2}$ and $k$ others centered at $-\frac{1}{k}N^{3/2}$, while all other $N-k$ eigenvalues are on the order of $N$. We also see that when we have an antricommutator of a k-checkerboard matrix and a j-checkerboard matrix then we have one eigenvalue centered at $\frac{1}{jk}N^2$ as well as $k-1$ eigenvalues centered at $\frac{1}{k}\sqrt{1-\frac{1}{j}}N^{3/2}$, with $k-1$ eigenvalues $-\frac{1}{k}\sqrt{1-\frac{1}{j}}N^{3/2}$ and similarly there are $j-1$ eigenvalues centered at each of $\pm \frac{1}{j}\sqrt{1-\frac{1}{k}}N^{3/2}$ with all other eigenvalues on the order of $N$. These facts can be rigorously shown by Weyl's inequality.

\begin{lemma}[Weyl's inequality]\cite{Weyl}
    Let $H,P$ be two $N\times N$ Hermitian matrices and let the eigenvalues of $H,P$, and $H+P$ be arranged in increasing order (i.e., $\lambda_N(H)\geq \lambda_{N-1}(H)\geq \cdots\geq \lambda_1(H)$). Then for any pair of integers $j,k$ such that $1\leq j,k\leq n$ and $j+k\geq n+1$ we have
    \begin{align}
       \lambda_{j+k-n}(H+P)\leq \lambda_j(H)+\lambda_k(P),
    \end{align}
    and for any pair of integers $j,k$ such that $1\leq j,k\leq n$ and $j+k\leq n+1$
    \begin{align}
    \lambda_j(H)+\lambda_k(P)\leq \lambda_{j+k-1}(H+P).
    \end{align}
\end{lemma}

%Then we can look at the blips in the cases of the anticommutator of the k-checkerboard and GOE. By Weyl's inequality we can see that for the anticommutator of the GOE and k-checkerboard the bulk is on the order of $N$ while there are two blips of size $k$ centered at $\pm \frac{1}{k}N^{3/2}$ so we can use the method of using a weight function as in \cite{split} and use this weight function to calculate the moments and distribution of the blip in this case. To do this we define weight function 
%\begin{align}
%f_1^{2n}(x) = \left(\frac{x(2-x)(x+1)(3-x)}{4}\right)^{2n}
%\end{align}

After choosing an appropriate weight function that concentrates on one blip at a time, we define the empirical blip measure for $\{\textup{GOE, }k\textup{-checkerboard}\}$ and $\{k-\textup{-checkerboard, }j\textup{-checkerboard}\}$ as follows.

\begin{definition}
Let $w_s=\frac{(-1)^{s+1}}{k}$ for $s\in \{1,2\}$. Then the \textbf{empirical blip spectral measure} associated to the anticommutator of an $N\times N$ GOE and $k$-checkerboard $\{A_N,B_N\}$ around $w_sN^{3/2}$ is
\begin{align}
\mu_{\{A_N,B_N\},s}(x) \ = \ \frac{1}{k}\sum_{\lambda\textup{ eigenvalues}}f_s^{2n}\left(\frac{\lambda}{w_sN^{3/2}}\right)\delta\left(\frac{x-\left(\lambda-w_sN^{3/2}\right)}{N}\right),
\end{align}
where $n(N)$ is a function satisfying $\lim_{N\rightarrow\infty}n(N)=\infty$ and $n(N)=\log\log(N)$ and
\begin{align}
f_s^{2n}(x) \ := \ \left(\frac{x(2-x)(x+1)(3-x)}{4}\right)^{2n}.
\end{align}
\end{definition}

\begin{definition}
The \textbf{empirical largest blip spectral measure} associated to the anticommutator of an $N\times N$ k-checkerboard and j-checkerboard  $\{A_N, B_N\}$, where $\textup{gcd}(k, j)=1$ and $jk\mid N$, is
\begin{align}
\mu_{\{A_N,B_N\}}(x)=\sum_{\lambda\textup{ eigenvalues}}g_0^{2n}\left(\frac{jk\lambda}{2N^2}\right)\delta\left(x-\left(\frac{\lambda-\frac{2}{jk}N^2}{N}\right)\right).
\end{align}
Let $w_s=\frac{(-1)^{s+1}}{k}\sqrt{1-\frac{1}{j}}$ and $h_s=k$ for $s\in \{1, 2\}$ and $w_s=\frac{(-1)^{s+1}}{j}\sqrt{1-\frac{1}{k}}$ and $h_s=j$ for $s\in \{3, 4\}$ and $g_s$ as defined in the Appendix \ref{intermediaryblip}. Then the \textbf{empirical intermediary blip spectral measure} associated to $\{A_N, B_N\}$ around $w_sN^{3/2}$ is
\begin{align}
\mu_{\{A_N, B_N\}, s}(x) \ = \ \frac{1}{h_s}\sum_{\lambda\textup{ eigenvalues}}g_s^{2n}\left(\frac{\lambda}{w_sN^{3/2}}\right)\delta\left(\frac{x-\left(\lambda-w_sN^{3/2}\right)}{N}\right).
\end{align}
We again require that $n(N)$ is a function satisfying $\lim_{N\rightarrow\infty}n(N)=\infty$ and $n(N)=\log\log(N)$.
\end{definition}

After applying the eigenvalue trace lemma to the empirical blip measures and identifying the configuration that has the highest contribution to the moment, we obtain an explicit formula for the moments of the blips of $\{\textup{GOE, }k\textup{-checkerboard}\}$ and the largest blip of $\{k\textup{-checkerboard, }j\textup{-checkerboard}\}$.

\begin{theorem}\label{GOE-checkerboard Moments}
The expected $m^{th}$ moment associated to the empirical blip spectral measure is
\begin{align}
\mathbb{E}\left[\mu_{\{A_N,B_N\},1}^{(m)}\right] \ = \ 2\left(\frac{1}{k}\right)^{m+1}\mathbb{E}_k[\textup{Tr }C^m].
\end{align}
\end{theorem}

\begin{theorem}
The $m^{th}$ moment of the largest blip spectral measure is 
\begin{align*}
\mathbb{E}\left[\mu_{\{A_N,B_N\}}^{(m)}\right] \ = \ \sum_{\substack{m_{1a}+m_{1b}+m_{2a}+m_{2b}=m; \\ m_{1a},m_{1b}\textup{ even}}}C(m, m_{1a}, m_{2a}, m_{1b}, m_{2b})\left(k\sqrt{1-\frac{1}{k}}\right)^{m_{1a}+2m_{2a}}\left(j\sqrt{1-\frac{1}{j}}\right)^{m_{1b}+2m_{2b}},
\end{align*}
where $C(m, m_{1a}, m_{2a}, m_{1b}, m_{2b}):=m!\left(\frac{2}{jk}\right)^m\frac{ 2^{\frac{m_{1a}+m_{1b}}{2}-2(m_{2a}+m_{2b})}m_{1a}!!m_{1b}!!}{m_{1a}!m_{1b}!m_{2a}!m_{2b}!}$.
\end{theorem}

%\begin{definition}
%The \textbf{empirical blip spectral measure} associated to the anticommutator of an $N\times N$ GOE and k-checkerboard $\{A_N,B_N\}:= A_NB_N+B_NA_N$ around $w_iN^{3/2}$ is
%\begin{align}
%\mu_{\{A_N,B_N\},i} \ = \ \frac{1}{k}\sum_{\lambda}f_i^{2n}\left(\frac{\lambda}{w_iN^{3/2}}\right)\delta\left(\frac{x-\left(\lambda-w_iN^{3/2}\right)}{N}\right).
%\end{align}
%\end{definition}

%\textcolor{red}{Maybe add a picture of the graph for k-checkerboard and GOE}

%Then we also consider the anticommutator of a $j$-checkerbard and k-checkerboard where $j$ and $k$ are coprime. In this case we see that we have three regimes, the bulk on the order of $N$, four intermediate blips, two of size $j-1$ centered at $\pm \frac{1}{j}\sqrt{1-\frac{1}{k}}N^{3/2}$, two of size $k-1$ centered at $\pm \frac{1}{k}\sqrt{1-\frac{1}{j}}N^{3/2}$, and one far away blip of size $1$ centered at $\frac{2}{jk}N^2$. We are unable to use the weight function approach for the intermediate blips since they require a very restricted weight function but we work to develop this weight function. We are able to use the weight function to approach to find the distribution of the far away blip. The weight function for this is simply $x^{2n}(2-x)^{2n}$ and we similarly define the far away blip spectral measure to find this distribution.

%\begin{definition}[Largest Blip Spectral Measure]
%The largest blip spectral measure for k-checkerboard matrix $A_N$ and j-checkerboard matrix $B_N$ is defined as 
%\begin{align}
%\mu_{A,B,N}=\sum_{\lambda\text{ eigenvalue of }\{A_N,B_N\}}f_{n(N)}\left(\frac{jk\lambda}{2N^2}\right)\delta\left(x-\left(\frac{\lambda-\frac{2}{jk}N^2}{N}\right)\right)
%\end{align}
%where $f_{n(N)}$ is the weight function $x^{2n}(2-x)^{2n}$ and $n(N)=O(\log \log N)$. Also note that we can write the expansion of the weight function as $\sum_{\alpha=2n}^{4n}c_\alpha x^{\alpha}$.
%\end{definition}

%\textcolor{red}{maybe put a picture of the histogram for the k-checkerboard and j-checkerboard}
%%%%%%%%%%%%%%%%%%%%%%%%%%%%%%%%%%%%%%%%%%%%%%%%%%%%%%%%%%%%%%%%%%%%%%%%%%%%%%%%%%%%%%%%%%%%%%%%%%%%%%%%%%%%%%%%%%%%%%%%%%%%%%%%%%%
%%%%%%%%%%%%%%%%%%%%%%%%%%%%%%%%%%%%%%%%%%%%%%%%%%%%%%%%%%%%%%%%%%%%%%%%%%%%%%%%%%%%%%%%%%%%%%%%%%%%%%%%%%%%%%%%%%%%%%%%%%%%%%%%%%%%
%%%%%%%%%%%%%%%%%%%%%%%%%%%%%%%%%%%%%%%%%%%%%%%%%%%%%%%%%%%%%%%%%%%%%%%%%%%%%%%%%%%%%%%%%%%%%%%%%%%%%%%%%%%%%%%%%%%%%%%%%%%%%%%%%%%%
\section{Moments of the anticommutator of Random Matrices}\label{sec: anticommutator Combinatorics}

In this section, we provide formulae for the moments of the following anticommutator ensembles: $\{\textup{GOE,}\\ \textup{GOE}\}$, $\{\textup{PTE, PTE}\}$, $\{\textup{GOE, PTE}\}$, $\{\textup{GOE}, k\textup{-BCE}\}$, and $\{k\textup{-BCE}, k\textup{-BCE}\}$. Due to the similarities of the structures of GOE and checkerboards, we leave the bulk moment calculation of $\{\textup{GOE, }k\textup{-checkerboard}\}$ and $\{k\textup{-checkerboard, }j\textup{-checkerboard}\}$ in Appendix \ref{AppendixMomentsAnti}. To better understand the combinatorics of moment calculation, we first go through some notations and results (see \cite{MS} for more details). 

\begin{definition}[$(\ell,m)$-configurations]\label{ellkconfig}
Let $\Sigma=\{\alpha_1,\alpha_2,\cdots, \alpha_\ell\}$ be a finite alphabet and $\phi_\ell:=\alpha_1\alpha_2\dots\alpha_\ell$. Define the action of the symmetric group $S_\ell$ on $\phi_\ell$ as $\sigma \circ \phi_\ell:=\alpha_{\sigma(1)}\alpha_{\sigma(2)}\cdots \alpha_{\sigma(\ell)}$ for all $\sigma\in S_\ell$. Then, a \textbf{$(\ell,m)$-configuration} is a string of length $m$ comprised of the concatenation of group actions on some $\phi_\ell$, i.e., $\sigma_1\circ \phi_\ell\sigma_2\circ \phi_\ell\cdots\sigma_m \circ \phi_\ell$. We denote the set of all $(\ell,m)$-configurations as $\mathcal{C}_{\ell,\ell m}$.
\end{definition}

For our purpose, we simply set $\Sigma=\{a,b\}$ and restrict our attention to $(2,m)$-configurations.

\begin{definition}[Partitions w.r.t. a configuration]
For positive integers $n$ and $m$, let $[n]=\{1, 2, \dots, n\}$ and $C=c_1c_2\cdots c_m$ be a $(2,m)$-configuration, where $c_i\in\{a_i,b_i\}$ for all $1\leq i\leq m$ under the restriction in Definition \ref{ellkconfig} that $(c_{2s-1},c_{2s})\in\{(a_{2s-1},b_{2s}),(b_{2s-1},a_{2s})\}$. Then, a partition with respect to $C$, $\pi_C=(V_1,\dots, V_t)$, is a tuple of subsets of $[2n]$ such that the following holds: 
\begin{enumerate}
\item $V_i\neq\emptyset$ for all $1\leq i\leq t$,
\item $V_1\cup\cdots\cup V_t=[2n]$,
\item $V_i\cap V_j=\emptyset$ for $i\neq j$,
\item For all $1\leq i\leq t$ and $i_1,i_2\in V_i$, $\{c_{i_1},c_{i_2}\}\in\{\{a_{i_1},a_{i_2}\},\{b_{i_1},b_{i_2}\}\}$.
\end{enumerate}
Let $\mathcal{P}_C(2n)$ denote the set of all partitions with respect to $C$ of $[2n]$. We call $V_1, V_2, \dots, V_t$ \textbf{blocks} of $\pi_C$. A partition is called a \textbf{pairing} (or matchings) if each block is of size 2. We denote all the pairings with respect to $C$ of $[2n]$ as $\mathcal{P}_{2,C}(2n)$.
\end{definition}

Note that this definition can be easily extended to any arbitrary subset $S\subseteq [n]$ and configuration $C_S$ (indexed by $S$), where $S$ is not necessarily equal to $[k]$ for any $k\in \mathbb{N}$.

\begin{definition}[Non-crossing partitions w.r.t. a configuration]
A partition with respect to $C$, $\pi=(V_1,\dots, V_t)$, of $[2n]$ is \textbf{crossing} if there exists blocks $V$ and $W$ with $i, k\in V$ and $j, l\in W$ such that $i<j<k<l$. We denote the set of non-crossing partitions with respect to $C$ of $[2n]$ by $NC_C(2n)$ and the set of non-crossing pairings with repsect to $C$ of $[2n]$ by $NC_{2,C}(2n)$.
\end{definition}



\begin{proposition}[Wick's formula] For $\pi\in\mathcal{P}_2(2m)$, let $\mathbb{E}_\pi(X_1,\dots, X_{2m})=\prod_{(r,s)\in\pi}\mathbb{E}(X_rX_s)$. Let $(X_1, \dots, X_n)$ be a real Gaussian random vector. Then
\begin{align}
\mathbb{E}[X_{i_1}, \dots, X_{i_{2m}}] \ = \ \sum_{\pi\in \mathcal{P}_2(2m)}\mathbb{E_\pi}[X_{i_1},\dots, X_{i_{2m}}]
\end{align}
for any $i_1,\dots, i_{2m}\in [n]$.
\end{proposition}

In moment calculation, it is often important to characterize the pairings that contribute in the limit, as it not only simplifies the calculation, but also allows us to view the moments as known combinatorial objects (i.e. the $m$\textsuperscript{th} Catalan number is the $2m$\textsuperscript{th} moment of the limiting distribution of the GOE). To facilitate this characterization, we extend the method of genus expansion originally used in the moment calculation of the GUE (see \cite{MS} section 1.7 for more details) to our ensembles.

\subsection{Moments of $\{\textup{GOE, GOE}\}$}\label{momentGOEGOE} In general, due to the structure of the anticommutator, a cyclic product $c_{i_1i_2}c_{i_2i_3}\cdots c_{i_{2m}i_1}$ in the $m$\textsuperscript{th} moment of the anticommutator of any two arbitrary random matrices has the restriction that $(c_{i_{2\ell-1}i_{2\ell}}, c_{i_{2\ell}i_{2\ell+1}})\in \{(a_{i_{2\ell-1}i_{2\ell}}, b_{i_{2\ell}i_{2\ell+1}}),(b_{i_{2\ell-1}i_{2\ell}}, a_{i_{2\ell}i_{2\ell+1}})\}$ for all $1\leq \ell\leq m$. In other words, a cyclic product $c_{i_1i_2}c_{i_2i_3}\cdots c_{i_{2m}i_1}$ is of a valid configuration iff it is a 2-configuration of the alphabet $\Sigma=\{a,b\}$. Now, let $A_N=(a_{ij})$ and $B_N=(b_{ij})$ be independent $N\times N$ GOE's with $\mathbb{E}[a_{ij}^2]=\mathbb{E}[b_{ij}^2]=1$ and consider the $m$\textsuperscript{th} moment of $A_NB_N+B_NA_N$. With a slight abuse of notation, we identify a $(2,2m)$-configuration $C=c_1c_2\cdots c_{2m}$ with the cyclic product $c_{i_1i_2}c_{i_2i_3}\cdots c_{i_{2m}i_1}$. Then,
\begin{align}
M_m(N) \ = \ \frac{1}{N^{m+1}}\mathbb{E}[\textup{Tr}(A_NB_N+B_NA_N)^m] \ = \ \frac{1}{N^{m+1}}\sum_{C\in\mathcal{C}_{2,2m}}\sum_{1\leq i_1,\cdots, i_m\leq N}\mathbb{E}[c_{i_1i_2}c_{i_2i_3}\dots c_{i_{2m}i_1}],  
\end{align}

We apply genus expansion to each $\mathbb{E}[c_{i_1i_2}c_{i_2i_3}\dots c_{i_{2m}i_1}]$. The argument that follows is essentially the same argument as the genus expansion of the $2m$\textsuperscript{th} moment of the GUE, since treating $a$'s and $b$'s both as $c$'s while ensuring that they are matched within themselves preserves the ``non-crossing'' property of pairings that contribute in the limit.

Now, as $N\rightarrow\infty$, we have $M_m(N)=0$ when $m$ is odd, since by standard argument the contribution from each type of configuration is $O(N^{m})$, but the number of types of configurations depends only on $m$. When $m$ is even, by Wick's formula
\begin{align}
\mathbb{E}[c_{i_1i_2}c_{i_2i_3}\cdots c_{i_{2k}i_1}] \ = \ \sum_{
\pi_C\in\mathcal{P}_{2,C}(2m)}\mathbb{E}_{\pi_C}[c_{i_1i_2},c_{i_2i_3},\dots, c_{i_{2m}i_1}].
\end{align}
Since $\mathbb{E}[c_{i_ri_{r+1}}c_{i_si_{s+1}}]=1$ when $i_r=i_{s+1}$ and $i_{r+1}=i_{s}$ and is 0 otherwise, then $\mathbb{E}[c_{i_1i_2}c_{i_2i_3}\cdots c_{i_{2m}i_1}]$ is the number of pairings $\pi_C$ with respect to $C$ of $[2m]$ such that $i_r=i_{s+1}$, $i_{r+1}=i_s$, and $a$'s and $b$'s are matched within themselves (i.e., an $a$ is not matched with a $b$). Now, we think of a tuple of indices $(i_1,\dots, i_{2m})$ as a function $i:[2m]\rightarrow [N]$ and write the pairing $\pi_C=\{(r_1,s_1), (r_2, s_2), \cdots, (r_{k},s_{m})\}$, as the product of transpositions $(r_1,s_1)(r_2,s_2)\dots(r_k,s_m)$. We also take $\gamma_{2m}$ to be the cycle $(1, 2, 3, \dots, 2m)$. If $\pi_CC$ is a pairing of $[2m]$ and $(r,s)$ is a pair of $\pi_C$, then we express our conditions $i_r=i_{s+1}$ and $i_s=i_{r+1}$ as $i(r)=i(\gamma_{2m}(\pi_C(r)))$ and $i(s)=i(\gamma_{2k}(\pi_C(s)))$ respectively. Hence, $\mathbb{E}_{\pi_C}[c_{i_1i_2}c_{i_2i_3}\cdots, c_{i_{2m}i_1}]=1$ if $i$ is constant on the orbits of $\gamma_{2m}\pi_C$ (e.g. $i(r)=i(s+1)$) and $0$ otherwise. Let $\#(\sigma)$ denote the number of cycles of a permutation $\sigma$, then
\begin{align}
M_m(N) \ = \ \frac{1}{N^{m+1}}\mathbb{E}[\textup{Tr}(A_NB_N+B_NA_N)^m] &\ = \ \frac{1}{N^{m+1}}\sum_{C\in \mathcal{C}_{2,2m}}\sum_{\pi_C\in \mathcal{P}_{2,C}(2m)}N^{\#(\gamma_{2m}\pi_C)}.
\end{align}

\begin{proposition}[\cite{MS}]\label{non-crossing}
If $\pi$ is a pairing of $[2m]$ then $\#(\gamma_{2m}\pi)\leq m-1$ unless $\pi$ is non-crossing in which case $\#(\gamma_{2m}\pi)=m+1$.
\end{proposition}

\begin{corollary}
As $N\rightarrow \infty$,
\begin{align}\label{genusGOEGOE}
M_m \ = \ \lim_{N\rightarrow\infty} M_m(N) \ = \ \sum_{C\in \mathcal{C}_{2,2m}}\sum_{\pi_C\in NC_{2,C}(2m)}1.
\end{align}
\end{corollary}

Given genus expansion formula \ref{genusGOEGOE}, we are then able to obtain a recurrence relation for the even moment of $\{\textup{GOE, GOE}\}$.

\begin{lemma}\label{GOE-GOE moment recurrence}
The $2m$\textsuperscript{th} moment $M_{2m}$ of $\{\textup{GOE, GOE}\}$ is given by $M_{2m}=2f(m)$, where $f(0)=f(1)=1$, $g(1)=1$, and
\begin{align}
f(m) &\ = \ 2\sum_{j=1}^{m-1}g(j)f(m-j) + g(m), \\
g(m) &\ = \ 2f(m-1) + \sum_{\substack{0\leq x_1,x_2\leq m-2\\ x_1+x_2\leq m-2}}(1+\mathbb{1}_{x_1>0})(1+\mathbb{1}_{x_2>0})f(x_1)f(x_2)g(m-1-x_1-x_2).
\end{align}
\end{lemma}

\begin{proof} Let $f(m)$ be the number of non-crossing pairings with respect to all $(2, 4m)$-configurations starting with an $a$, and $g(m)$ be the number of non-crossing pairings with respect to all $(2, 4m)$-configurations starting and ending with an $a$ such that these two $a$'s are matched together (i.e., a configuration $a_{i_1i_2}b_{i_2i_3}\cdots \\b_{i_{4m-1}i_{4m}}a_{i_{4m}i_1}$ with $i_{4m}=i_2$).

%From these definitions, we have that $f(k)=g(k) + h(k)$, where $h$ is the number of non-crossing partitions such that $a_{i_1,i_2}$ is not paired with $a_{i_{4k},i_1}$. 
We first find the recurrence relation for $f(m)$. We know that $a_{i_1i_2}$ is matched with some $a_{i_{4j}i_{4j+1}}$ with $j\leq m$ (in case when $j=m$, we identify $4m+1$ as $1$) since there should be an even number of both $a$ and $b$ terms between $a_{i_1i_2}$ and $a_{i_{4j}i_{4j+1}}$ to ensure non-crossing pairings. When $j=m$, the number of non-crossing pairings is just $g(m)$ by definition. When $j<m$, the number of non-crossing pairings within $a_{i_1,i_2}b_{i_2,i_3}\cdots b_{i_{4j-1}i_{4j}}a_{i_{4j}i_{4j+1}}$ is $g(j)$. We multiply this by the number of non-crossing pairings within the rest of the cyclic product which have no restrictions and is therefore simply $2f(m-j)$, with the 2 accounting for starting with either an $a$ or $b$. Thus, summing over all possible $j$'s, we have
\begin{align}
f(m) \ = \ 2\sum_{j=1}^{m-1}g(j)f(m-j)+g(m).
\end{align}

%The remaining indices $4j+1$ through to $4m$ can be paired up without any restrictions, which is equivalent to $2f(m-j)$ by definition, with the $2$ accounting for the choice of starting with either an $a$ or a $b$. Since $j$ can take any value from $1$ to $m-1$, summing over these possibilities gives $h(m)=2\sum_{j=1}^{m-1}g(j)f(m-j)$.

%we must have two more $b$'s than $a$'s between the configuration as explained in the next sentence. We also know that between indices $i_2,i_3,...,i_{2j-1}$ we must have two more $b$'s than $a$'s because otherwise we would get that after $a_{i_{4j},i_{4j+1}}$ there would be two more $a$'s than $b$'s and if there were any more than two more $b$'s than $a$'s this would also contradict the conditions}
%Similarly, we know that $b_{i_2 i_3}$ is paired with some $b_{i_{4j-1}i_{4j}}$ and $b_{i_{4m-1}i_{4m}}$ with some $b_{i_{4\ell-2}i_{4\ell-1}}$ with $1\leq j\leq m$, $2\leq \ell\leq m$. 

Similarly, we know that either $b_{i_2 i_3}$ is matched with $b_{i_{4m-1}i_{4m}}$, or $b_{i_2 i_3}$ is matched with $b_{i_{4x_1+3}i_{4x_1+4}}$ and $b_{i_{4m-1}i_{4m}}$ is matched with $b_{i_{4m-4x_2-2}i_{4m-4x_2-1}}$, with $4x_1+4<4m-4x_2-2$, or $x_1+x_2\leq m-2$. In the first case, since there are no restrictions on the $4k-4$ terms between $b_{i_2 i_3}$ and $b_{i_{4m-1}i_{4m}}$, the number of non-crossing pairings is $2f(m-1)$. In the second case, the number of non-crossing pairings of terms between $b_{i_2 i_3}$ and $b_{i_{4x_1+3}i_{4x_1+4}}$ is $(1+\mathbb{1}_{x_1>0})f(x_1)$, the number of non-crossing pairings of terms between $b_{i_{4m-4x_2-2}i_{4m-4x_2-1}}$ and $b_{i_{4m-1}i_{4m}}$ is $(1+\mathbb{1}_{x_2>0})f(x_2)$, with the indicator functions accounting for the intermediary terms starting with either an $a$ or a $b$, and lastly the number of non-crossing pairings of terms between $b_{i_{4x_1+3}i_{4x_1+4}}$ and $b_{i_{4m-4x_2-2}i_{4m-4x_2-1}}$ is $g(m-1-x_1-x_2)$. Thus, %Summing over all possible $x_1$ and $x_2$'s in the second case and adding this to the number of possibilities in the first case gives
\begin{align}
    g(m) \ = \ 2f(m-1) + \sum_{\substack{0\leq x_1,x_2\leq m-2\\ x_1+x_2\leq m-2}}(1+\mathbb{1}_{x_1>0})(1+\mathbb{1}_{x_2>0})f(x_1)f(x_2)g(m-1-x_1-x_2).
\end{align}

% It remains to find $g(m)$. Since we start and end with an $a$, we must have the terms $b_{i_2,i_3}$ and $b_{i_{4k-1},i_{4k}}$. If $b_{i_2,i_3}$ and $b_{i_{4k-1},i_{4k}}$ are also paired together, the remaining indices are essentially free, or in other words the remaining $4k-4$ terms can be paired up without any constraints, which adds a term of $2f(k-1)$. If $b_{i_2,i_3}$ and $b_{i_{4k-1},i_{4k}}$ are \textit{not} paired together, then $b_{i_2,i_3}$ is matched with some $b_{i_{4x_1+3},i_{4x_1+4}}$ and $b_{i_{4k-1},i_{4k}}$ with some $b_{i_{4k-4x_2-1},i_{4k-4x_2-1}}$, with $4x_1+3<4k-4_{x_2}-1$. Rewriting gives $x_1+x_2<k-1$, from which we obtain the sum $\sum_{\substack{0\leq x_1,x_2<k-1\\ x_1+x_2<k-1}}f(x_1)f(x_2)g(k-1-x_1-x_2)$.  We get a factor of 2 if $x_1>0$ since there are no restrictions on whether the terms between $b_{i_2,i_3}$ and $b_{i_{4x_1+3},i_{4x_1+4}}$ start with $a$ or $b$. The same is true if $x_2 > 0$. 

We have now defined our recurrence for $f(m)$, which counts the number of non-crossing pairings with respect to $(2, 4m)$-configurations starting with an $a$. Since general $(2, 4m)$-configurations can start with either an $a$ or a $b$, we multiply $f(m)$ by $2$ to get all possible non-crossing pairings of $(2, 4m)$-configurations, and we arrive at the even moments being $M_{2m}=2f(m)$.
\end{proof} 

A natural extension of the anticommutator $\{A_N,B_N\}$ is the \textbf{$\ell$- anticommutator} of $\ell$ matrix ensembles $A^{(1)}_N,A^{(2)}_{N},\dotsc, A^{(\ell)}_{N}$, defined as
\begin{align}
\{A^{(1)}_N, A^{(2)}_N, \dotsc, A^{(\ell)}_N\} \ := \ \sum_{\sigma\in S_\ell} A^{(\sigma(1))}_N A^{(\sigma(2))}_N \cdots A^{(\sigma(\ell))}_N.
\end{align}
%Note that every one of these products corresponds to a permutation of these matrices that determines the ordering. This means that the $\ell$-anticommutator can be written as 
%\[
%\sum_{\sigma\in S_\ell}A_{\sigma(1)}A_{\sigma(2)}...A_{\sigma(\ell)}.
%\]
%\begin{definition}
%A natural extension of the anticommutator is the \textbf{$\ell$-anticommutator} of matrices $A_1,A_2,\dotsc,A_\ell$ which is the sum of all possible products containing one of each matrix. Note that every one of these products corresponds to a permutation of these matrices that determines the ordering. This means that the $\ell$-anticommutator can be written as 
%\[
%\sum_{\sigma\in S_\ell}A_{\sigma(1)}A_{\sigma(2)}...A_{\sigma(\ell)}.
%\]
%\end{definition}
By employing the same method as in the proof of Lemma \ref{GOE-GOE moment recurrence}, we are able to obtain a recurrence relation for the moments of the $\ell$-anticommutator. Now, however, instead of two interdependent recurrence relations, we have $\ell$ interdependent recurrence relations. We state the results below and leave the details of the proof in Appendix
\ref{AppendixMomentsAnti}.

%\begin{theorem}
%For sequences $f_0,f_1,\dotsc,f_\ell$ defined such that $f_0(0)=1$, $f_1(0)=f_2(0)=\cdots = f_\ell(0) = 0$, and $f_0(1)=f_1(1)=f_2(1)=\cdots=f_\ell(1)=1$, and with recurrence relations for $k>1$ given by
%\[
%f_\ell(k)= k!\cdot f_0(k-1)
%\]
%\[
%f_{\ell-1}(k)=f_\ell(k)+\sum_{\substack{0\leq x_1,x_2<k-1\\ x_1+x_2<k-1}} (\ell-1)!(1+(\ell!-1)\cdot \mathbb{1}_{x_1>0})(1+(\ell!-1)\cdot \mathbb{1}_{x_2>0})f_0(x_1)f_0(x_2)f_1(k-x_1-x_2-1)
%\]
%and for any $0< m<\ell-1$ the recurrence
%\[
%f_m(k)=f_{m+1}(k)+\sum_{\substack{1\leq x_1,x_2<k\\ x_1+x_2<k}}(\ell-m)!(m-1)!f_m(x_1)f_m(x_2)f_{\ell-m+1}(k-x_1-x_2)
%\]
%and finally
%\[
%f_0(k)=f_1(k)+k!\sum_{j=1}^{k-1}f_1(j)f_0(k-j),
%\]
%the $2k^{th}$ moment of the $\ell$-anticommutator is 
%\[
%M_{2k}=\ell! \cdot f_0(k)
%\]
%\end{theorem}

%\begin{proof}
%The proof follows in the same way as with the $2$-anticommutator. We get that if two terms are matched together, their indices must sum to $1$ mod $2\ell$. Moreover, $f_m(k)$ counts the number of sequences of length $2k\ell$ such that the first $\ell$ terms are $A_1A_2\cdots A_\ell$ and the first $m$ terms are matched with the last $m$ terms, which must be in the same \textcolor{red}{(reverse?)} order. Then, if $A_{m+1}$ (the $m+1^{\textup{th}}$ term from the front) is matched to the $m+1^{\textup{th}}$ term from the end, the number of such configurations is counted by $f_{m+1}(k)$, which is the first term in our expression for $f_m(k)$. Otherwise, $A_{m+1}$ must be matched to a term in the middle, of index $2\ell x_1-m$, with the number of internal matchings between them given by $f_m(x_1)$. Similarly, the $m+1^{\textup{th}}$ term from the end must be matched to a term with index $2\ell (k-x_2)+m$, leading to $f_m(x_2)$ possibilities. We add an extra factor of $(\ell-m)!$ since the final $\ell-m$ terms can be ordered arbitrarily (with the final $m$ terms fixed due to their being matched with the first $m$ terms). There are $2m+2 \mod{2k}$ terms between $2\ell x_1-m$ and $2\ell (k-x_2)+m+2$, which is equivalent to fixing $\ell-m + 1$ terms on the outside, yielding a factor of $f_{\ell-m+1}(k-x_1-x_2)$. The $m-1$ terms on the inside can be arranged in $(m-1)!$ ways which completes the summand in the second component of $f_m(k)$. Note that the $f_{\ell-1}(k)$ case is slightly different but works in the same way as described in the $2$-anticommutator case. Furthermore, the $f_0$ case works in exactly the same way, thus proving the formula for the recurrence.

%Observe that $f_0(k)$ represents the number of non-crossing partitions with $2k\ell$ terms where the first $\ell$ terms are fixed to be $A_1A_2\cdots A_\ell$. Applying any permutation to these $l$ terms would preserve the non-crossing property of these partitions and count \textit{all} possible non-crossing partitions with $2k\ell$ terms. We know from earlier lemmas that the even moments are given by precisely this number of non-crossing partitions and thus we arrive at $M_{2k}=\ell! \cdot f_0(k)$.
%\end{proof}

\subsection{Moments of $\{\textup{PTE, PTE}\}$}\label{momentsPTPT} The palindromic Toeplitz ensemble is introduced by Massey-Miller-Sinsheimer in \cite{palindromicToeplitz} to remove the Diophantine obstruction encountered in the moment calculation of Toeplitz ensemble in \cite{Toeplitz}. Essentially, the additional symmetry in the structure of palindromic Toeplitz allows almost all free matching of all the terms in the cyclic product to have consistent choice of indexing and contribute in the limit. Interestingly, the $2m$\textsuperscript{th} moment of palindromic Toeplitz is $(2m-1)!!$, which is exactly the $2m$\textsuperscript{th} moment of standard Gaussian. The moment calculation of palindromic Toeplitz can be naturally extended to that of $\{\textup{Palindromic Toeplitz, Palindromic Toeplitz}\}$. By standard argument, the odd moments of $\{\textup{palindromic Toeplitz, palindromic Toeplitz}\}$ vanish in the limit. For even moments, we can view each cyclic product in the $2m$\textsuperscript{th} moment of $\{\textup{palindromic Toeplitz, palindromic Toeplitz}\}$ as a cyclic product in the $4m$\textsuperscript{th} moment of palindromic Toeplitz. Thus, the matching in each cyclic product is again free, giving us the following genus expansion formula:
\begin{align}\label{genusPalindromicPalindromic}
M_m \ = \ \lim_{N\rightarrow \infty}M_m(N) \ = \ \sum_{C\in\mathcal{C}_{2,2m}}\sum_{\pi_C\in P_{2,C}(2m)}1.
\end{align}

\begin{theorem}
The $2m$\textsuperscript{th} moment $M_{2m}$ of $\{\textup{PTE, PTE}\}$ is $2^m((2m-1)!!)^2$.
\end{theorem}
\begin{proof}
Since there are the number of $(2, m)$ configurations is $2^m$ and each configuration has $(2m-1)!!$ ways of matching up the $a$'s and $(2m-1)!!$ ways of matching up the $b$'s, then by \eqref{genusPalindromicPalindromic} we have
\begin{align}
M_{2m} \ = \ 2^m((2m-1)!!)^2.
\end{align}
\end{proof}

%\textbf{Claim:} If $A,B$ are both palindromic Toeplitz matrices as defined above the moments of the anticommutator, $AB+BA$, denoted $M_k$ for the $k$th moment are $0$ if $k$ is odd and $2^k\cdot ((k-1)!!)^2$ if $k$ is even.

%\textbf{Proof Sketch:} For $k$ even we can just use $2k$ instead and prove that $M_{2k}=2^{2k}\cdot ((2k-1)!!)^2$. By the eigenvalue trace lemma we know that $M_k=\frac{1}{N^{k+1}}\mathbb{E}[\text{Tr}((AB+BA)^k)]$ and by binomial expansion we see that this can be split up into terms of the form $\mathbb{E}[\text{Tr}(ABBABAAB...BAAB)]$ with $k$ pairs of $AB$s or $BA$s in a row. Also we can ignore the $N^{k+1}$ term until we normalize at the end. In any case we see that $\mathbb{E}[\text{Tr}(ABBABAAB...BAAB)] $ \newline  $=\sum_{1\leq i_1,i_2,...,i_{2k}\leq N}\mathbb{E}[a_{i_1,i_2}b_{i_2,i_3}b_{i_3,i_4}a_{i_4,i_5}...a_{i_{2k-1},i_{2k}}b_{i_{2k},i_1}]$. Clearly, since all of the matrix entries are assumed to have mean $0$, we need to have everything in pairs at least and if any of them are in a triple or more we see that the number of degrees of freedom is $\leq \frac{2k-1}{2}+1$ where the $2k-1$ comes from having $2k-3$ indices in pairs and at least one triple that adds $+1$ and another $+1$ from the choice of the first term $i_1$, so this proves that these must be pairs. Note that the indices being $a$ and $b$ doesn't matter here but it will when we need to count pairs.

%So now we see that for odd moments we have products of the form $ABABBABA...AB$ where the number of $A$ terms is odd so in our formula $\mathbb{E}[\text{Tr}(ABBABAAB...BAAB)]=\sum_{1\leq i_1,i_2,...,i_{2k}\leq N}\mathbb{E}[a_{i_1,i_2}b_{i_2,i_3}b_{i_3,i_4}a_{i_4,i_5}...a_{i_{2k-1},i_{2k}}b_{i_{2k},i_1}]$ we have an odd number of $a_{i_\ell,i_{\ell+1}}$ terms so we cannot properly pair these and the $a$ terms are independent of the $b$ terms so we cannot have crossing pairs so this proves that all of the odd moments are $0$.

%Now we can move to even moments. Here we can write $M_{2k}$ and we see that we have a sum of the form $\mathbb{E}[\text{Tr}(ABBABAAB...BAAB)]=\sum_{1\leq i_1,i_2,...,i_{4k}\leq N}\mathbb{E}[a_{i_1,i_2}b_{i_2,i_3}b_{i_3,i_4}a_{i_4,i_5}...a_{i_{4k-1},i_{2k}}b_{i_{4k},i_1}]$ and if we have these properly paired up they would contribute $1$ since the variance of all of these terms is assumed to be $1$. So we can count the number of ways to pair up terms. We know that $a$ terms can only be paired with other $a$ terms and there are $2k$ of these terms so it is well known that there are $(2k-1)!!$ ways to pair up these terms, similarly for the $b$ terms there are also $(2k-1)!!$ ways to pair. Once we have these paired up we can assign the actual indices $i_1,i_2,...,i_{4k}$ and from the Toeplitz matrix paper we see that the number of ways to assign these indices given the pairs is on the order of $1N^{k+1}$ with coefficient $1$, note that this step is independent of choices of $A$s and $B$s so we can lift this argument directly. So this means that for a specific list of ABABAB...BA we have $\mathbb{E}[\text{Tr}(ABBABAAB...BAAB)]=((2k-1)!!)^2\cdot N^k$. Now note that there are a total of $2^{2k}$ configurations since we are expanding $(AB+BA)^{2k}$ so inductively there are $2^n$ terms in $(AB+BA)^n$ and the expansion of $(AB+BA)^{n+1}$ are just the configurations for $(AB+BA)^{n}$ but with an $AB$ or $BA$ at the end which multiplies a factor of $2$. So adding these together we get that $M_{2k}=\frac{1}{N^k}\mathbb{E}[\text{Tr}(AB+BA)^k]=\frac{2^{2k}}{N^{k+1}}\sum_{1\leq i_1,i_2,...,i_{4k}\leq N}\mathbb{E}[a_{i_1,i_2}b_{i_2,i_3}b_{i_3,i_4}a_{i_4,i_5}...a_{i_{4k-1},i_{2k}}b_{i_{4k},i_1}]=2^{2k}\cdot ((2k-1)!!)^2$ so this proves the moments of this distribution.

\subsection{Moments of $\{\textup{GOE, PTE}\}$}\label{subsectionGOEpalindromicToeplitz}
So far, we've only been looking at \textbf{homogeneous} anticommutator ensembles $\{A_N, B_N\}$, i.e., $A_N$ and $B_N$ are the same ensembles. Genus expansions of $\{\textup{GOE, GOE}\}$ and $\{\textup{PTE, PTE}\}$ suggest that in general, genus expansion of a homogeneous anticommutator ensemble $\{A_N, B_N\}$ is an easy generalization of the genus expansion of $A_N$ (or $B_N$). A natural question to ask is: what does genus expansion of an \textbf{inhomogeneous} anticommutator ensembles $\{A_N, B_N\}$ (i.e., $A_N$ and $B_N$ are different ensembles) look like? In this section, we turn our attention to an inhomogeneous anticommutator ensemble, namely $\{\textup{GOE, PTE}\}$. Interestingly, we shall see that the matching properties of GOE and palindromic Toeplitz are well preserved under the anticommutator operator, that is, the contributions to the moments of $\{\textup{GOE, PTE}\}$ in the limit come solely from non-crossing matchings of the GOE terms and free matchings of the palindromic Toeplitz terms that don't cross the matchings of the GOE terms.

Similar to the previous examples, we have that the $m$\textsuperscript{th} moment of $\{A_N, B_N\}$ is given by
\begin{align}\label{expansionGOEPT}
M_m(N) \ = \ \frac{1}{N^{m+1}}\sum_{C\in\mathcal{C}_{2, 2m}}\sum_{1\leq i_1,\cdots, i_m\leq N}\mathbb{E}[c_{i_1i_2}c_{i_2i_3}\cdots c_{i_{2m}i_1}].
\end{align}


%To find the moments of the anticommutator of a GOE and a palindromic Toeplitz, for the section we assume that $A$ is a GOE and $B$ is a palindromic Toeplitz random matrix then we consider $AB+BA$ and find its moments. We can use the eigenvalue trace lemma for $\text{Tr}(AB+BA)^k=\sum_{1\leq i_1,i_2,...,i_{2k}\leq N}c_{i_1,i_2}c_{i_2,i_3}...c_{2k,1}$ where the $c$s are $a$s or $b$s where every pair of consecutive terms is $ab$ or $ba$ in order. Then to find the nonzero contributions we must pair up these terms such that $a$ terms are only matched with other $a$s and same with $b$s. In the case of $\{\text{GOE,GOE}\}$ we know that from proposition \ref{non-crossing} the matchings for the GOE must be non-crossing while in the case of $\{\text{Palindromic Toeplitz,Palinddromic Toeplitz}\}$, from \cite{palindromicToeplitz} the matchings are free to cross in all but $O\left(\frac{1}{N}\right)$ cases which tends to $0$. For the case of the anticommutator we show that any pair of matchings between $b$s can cross, but any matching between $b$s cannot cross and matching of $a$s and no pairs of matchings between $a$s can cross.

%\begin{definition}\textbf{(Layers)}
%For any matching of the $a$ terms in our cyclic product, a layering of $b$s is a partition of all of the $b_{i_j,i_{j+1}}$ terms into layers $B_1\cup B_2\cup\cdots \cup B_\alpha$ such that any $b$s in the same layer $B_i$ can be matched such that any pair of $b$ terms in $B_i$ can be matched without crossing any of the $a$ matchings and $B_i$ is the maximal such mathcing
%\end{definition}

\begin{definition}
For positive integers $n$ and $m$, let $C=c_1c_2\cdots c_{2m}$ be a $(2, m)$-configuration, where $c_i\in\{a_i, b_i\}$ for all $1\leq i\leq 2m$ under the restriction that $(c_{2s-1},c_{2s})\in \{(a_{2s-1}, b_{2s}), (b_{2s-1}, a_{2s})\}$ and $S\subseteq [2n]$ be the set of all the indices of the $a$'s. Let $\pi_S$ be a partition of $S$. Then a \textbf{layer of $[2n]$ with respect to $S$} is a maximal subset $B^{(i)}_S\subseteq [2n]\setminus S$ such that for any $j, k\in B^{(i)}_S$, there doesn't exist $(p, q)\in \pi_S$ such that $j<p<k<q$ or $p<j<q<k$. It's clear from definition that distinct layers must be disjoint. Then we denote the union of all the layers with respect to $S$ by $B_S:=\cup_{i=1}^t B^{(i)}_S$, where $t$ is the total number of layers.
\end{definition}

\begin{lemma}\label{PT-GOE lemma for bs}
%Consider the expansion of $(AB+BA)^{2k}=\sum_{1\leq i_1,i_2,...,i_{2k}\leq N}c_{i_1,i_2}c_{i_2,i_3}...c_{2k,1}$ as described in at the start of this subsection. For any matching of $a$s that doesn't cross itself and corresponding layering $B_1\cup B_2\cup\cdots \cup B_\alpha$, over all matchings of $b$s the maximal number of ways to choose all of the indices for the $b$ terms is $N^{k+\alpha}+O(N^{k+\alpha-1})$ which is achieved by matching all of the $b$s within the same layer together.

Consider a cyclic product in \eqref{expansionGOEPT}. Let $S$ be the set of all the indices of the $a$'s, $\pi_S$ be a matching of the $a$'s and $\pi_{[2n]\setminus S}$ be a matching of the $b$'s. If $\pi_S$ is non-crossing, then the matching $\pi_S\circ\pi_{[2n]\setminus S}$ contribute to \eqref{expansionGOEPT} in the limit if and only if there exists $i$ such that $j, k\in B^{(i)}_S$ for each $(j, k)\in \pi_{[2n]\setminus S}$, i.e. every layer is matched within itself. For the $2m^{th}$ moment, the number of ways to assign indices for all the $t$ layers is $N^{m+t}+O(N^{m+t-1})$.
\end{lemma}

\begin{proof}
%We use the approach of \cite{palindromicToeplitz} and consider the equations we get from matching all of the $b$s given the matching of $a$s. First we see that for every layer of $b$s together they are bounded by the $a$s on the outside. In particular a layer with $2m$ total $b$s can be written as 
%\begin{align}
%a_{i_{j_1},i_{j_1+1}}b_{i_{j_1+1},i_{j_1+2}}b_{i_{j_2},i_{j_2+1}}a_{i_{j_2+1},i_{j_2+2}}a_{i_{j_3},i_{j_3+1}}b_{i_{j_3+1},i_{j_3+2}}b_{i_{j_4},i_{j_4+1}}\dots b_{i_{j_{2m},i_{j_{2m}+1}}}a_{i_{j_{2m}+1},i_{j_{2m}+2}}.
%\end{align}

%As shown in the picture the matchings for $a$s carry through the $b$s since whenever we have $a_{i_m,i_{m+1}}$ matched with $a_{i_n,i_{n+1}}$ we get that $i_n=i_{m+1}$ and $i_m=i_{n+1}$ because otherwise there are too many degrees of freedom lost total using the same reasoning used in \cite{palindromicToeplitz}. So then from the matchings of $a$s shown in the picture, $i_{j_1+1}=i_{j_{2m}+1}$, $i_{j_2+1}=i_{j_3+1}$, and generally $i_{j_{\ell}+1}=i_{j_{\ell+1}+1}$ for $j_{2m+1}=j_1$. Similarly we get also have $i_{j_1+2}=i_{j_2}$ and this is also extended to $i_{j_\ell+2}=i_{j_{\ell+1}}$. So this shows that choosing the indices for any single layer is a cyclic product $b_{i_{j_1+1},i_{j_2}}b_{i_{j_2},i_{j_3+1}}b_{i_{j_3+1},i_{j_4}}...b_{i_{j_{2m}},i_{j_1+1}}$. So choosing these indices of this layer is equivalent to just choosing the indices of any cyclic product of $2m$ which is $N^{m+1}$ from \cite{palindromicToeplitz}.

First, observe that each layer $B^{(i)}_{S}$ can be thought of as a cyclic product. For example, consider the following layer $B^{(i)}_S$ consisting of $2\ell$ $b$'s. For clarity, we include some of the $a$'s to highlight how the matching the $a$'s give rise to the cyclic product:
\begin{align}
& a_{i_{j_1-1}i_{j_1}}b_{i_{j_1}i_{j_1+1}}a_{i_{j_1+1}i_{j_1+2}}\cdots a_{i_{j_2-1}i_{j_2}}b_{i_{j_2}i_{j_2+1}}a_{i_{j_2+1}i_{j_2+2}}\cdots \nonumber \\&a_{i_{j_3-1}i_{j_3}}b_{i_{j_3}i_{j_3+1}}a_{i_{j_3+1}i_{j_3+2}}\cdots b_{i_{j_{2\ell}}i_{j_{2\ell}+1}}a_{i_{j_{2\ell}+1}i_{j_{2\ell}+2}}\cdots.
\end{align}

Since the $b$'s form a layer, then for every neighboring two $b$'s, the inner adjacent two $a$'s must be paired together. For example, $a_{i_{j_1+1}i_{j_1+2}}$ and $a_{i_{j_2-1}i_{j_2}}$, which are adjacent $b_{i_{j_1}i_{j_1+1}}$ and $b_{i_{j_2}i_{j_2+1}}$, must be paired together to ensure that all the $b$'s form a layer. Hence, the indices must satisfy the relations $i_{j_1}=i_{j_{2\ell+1}}$, $i_{j_1+1}=i_{j_2}$, $i_{j_2+1}=i_{j_3}$, $\dots$, $i_{j_{2\ell-1}+1}=i_{j_{2\ell}}$, which allows us to think of $B^{(i)}_S$ as $b_{i_1i_2}b_{i_2i_3}\cdots b_{i_{2\ell}i_1}$. Let $\#(B^{(i)}_S)$ be the number of $b$'s in the layer $B^{(i)}$. For each cyclic product, if the matching of all the $b$'s is within each layer, then the number of ways to choose indices for all the $b$'s is $\prod_{i=1}^t (N^{\#(B^{(i)}_{S})/2+1}+O(N^{\#(B^{(i)}_S)/2}))=N^{m+t}+O(N^{m+t-1})$ by \cite{palindromicToeplitz}.

%So now if we suppose that all of the matchings are within each layer, then the number of ways to choose all of these indices is $\prod_{i=1}^\alpha (N^{m(B_i)/2+1} + O(N^{m(B_i)/2}))$ where $m$ is the total number of $b$s divided by $2$ (here we assume for the sake of simplicity that all of the $m$ terms are even but this will be shown in the next paragraph. So taking this product we see that the total number of ways to assign these indices is on the order of $N^{\alpha+\sum_i m(B_i)/2}$ but since the $B_i$ all partition the $b$s there are always $2k$ total $b$s which means that the sum is $N^{k+\alpha-1}$ plus lower order terms which we can ignore.

We move on to the case where the matchings of the $b$'s are across different layers. Now, for two arbitrary layers $B^{(i_1)}_S$ and $B^{(i_2)}_S$, suppose that all the $b$'s are paired within these two layers except for at least two $b$'s that are paired across these two layers. Due to the special structure of palindromic Toeplitz, if $b_{i_ji_{j+1}}$ and $b_{i_ki_{k+1}}$ are paired together, then the indices must satisfy the equation $i_{j+1}-i_j+i_{k+1}-i_k=C_j$ for some $C_j\in\{0, \pm (N-1)\}$. Hence, similar to \cite{palindromicToeplitz}, we can think of the matching of all the indices as a system of $M:=(m(B^{(i_1)}_S)+m(B^{(i_2)}_S)/2$ equations, where each index appears exactly twice. After labeling these equations, we pick any equation as $\textup{eq}(M)$, and choose an index that has occured only once. Then, we select the equation in which this index first appeared and label this equation as $\textup{eq}(M-1)$. This index is one of our dependent indices and guarantees consistency choice of indices for the other indices in $\textup{eq}(M-1)$. We can continue this process, and at stage $s$, if at least one index of $\textup{eq}(M-s)$ has occured only once among $\textup{eq}(M-s), \textup{eq}(M-s+1), \dots, \textup{eq}(M)$, then we can choose such an index as one of our dependent indices and continue this process. The only way to terminate this process at stage $s<M-1$ is for all the indices among $\textup{eq}(M-s), \textup{eq}(M-s+1), \dots, \textup{eq}(M)$ to occur twice, which implies that each layer is paired within itself, a contradiction. Hence, if at least two $b$'s are paired across these two layers, then the number of dependent indices is $M/2-1$ and the number of ways to choose indices for all the $b$'s is $N^{M/2+1}$. This is a lower order term compared to the case where each layer is paired within itself, which gives that the number of ways to choose indices for all the $b$'s is $N^{M/2+2}$.

%Now we can consider the case where there are matchings of $b$s across different layers. First suppose that we have two layers $B_1,B_2$ where all $b$s are matched within the two layers but there is at least one matching such that there is one $b$ in $B_1$ and one in $B_2$. Then we use the method used by \cite{palindromicToeplitz} and write the cyclic products $b_{i_{1,1},i_{1,2}}b_{i_{1,2},i_{1,3}}b_{i_{1,3},i_{1,4}}...b_{i_{1,m(B_1)},i_{1,1}}b_{i_{2,1},i_{2,2}}b_{i_{2,2},i_{2,3}}...b_{i_{2,m(B_2)},i_{2,1}}$ then we get the that there are $m(B_1)+m(B_2)$ equations determined by the indices and the $C_j$ terms as defined in the definition of palindromic Toeplitz matrices as defined in \ref{Def-palindromicToeplitz} that we need to find the number of solutions for. Then we can use the method in the proof of lemma 2.9 in \cite{palindromicToeplitz} where we label the equations starting with $\text{eq}((m(B_1)+m(B_2))/2)$ and then choosing the previous equation $\text{eq}((m(B_1)+m(B_2))/2-1)$ that shares one index and choosing $C_j$ depending on the sum of the other indices. Then we can choose $\text{eq}((m(B_1)+m(B_2))/2-2)$ as the equation sharing one index, note that we can always choose such indices because the pairs of indices for equations are $(i_{1,1},i_{1,2}),(i_{1,2},i_{1,3}),...,(i_{1,m(B_1)},i_{1,1}),(i_{2,1},i_{2,2}),(i_{2,2},i_{2,3}),...,(i_{2,m(B_2)},i_{2,1})$ and every index is in two pairs and from the proof of lemma 2.9 in \cite{palindromicToeplitz} the only way to terminate this process before they are chosen is to either have all of $(i_{1,1},i_{1,2}),(i_{1,2},i_{1,3}),...,(i_{1,m(B_1)},i_{1,1})$ or $(i_{2,1},i_{2,2}),(i_{2,2},i_{2,3}),...,(i_{2,m(B_2)},i_{2,1})$ but this implies that all the matchings are within the layers which contracts our assumption. Therefore we can choose equations this way and then the lemmas 2.10 and 2.11 from \cite{palindromicToeplitz} apply directly and we see that the number of ways to choose the indices is $O(N^{(m(B_1)+m(B_2))/2+1})$, however, if we matched within layers we get that the number of ways to choose indices is $N^{(m(B_1)+m(B_2))/2+2} + O(N^{(m(B_1)+m(B_2))/2+1})$, this logic naturally extends to having matching $b$s in many different layers and proves that it suffices to match within individual layers. Therefore the lengths of the layers must be even since otherwise we wouldn't be able to match all of the $b$s within layers.

%Finally we consider the case where some matchings of $a$s are crossing. Assume that we have one crossing of $a$s then we see that the indices of the layers are no longer matched in equations so it is impossible to have a consistent matching within the layer since the indices no longer match, so then we see that one matching creates three layers, $B_1,B_2,B_3$ in which we can match the equations but this gives the total number of ways to assign the indices for $B$ as $N^{(m(B_1)+m(B_2)+m(B_3))/2+1}$ which means that the total number of matchings in this case is $O(N^{k+\alpha-1})$ which means that having any crossing for $a$s automatically decreases the power of $N$ so this lemma combined with the next one shows that we cannot have any crossings of $a$s. \textcolor{red}{If this is unclear you can write the example of abbaabbaabba where the first a is matches with the 4th, the 2nd with the 5th and 3rd with 6th then all of the bs have to be combined into 6 equations together}

Finally, we consider the case where the matchings of the $a$'s cross each other. If a matching of two $a$'s cross another matching of two $a$'s, then we automatically have three layers $B^{(i_1)}_S, B^{(i_2)}_S$, and $B^{(i_3)}_S$. Due to the mismatch, we can no longer view different layers as independent cyclic products, but all three layers as a single cyclic product. The total number of ways to assign the indices for the three layers is $N^{(m(B^{(i_1)}_S)+m(B^{(i_2)}_S+m(B^{(i_3)}))/2)+1}$, which is a lower order term compared to the case where each of the three layers is matched within itself. Thus, the number of ways to assign indices for all the layers is $O(N^{m+t-1})$, which is again a lower order term. 
\end{proof}

\begin{lemma}\label{PT-GOE lemma for as}
With the same notation as in Lemma \ref{PT-GOE lemma for bs}, regardless of whether $\pi_S$ is non-crossing or not, the number of ways to assign the remaining indices for the $2m$\textsuperscript{th} moment is $N^{m+1-t}+O(N^{m-t})$.
\end{lemma}

\begin{proof}
For a fixed $m$, when a cyclic product has only one layer, the only possible configurations for the cyclic product are $abba\cdots abba$ or $baab\cdots baab$; moreover, all the $a$'s must be matched in adjacent pairs. Since there is one free index for each adjacent pair of $a$, then the number of ways to assign the remaining indices is $N^{m}=N^{(m+1)-1}$. This proves the base case.

%We can prove this by induction on the number of total layers for a fixed $k$. Since we handle all of the indices that are in any of the $b$ terms in lemma \ref{PT-GOE lemma for bs} it suffices to count the number of choices for indices that are between adjacent pairs of $a$s, in particular it suffices to show that for every extra layer we get the number of choices for there . If there is one layer then all of the $a$s must be paired to adjacent $a$s because otherwise if we pair some $a$s with any $b$s between them they wouldn't be able to match with any $b$s on the outside. So there are $k$ pairs of $a$s so there are $k$ total internal indices and each of them are free since the $a$s are all matched to the one adjacent which gives $N^k=N^{(k+1)-1}$ choices for the internal indices which proves our base case. Note that from this description it shows that every layer is of the form $baabbaabbaab\dots baab$ or of the form $abbaabbaabba...abba$

%Now, we can consider how we combine different layerings. We can consider this in two cases crossing and non-crossing. In the case where we add a layer such that no pair of $a$s cross we prove that the number of ways to choose the indices between $a$s is always the desired $N^{(k+1)-\alpha}+O(N^{k-\alpha})$ but if there is a crossing then the number of ways to choose the indices is at most $N^{(k+1)-\alpha}$ which combined with the previous lemma shows that we cannot have any crossings in our matchings of $a$ terms.

When a cyclic product $C$ has two layers $B^{(1)}_S$ and $B^{(2)}_S$, suppose that layer $B_S^{(1)}$ is contained in the cyclic product $C_1$ with $2k_1$ total $b$'s and layer $B_S^{(2)}$ is contained in the cyclic product $C_2$ with $2k_2$ total $b$'s, where $C_1\cap C_2=\emptyset$. We can think of $C$ as inserting $C_2$ into $C_1$. From the base case, $C_1$ and $C_2$ are either $abba\cdots abba$ or $baab\cdots baab$. Then, without loss of generalilty, suppose that $C_1$ has the configuration $abba\cdots abba$. If $C_2$ is inserted between two $a$'s in $C_1$, then it must have the configuration $abba\cdots abba$, otherwise $C$ has only one layer instead of two layers. Let $C_2$ be $a_{i'_{1}i'_{2}}b_{i'_{2}i'_{3}}\cdots b_{i'_{4k-1}i'_{4k}}a_{i'_{4k}i'_{1}}$ and surrounded by $a_{i_\ell i_{\ell+1}}$ and $a_{i_{\ell+1}i_{\ell+2}}$ in $C_1$. Since $a_{i'_{1}i'_{2}}$ and $a_{i'_{4k}i'_{1}}$ as well as $a_{i_\ell i_{\ell+1}}$ and $a_{i_{\ell+1}i_{\ell+2}}$ are no longer adjacent, then we lose one additional degrees of freedom and the number of ways to assign the remaining indices is $k_1+k_2-1$. If $C_2$ is inserted between two $b$'s in $C_1$, then it can either be $abba\cdots abba$ or $baab\cdots baab$. Similarly, we can see that the number of ways to assign the remaining indices is $k_1+k_2-1$. Similar constructions follow when we have an arbitrary number of layers in the cyclic product, and whenever we get another layer we lose one extra degree of freedom, giving us $N^{(m+1)-t}+O(N^{m-t})$ ways of assigning the remaining indices for $t$ layers.

%is of the configuration $baab\cdots baab$, Consider the non-crossing case. First we consider combining two layers of with $k_1$ total indices determined by adjacent $a$s and $k_2$ indices determined by adjacent $a$s respectively, first we can write the layer of length $k_1$ in cyclic product form as $a_{i_{1,1},i_{1,2}}b_{i_{1,2},i_{1,3}}b_{i_{1,3},i_{1,4}}a_{i_{1,4},i_{1,5}}\dots b_{i_{1,4k_1-1},i_{1,4_{k_1}}}a_{i_{1,4k_1},i_{1,1}}$ and similarly for the $k_2$ configuration then we can put the layer with $k_2$ pairs of adjacent $a$s into this configuration. Note that by the base case we can put this in either the $baab$ orientation of the $abba$ orientation, note that if we put the $baab$ orientation between two $b$s we just get a single layer with $k_1+k_2$ instead of two distinct layers so we discard this case. So now we consider the cases, if we put the $k_2$ layer in between two $b$s they must be in the $abba$ orientation but since the $a_{i_{2,1},i_{2,2}}$ term and $a_{i_{2,4k_2},i_{2,1}}$ no longer have $i_{2,1}$ automatically matched we get that there are $N^{k_1+k_2-1}$ ways to assign indices in this case. Similarly in the $abba$ orientation if we put the $k_2$ layer between two $a$s we see that both the $a_{i_{2,1},i_{2,2}}$ term and $a_{i_{2,4k_2},i_{2,1}}$ no longer have $i_{2,1}$ automatically matched and the $a_{i_{1,4\ell},i_{1,4\ell+1}}$ and $a_{i_{1,4\ell+1},i_{1,4\ell+2}}$ no longer have $i_{1,4\ell+1}$ adjacent but note that in this case if the $k_2$ layer is in $abba$ orientation we get that we automatically have $i_{1,4\ell+1}=i_{2,1}$ by adjacency so we only lose one extra degree of freedom total so we still get $k_1+k_2-1$ degrees of freedom. If we are instead in the $baab$ orientation we get that the cyclic product is $b_{i_{1,1},i_{1,2}}a_{i_{1,2},i_{1,3}}a_{i_{1,3},i_{1,4}}b_{i_{1,4},i_{1,5}}\dots a_{i_{1,4k_1-1},i_{1,4_{k_1}}}b_{i_{1,4k_1},i_{1,1}}$ so we get that the internal matchings for the $a$s in the $k_2$ layer are preserved while there is one power of $N$ lost since we no longer have the $i_{1,4\ell+1}$ terms matched in the $k_1$ layer which also gives $k_1+k_2-1$. For any number of layers this combination construction follows and shows that every time we get another layer we lose an extra degree of freedom proving that we always have $N^{(k+1)-\alpha}+O(N^{k-\alpha})$ choices for the indices between adjacent $a$s.
\end{proof}

By \eqref{PT-GOE lemma for bs} and \eqref{PT-GOE lemma for as}, for the $2m$\textsuperscript{th} moment, the number of ways to assign all the indices is $N^{2m+1}+O(N^{2m})$ when $\pi_S$ is non-crossing and every layer is matched within itself, and $O(N^{2m})$ otherwise. In other words, in the limit, the only contributions come from non-crossing matchings of the $a$'s and matchings of the $b$'s within the same layer. This leads us to Theorem \ref{sigmarecurrence}, as follows.

\begin{theorem}\label{sigmarecurrence}
The $2m$\textsuperscript{th} moment of $\{\textup{GOE, PTE}\}$ is given by $\sigma_{2m, 0, m}$, where $\sigma_{n, s, k}$ is given by the conditions
\begin{enumerate}
\item $\sigma_{n, s, k}=0$ if $k<0$,
\item $\sigma_{n, s, k}=0$ if $s+k>n$,
\item $\sigma_{n, s, 2k+1}=0$,
\item $\sigma_{n, s, 0}=(2n-1)!!$,
\end{enumerate}
and the recurrence relation
\begin{align}
\sigma_{n, s, 2k} \ = \ \sum_{p=s+1}^n \sum_{q=p+1}^n \sum_{r=0}^{2k}\left[
\sigma_{n-q+p, p, r}\cdot\sigma_{q-p-1, 0, 2k-2-r}+\sigma_{n-q+p-1, p-1, r}\cdot\sigma_{q- p, 1, 2k-2-r}
\right].
\end{align}
\end{theorem}

\begin{proof}
Let $\sigma_{n, s, k}$ be the total number of matchings of any cyclic products of $a$'s and $b$'s of length $2n$ that starts with at least $s$ adjacent pairs of $bb$ and has $k$ adjacent pairs of $ab$ and $ba$ in total. It's clear that the $2m$\textsuperscript{th} moment of $\{\textup{GOE, PTE}\}$ is given by $\sigma_{2m, 0, m}$ and conditions (1), (2), (3) trivially follows from the definition. Moreover, since $\sigma_{n, s, 0}$ is the number of matchings of cyclic products of $b$'s of length $2n$, where the matchings of $b$'s are free. Then, $\sigma_{n,s,0}=(2n-1)!!$. 
%Note that from a previous lemma \ref{} we know that if $A$ is a GOE and $B$ is a palindromic Toeplitz matrix then in our expansion for the $2m$th moment $c_{i_1,i_2}c_{i_2,i_3}...c_{i_{4m},i_1}$ the $a$s must be paired up such that they are non crossing and the $b$s are paired up so they can cross each other but they cannot cross any of the $a$s. I claim that $\sigma_{2n,s,\ell}$ is the number of possible matchings (given the restriction described in the previous sentence) of strings of length $2n$ made up of $AB,BA,$ or $BB$ with $\ell$ of the pairs of $AB$, or $BA$ terms. So if we can find $\sigma_{4m,2m}$ it gives the possible matchings of the expansion for $(AB+BA)^m$ and therefore gives the $m$th moment as desired. So it suffices to prove that $\sigma_{2n,\ell}$ correctly counts the given property.
%If we have a list of $n$ elements where each element is an $AB,BA$, or $BB$ with $\ell$ terms that are $AB,BA$. Note that for $\ell=0$ the matchings of $B$ are free so the number of ways to pair is simply $(2n-1)!!$ which proves the base case. The other initial conditions follow by definition so now it suffices to prove the recurrence relation. We can do this by considering where the first $A$ matches and count the matchings between the first $A$ and its matching and the matchounts outside of the first pairs of $A$s since we know that they cannot cross.

Now, we move on to prove the recurrence relation for $\sigma_{n, s, 2k}$. Suppose that the $p$\textsuperscript{th} adjacent pair is the first occurrence of $ab$ or $ba$ pair and that the $a$ is paired with another $a$ in the $q$\textsuperscript{th} block. Since no matchings can cross the matching of two $a$'s, then if the $p$\textsuperscript{th} block is $ab$, the $q$\textsuperscript{th} block must be $ba$, and vice versa. In both cases, the matching of the two $a$'s split the cyclic product into two smaller cyclic product, as illustrated in the following example:

\begin{example}
If $(n, s, k)=(5, 1, 2)$, then an example of a cyclic product with $p=3$ and $q=5$ is $b_{i_1i_2}b_{i_2i_3}b_{i_3i_4}b_{i_4i_5}b_{i_5i_6}a_{i_6i_7}b_{i_7i_8}b_{i_8i_9}a_{i_9i_{10}}b_{i_{10}i_1}$. The matching of the $a$'s partitions the cyclic product into two smaller cyclic products $b_{i_7i_8}b_{i_8i_9}$ (inner cyclic product) and $b_{i_{10}i_1}b_{i_1i_2}b_{i_2i_3}b_{i_3i_4}b_{i_4i_5}b_{i_5i_6}$ (outer cyclic product), where a term from either smaller cyclic product is paired with another term in the same smaller cyclic product.
\end{example}

%We move on to prove the recurrence relation. Let $p$ be the 
    %first occurence of any block that has an $A$ and $q$ the block in which 
    %the $A$ in the $p$th block matches to. For example, if $(n, s, k) = (5, 1, 2)$, 
    %here is an example word with the matching with $p = 3, q = 5$. 

    %\newcommand{\red}[1]{\color{red} {#1} \color{black}}

    %\begin{align}
    %W \ = \
    %BB \, BB \, B\red{A} \, BB \, \red{A}B
    %\end{align}

    %Note that the matching between the $A$ blocks divide into two types. 
    %Type 1 matching is $BA \, AB$ and Type 2 paring is $AB \, BA$ both in order. 
    %matching the two $A$'s split the word into two sub-words, the word 
    %outside the $AA$ block and the word between the $AA$ block. So for the previous example, 

    %\begin{align}
    %    W_1 \ = \ BB \, BB \, BB\space  \& \space W_2 \ = \  BB
    %\end{align}
    %where $W_1$ is outside the $A$ matching and $W_2$ is between the $A$ matchings. 

    If the $p$\textsuperscript{th} and the $q$\textsuperscript{th} adjacent pairs are both $ba$, then the matching of the $a$'s partitions the cyclic product into two smaller cyclic products $C'_{1}$ (inner cyclic product)  and $C'_{2}$ (outer cyclic product), where $C'_{1}$ is of length $2(q-p-1)$ and has no restrictions on the number of starting adjacent pairs of $bb$ and $C'_2$ is of length $2(n-(q-p))$ and starts with at least $p$ adjacent pairs of $bb$. Then the total number of matchings is $\sigma_{q-p-1, 0, r}$ for $C'_{1}$ and $\sigma_{n-q+p, p, 2k-2-r}$ for some $r$. Similarly, if $p$\textsuperscript{th} and the $q$\textsuperscript{th} adjacent pairs are both $ab$, then the total number of matchings is $\sigma_{q-p, 1, r}$ for $C'_{1}$ and $\sigma_{n-q+p-1, p-1, 2k-2-r}$ for $C'_{2}$.
    %For matching Type 1, the value of $s$ increases by 1 after the splitting for 
    %the outer word. For matching Type 2, the value of $s$ increases from zero to 1 
    %after the split. The inner word and the outer word can be considered 
    %independent. The important obsevation to deduce the latter fact is to observe 
    %that the matching number is equivalent for the following two blocks. 
   % \begin{align}
    %B \, [\textnormal{Some Blocks}] \, B
    %\end{align}
 %\begin{align}
    %BB \, [\textnormal{Some Blocks}]
    %\end{align}

    %With these fact in mind, we count the contribution of 
    %Type 1 and Type 2 matchings for fixed $p, q$. 
    %For Type 1, the contribution is 
    %\begin{align}
    %\sigma_{n - q + p, p, r} \cdot \sigma_{q - p - 1, 0, 2k - 2 - r}
    %\end{align}
    %For Type 2, the contribution is 
    %\begin{align}
    %\sigma_{n - q + p - 1, p - 1, r} \cdot \sigma_{q - p, 1, 2k - 2 - r}
    %\end{align}
    Summing over all possible $p$'s, $q$'s and $r$'s, we have 
    \begin{align}
    \sigma_{n, s, 2k} \ = \ \sum_{p = s + 1}^{n}\sum_{q = p + 1}^{n}\sum_{r = 0}^{2k}
    \left[\sigma_{n - q + p, p, r} \cdot \sigma_{q - p - 1, 0, 2k - 2 - r}+ \sigma_{n - q + p - 1, p - 1, r} \cdot \sigma_{q - p, 1, 2k - 2 - r}\right].
    \end{align}
    \end{proof}

\subsection{Moments of $\{\textup{GOE}, k\textup{-BCE}\}$ and $\{k\textup{-BCE}, k\textup{-BCE}\}$}

The real symmetric $k$-block circulant ensemble is introduced by Kolo$\breve{{\rm g}}$lu-Kopp-Miller in \cite{Block Circulant} and possesses an even more complicated symmetry structure than the palindromic Toeplitz ensemble: not only do entries on different diagonals satisfy relations analogous to the palindromic Toeplitz ensemble, but the entries on the same diagonal also appear periodically due to the $k$-block structure. Because of its complicated structure, the $2m\textsuperscript{th}$ moments of the spectral distribution are not given explicitly, but expressed in terms of the number of pairings of the edges of a $2m$-gon which give rise to a genus $g$ surface. Similar to the palindromic case, we can extend the moment calculation of $k$-block circulant ensemble to that of $\{k\textup{-BCE}, k\textup{-BCE}\}$ and $\{\textup{GOE}, k\textup{-BCE}\}$.

Suppose that $b_{i_si_{s+1}}$ and $b_{i_ti_{t+1}}$ are entries from an $N\times N$ real symmetric $k$-block circulant matrix, then $b_{i_si_{s+1}}$ and $b_{i_ti_{t+1}}$ are matched iff either of the following relations hold:
\begin{enumerate}
\item $i_{s+1}-i_s=i_{t+1}-i_t+C_s$ and $i_s\equiv i_{t}\textup{ (mod $k$)}$, or
\item $i_{s+1}-i_s=-(i_{t+1}-i_t)+C_s$ and $i_s\equiv i_{t+1}\textup{ (mod $k$)}$,
\end{enumerate}
where $C_s\in \{0, \pm N\}$. The difference in sign in the two relations above allows us to think of the matching of $(s,s+1)$ and $(t,t+1)$ as having the same or different orientations. For both $\{k\textup{-BCE}, k\textup{-BCE}\}$ and $\{\textup{GOE, k\textup{-BCE}}\}$, we can apply the same argument from \cite{Toeplitz}, \cite{palindromicToeplitz}, and \cite{Block Circulant} to show that the total contribution of all the pairings with at least one matching of the same orientation is $O(1/N)$. Hence, it suffices to consider those pairings with matchings of the same orientation, i.e. $i_{s+1}-i_s=-(i_{t+1}-i_t)+C_s$ and $i_s\equiv i_t\textup{ (mod $k$)}$. By assumption $k=o(N)$, then the modular restrictions do not reduce the total degrees of freedom. Hence, analogous to the palindromic Toeplitz case, we can think of the pairing of terms of in the $2m\textsuperscript{th}$ moment of an $N\times N$ real symmetric $k$-block circulant matrix as a system of $m$ linear equations each of the form $i_{s+1}-i_s=-(i_{t+1}-i_t)+C_s$. This gives us $m+1$ free indices with $m-1$ dependent indices and constants $C_{s}\in \{0,\pm N\}$ uniquely determined, except for a lower order term of choices of free indices.

Using the idea of layers developed in subsection \ref{subsectionGOEpalindromicToeplitz}, we see that if $\pi$ is a pairing of $\{\textup{GOE}, k\textup{-BCE}\}$, then $\pi$ contributes to the moment of $\{\textup{GOE}, k\textup{-BCE}\}$ iff the GOE terms are matched non-crossing and the real symmetric $k$-block circulant terms are matched within each layer. Now, consider a pairing $\pi$ in the $2m\textsuperscript{th}$ moment of $\{\textup{GOE}, k\textup{-BCE}\}$. We identify the matched indices in the same congruence class modulo $k$ by the equivalence relation $\sim$. For example, if $a_{i_si_{s+1}}$ and $a_{i_ti_{t+1}}$ are matched, i.e. $i_s=i_{t+1}$ and $i_{s+1}=i_t$, then $i_s\sim i_{t+1}$ and $i_{s+1}\sim i_t$. If $b_{i_si_{s+1}}$ and $b_{i_ti_{t+1}}$ are matched, i.e. $i_{s+1}-i_s=-(i_{t+1}-i_t)+C_s$ and $i_s\equiv i_{t+1}\textup{ (mod $k$)}$, then we also have $i_{s+1}\equiv i_t\textup{ (mod $k$)}$. Hence, we still have $i_s\sim i_{t+1}$ and $i_{s+1}\sim i_t$. We see that the number of equivalence classes of indices of the pairing $\pi$ is $\#(\gamma_{4m}\pi)$. For each equivalence class, there are $k$ ways to choose congruence classes. So the number of ways to choose congruence class for all the indices is $k^{\#(\gamma_{4m}\pi)}$. Since there are $N/k$ choices for indices for each congruence class, then $2m\textsuperscript{th}$ moment of $\{\textup{GOE}, k\textup{-BCE}\}$ is given by
\begin{align*}
M_{2m} \ = \ \sum_{C\in\mathcal{C}_{2,4m}}\sum_{\pi_C\in NCF_{2,C}(4m)} k^{\#(\gamma_{4m}\pi)-(2m+1)},
\end{align*}
where $NCF_{2,C}(4m)$ denotes the set of all the pairings with respect to $C$ of $[4m]$ where the GOE terms are matched non-crossingly and the palindromic Toeplitz terms are matched freely without crossing the matchings of the GOE terms. Similarly, the $2m\textsuperscript{th}$ moment of $\{k\textup{-BCE}, k\textup{-BCE}\}$ is given by
\begin{align*}
M_{2m} \ = \ \sum_{C\in \mathcal{C}_{2,4m}}\sum_{\pi_C\in \mathcal{P}_{2,C}(4m)}k^{\#(\gamma_{4m}\pi)-(2m+1)}.
\end{align*}

%\begin{definition} [Equivalence relation $\approx$ and $\simeq$]

   % We say $(i, j) \approx (i', j')$ if and only if 
   % \begin{equation}
   %     i \ = \ j' \textAnd j \ = \ i'.
   % \end{equation}

   % Also, 
%$(i, j) \simeq (i', j')$ if and only if 
   % \begin{eqnarray}
   %     i - j \ \equiv \ j' - i' \mod N \nonumber\\
   %    i \ \equiv \ j' \textAnd j \ \equiv \ i' \mod m.
   % \end{eqnarray}
   % The value of $N, m$ are implied from context. 
%\end{definition}


%\begin{definition}[Valid Configuration]
%A \textit{Valid Configuration} of length \(2k\) is composed of \(k\) 2-blocks, where each block is one of \(\{AB, BA\}\). 
%We denote the set of all valid configurations of length $2k$ as 
%$\PW(2k)$.


%For example, when \(k = 3\),
%\[
%W \ = \ AB \, BA \, AB \, BA \in \PW(8)
%\]
%is an example of a valid configuration of length 6. To 
%refer to the specific index of the configuration, use the superscript. 
%For example, $W^{3} \ =\ B$. 

%\end{definition}

%\begin{definition}[Combining pairings]
%    Suppose we are given $W \in \PW(4k)$ and two pairings 
%    $\pi, \delta \in \mathcal{P}[2k]$. We denote the 
%    combined pairing of $\pi, \delta$ with respect 
%    to the configuration $W$ as 
%    \[
 %       \pi *_W \delta,
%    \]
%    where the combined pairing denotes an element in $\mathcal{P}[4k]$ 
%    where the composition between $A$'s are specified by $\pi$ and 
%    composition between $B$'s are specified by $\delta$. 

%    For example if 
%    \[
%    \pi \ = \ (1 2) (3 4) \textAnd 
%    \delta \ = \ (12) (34)
%    \]
%    the combined pairing is 
%    \[
%    \pi *_W \delta \ = \ (1 4)(2 3)(5 8)(67).
%    \]
%\end{definition}
    

%====For GOE x BC====

%We wish to compute $\mu_N^{(2k)}$, the $2k$\textsuperscript{th} moment of 
%the anticommutator product of ensemble $A$ which is a GOE 
%and ensemble $B$ which is a m-block circulant matrix, 
%where both $A, B$ are of order $N$. It is straightforward to 
%verify the following. 

%\begin{prop}[Even moment as configurations] The even moments for general configurations are
%\label{thm:baseForm}
%\begin{equation}
%\mu_N^{(2k)} \ = \ \sum_{W\in \PW(2k)}\sum_{1 \leq i_1 , \dots, i_{4k} \leq N}\sum_{\pi \in \mathcal P[2k]}\sum_{\delta \in \NC(2k)} \mathbb{E}_{(\pi*_W\delta)}\left(\prod_{l = 1}^{4k} W^{l}_{i_l i_{l + 1}}\right) \mathbb{1}_{(\pi*_W\delta)}.
%\end{equation}
%\end{prop}

%We first present the formula for the even moments. 

%\begin{theorem}[GOE anticommuted with Block Circulant]\label{GOEBC}\label{thm:GOEBC} For the GOE anticommuted with Block Circulant we get that the moments are
%\begin{equation}
%\mu_N^{(2k)} \ = \ \sum_{W \in \PW(2k)}\sum_{\pi \in \mathcal P[2k]}\sum_{\delta \in \NC(2k)}m^{\#((\pi *_W \delta) \circ \gamma_{2k} )}\left(\frac 1 m\right)^{2k + 1}
%\mathbb{1}_{(\pi*_w\delta)}.
%\end{equation}
%\end{theorem}

%Note that here $\gamma_n$ denotes a permutation of the cannonical set $[n]$ where $\gamma_n(x) = x + 1 \mod n$. 

%\begin{theorem}[Block Circulant antcommuted with Block Circulant]\label{thm:BCBC}For the Block Circulant anticommuted with Block Circulant we get that the moments are
%\begin{equation}
%\mu_N^{(2k)} \ = \ \sum_{W \in \PW(2k)}\sum_{\pi \in \mathcal P[2k]}\sum_{\delta \in  \mathcal P[2k]}m^{\#((\pi *_W \delta) \circ \gamma_{2k} )}\left(\frac 1 m\right)^{2k + 1}.
%\end{equation}
%\end{theorem}

%To prove these two theorems, we first establish 
%the following propositions. 

%\begin{prop} [Rules for pairing] \label{thm:pairingRule}
%For a pairing of each valid configuration, each of the compositions must match $A$'s to $A$'s and $B$'s to $B$'s. Moreover, the congruence classes of the indicies are confirmed once the two matricies are matched. That is, if $A_{i_s, i_{s + 1}}$ is matched with $A_{i_t, i_{t + 1}}$, then \begin{equation}
%(i_s, i_{s + 1}) \ \approx \ (i_t, i_{t + 1}).
%\end{equation}
%Also, 
%if $B_{i_s, i_{s + 1}}$ is matched with 
%$B_{i_t, i_{t + 1}}$, then 
%\begin{equation}
%(i_s, i_{s + 1}) \ \simeq \  (i_t, i_{t + 1}).
%\end{equation}
%\end{prop}
%\begin{proof}
%Introduce the signed variable $\epsilon_j$. Adding up the signed difference allows us to find that if any one of the signs are nonzero, the 
%degree of freedom reduces. \footnote{
%For a detailed explanation, refer to Lemma 2.6 of \cite{palindromicToeplitz}.
%}
%\end{proof}


%From now on, we focus on the \textbf{pairings}, that is, an element 
%of a 2-partition of the cannonical set. 
%Call a 2-block of this partition a \textbf{matching}. 
%For example, 
%for a permutation $(13)(24) \in \mathcal{P}[4]$, $(13)$ is considered 
%as a matching of the permutation. Also, if two matchings 
%$(i, j), (k, l)$ %exists within a pairing where $i < j < k < l$, we 
%say that the two matchings cross and hence the pairing is \textbf{crossing}. 

%\begin{prop} [GOE and BC pairing rules]
%Let $A$ be the GOE ensemble in the anticommutator. The matchings of $A$ must not cross with any other matchings. The matching between Block Circulant matricies can cross, and the crossings do not reduce the degree of freedom. 
%\end{prop}
%\begin{proof}
%The matching of $A$'s in each pairing slices the entire configuration. For example, consider the valid configuration $W=ABBABAAB$, where the pairing is given as $\pi=(1 4)(2 3)(5 8) (6 7)$. The composition $(14)$ slices the configuration into $W_1=BB$ and $W_2'=BAAB$,
%   where each configuration is extracted from between ($W_1$) and 
%    outside ($W_2$) the matching $W^1 = W^4 = A$. Call 
%   this matching of $A$'s as the slicing matching. Upon inspection of proposition \ref{thm:baseForm}, we know that the contribution of the moment of this word and pairing 
%   is the number of sequence $i$'s that satisfy the following modular restriction. 

%   \begin{eqnarray}\label{eqn:scrapModRest}
%       (i_1, i_2) \ \approx \ (i_4, i_5) \textAnd \nonumber\\ 
%       (i_2, i_3) \ \simeq \ (i_3, i_4) , \ 
%       (i_5, i_6) \ \simeq \ (i_8, i_1) , \ 
%       (i_6, i_7) \ \approx \ (i_7, i_8)
%   \end{eqnarray}

%   The condition in the first line implies that $i_1 = i_5$ and $i_2 = i_4$. For better analysis, introduce two 
%   auxillary sequence $j, k$ where the sequence elements 
%   are explicitly defined as follows. 

%   \begin{eqnarray}
%       (j_1, j_2) \ = \ (i_2, i_3) \nonumber \\ 
%       (k_1, k_2, k_3, k_4) \ = \ (i_5, i_6, i_7, i_8)
%   \end{eqnarray}

%   Under the substitution, the modular restrictions described in the second line of \ref{eqn:scrapModRest} converts 
%   to 
%   \begin{eqnarray}
%       (j_1, j_2) \ \simeq \ (j_2, j_1) \nonumber \\ 
%       (k_1, k_2) \ \simeq \ (k_4, k_1), \ 
%       (k_2, k_3) \ \approx \ (k_3, k_4)
%   \end{eqnarray}
%   It is easy to observe that these are exactly the modular 
%   restrictions enforced by $W_1$ and $W_2'$. We have demonstrated that the contribution of a certain configuration along with a matching that respects the configuration can be computed as a product of two independent sub-configurations divided by the slicing matching and the corresponding sub-matching. 
   
%   Furthermore, the slicing matching $(67)$ slices $W_2'$ 
%   into another configuration. 
%   \[
%       W_2 \ = \ BB
%   \]

 %  The observation has two implications. The first implication 
 %  is that any matching that crosses with the slicing 
 %  matching reduces a degree of freedom. The crossings link the two otherwise independent sub-configurations. 
%   Hence, crossings 
%   with slicing matching, which can be any pairings between $A$'s which have a 
%   crossing, 
 %  result in a vanishing contribution. 

%   The second implication is that any pairing where the 
%   slicing compositions do not cross with other compositions 
%   always have a positive nonzero contribution. After reducing the entire 
%   configuration according to all its slicing matching, we 
%   we are left with finite number of sub-configurations that are comprised solely 
%   of $B$'s. For the configuration $W$, the remaining configurationss are $W_1, W_2$. 

%   Composition between $B$'s lose one degree of freedom, regardless of 
%   crossings. So these always have a contribution.
%\end{proof}

%Finally, we present a proof of theorem \ref{thm: GOEBC}.
%\begin{proof} Finally, we provide the proof for Theorem \ref{GOEBC}. From proposition \ref{thm:baseForm}, we recognize that it suffices to count the number of integer sequences 
%$i_1, \dots, i_{4k}$ that satisfy the pairing restrictions. Fix a pairing $\pi$ that matches all the GOE $A$'s and a pairing $\delta$  that pairs the Block Circulnat $B$'s. We first configure the modular residue of $i$'s mod $m$. Clearly, by proposition \ref{thm:pairingRule}, the number of such configurations are \footnote{Refer to \cite{MS} Section 1.7 for details, where $\#(\pi)$ denotes the number of orbits of the permutation $\pi$} $m^{\# ((\pi *_W \delta)\circ \gamma_{2k})}$
%Move on to choose the value of $\lfloor i / m \rfloor$. 
%We know that as long as the slicing matchings of $A$ do not cross with other matchings, the degree of freedom is not reduced. Otherwise, the contribution is can be ignored at the limit $N \rightarrow \infty$. Thus, the ways to choose 
%$\lfloor i / m \rfloor$ is $\left(\frac 1 m\right)^{2k + 1}
%\mathbb{1}_{(\pi*_w\delta)}$, where $\mathbb{1}_{(\pi*_w\delta)}$ is defined to be $1$ if and only if 
%the pairing $(\pi*_w\delta)$ is non-crossing in the sense of proposition 
%\ref{thm:pairingRule} and zero otherwise. 

%The variance of all the random variables involved in the matricies are 
%fixed to be $1$. Thus, from proposition \ref{thm:baseForm}, we obtain 
%\label{thm: GOEBC}
%\begin{equation}
%\mu_N^{(2k)} \ = \ \sum_{W \in \PW(2k)}
%\sum_{\pi \in \mathcal P[2k]}\sum_{\delta \in \NC(2k)}m^{\#((\pi *_W \delta) \circ \gamma_{2k} )}\left(\frac 1 m\right)^{2k + 1}\mathbb{1}_{(\pi*_w\delta)}.
%\end{equation}
%\end{proof}

%\begin{remark}
%Using a result from group theory, we can rewrite the number of orbits of 
%a permutation as a genus of a graph that embeds the permutation, and hence 
%the following formula holds:
%\begin{equation}
%\mu_N^{(2k)} \ = \ \sum_{W, \pi, \delta} m^{-2g} \mathbb{1}_{(\pi*_w\delta)},
%\end{equation}
%where $g$ is the minimum genus of the graph associated to $((\pi *_W \delta) \circ \gamma_{2k} )$. 
%\end{remark}
%Via matlab, it is possible to in

\section{The Blip Spectral Measure of anticommutators}
In this section, we consider the blip spectral measures of two ensembles: $\{\textup{GOE, }k\textup{-checkerboard}\}$ and $\{k\textup{-checkerboard, }j\textup{-checkerboard}\}$. We always assume that $\textup{gcd}(k, j)=1$ and $jk\mid N$, which as we will see later on is crucial to the structure of the anticommutator. Even though the bulk eigenvalues of these two ensembles are both of order $O(N)$ (with largest and smallest eigenvalues $\Theta(N)$), we observe drastically different splitting behaviors: while $\{\textup{GOE, }k\textup{-checkerboard}\}$ has blip eigenvalues only of order $\Theta(N^{3/2})$, $\{k\textup{-checkerboard, }j\textup{-checkerboard}\}$ has blip eigenvalues of order $\Theta(N^2)$ and $\Theta(N^{3/2})$. Specifically, $\{\textup{GOE}, k\textup{-checkerboard}\}$ has $2k$ eigenvalues of order $\Theta(N^{3/2})$ (called the \textbf{blip} eigenvalues), of which $k$ are $\frac{N^{3/2}}{k}+O(N)$ and $k$ are $-\frac{N^{3/2}}{k}+O(N)$; by contrast $\{k\textup{-checkerboard}, j\textup{-checkerboard}\}$ has 1 eigenvalue of order $\Theta(N^2)$ (called the \textbf{largest blip} eigenvalue) at $\frac{2}{jk}N^2+O(N)$ and $2k+2j-4$ eigenvalues of order $\Theta(N^{3/2})$ (called the \textbf{intermediary blip} eigenvalues), of which two intermediary blips each containing $k-1$ eigenvalues are $\pm \frac{1}{k}\sqrt{1-\frac{1}{j}}N^{3/2}+O(N)$ and two intermediary blips each containing $j-1$ eigenvalues are $\pm\frac{1}{j}\sqrt{1-\frac{1}{k}}N^{3/2}+O(N)$. For proofs of different regimes, see Appendix \ref{multipleregimes}.

\subsection{Structural Preliminaries}

We first define the empirical blip spectral measure using appropriate weight functions and reduce the blip moment calculation to combinatorics. Next, using the language developed in \cite{split}, we identify the types of cyclic products that contribute to the expected $m$\textsuperscript{th} moments of the blip spectral measures of both ensembles. Then, we explicitly obtain the expected $m$\textsuperscript{th} moments of the blip spectral measures of $\{\textup{GOE, }k\textup{-checkerboard}\}$ and of $\{k\textup{-checkerboard, }j\textup{-checkerboard}\}$ around the far away blip. Finally, we highlight the combinatorial challenge that we encounter in the calculation of the expected $m$\textsuperscript{th} moments of the blip spectral measures of $\{k\textup{-checkerboard, }j\textup{-checkerboard}\}$ around the intermediary blips.

\begin{definition}
Let $w_s=\frac{(-1)^{s+1}}{k}$ for $s\in \{1,2\}$. Then the \textbf{empirical blip spectral measure} associated to the anticommutator of an $N\times N$ GOE and k-checkerboard $\{A_N,B_N\}$ around $w_sN^{3/2}$ is
\begin{align}
\mu_{\{A_N,B_N\},s}(x) \ = \ \frac{1}{k}\sum_{\lambda\textup{ eigenvalues}}f_s^{2n}\left(\frac{\lambda}{w_sN^{3/2}}\right)\delta\left(\frac{x-\left(\lambda-w_sN^{3/2}\right)}{N}\right),
\end{align}
where $n(N)$ is a function satisfying $n(N)=\log\log(N)$ (note that when we use $n$ in this section we are referring to $n(N)$) and
\begin{align}
f_s^{2n}(x) \ := \ \left(\frac{x(2-x)(x+1)(3-x)}{4}\right)^{2n}.
\end{align}
\end{definition}

\begin{definition}
The \textbf{empirical largest blip spectral measure} associated to the anticommutator of an $N\times N$ k-checkerboard and j-checkerboard  $\{A_N, B_N\}$, where $\textup{gcd}(k,j)=1$ and $jk\mid N$, is
\begin{align}
\mu_{\{A_N,B_N\}}(x)=\sum_{\lambda\textup{ eigenvalues}}g_0^{2n}\left(\frac{jk\lambda}{2N^2}\right)\delta\left(x-\left(\frac{\lambda-\frac{2}{jk}N^2}{N}\right)\right),
\end{align}
where $g^{2n}_0(x):=x^{2n}(2-x)^{2n}$.
Let $w_s=\frac{(-1)^{s+1}}{k}\sqrt{1-\frac{1}{j}}$ and $h_s=k$ for $s\in \{1, 2\}$ and $w_s=\frac{(-1)^{s+1}}{j}\sqrt{1-\frac{1}{k}}$ and $h_s=j$ for $s\in \{3, 4\}$ and $g_s$ as defined in Appendix \ref{intermediaryblip}. Then the \textbf{empirical intermediary blip spectral measure} associated to $\{A_N, B_N\}$ around $w_sN^{3/2}$ is
\begin{align}
\mu_{\{A_N, B_N\}, s}(x) \ = \ \frac{1}{h_s}\sum_{\lambda\textup{ eigenvalues}}g_s^{2n}\left(\frac{\lambda}{w_sN^{3/2}}\right)\delta\left(\frac{x-\left(\lambda-w_sN^{3/2}\right)}{N}\right).
\end{align}
We again require that $n(N)$ is a function satisfying $\lim_{N\rightarrow\infty}n(N)=\infty$ and $n(N)=\log\log(N)$.
\end{definition}

\begin{remark}
Note that we never use this blip spectral measure since the combinatorics needed to do the calculations become too complex. We would need some strong algebraic and combinatorial tools to simplify the large sums we have so we were unable to complete the analysis of the intermediate blip despite creating a weight function with the desired properties. The difficulties of the calculations needed to reduce these sums is discussed in Appendix \ref{intermediaryblip}.
\end{remark}

% We first consider the expected $m^{th}$ moments of the empirical blip spectral measure associated to $\{\textup{GOE, k-checkerboard}\}$ around $w_1N^{3/2}$. Since $w_2=-w_1$, then as we shall see, the moments only differ by a change of sign. Note that the weight polynomial $f_1^{2n}(x)$ can be written as $\sum_{\alpha=2n}^{8n}c_\alpha x^\alpha$.

We first consider empirical blip spectral measure associated to $\{A_N, B_N\}$ around $\frac{N^{3/2}}{k}$. As we shall see later in this section and by symmetry and from \ref{MomentCalc} with \ref{MomentCalc2}, the empirical blip spectral measure associated to $\{A_N, B_N\}$ around $-\frac{N^{3/2}}{k}$ is the same as that around $\frac{N^{3/2}}{k}$. Since the weight polynomial $f_1^{2n}(x)$ can be written as $\sum_{\alpha=2n}^{8n}c_\alpha x^\alpha$, where $c_\alpha\in\mathbb{R}$, then by eigenvalue trace lemma, the expected $m$\textsuperscript{th} moment of the empirical blip spectral measure associated to $\{A_N,B_N\}$ around $\frac{N^{3/2}}{k}$ is
\begin{align}
\mathbb{E}\left[\mu_{\{A_N,B_N\},1}^{(m)}\right] &\ = \ \mathbb{E}\left[\frac{1}{k}\sum_{\lambda\textup{ eigenvalues}}\sum_{\alpha=2n}^{8n}c_\alpha\left(\frac{k\lambda}{N^{3/2}}\right)^\alpha\left(\frac{\lambda-w_1N^{3/2}}{N}\right)^{m}\right] \nonumber \\
&\ = \ \mathbb{E}\left[\frac{1}{k}\sum_{\alpha=2n}^{8n}c_\alpha\left(\frac{k}{N^{3/2}}\right)^\alpha\frac{1}{N^m}\sum_{i=0}^m \binom{m}{i}\left(-\frac{N^{3/2}}{k}\right)^{m-i}\textup{Tr}(\{A_N,B_N\}^{\alpha+i})\right] \nonumber \\
&\ = \ \frac{1}{k}\sum_{\alpha=2n}^{8n}c_\alpha\left(\frac{k}{N^{3/2}}\right)^\alpha\frac{1}{N^m}\sum_{i=0}^m \binom{m}{i}\left(-\frac{N^{3/2}}{k}\right)^{m-i}\mathbb{E}[\textup{Tr}(\{A_N,B_N\}^{\alpha+i})]\label{MomentCalc}.
\end{align}

Let $g_0^{2n}(x)=\sum_{\beta=2n}^{4ln}d_\beta x^{\beta}$, then similarly by eigenvalue trace lemma, the expected $m$\textsuperscript{th} moment of the empirical largest blip spectral measure associated to $\{A_N, B_N\}$ is
\begin{align}\label{momentOfCNDN}
\mathbb{E}\left[\mu^{(m)}_{\{A_N, B_N\}}\right] &\ = \ \mathbb{E}\left[\sum_{\lambda\textup{ eigenvalues}} \sum_{\beta=2n}^{4nl}d_\beta\left(\frac{jk\lambda}{2N^2}\right)^\beta \left(\frac{\lambda-\frac{2}{jk}N^2}{N}\right)^m\right] \nonumber \\
&\ = \ \mathbb{E}\left[\sum_{\beta=2n}^{4nl}d_\beta\left(\frac{jk}{2N^2}\right)^\beta\frac{1}{N^m} \sum_{i=0}^m\binom{m}{i}\left(-\frac{2}{jk}N^2\right)^{m-i}\textup{Tr}(\{C_N, D_N\}^{\beta+i})\right] \nonumber \\
&\ = \ \sum_{\beta=2n}^{4nl}d_\beta \left(\frac{jk}{2N^2}\right)^\beta \frac{1}{N^m}\sum_{i=0}^m\binom{m}{i}\left(-\frac{2}{jk}N^2\right)^{m-i}\E{\text{Tr}\{C_N,D_N\}^{\beta+i}}.
\end{align}

We know that for an $N\times N$ anticommutator ensemble $\{X_N, Y_N\}$, the $(\alpha+i)$\textsuperscript{th} expected moment is
\begin{align}
\mathbb{E}[\textup{Tr}(\{X_N, Y_N\}^{\alpha+i})] \ = \ \sum_{1\leq i_1, \cdots, i_2m\leq N}\mathbb{E}[c_{i_1i_2}\cdots c_{i_{2m}i_1}].
\end{align}

Hence, the calculation of the blip moment has now been transformed into a combinatorics problem of counting different types of products of entries. For the rest of this section, we use $a$ to denote a non-weight term of $A_N$, $w$ a weight term of $A_N$, $b$ a non-weight term of $B_N$, $v$ a weight term of $B_N$, and $c$ any term of $A_N$ or $B_N$.

\begin{defi}
A \textbf{block} is a set of adjacent $a$'s and $b$'s surrounded by $w$'s and $v$'s in a cyclic product, where the last term of a cyclic product is considered to be adjacent to the first. We refer to a block of length $\ell$ as an $\ell$ block or sometimes a block of size $\ell$.
\end{defi}

\begin{defi}
A \textbf{weight block} is a set of adjacent $w$'s and $v$'s surrounded by $a$'s and $b$'s in a cyclic product. We similarly refer to a weight block of length $\ell$ as an $\ell$ weight block or sometimes a weight block of size $\ell$.
\end{defi}

\begin{defi}
An \textbf{adjacent pair} is a pair of adjacent entries of the form $c_{i_{2\ell-1}i_{2\ell}}$, where the first term starts with an odd index.
\end{defi}

\begin{defi}
A \textbf{weight pair} is a pair of adjacent weight terms $c_{i_{2\ell -1}i_{2\ell}}c_{i_{2\ell}i_{2\ell+1}}$. Due to the structure of anticommutator, $\{c_{i_{2\ell -1}i_{2\ell}}, c_{i_{2\ell}i_{2\ell+1}}\}\in \{\{w_{i_{2\ell -1}i_{2\ell}}, v_{i_{2\ell}i_{2\ell+1}}\}, \{v_{i_{2\ell -1}i_{2\ell}},w_{i_{2\ell}i_{2\ell+1}}\}\}$.
\end{defi}

\begin{defi}
An \textbf{mixed pair} is a pair of adjacent weight and non-weight terms $c_{i_{2\ell-1}i_{2\ell}}c_{i_{2\ell}i_{2\ell+1}}$. Due to the structure of anticommutator, 
\begin{align}
\{c_{i_{2\ell-1}i_{2\ell}}, c_{i_{2\ell}i_{2\ell+1}}\}\in \{\{a_{i_{2\ell-1}i_{2\ell}}, v_{i_{2\ell}i_{2\ell+1}}\}, \{v_{i_{2\ell-1}i_{2\ell}}, a_{i_{2\ell}i_{2\ell+1}}\}, \{b_{i_{2\ell-1}i_{2\ell}}, w_{i_{2\ell}i_{2\ell+1}}\}, \{w_{i_{2\ell-1}i_{2\ell}}, b_{i_{2\ell}i_{2\ell+1}}\}\}.
\end{align}
\end{defi}

\begin{defi}
A \textbf{configuration} is a set of all cyclic products for which it is specified (a) how many blocks there are and what each of them compose of (e.g., a block of $abba$); and (b) in what order these blocks appear (up to cyclic permutation); However, it is not specified how many $w$'s and $v$'s there are between each block.
\end{defi}

\begin{defi}
A \textbf{congruence configuration} is a configuration together with a choice of the congruence class modulo $k$ every index.
\end{defi}

\begin{defi}
Given a configuration, a \textbf{matching} is an equivalence relation $\sim$ on the $a$'s and $b$'s in the cyclic product which constrains the way of indexing: for any $c_{i_\ell i_{\ell+1}}$ and $c_{i_ti_{t+1}}$, if $\{c_{i_\ell i_{\ell+1}}, c_{i_ti_{t+1}}\}\in \{\{a_{i_\ell i_{\ell+1}},a_{i_ti_{t+1}}\}, \{b_{i_\ell i_{\ell+1}}, b_{i_t i_{t+1}}\}\}$, then $\{i_\ell, i_{\ell+1}\}=\{i_t, i_{t+1}\}$ if and only if $c_{i_\ell i_{\ell+1}}\sim c_{i_ti_{t+1}}$.
\end{defi}

\begin{defi}
Given a configuration, matching, and length of the cyclic product, then an \textbf{indexing} is a choice of
\begin{enumerate}
\item the (positive) number of $w$'s and $v$'s between each pair of adjacent blocks (in the cyclic sense), and
\item the integer indices of each $a$, $b$, $w$, $v$ in the cyclic product.
\end{enumerate}
\end{defi}

\begin{defi}
A \textbf{configuration equivalence} $\sim_C$ is an equivalence relation on the set of all configurations such that for any configurations $C_1, C_2$, $C_1\sim_C C_2$ if and only they have the same blocks (but they may have different number and arrangement of $w$'s and $v$'s between their blocks). Every equivalence class under $\sim_C$ is called an \textbf{$S$-class}, specified by the blocks in all the configurations in the equivalence class. 
\end{defi}

The following lemma characterizes the $S$-class with the highest degree of freedom and boils down the blip moment calculation for both $\{\textup{GOE, }k\textup{-checkerboard}\}$ and $\{k\textup{-checkerboard, }j\textup{-checkerboard}\}$ to consideration of some nice configurations.

\begin{lemma}\label{blockslemma}
Fix $m\geq 1$, consider all the $S$-classes with $|S|=m$. Then a $S$-class with a matching $\sim$ yields the highest degrees of freedom iff it satisfies the following conditions.
\begin{enumerate}
\item It consists only of the following blocks: (i) 1-block of $a$; (ii) 1-block of $b$; (iii) 2-block of $aa$, (iv) 2-block of $bb$.
\item Each 1-block is paired up to another 1-block and the two terms in each 2-block are paired up with each other.
\end{enumerate}
\end{lemma}

\begin{proof}
Similar to Lemma 3.14 of \cite{split}, we see that when a 1-block of $a$ (resp. $b$) is paired up with another 1-block of $a$ (resp. $b$) or when the letters in a 2-block of $a$'s (resp. $b$'s) are paired up with each other, there is one degree of freedom lost per block. Now, fix a configuration $\mathcal{C}$ with $\alpha$ from the $a$'s and $\beta$ from the $b$'s and a matching $\sim$. Suppose that $\sim$ partitions all the $a$'s into equivalence classes $\mathcal{E}_1, \cdots, \mathcal{E}_{s_a}$ and $\mathcal{E}'_1, \cdots, \mathcal{E}'_{s_b}$. Then, without any matching restrictions, the degrees of freedom of $\mathcal{C}$ is
\begin{align}
\mathcal{\Tilde{F}}_{\mathcal{C}} \ = \ \sum_{\textup{blocks } \mathcal{B}}(\textup{len}(\mathcal{B})+1) \ = \ \alpha+\beta+m.
\end{align}
To find the actual degree of freedom $\mathcal{F}_\mathcal{C}$ of $\mathcal{C}$, we can choose two indices from each equivalence class. However, adjacent $a$'s and $b$'s from different equivalence classes (which we call cross-overs) place restrictions on the indices and cause additional loss of degrees of freedom. Let the loss of degrees of freedom due to cross-overs be $\gamma$, then $\mathcal{F}_\mathcal{C}=2s_a+2s_b-\gamma$. Thus, the degree of freedom lost per block is
\begin{equation}\label{degreeoffreedom}
\mathcal{\overline{L}}_\mathcal{C} \ = \ \frac{\mathcal{\Tilde{F}_C}-\mathcal{F}_C}{m} \ = \ 1+\frac{\alpha+\beta+\gamma-2s_a-2s_b}{m}.
\end{equation}
Since $|\mathcal{E}_{i}|,|\mathcal{E}_{j}'|\geq 2$ for $1\leq i\leq s_a$ and $1\leq j\leq s_b$, then $s_a\leq \frac{\alpha}{2}$ and $s_b\leq \frac{\beta}{2}$, and so $\mathcal{\overline{L}}\geq 1$. We've shown that if $\mathcal{C}$ satisfies the conditions (1) and (2), then $\mathcal{\overline{L}}_\mathcal{C}=1$. Hence, it suffices to show that if $\mathcal{C}$ with a matching $\sim$ loses one degree of freedom per block (or equivalently, satisfies $\frac{\alpha+\beta+\gamma}{s_a+s_b}=2$), then it must satisfy the conditions (1) and (2). Since $|\mathcal{E}_{i}|,|\mathcal{E}'_{j}|\geq 2$, we get $\alpha\geq 2s_a$ and $\beta\geq 2a_b$. Hence, if some $|\mathcal{E}_{i}|>2$ or $|\mathcal{E}'_{j}|> 2$, then $\frac{\alpha+\beta+\gamma}{s_a+s_b}>2$. Moreover, if $\gamma>0$, then $\frac{\alpha+\beta+\gamma}{s_a+s_b}>2$. Therefore, if $\mathcal{C}$ with a matching $\sim$ loses one degree of freedom per block, then all the blocks are paired up and there can be no cross-overs from different equivalence classes. Thus, the only possible $S$-classes and matchings are those satisfying conditions (1) and (2).
\end{proof}

%\textcolor{red}{To do: Lemma \ref{blockslemma} potentially needs revision}
%\begin{definition}
%An $S_{ab}$-class is a 4-tuple $(m_{1a},m_{2a},m_{1b},m_{2b})$ where $m_{ic}$ represents the number of $i$-blocks of variable $c\in \{a,b\}$. We know from a previous lemma that only configurations where blocks are either $1$-blocks or $2$-blocks with letters matched with each other contribute to the trace calculation.
%\end{definition}

%$\{c_{i_{2\ell-1}i_{2\ell}}, c_{i_{2\ell}i_{2\ell+1}}\}$, which are $\{a_{i_{2\ell-1} i_{2\ell}}, b_{i_{2\ell}i_{2\ell+1}}\}$, $\{b_{i_{2\ell-1} i_{2\ell}}, a_{i_{2\ell}i_{2\ell+1}}\}$, $\{a_{i_{2\ell-1}i_{2\ell}}, v_{i_{2\ell}i_{2\ell+1}}\}$, $\{v_{i_{2\ell-1}i_{2\ell}}, a_{i_{2\ell}i_{2\ell+1}}\}$, $\{b_{i_{2\ell-1} i_{2\ell}}, w_{i_{2\ell}i_{2\ell+1}}\}$, $\{w_{i_{2\ell-1} i_{2\ell}}, b_{i_{2\ell}i_{2\ell+1}}\}$, $\{v_{i_{2\ell-1}i_{2\ell}}, w_{i_{2\ell}i_{2\ell+1}}\}$, or $\{w_{i_{2\ell-1}i_{2\ell}}, v_{i_{2\ell}i_{2\ell+1}}\}$. 
Due to the structure of anticommutator, there are eight possibilities for each adjacent pair: $ab$, $ba$, $av$, $va$, $bw$, $wb$, $vw$, $wv$. Hence, after specifying the 1-block, we know the other term in the mixed pair, i.e. if the 1-block is of the form $c_{i_{2\ell-1}i_{2\ell}}$ and $c_{i_{2\ell-1}i_{2\ell}}=a_{i_{2\ell-1}i_{2\ell}}$ (resp. $c_{i_{2\ell-1}i_{2\ell}}=b_{i_{2\ell-1}i_{2\ell}}$), then $c_{i_{2\ell}i_{2\ell+1}}=v_{i_{2\ell}i_{2\ell+1}}$ (resp. $c_{i_{2\ell}i_{2\ell+1}}=w_{i_{2\ell}i_{2\ell+1}}$); similar conclusion holds for 1-block of the form $c_{i_{2\ell}i_{2\ell+1}}=a_{i_{2\ell}i_{2\ell+1}}$ (resp. $c_{i_{2\ell}i_{2\ell+1}}=b_{i_{2\ell}i_{2\ell+1}}$). Moreover, after specifying the 2-block, we know both of its adjacent terms (or the two mixed pairs the 2-block belongs to), i.e. every 2-block is of the form $c_{i_{2\ell}i_{2\ell+1}}c_{i_{2\ell+1}i_{2\ell+2}}$, and if $\{c_{i_{2\ell}i_{2\ell+1}}, c_{i_{2\ell+1}i_{2\ell+2}}\}=\{a_{i_{2\ell}i_{2\ell+1}}, a_{i_{2\ell+1}i_{2\ell+2}}\}$, then $\{c_{i_{2\ell-1}i_{2\ell}}, c_{i_{2\ell+2}i_{2\ell+3}}\}=\{v_{i_{2\ell-1}i_{2\ell}}, v_{i_{2\ell+2}i_{2\ell+3}}\}$; if $\{c_{i_{2\ell}i_{2\ell+1}}, c_{i_{2\ell+1}i_{2\ell+2}}\}=\{b_{i_{2\ell}i_{2\ell+1}}, b_{i_{2\ell+1}i_{2\ell+2}}\}$, then $\{c_{i_{2\ell-1}i_{2\ell}}, c_{i_{2\ell+2}i_{2\ell+3}}\}=\{w_{i_{2\ell-1}i_{2\ell}}, w_{i_{2\ell+2}i_{2\ell+3}}\}$. We also know that a weight pair $\{c_{i_{2\ell-1}i_{2\ell}}, c_{i_{2\ell}i_{2\ell+1}}\}\in \{\{w_{i_{2\ell-1}i_{2\ell}}, v_{i_{2\ell}i_{2\ell+1}}\},\{v_{i_{2\ell-1}i_{2\ell}}, w_{i_{2\ell}i_{2\ell+1}}\}\}$. Thus, we can view specifying a cyclic product of length $2\eta$ as only specifying $\eta$ terms, where specifying one term of the pair $c_{i_{2\ell-1}i_{2\ell}}c_{i_{2\ell}i_{2\ell+1}}$ and whether the pair is weight/non-weight or not uniquely determines the other term of the pair.

\subsection{$\{\textup{GOE, }k\textup{-checkerboard}\}$} %For the anticommutator of GOE and k-checkerboard, the bulk is of order $O(N)$ and the blips on either side of the bulk (each of which contains $k$ eigenvalues) are of order $O(N^{3/2})$. This can be easily justified using Weyl's inequality, by separating out the contribution of the bulk and the blip: $\{A_N,B_N\}=A_N(\overline{B_N}+\Tilde{B_N})+(\overline{B_N}+\Tilde{B_N})A_N=A_N\overline{B_N}+\overline{B_N}A_N+\Tilde{B_N}\overline{B_N}+\overline{B_N}\Tilde{B_N}$, where $\Tilde{B_N}$ is the weight matrix and $\overline{B_N}:=B_N-\Tilde{B_N}$. More precisely, the two blips are at $\pm\frac{1}{k}N^{3/2}+O(N)$, respectively.

%Since there are two blips, we need to use the following weight function to remove the contribution from other blips than the one we're looking at. Let $w_1=\frac{1}{k}$ and $w_2=-\frac{1}{k}$, then

%\begin{align}
%f_1^{2n}(x) &\ = \ \left(\frac{x(2-x)(x-\frac{w_2}{w_1})(2-x-\frac{w_2}{w_1})}{(1-\frac{w_2}{w_1})^2}\right)^{2n} \\
%&\ = \ \left(\frac{x(2-x)(x+1)(3-x)}{4}\right)^{2n}\end{align}


%\begin{definition}
%The \textbf{empirical blip spectral measure} associated to the anticommutator of an $N\times N$ GOE and k-checkerboard $\{A_N,B_N\}:= A_NB_N+B_NA_N$ around $w_iN^{3/2}$ is
%\begin{align}
%\mu_{\{A_N,B_N\},i} \ = \ \frac{1}{k}\sum_{\lambda}f_i^{2n}\left(\frac{\lambda}{w_iN^{3/2}}\right)\delta\left(\frac{x-\left(\lambda-w_iN^{3/2}\right)}{N}\right)
%\end{align}
%\end{definition}

%We set $i=1$ and look at the blip centered at $\frac{1}{k}N^{3/2}+O(N)$ first. Note that the polynomial $f_1^{2n}(x)$ can be written as $\sum_{\alpha=2n}^{8n}c_\alpha x^\alpha$. Then the expected $m$-th moment associated with the empirical blip spectral measure is
%\begin{align}
%\mathbb{E}\left[\mu_{\{A_N,B_N\},1}^{(m)}\right] &\ = \ \mathbb{E}\left[\frac{1}{k}\sum_{\lambda}\sum_{\alpha=2n}^{8n}c_\alpha\left(\frac{k\lambda}{N^{3/2}}\right)^\alpha\left(\frac{\lambda-w_1N^{3/2}}{N}\right)^{m}\right] \\
%&\ = \ \mathbb{E}\left[\frac{1}{k}\sum_{\alpha=2n}^{8n}c_\alpha\left(\frac{k}{N^{3/2}}\right)^\alpha\left(\frac{1}{N^m}\sum_{i=0}^m \binom{m}{i}\left(-\frac{N^{3/2}}{k}\right)^{m-i}\textup{Tr}(\{A_N,B_N\}^{\alpha+i})\right)\right] \\
%&\ = \ \frac{1}{k}\sum_{\alpha=2n}^{8n}c_\alpha\left(\frac{k}{N^{3/2}}\right)^\alpha\frac{1}{N^m}\sum_{i=0}^m \binom{m}{i}\left(-\frac{N^{3/2}}{k}\right)^{m-i}\mathbb{E}[\textup{Tr}(\{A_N,B_N\}^{\alpha+i})]
%\end{align}

\begin{lemma}\label{Sclasscontribution} For $m_1, m_2\in \Z_{\geq 0}$, the total contribution to $\mathbb{E}[\textup{Tr}\{A_N, B_N\}^\eta]$ of an $S$-class with $m_1$ 1-blocks of $a$'s and $m_2$ 2-blocks of $a$ is
\begin{align}
\frac{p(\eta)N^{\frac{3}{2}\eta -\frac{1}{2}m_1}}{k^{\eta}}\mathbb{E}_k[\textup{Tr }C^{m_1}]+O\left(N^{\frac{3}{2}\eta -\frac{1}{2}m_1-1}\right),
\end{align}
where $p(\eta)=\frac{2\eta^{m_1}}{m_1!}+O(\eta^{m_1-1})$ and $C$ is a $k\times k$ Gaussian Wigner matrix.
\end{lemma}

\begin{proof}
We begin by noting that by \ref{blockslemma}, an $S$-class containing any $b$ would have fewer degrees of freedom and hence would contribute at most $O(N^{\frac{3}{2}\eta-\frac{m_1}{2}-1})$. Thus, it suffices to consider the case when $m_2=\frac{\eta-m_1}{2}$ and there are no $b$'s. The rest of the proof is divided into two parts: we first count the number of ways to arrange a prescribed number of blocks into a cyclic product of length $2\eta$; we then count the number of ways to pair together $1$-blocks and assign indices that are consistent throughout the cyclic product.

Given $m_1=o(\eta)$, we claim that the number of ways $q(\eta)$ of arranging $m_1$ $1$-blocks and $\frac{\eta-m_1}{2}$ $2$-blocks of $a$'s into a cyclic product of length $2\eta$ is $\frac{2\eta^{m_1}}{m_1!}+O(\eta^{m_1-1})$.

Indeed, there are two ways to choose the $\frac{\eta-m_1}{2}$ $2$-blocks since we can either start with $aw$ or $wa$, and there are $2^{m_1}\binom{\frac{\eta-m_1}{2}}{m_1}=\frac{\eta^{m_1}}{m_1!} + O(\eta^{m_1-1})$ ways to choose the $m_1$ 1-blocks between adjacent $2$-blocks assuming that $m_1=o(\eta)$. Moreover, as we shall see later in the proof, we require that mixed pairs containing the $1$-blocks are not placed adjacent to each other, which is possible since the number of ways of having at least one $2$-block formed from adjacent mixed pairs is $2\left(\frac{\eta-m_1}{2}\right)\binom{\frac{\eta-m_1}{2}-2}{m_1-2}=O(\eta^{m_1-1})$ and thus a lower order term. Hence, overall we get $\frac{2\eta^{m_1}}{m_1!}+O(\eta^{m_1-1})$ ways to arrange the prescribed blocks.
%and that no two 1-blocks are adjacent to each other in which case contributions would be of lower order. Each 1-block placed introduces a factor of 2 since it can be written as either $av$ or $va$. Hence, overall we get $2^{m_1+1}\cdot \frac{(\eta/2)^{m_1}}{m_1!} + O(\eta^{m_1-1})=\frac{2\eta^{m_1}}{m_1!}+O(\eta^{m_1-1})$ ways to arrange prescribed blocks.

% First we count the number of ways we can have a list of $m_1$ 1-blocks and $\frac{\eta-m_1}{2}$ 2-blocks, first we can place the $\frac{\eta-m_1}{2}$ 2-blocks and then place the one blocks between the $w$s on the edges of the $2$-blocks. Note that there are $2$ ways to place the $2$-blocks since they are just mixed $aw$ and $wa$ terms we can start with either and they would fix the rest. So then since there are already $\frac{\eta-m_1}{2}$ 2-blocks placed we have $\binom{\frac{\eta-m_1}{2}}{m_1}=2\cdot \frac{(\eta/2)^{m_1}}{m_1!} + O(\eta^{m_1-1})$ since we assume that $m$ is not on the order of $\eta$, note that this assumes that now two $1$-blocks are adjacent since if we were to have two 1-blocks being adjacent this would contribute $\frac{\eta-m_1}{2}\cdot \binom{\frac{\eta-m_1}{2}}{m_1-2}=O(\eta^{m_1-1})$ so these cases contribute a lower order. Also note that for any 1-block we can make it either $aw$ or $wa$ without restriction because they always go between $w$s. So this multiplies a factor of $2$ for every 1-block which gives $2^{m_1+1}\cdot \frac{(\eta/2)^{m_1}}{m_1!} + O(\eta^{m_1-1})=\frac{2\eta^{m_1}}{m_1!}+O(\eta^{m_1-1})$ ways to choose the locations of all $1$-blocks.

Now, we observe that the second and the first index respectively of two adjacent $1$-blocks are congruent mod $k$, as illustrated in the example below.

\begin{example}
    Consider the configuration
$$\cdots v_{i_1i_2}a_{i_2i_3}v_{i_3i_4}v_{i_4i_5}a_{i_5i_6}a_{i_6i_7}v_{i_7i_8}a_{i_8i_9}\cdots.$$
Since $a$'s within a $2$-block are matched together, we have $i_5=i_7$ with $i_6$ being free, and that $$i_3 \equiv i_4 \equiv i_5 \equiv i_7 \equiv i_8 \textup{ (mod $k$)}.$$ 
\end{example}

Thus, all the indices of terms between a pair of $1$-blocks, except for those within $2$-blocks, share a congruence class.  The number of ways to specify the congruence classes of $1$-blocks and to pair the $1$-blocks up is 
\begin{align}
\sum_{1\leq i_1,i_2,...,i_{m_1}\leq k}\mathbb{E}[a_{i_1i_2}a_{i_2i_3}...a_{i_{m_1}i_1}],
\end{align}
where each $c_{ij}\sim \mathcal{N}(0,1)$. The above expression is simply the $m_1$\textsuperscript{th} expected moment of $k\times k$ GOE.

% Now we make the observation that between any two one blocks all indices are equivalent mod $k$ other than the indices that are between the two blocks, this can be seen by the series $w_{i_1,i_2}a_{i_2,i_3}w_{i_3,i_4}w_{i_4,i_5}a_{i_5,i_6}a_{i_6,i_7}w_{i_7,i_8}a_{i_8,i_9}...$ we see that $i_5=i_7$ since the 2-block must be matched so we also get $i_3\equiv i_4\equiv i_5\equiv i_7\equiv i_8\mod{k}$ since the condition for having a $w$ instead of a $b$ is equivalence mod $k$, also note that index $i_6$ is free since the equivalence $i_5=i_7$ is sufficient to show that $a_{i_5,i_6}=a_{i_6,i_7}$. So first we can fix the equivalence class mod $k$ for all of the terms between every pair of 1-blocks. Since the 1-blocks must also be paired up they are also paired up mod $k$ we see that we can write the number of ways to pair these up specifying the matching is
% \[
% \sum_{1\leq i_1,i_2,...,i_{m_1}\leq k}\mathbb{E}[c_{i_1,i_2}c_{i_2,i_3}...c_{i_{m_1},i_1}]
% \]
% with every $c_{ij}\sum\mathcal{N}(0,1)$ which is just $\mathbb{E}\text{Tr}C^{m_1}$ where $c$ is a Gaussian Wigner matrix. So this specifies the congruence class mod $k$ for all of the indices except those that are in the middle of a 2-block.

Since the $2m_1$ 1-blocks are paired together, then there are only $m_1$ free indices. As the congruence class of these indices are fixed, the number of choices of these indices is $(\frac{N}{k})^{m_1}$. Similarly, the number of choices of indices for all the 2-blocks is $(\frac{N^2}{k})^{\frac{\eta-m_1}{2}}$, since the indices of each 2-block $a_{i_\ell i_{\ell+1}}a_{i_{\ell+1}i_{\ell+2}}$ must satisfy $i_\ell=i_{\ell+2}$, and there are $\frac{N}{k}$ choices for $i_\ell=i_{\ell+2}$ whose congruence class is fixed and $N$ choices for $i_2$ that is free. The remaining indices are those of the weight blocks, which must satisfy congruence mod $k$ and hence are each restricted to $\frac{N}{k}$ choices. By the structure imposed by the anticommutator, the total number of indices of all weight blocks is $\eta-\left(\left(\frac{\eta-m_1}{2}\right)+m_1\right)=\frac{\eta-m_1}{2}$. Thus, the total number of ways to assign indices is
$\left(\frac{N}{k}\right)^{m_1} \left(\frac{N^2}{k}\right)^{\frac{\eta-m_1}{2}}\left(\frac{N}{k}\right)^{\frac{\eta-m_1}{2}}=\frac{N^{\frac{3}{2}\eta-\frac{1}{2}m_1}}{k^\eta}$. After combining all these pieces, we arrive at the desired result for the contribution of a fixed $S$-class.
\end{proof}

In the expected $m$\textsuperscript{th} moment calculation, the following two combinatorial equalities from \cite{split} are extremely useful for cancelling the contribution of $S$-classes with fewer than $m$ blocks.
\begin{lemma}\label{inequalities}
For any $0\leq p<m$,
\begin{align}
\sum_{i=0}^m(-1)^i\binom{m}{i}i^p &\ = \ 0, \\
\sum_{i=0}^m(-1)^{m-i}\binom{m}{i}i^m &\ = \ m!.
\end{align}
\end{lemma}

Observe that if $m_1>m$, then by Lemma \ref{Sclasscontribution} the contribution of an $S$ class with $m_1$ 1-block is
\begin{align}
&\frac{1}{k}\sum_{\alpha=2n}^{8n}c_\alpha\left(\frac{k}{N^{3/2}}\right)^\alpha\frac{1}{N^m}\sum_{i=0}^m\binom{m}{i}\left(-\frac{N^{3/2}}{k}\right)^{m-i}p(\alpha+i)\frac{N^{\frac{3}{2}(\alpha+i)-\frac{1}{2}m_1}}{k^{\alpha+i}} \nonumber \\
&\ = \ \frac{C_{k,m}}{N^{\frac{1}{2}(m_1-m)}}\sum_{\alpha=2n}^{8n}c_\alpha \sum_{i=0}^m\binom{m}{i}(-1)^{m-i}p(\alpha+i) \nonumber \\
&\ = \ \frac{C_{k,m}}{N^{\frac{1}{2}(m_1-m)}}\sum_{\alpha=2n}^{8n}c_\alpha \sum_{i=0}^m\binom{m}{i}(-1)^{m-i}\left(\frac{2(\alpha+i)^{m_1}}{m_1}+O\left((\alpha+i)^{m_1-1}\right)\right) \nonumber \\
&\ = \ \frac{C_{k,m,m_1}}{N^{\frac{1}{2}(m_1-m)}}\sum_{\alpha=2n}^{8n}c_\alpha\alpha^{m_1}.
\end{align}
Since $f_1^{2n}(x) \ = \ \left(\frac{x(2-x)(x+1)(3-x)}{4}\right)^{2n}$, then $|c_\alpha|\ll C_0^{2n}$ for some $C_0>0$. Moreover, $\alpha\ll\log\log(N)$, then for some $\epsilon>0$ 
\begin{align}
\sum_{\alpha=2n}^{8n}c_\alpha \alpha^{m_1}\ll n^{m_1+1}C_0^{2n}\ll (\log\log(N))^{m_1+1}\log(N)\ll N^{1/2(m_1-m)-\epsilon}
\end{align}

Hence, as $N\rightarrow 0$, the contribution of an $S$-class with $m_1>m$ total $a$ blocks and $m_2$ total $aa$ blocks is negligible. Moreover, if $m_1<m$, then the contribution of an $S$-class with $m_1$ total $a$ blocks is
\begin{align}
&\frac{1}{k}\sum_{\alpha=2n}^{8n}c_\alpha\left(\frac{k}{N^{3/2}}\right)^\alpha\left(\frac{1}{N^m}\sum_{i=0}^m\binom{m}{i}\left(-\frac{N^{3/2}}{k}\right)^{m-i}p(\alpha+i)\left(\frac{N^{\frac{3}{2}(\alpha+i)-\frac{1}{2}m_1}}{k^{\alpha+i}}\right)\right) \nonumber \\
&\ = \ \frac{C_{k,m}}{N^{\frac{1}{2}(m_1-m)}}\sum_{\alpha=2n}^{8n}c_\alpha\sum_{i=0}^m\binom{m}{i}(-1)^ip(\alpha+i) \nonumber \\
&\ = \ \frac{C_{k,m}}{N^{\frac{1}{2}(m_1-m)}}\sum_{\alpha=2n}^{8n}c_\alpha \sum_{q=0}^{m_1}c_q \alpha^{m_1-q} \sum_{i=0}^m(-1)^i\binom{m}{i} i^q \ = \ 0.
\end{align}
Thus, we must have $m_1=m$.

\begin{theorem}\label{GOE-checkerboard Moments}
The expected $m$\textsuperscript{th} moment associated to the empirical blip spectral measure is
\begin{align}
\mathbb{E}\left[\mu_{\{A_N,B_N\},1}^{(m)}\right] \ = \ 2\left(\frac{1}{k}\right)^{m+1}\mathbb{E}_k[\textup{Tr }C^m].
\end{align}
\end{theorem}
\begin{proof}
By the discussion above, we know that $m_1=m$. Then
\begin{align}
&\mathbb{E}\left[\mu_{\{A_N,B_N\},1}^{(m)}\right] \\
&\ = \ \frac{1}{k}\sum_{\alpha=2n}^{8n}c_\alpha\left(\frac{k}{N^{3/2}}\right)^\alpha\frac{1}{N^{m+\frac{1}{2}m}} \sum_{i=0}^m \binom{m}{i}(-1)^{m-i}\left(\frac{N^{3/2}}{k}\right)^{m+\alpha}\frac{2(\alpha+i)^m}{m!}\mathbb{E}_k[\textup{Tr}C^{m}] \nonumber \\
&\ = \ \frac{2}{m!}\left(\frac{1}{k}\right)^{m+1}\mathbb{E}_k[\textup{Tr}C^m]\sum_{\alpha=2n}^{8n}c_\alpha\sum_{i=0}^m \binom{m}{i}(-1)^{m-i}(\alpha+i)^m \nonumber \\
&\ = \ \frac{2}{m!}\left(\frac{1}{k}\right)^{m+1}\mathbb{E}_k[\textup{Tr}C^m] \sum_{\alpha=2n}^{8n}c_\alpha\sum_{i=0}^m \binom{m}{i}(-1)^{m-i}\sum_{p=0}^m \binom{m}{p}\alpha^{p}i^{m-p} \nonumber \\
&\ = \ \frac{2}{m!}\left(\frac{1}{k}\right)^{m+1}\mathbb{E}_k[\textup{Tr}C]\sum_{\alpha=2n}^{8n}\sum_{p=0}^m\binom{m}{p}c_\alpha\alpha^p\sum_{i=0}^m\binom{m}{i}(-1)^{m-i}i^{m-p}.\label{MomentCalc2}
\end{align}
Since the inner sum is 0 if $p>0$ and $m!$ if $p=0$ by Lemma \ref{inequalities} and $f_1^{(2n)}(1)=\sum_{\alpha=2n}^{8n}c_\alpha=1$, then
\begin{align}
\mathbb{E}\left[\mu_{\{A_N,B_N\},1}^{(m)}\right] &\ = \ \frac{2}{m!}\left(\frac{1}{k}\right)^{m+1}\mathbb{E}_k[\textup{Tr}C^m]\sum_{\alpha=2n}^{8n}c_\alpha m! \nonumber \\
&\ = \ 2\left(\frac{1}{k}\right)^{m+1}\mathbb{E}_k[\textup{Tr}C^m].
\end{align}
\end{proof}


%\textcolor{red}{To do: need to add the error term too}
\subsection{Moments of $\{k\textup{-checkerboaord, }j\textup{-checkerboard}\}$}

\begin{prop}\label{j,k-checkerboard trace}
For $m_{1a}, m_{2a}, m_{1b}, m_{2b}\in \mathbb{Z}_{\geq 0}$, define $m_1:=m_{1a}+m_{1b}$ and $m_2:=m_{2a}+m_{2b}$. If $m_1+m_2=o(\eta)$, then the total contribution to $\E{\textup{Tr}\{A_N, B_N\}^\eta}$ of an $S$-class with $m_{1a}$ 1-blocks of $a$, $m_{1b}$ 1-blocks of $b$, $m_{2a}$ 2-blocks of $a$, and $m_{2b}$ 2-blocks of $b$ is 
\begin{equation*}
\resizebox{1.0\hsize}{!}{$\frac{2^{\eta-2m_2}\eta^{m_1+m_2}}{m_{1a}!m_{1b}!m_{2a}!m_{2b}!}2^{\frac{m_1}{2}}(m_{1a})!!(m_{1b})!!\left(\frac{1}{k}\right)^{\eta-m_{1a}-2m_{2a}}\left(\frac{1}{j}\right)^{\eta-m_{1b}-2m_{2b}}\left(1-\frac{1}{k}\right)^{\frac{m_{1a}}{2}+m_{2a}}\left(1-\frac{1}{j}\right)^{\frac{m_{1b}}{2}+m_{2b}}N^{2\eta-(m_1+m_2)}$}.
\end{equation*} 
\end{prop}

\begin{proof}
The proof is divided into two parts. First, we count the number of ways to arrange the prescribed blocks into the cyclic product of length $2\eta$ and assign the weight pairs; second, we count the number of ways to pair up all the 1-blocks and assign indices that ensures consistent indexing throughout the cyclic product.

Given $m_1+m_2=o(\eta)$, we claim that the number of ways $q(\eta)$ of arranging the prescribed blocks into the cyclic product of length $2\eta$ and assign the adjacent weight pairs is $\frac{2^{\eta-2m_2}\eta^{m_1+m_2}}{m_{1a}!m_{1b}!m_{2a}!m_{2b}!}+O(2^\eta\eta^{m_1+m_2-1})$. 
Naively, $q(\eta)$ is simply
\begin{align}
\binom{\eta-m_2}{m_1+m_2}\binom{m_1+m_2}{m_2}2^{\eta-2m_2}\binom{m_1}{m_{1a}}\binom{m_2}{m_{2a}} \ = \ \frac{2^{\eta-2m_2}\eta^{m_1+m_2}}{m_{1a}!m_{1b}!m_{2a}!m_{2b}!}+O(2^\eta\eta^{m_1+m_2-1}).
\end{align}
%Due to the structure of anticommutator, there are eight possibilities for $\{c_{i_{2\ell-1}i_{2\ell}}, c_{i_{2\ell}i_{2\ell+1}}\}$, which are $\{a_{i_{2\ell-1} i_{2\ell}}, b_{i_{2\ell}i_{2\ell+1}}\}$, $\{b_{i_{2\ell-1} i_{2\ell}}, a_{i_{2\ell}i_{2\ell+1}}\}$, $\{a_{i_{2\ell-1}i_{2\ell}}, v_{i_{2\ell}i_{2\ell+1}}\}$, $\{v_{i_{2\ell-1}i_{2\ell}}, a_{i_{2\ell}i_{2\ell+1}}\}$, $\{b_{i_{2\ell-1} i_{2\ell}}, w_{i_{2\ell}i_{2\ell+1}}\}$, $\{w_{i_{2\ell-1} i_{2\ell}}, b_{i_{2\ell}i_{2\ell+1}}\}$, $\{v_{i_{2\ell-1}i_{2\ell}}, w_{i_{2\ell}i_{2\ell+1}}\}$, or $\{w_{i_{2\ell-1}i_{2\ell}}, v_{i_{2\ell}i_{2\ell+1}}\}$. Hence, after specifying the 1-block, we know the mixed pair it belongs to, i.e. if the 1-block is of the form $c_{i_{2\ell-1}i_{2\ell}}$ and $c_{i_{2\ell-1}i_{2\ell}}=a_{i_{2\ell-1}i_{2\ell}}$ (resp. $c_{i_{2\ell-1}i_{2\ell}}=b_{i_{2\ell-1}i_{2\ell}}$), then $c_{i_{2\ell}i_{2\ell+1}}=v_{i_{2\ell}i_{2\ell+1}}$ (resp. $c_{i_{2\ell}i_{2\ell+1}}=w_{i_{2\ell}i_{2\ell+1}}$); similar conclusion holds for 1-block of the form $c_{i_{2\ell}i_{2\ell+1}}=a_{i_{2\ell}i_{2\ell+1}}$ (resp. $c_{i_{2\ell}i_{2\ell+1}}=b_{i_{2\ell}i_{2\ell+1}}$). Moreover, after specifying the 2-block, we know both of its adjacent entries (or the two mixed pairs the 2-block belongs to), i.e. every 2-block is of the form $c_{i_{2\ell}i_{2\ell+1}}c_{i_{2\ell+1}i_{2\ell+2}}$, and if $\{c_{i_{2\ell}i_{2\ell+1}}, c_{i_{2\ell+1}i_{2\ell+2}}\}=\{a_{i_{2\ell}i_{2\ell+1}}, a_{i_{2\ell+1}i_{2\ell+2}}\}$, then $\{c_{i_{2\ell-1}i_{2\ell}}, c_{i_{2\ell+2}i_{2\ell+3}}\}=\{v_{i_{2\ell-1}i_{2\ell}}, v_{i_{2\ell+2}i_{2\ell+3}}\}$; if $\{c_{i_{2\ell}i_{2\ell+1}}, c_{i_{2\ell+1}i_{2\ell+2}}\}=\{b_{i_{2\ell}i_{2\ell+1}}, b_{i_{2\ell+1}i_{2\ell+2}}\}$, then $\{c_{i_{2\ell-1}i_{2\ell}}, c_{i_{2\ell+2}i_{2\ell+3}}\}=\{w_{i_{2\ell-1}i_{2\ell}}, w_{i_{2\ell+2}i_{2\ell+3}}\}$. We also know that a weight pair $\{c_{i_{2\ell-1}i_{2\ell}}, c_{i_{2\ell}i_{2\ell+1}}\}\in \{\{w_{i_{2\ell-1}i_{2\ell}}, v_{i_{2\ell}i_{2\ell+1}}\},\{v_{i_{2\ell-1}i_{2\ell}}, w_{i_{2\ell}i_{2\ell+1}}\}\}$. Thus, we can view specifying a cyclic product of length $2\eta$ as only specifying $\eta$ entries, where specifying one term of the pair $c_{i_{2\ell-1}i_{2\ell}}c_{i_{2\ell}i_{2\ell+1}}$ and whether the pair is weight/non-weight or not uniquely determines the other term of the pair.


That is, we first choose all the $m_1+m_2$ blocks, viewing each of the $m_2$ 2-block as a $1$-block (where $\eta-m_2$ comes from), which can be done in $\binom{\eta-m_2}{m_1+m_2}$ ways. Then, we choose the $m_2$ 2-blocks from all the $m_1+m_2$ blocks, $m_{1a}$ 1-blocks of $a$ from all the $m_1$ 1-blocks, $m_{2a}$ 2-blocks of $a$ from all the $m_2$ 2-blocks, and finally specifying the mixed pairs the 1-blocks belong to and the weight pairs, all of which can be done in $\binom{m_1+m_2}{m_2}2^{\eta-2m_2}\binom{m_1}{m_{1a}}\binom{m_2}{m_{2a}}$ ways.

However, this naive counting method fails to account for the restriction that different mixed pairs of $a$ and $v$ and of $b$ and $w$ cannot be placed adjacent to each other to form a 2-block (e.g. if two mixed pairs $va$ and $av$ are adjacent to each other, then we have a 2-block of $a$). The number of ways of having at least one 2-block formed from different mixed pairs of $a$ and $v$ and of $b$ and $w$ is $2^{\eta-2m_2-1}(\eta-m_2)\binom{\eta-m_2-2}{m_1+m_2-2}\binom{m_1+m_2}{m_2}\binom{m_1}{m_{1a}}\binom{m_2}{m_{2a}}=O(2^\eta \eta^{m_1+m_2-1})$ if $m_1+m_2=o(\eta)$. Thus,
\begin{align}
q(\eta) \ = \ \frac{2^{\eta-2m_2}\eta^{m_1+m_2}}{m_{1a}!m_{1b}!m_{2a}!m_{2b}!}+O(2^\eta\eta^{m_1+m_2-1}).
\end{align}

Similarly, we can also guarantee that none of the 1-blocks are adjacent to each other, since the number of ways of having at least two adjacent 1-blocks is $2^{\eta-2m_2}(\eta-m_2)\binom{\eta-m_2-2}{m_1+m_2-2}\binom{m_1+m_2}{m_2}\binom{m_1}{m_{1a}}\binom{m_2}{m_{2a}}=O(2^\eta \eta^{m_1+m_2-1})$ if $m_1+m_2=o(\eta)$.

%first, we count the number of ways to arrange $m_{1a}$ 1-block of $a$, $m_{2a}$ 2-block of $a$, $m_{1b}$ 1-block of $b$, $m_{2b}$ 2-block of $b$ into a cyclic 
% First we see that the number of ways to choose where the blocks are is on the order of $\frac{2^{m_1}\eta^{m_1+m_2}}{m_{a1}!m_{b1}!m_{a2}!m_{b2}!}$, note that we choose $\eta$ to be large so we ignore the lower order terms. First, we can choose where the $2$-blocks are. There are $m_2$ total $2$-blocks, but we have the restriction that for any $2$-block must have its first $a$ or $b$ at an even index due to the expansion of $(AB+BA)^\eta$. So since we are calculating the $\eta$th moment, there are $2\eta$ terms, so there are $\eta$ possible starting points, also note that no two $2$-blocks can be directly adjacent, note that two $2$-blocks have at least one index between them that wiell be free, note that this is important for when we actually calculate the number of ways we can assign indices. So there are originally $\binom{\eta}{m_2}=\frac{\eta^{m_2}}{m_2!}+O(\eta^{m_2-1})$ choices for the $2$-blocks but since we cannot have any pair next to each other we get that if we fix two $2$-blocks to be adjacent we get that we must subtract at most $\eta\cdot \binom{\eta-2}{m_2-2}=O(\eta^{m_2-1})$ which means that the dominating term is $\frac{\eta^{m_2}}{m_2!}$ so then we can separate these into $a$-blocks and $b$-blocks which we can do arbitrarily giving \[\frac{\eta^{m_2}}{m_2!}\cdot\binom{m_2}{m_{2a}}=\frac{\eta^{m_2}}{m_{2a}!m_{2b}!}.\]

%So now that we have placed the $2$-blocks we can place the $1$-blocks note that the $1$-blocks do not have any restriction on the parity of the starting index and can start at any remaining open index. We see that every $2$-block takes up $4$ spaces since they are all of the form $waaw$ or $wbbw$ so to choose the remaining squares directly there are $\binom{2\eta-4m_2}{m_1}=\frac{2^{m_1}\eta^{m_1}}{m_1!}$ but we again have the restriction that no two of $1$-blocks are directly adjacent or even less than three $w$s apart (i.e. $wabw$) so if we set two $1$-blocks to be within three adjacent places we subtract a term on the order of $(\eta-4m_2)\cdot 3\binom{\eta-4m_2-2}{m_1-2}=O(\eta^{m_1-1})$ which is much smaller than the dominating term. So there are $\frac{2^{m_1}\eta^{m_1}}{m_1!}$ ways to choose the places where we have $1$-blocks and we must separate them into $a$'s or $b$'s arbitrarily so we multiply by $\binom{m_1}{m_{1a}}$ and get 
%\begin{align}
%\frac{2^{m_1}\eta^{m_1}}{m_1}\cdot \binom{m_1}{m_{1a}}=\frac{2^{m_1}\eta^{m_1}}{m_{1a}!m_{1b}!}
%\end{align}
%So this proves that the number of ways to assign all the $1$ and $2$-blocks for $S$ is 
%\begin{align}
%\frac{2^{m_1}\eta^{m_1+m_2}}{m_{a1}!m_{b1}!m_{a2}!m_{b2}!}.
%\end{align}

Now, we count the number of ways to assign the indices for the $S$-class. In contrast to \cite{split} where there are restrictions on the indices of the 1-blocks, we demonstrate in the example below that we can remove such restrictions.

\begin{example}
Consider a weight block surrounded by two non-weight terms $c_{i_{\ell-1} i_{\ell}} c_{i_{\ell}i_{\ell+1}}\cdots c_{i_{t}i_{t+1}} c_{i_{t+1} i_{t+2}}$, where $t-\ell$ is sufficiently large. We assume without loss of generality that $c_{i_{\ell-1} i_{\ell}}=a_{i_{\ell-1} i_{\ell}}$ and $c_{i_{t+1} i_{t+2}}=a_{i_{t+1} i_{t+2}}$. After specifying $i_{\ell}$, if $c_{i_{\ell}i_{\ell+1}}=w_{i_{\ell}i_{\ell+1}}$, then $i_{\ell}\equiv i_{\ell+1}\textup{ (mod $k$)}$ and there are $\frac{N}{k}$ choices of indices for $i_{\ell+1}$; if $c_{i_{\ell}i_{\ell+1}}=v_{i_{\ell}i_{\ell+1}}$, then $i_{\ell}\equiv i_{\ell+1}\textup{ (mod $j$)}$ and there are $\frac{N}{j}$ choices of indices for $i_{\ell+1}$. After specifying $i_{t+1}$, there are similar number of choice of indices for $i_{t}$. Since $t-\ell$ is sufficiently large, then with high probability there exists $\ell+1\leq s\leq t-2$ such that $\{c_{i_s i_{s+1}}, c_{i_{s+1}i_{s+2}}\}=\{w_{i_s i_{s+1}}, v_{i_{s+1}i_{s+2}}\}$. We can specify the indices $i_{\ell+2}, \dots, i_s$ and $i_{s+2}, \dots, i_{t-1}$ the same way as before. Then we have $i_{s+1}\equiv i_s\textup{ (mod $k$)}$ and $i_{s+1}\equiv i_{s+2}\textup{ (mod $j$)}$. Since $\textup{gcd}(k, j)=1$ and $jk\mid N$, then by Chinese remainder theorem, there are $\frac{N}{kj}$ choices of indices $i_{s+1}$. If the number of $w$'s and $v$'s in the weight block is $r$ and $t-\ell+1-r$, respectively, then there are $\left(\frac{1}{k}\right)^{r}\left(\frac{1}{j}\right)^{t-\ell+1-r}N^{t-\ell}$ ways of specifying the $i_{\ell+1}, \dots, i_{t}$. Thus, regardless of the indices we specify for the two non-weight terms surrounding a weight block, we can guarantee with high probability consistency of indexing throughout the weight block.
\end{example}

We know that the total number of $w$'s and $v$'s in a cyclic product of length $2\eta$ is $\eta-m_{1a}-2m_{2a}$ and $\eta-m_{1b}-2m_{2b}$, respectively. Then the number of choices of congruence classes of indices for all the $w$'s and $v$'s has the corresponding factors $\left(\frac{1}{k}\right)^{\eta-m_{1a}-2m_{2a}}$ and $\left(\frac{1}{j}\right)^{\eta-m_{1b}-2m_{2b}}$. Now, by Lemma \ref{blockslemma}, each 1-block is paired up with another 1-block and the two terms of each 2-block are paired up with each other. Moreover, the indices of any non-weight terms $a_{i_\ell i_{\ell+1}}$ and $b_{i_t i_{t+1}}$ must satisfy the modular restrictions $i_\ell\not\equiv i_{\ell+1}\textup{ (mod $k$)}$ and $i_t\not\equiv i_{t+1}\textup{ (mod $j$)}$. Then similarly, the number of choices of congruence classes of indices for all the $a$'s and $b$'s has the corresponding factors $\left(1-\frac{1}{k}\right)^{\frac{m_{1a}}{2}+m_{2a}}$ and $\left(1-\frac{1}{j}\right)^{\frac{m_{1b}}{2}+m_{2b}}$. Since the loss of degrees of freedom per block in a contributing configuration is 1, then the contribution of actually specifying all the indices is $N^{2\eta-(m_1+m_2)}$. Thus, the number of ways of assigning the indices that guarantees consistent indexing is
\begin{align}
\left(\frac{1}{k}\right)^{\eta-m_{1a}-2m_{2a}}\left(\frac{1}{j}\right)^{\eta-m_{1b}-2m_{2b}}\left(1-\frac{1}{k}\right)^{\frac{m_{1a}}{2}+m_{2a}}\left(1-\frac{1}{j}\right)^{\frac{m_{1b}}{2}+m_{2b}}N^{2\eta-(m_1+m_2)}.
\end{align}
%So now it only remains to count the number of ways to assign indices. By a previous lemma, given the total number of blocks, the $S$-classes with the highest degree of freedom satisfy that the average loss of degree of freedom per block is 1. Since there are $|S|$ blocks, and the loss of degree of freedom comes solely from matching $a$'s ans $b$'s in those blocks, then naively the total number of ways to assign indices to an arbitrary cyclic product in the $S$-class is $N^{2\eta-|S|}$. However, this fails to account for the following restrictions on the indices: (1) indices of a weight $w_{i_ri_{r+1}}$ from $A$ satisfies $i_r\equiv i_{r+1}\textup{ (mod $k$)}$; (2) indices of a weight $v_{i_ri_{r+1}}$ from $B$ satisfies $i_r\equiv i_{r+1}\textup{ (mod $j$)}$; (3) indices of a non-weight $a_{i_ri_{r+1}}$ satisfies $i_{r}\not\equiv i_{r+1}\textup{ (mod $k$)}$; (4) indices of a non-weight $b_{i_ri_{r+1}}$ satisfy $i_{r}\not\equiv i_{r+1}\textup{ (mod $j$)}$.

%We first turn to matching up the $a$'s and $b$'s within their 1 and 2-blocks. From before, we know that the $a$'s and $b$'s within 2-blocks must be matched together. For the 1-blocks, there are $m_{1a}!!$ ways of matching up the $m_{1a}$ 1-blocks of $a$, and similarly $m_{1b}!!$ ways of matching up the $m_{1b}$ 1-blocks of $b$. For each pair of 1-blocks, there are two ways of assigning their indices. For example, if $a_{i_r i_{r+1}}$ is paired with $a_{i_s i_{s+1}}$, we can either set $i_r = i_s$ and $i_{r+1} = i_{s+1}$, or $i_r = i_{s+1}$ and $i_{r+1} = i_s$. With $m_1 = m_{1a}+m_{1b}$ total 1-blocks, this contributes a term of $2^{\frac{m_1}{2}}$. 

%To incorporate the above restrictions in assigning indices, we first look at those of the non-weight $a$'s and $b$'s. For each pair of $a$'s, after assigning the first index, the second must \textit{not} be congruent to the first mod $k$. This reduces the number of possibilities for the second index from $N$ to $N\left(1-\frac{1}{k}\right)$, or, since we are collecting all our $N$'s in our naive expression $N^{2\eta-|S|}$, this introduces a factor of $\left(1-\frac{1}{k}\right)$ per pair of $a$'s. We have in total $\frac{m_{1a}}{2}+m_{2a}$ pairs of $a$'s and hence, restrictions in assigning the indices of all $a$'s yield overall a term of $\left(1-\frac{1}{k}\right)^{\frac{m_{1a}}{2}+m_{2a}}$. A parallel argument follows for the $b$'s whose indices must not be congruent to each other mod $j$, which yields a term of $\left(1-\frac{1}{j}\right)^{\frac{m_{1b}}{2}+m_{2b}}$.

%We now look at the restrictions in assigning indices to weights $w$ and $v$. We show first that configurations containing a single weight or two weights of the same kind in isolation (e.g. $avva$ or $bwb$) may not allow for a consistent choice of indices given how the indices for our non-weight entries have been assigned in the previous paragraph.

%\begin{example}
    %Consider the configuration $$\cdots a_{i_1i_2}v_{i_2i_3}v_{i_3i_4}a_{i_4i_5}\cdots.$$ Then we must have $$i_2 \equiv i_3 \equiv i_4 \textup{ (mod $j$)}.$$ 
    %However, in assigning the indices of the $a$'s as described above, we only specified that $i_2 \not\equiv i_1 \textup{ (mod $j$)}$ and $i_4 \not\equiv i_5 \textup{ (mod $j$)}$ and not necessarily that $i_2 \equiv i_4 \textup{ (mod $j$)}$. Hence it may be that we end up with an inconsistent choice of indices.
%\end{example}

%We therefore exclude such configurations from our calculations, justifying this by noting that their contributions becomes small and negligible as $\eta$ gets large. 

%For weights in the remaining configurations, after specifying the first index of a weight $w$, the second index must be congruent to it mod $k$. This reduces the number of possibilities for this second index from $N$ to $\frac{N}{k}$ and again, with all the $N$'s collected in $N^{2\eta-|S|}$ and $\eta - m_{1a} - 2m_{2a}$ number of weights $w$, restrictions in assigning the indices of all $w$'s yield overall a term of $\left(\frac{1}{k}\right)^{\eta - m_{1a} - 2m_{2a}}$. Similarly, with weights $v$ whose indices must be congruent to each other mod $j$, we get a term of $\left(\frac{1}{j}\right)^{\eta - m_{1b} - 2m_{2b}}$.

Finally, since there are $m_{1a}$ 1-block of $a$ and $m_{1b}$ 1-block of $b$, and there are no restrictions on their indices, then the number of ways of matching up all the 1-blocks is $2^{\frac{m_1}{2}}(m_{1a})!!(m_{1b})!!$. Note that the $2^{\frac{m_1}{2}}$ factor is due to the fact that for any two paired 1-blocks $c_{i_\ell i_{\ell+1}}, c_{i_t, i_{t+1}}$, we either have $i_\ell=i_{t+1}$ and $i_{\ell+1}=i_t$ or $i_{\ell}=i_t$ and $i_{\ell+1}=i_{t+1}$. This completes the proof.
\end{proof}

%\subsection{Moments of $\{\textup{GOE, k-checkerboard}\}$}
%Now, we consider the limiting bulk and blip distribution of anticommutator of GOE and k-checkerboard.

%\begin{lemma} For the anticommutator of $A$ and $B$ where $A$ is a GOE and $B$ is $m$-checkerboard. For sequences $f,g$ with conditions $f(0)=f(1)=1$ and $g(1)=1$ and recurrence defined by

%\begin{align}
%f(k)=2\sum_{j=1}^{k-1}g(j)f(k-j) + g(k)
%\end{align}
%and 
%\begin{align}
%g(k)=2f(k-1) + \sum_{\substack{0\leq x_1,x_2<k-1\\ x_1+x_2<k-1}}(1+\mathbb{1}_{x_1>0})(1+\mathbb{1}_{x_2>0})f(x_1)f(x_2)g(k-1-x_1-x_2)
%\end{align}
%we get that the $2k$th moment of the bulk $k>0$ is $M_{2k}=2(1-\frac{1}{m})^{k}f(k)$.
%\end{lemma}

%\begin{proof}
%We note from \cite{Tao1} that the distribution of the bulk of the anticommutator of a GOE and $(m,1)$-checkerboard is given by the anticommutator of the GOE and the $(m,0)$-checkerboard. Note that since all of the non-weight elements of the $(m,0)$-checkerboard are independent mean $0$, variance $1$ variables and thus they must be paired similarly to how they are in lemma 2.1 of \cite{split}. So the if we choose the matchings for such an anticommutator the number of ways to choose these is the same as in \ref{GOE-GOE moment recurrence}, however we cannot have any of the weights appearing. If we choose the indices arbitrarily we see that there are $2k$ $b_{i_{j},i_{j+1}}$ terms but they are all paired up so there are $k$ distinct pairs of indices (note that if there were any fewer than $k$ there would be a loss of degrees of freedom), so the probability that none of these $k$ pairs of indices are weights is $\left(1-\frac{1}{m}\right)^k$ since there is a $\frac{1}{m}$ chance that the two arbitrarily chosen indices are equivalent mod $m$.
%\end{proof}

%\begin{corollary} For anticommutator of $A$ and $B$ where $A$ is a $n$-checkerboard and $B$ is $m$-checkerboard. For sequences $f,g$ with conditions $f(0)=f(1)=1$ and $g(1)=1$ and recurrence defined by

%\begin{align}
%f(k)=2\sum_{j=1}^{k-1}g(j)f(k-j) + g(k)
%\end{align}
%and 
%\begin{align}
%g(k)=2f(k-1) + \sum_{\substack{0\leq x_1,x_2<k-1\\ x_1+x_2<k-1}}(1+\mathbb{1}_{x_1>0})(1+\mathbb{1}_{x_2>0})f(x_1)f(x_2)g(k-1-x_1-x_2)
%\end{align}
%we get that the $2k$th moment of the bulk for $k>0$ is $M_{2k}=2(1-\frac{1}{m})^{k}(1-\frac{1}{n})^{k}f(k)$.
%\end{corollary}

%\begin{proof}
%Similarly to the previous lemma we see that by \cite{Tao1} the distribution of the bulk is the distribution of the anticommutator of a $(m,0)$-checkerboard and $(n,0)$-checkerboard. SO again we see that we cannot have any weights and the $a_{i_j,i_{j+1}}$ terms and $b_{i_\ell,i_{\ell+1}}$ terms must be paired up. Assuming that we ignore the possibility of having a weight from either of the checkerboard matrices, the number of ways to pair these is equivalent to the number of ways to pair the GOEs together which gives the same recurrence from \ref{GOE-GOE moment recurrence}. So now we only need to account for the fact that the indices must match such that there are no weights appearing. Similarly to the previous lemma we see that there are $k$ pairs of indices corresponding to $a$ terms and $k$ pairs of indices corresponding to $b$ terms. For each pair of indices corresponding to $a$ there is a $1-\frac{1}{n}$ chance that the pair doesn't correspond to a weight and similarly for every $b$ term there is a $1-\frac{1}{m}$ chance and we choose the pairs of indices arbitrarily so these are the probabilities for each of the $k$ pairs giving the extra factor of $(1-\frac{1}{m})^{k}(1-\frac{1}{n})^{k}$ in the final result.
%\end{proof}



%\begin{definition}[Largest Blip Spectral Measure]
%The largest blip spectral measure for k-checkerboard matrix $A_N$ and j-checkerboard matrix $B_N$ is defined as 
%\begin{align}
%\mu_{A,B,N}=\sum_{\lambda\text{ eigenvalue of }\{A_N,B_N\}}f_{n(N)}\left(\frac{jk\lambda}{2N^2}\right)\delta\left(x-\left(\frac{\lambda-\frac{2}{jk}N^2}{N}\right)\right)
%\end{align}
%where $f_{n(N)}$ is the weight function $x^{2n}(2-x)^{2n}$ and $n(N)=O(\log \log N)$. Also note that we can write the expansion of the weight function as $\sum_{\alpha=2n}^{4n}c_\alpha x^{\alpha}$.
%\end{definition}

\begin{theorem}
The $m$\textsuperscript{th} moment of the largest blip spectral measure is 
\begin{align*}
&\mathbb{E}\left[\mu_{\{A_N,B_N\}}^{(m)}\right] \ = \ \\
&\sum_{\substack{m_{1a}+m_{1b}+m_{2a}+m_{2b}=m; \\ m_{1a},m_{1b}\textup{ even}}}&C(m, m_{1a}, m_{2a}, m_{1b}, m_{2b})\left(k\sqrt{1-\frac{1}{k}}\right)^{m_{1a}+2m_{2a}}\left(j\sqrt{1-\frac{1}{j}}\right)^{m_{1b}+2m_{2b}},\end{align*}
where $C(m, m_{1a}, m_{2a}, m_{1b}, m_{2b}):=m!\left(\frac{2}{jk}\right)^m\frac{2^{\frac{m_{1a}+m_{1b}}{2}-2(m_{2a}+m_{2b})}m_{1a}!!m_{1b}!!}{m_{1a}!m_{1b}!m_{2a}!m_{2b}!}$.
\end{theorem}

\begin{proof}
%\begin{align}
%\E{\mu_{A,B,N}^{(m)}}& \ = \ \mathbb{E}\left[\sum_\lambda \sum_{\alpha=2n}^{4nl}c_\alpha\left(\frac{\lambda}{\frac{2}{jk}N^2}\right)^\alpha \left(\frac{\lambda-\frac{2}{jk}N^2}{N}\right)^m\right] \\
%& \ = \ \mathbb{E}\left[\sum_\lambda \sum_{\alpha=2n}^{4nl}c_\alpha\left(\frac{1}{\frac{2}{jk}N^2}\right)^\alpha\frac{1}{N^m} \left(\sum_{i=0}^m\binom{m}{i}\lambda^i\left(-\frac{2}{jk}N^2\right)^{m-i}\right)\right] \\
%& \ = \ \sum_{\alpha=2n}^{4nl}c_\alpha \left(\frac{1}{\frac{2}{jk}N^2}\right)^\alpha \frac{1}{N^m}\sum_{i=0}^m\binom{m}{i}\left(-\frac{2}{jk}N^2\right)^{m-i}\E{\text{Tr}\{A_N,B_N\}^{\alpha+i}}
%\end{align}

%Note that the first equality follows from binomial expansion and the second equality from the eigenvalue trace lemma.

By Lemma \ref{blockslemma}, it suffices to consider the contributions from $S$-classes with 1-blocks and 2-blocks. Applying Proposition \ref{j,k-checkerboard trace} to Equation \eqref{momentOfCNDN}, we have
\begin{align}
&\mathbb{E}\left[\mu^{(m)}_{\{A_N,B_N\}}\right] \ = \ \sum_{\beta=2n}^{4nl}d_\beta \left(\frac{jk}{2N^2}\right)^\beta \frac{1}{N^m}\sum_{i=0}^m\binom{m}{i}\left(-\frac{2}{jk}N^2\right)^{m-i}\left(\frac{2}{kj}\right)^{\beta+i}N^{2(\beta+i)-(m_1+m_2)}
\nonumber \\
&\sum_{\substack{m_{1a}, m_{1b}, m_{2a}, m_{2b} \\ m_{1a}, m_{1b}\textup{ even}}} \frac{2^{-2m_2+m_1/2}(m_{1a})!!(m_{1b})!!}{m_{1a}!m_{1b}!m_{2a}!m_{2b}!}\left(k\sqrt{1-\frac{1}{k}}\right)^{m_{1a}+2m_{2a}}\left(j\sqrt{1-\frac{1}{j}}\right)^{m_{1b}+2m_{2b}}(\beta+i)^{m_1+m_2}.
\end{align}
Similar to the blip moment calculation of $\{\textup{GOE, k-checkerbaord}\}$, we require $m_1+m_2\leq m$, since otherwise the contribution from $m_1+m_2>m$ vanishes in the limit. Moreover, by Lemma \ref{inequalities}, the sum $\sum_{i=0}^m\binom{m}{i}(-1)^{m-i}(\beta+i)^{m_1+m_2}$ vanishes except for the $m^{th}$ power of $i$. Hence, the contribution to the moment in the limit only comes from $m_1+m_2=m$. After combining terms and canceling out the dependency on $i$, and noting that $g_0^{2n}(1)=\sum_{\beta=2n}^{4ln}d_\beta x^\beta$, we have
\begin{align}
&\mathbb{E}\left[\mu^{(m)}_{\{A_N,B_N\}}\right] \ =\nonumber\\ \ &\sum_{\substack{m_{1a}+m_{1b}+m_{2a}+m_{2b}=m; \\ m_{1a},m_{1b}\textup{ even}}}C(m, m_{1a}, m_{2a}, m_{1b}, m_{2b})\left(k\sqrt{1-\frac{1}{k}}\right)^{m_{1a}+2m_{2a}}\left(j\sqrt{1-\frac{1}{j}}\right)^{m_{1b}+2m_{2b}}.
\end{align}
\end{proof}
%Now, when $m_1+m_2=|S|<m$, the power of $i$ is less than $m$ so by \ref{inequalities}, we can exclude these cases and only consider the case when $|S|=m$ and ignore all of the lower order terms in $(\alpha+i)^{|S|}$. Combining like terms, we see that all of the powers of $N$ and dependencies on $i$ cancel, reducing to
%\begin{equation*}
%\resizebox{1.0\hsize}{!}{$\sum_{\substack{m_{1a}+m_{1b}+m_{2a}+m_{2b}=m; \\ m_{1a},m_{1b}\textup{ even}}}\frac{m!2^{m+m_1/2-2m_2}m_{1a}!!m_{1b}!!}{m_{1a}!m_{1b}!m_{2a}!m_{2b}!}\left(\frac{1}{k}\right)^{m-m_{1a}-2m_{2a}} \left(\frac{1}{j}\right)^{m-m_{1b}-2m_{2b}} \left(1-\frac{1}{k}\right)^{m_{1a}/2+m_{2a}} \left(1-\frac{1}{j}\right)^{m_{1b}/2+m_{2b}}\sum_{\alpha=2n}^{4n}c_\alpha$.}
%\end{equation*}
%Lastly, since $f_n(1)=1$, by definition we have that $\sum_{\alpha=2n}^{4n}c_\alpha=1$ which proves our result.
%\subsection{$\{\textup{GOE, k-checkerboard}\}$} For the anticommutator of GOE and k-checkerboard, the bulk is of order $O(N)$ and the blips on either side of the bulk (each of which contains $k$ eigenvalues) are of order $O(N^{3/2})$. This can be easily justified using Weyl's inequality, by separating out the contribution of the bulk and the blip: $\{A_N,B_N\}=A_N(\overline{B_N}+\Tilde{B_N})+(\overline{B_N}+\Tilde{B_N})A_N=A_N\overline{B_N}+\overline{B_N}A_N+\Tilde{B_N}\overline{B_N}+\overline{B_N}\Tilde{B_N}$, where $\Tilde{B_N}$ is the weight matrix and $\overline{B_N}:=B_N-\Tilde{B_N}$. More precisely, the two blips are at $\pm\frac{1}{k}N^{3/2}+O(N)$, respectively.

%Since there are two blips, we need to use the following weight function to remove the contribution from other blips than the one we're looking at. Let $w_1=\frac{1}{k}$ and $w_2=-\frac{1}{k}$, then

%\begin{align}
%f_1^{2n}(x) &\ = \ \left(\frac{x(2-x)(x-\frac{w_2}{w_1})(2-x-\frac{w_2}{w_1})}{(1-\frac{w_2}{w_1})^2}\right)^{2n} \\
%&\ = \ \left(\frac{x(2-x)(x+1)(3-x)}{4}\right)^{2n}\end{align}


%\begin{definition}
%The \textbf{empirical blip spectral measure} associated to the anticommutator of an $N\times N$ GOE and k-checkerboard $\{A_N,B_N\}:= A_NB_N+B_NA_N$ around $w_iN^{3/2}$ is
%\begin{align}
%\mu_{\{A_N,B_N\},i} \ = \ \frac{1}{k}\sum_{\lambda}f_i^{2n}\left(\frac{\lambda}{w_iN^{3/2}}\right)\delta\left(\frac{x-\left(\lambda-w_iN^{3/2}\right)}{N}\right)
%\end{align}
%\end{definition}

%We set $i=1$ and look at the blip centered at $\frac{1}{k}N^{3/2}+O(N)$ first. Note that the polynomial $f_1^{2n}(x)$ can be written as $\sum_{\alpha=2n}^{8n}c_\alpha x^\alpha$. Then the expected $m$-th moment associated with the empirical blip spectral measure is
%\begin{align}
%\mathbb{E}\left[\mu_{\{A_N,B_N\},1}^{(m)}\right] &\ = \ \mathbb{E}\left[\frac{1}{k}\sum_{\lambda}\sum_{\alpha=2n}^{8n}c_\alpha\left(\frac{k\lambda}{N^{3/2}}\right)^\alpha\left(\frac{\lambda-w_1N^{3/2}}{N}\right)^{m}\right] \\
%&\ = \ \mathbb{E}\left[\frac{1}{k}\sum_{\alpha=2n}^{8n}c_\alpha\left(\frac{k}{N^{3/2}}\right)^\alpha\left(\frac{1}{N^m}\sum_{i=0}^m \binom{m}{i}\left(-\frac{N^{3/2}}{k}\right)^{m-i}\textup{Tr}(\{A_N,B_N\}^{\alpha+i})\right)\right] \\
%&\ = \ \frac{1}{k}\sum_{\alpha=2n}^{8n}c_\alpha\left(\frac{k}{N^{3/2}}\right)^\alpha\frac{1}{N^m}\sum_{i=0}^m \binom{m}{i}\left(-\frac{N^{3/2}}{k}\right)^{m-i}\mathbb{E}[\textup{Tr}(\{A_N,B_N\}^{\alpha+i})]
%\end{align}

%\begin{lemma}\label{Sclasscontribution} The total contribution to $\mathbb{E}\text{Tr}\{A_N,B_N\}^\eta$ of an $S_{ab}$ class $C$ with $m_1$ 1-blocks of $a$'s and $m_2\leq \frac{\eta-m_1}{2}$ 2-blocks of $a$'s is 
%\begin{align}
%p(\eta)\left(\left(\frac{N^{\frac{3}{2}\eta -\frac{1}{2}m_1}}{k^{\eta}}\right) + O\left(\frac{N^{\frac{3}{2}\eta -\frac{1}{2}m_1-1}}{k^\eta}\right)\right)\mathbb{E}_k\textup{Tr}C^{m_1}
%\end{align}

%Where $p(\eta)=\frac{2\eta^{m_1}}{m_1!}+O(\eta^{m_1-1})$ and $C$ is a $k\times k$ Gaussian Wigner matrix.
%\end{lemma}

%\begin{proof}
%First we note that by \ref{blockslemma} any $S_{ab}$-class with at least one $b$ would have fewer degrees of freedom meaning that they would contribute at most $O\left(\left(\frac{N}{k}\right)^{3/2 \eta - m_1/2-1}\right)$ so we only need to consider the case where $m_2=\frac{\eta-m_1}{2}$ and there are no $b$'s.

%First we count the number of ways we can have a list of $m_1$ 1-blocks and $\frac{\eta-m_1}{2}$ 2-blocks, first we can place the $\frac{\eta-m_1}{2}$ 2-blocks and then place the one blocks between the $w$s on the edges of the $2$-blocks. Note that there are $2$ ways to place the $2$-blocks since they are just mixed $aw$ and $wa$ terms we can start with either and they would fix the rest. So then since there are already $\frac{\eta-m_1}{2}$ 2-blocks placed we have $\binom{\frac{\eta-m_1}{2}}{m_1}=2\cdot \frac{(\eta/2)^{m_1}}{m_1!} + O(\eta^{m_1-1})$ since we assume that $m$ is not on the order of $\eta$, note that this assumes that now two $1$-blocks are adjacent since if we were to have two 1-blocks being adjacent this would contribute $\frac{\eta-m_1}{2}\cdot \binom{\frac{\eta-m_1}{2}}{m_1-2}=O(\eta^{m_1-1})$ so these cases contribute a lower order. Also note that for any 1-block we can make it either $aw$ or $wa$ without restriction because they always go between $w$s. So this multiplies a factor of $2$ for every 1-block which gives $2^{m_1+1}\cdot \frac{(\eta/2)^{m_1}}{m_1!} + O(\eta^{m_1-1})=\frac{2\eta^{m_1}}{m_1!}+P(\eta^{m_1-1})$ ways to choose the locations of all $1$-blocks.

%Now we make the observation that between any two one blocks all indices are equivalent mod $k$ other than the indices that are between the two blocks, this can be seen by the series $w_{i_1,i_2}a_{i_2,i_3}w_{i_3,i_4}w_{i_4,i_5}a_{i_5,i_6}a_{i_6,i_7}w_{i_7,i_8}a_{i_8,i_9}...$ we see that $i_5=i_7$ since the 2-block must be matched so we also get $i_3\equiv i_4\equiv i_5\equiv i_7\equiv i_8\mod{k}$ since the condition for having a $w$ instead of a $b$ is equivalence mod $k$, also note that index $i_6$ is free since the equivalence $i_5=i_7$ is sufficient to show that $a_{i_5,i_6}=a_{i_6,i_7}$. So first we can fix the equivalence class mod $k$ for all of the terms between every pair of 1-blocks. Since the 1-blocks must also be paired up they are also paired up mod $k$ we see that we can write the number of ways to pair these up specifying the matching is
%\[
%\sum_{1\leq i_1,i_2,...,i_{m_1}\leq k}\mathbb{E}[c_{i_1,i_2}c_{i_2,i_3}...c_{i_{m_1},i_1}]
%\]
%with every $c_{ij}\sum\mathcal{N}(0,1)$ which is just $\mathbb{E}\text{Tr}C^{m_1}$ where $c$ is a Gaussian Wigner matrix. So this specifies the congruence class mod $k$ for all of the indices except those that are in the middle of a 2-block.

%Now we can count the number of ways to assign indices. First we can assign the indices on all of the $1$-blocks which have $2m$ indices, but since they are already paired up and this matching is already assigned we have $m_1$ choices for these indices and since the congruence class of all of these indices is fixed there are $(\frac{N}{k})^{m_1}$ ways to choose this. Then we can assign the indices of the 2-blocks, we see that for a 2-block $a_{i_1,i_2}a_{i_2,i_3}$ we have $i_1=i_3$ whose congruence class mod $k$ is fixed so there are $\frac{N}{k}$ choices for this index and $i_2$ can be anything so there are $N$ choices which gives $\frac{N^2}{k}$ choices for every 2-block which is $(\frac{N^2}{k})^{\frac{\eta-m_1}{2}}$. Now we can chose the remaining indices which is just any index that isn't in any 1-block or 2-block so these are just indices between two $w$s which is equal to the number of times we have two adjacent $w$s. We see that by symmetry the number of times we have two adjacent $w$s is equal to the number of times we have two adjacent $a$'s which is just equal to the number of 2-blocks by definition. Each of these also must satisfy the equivalence mod $k$ so they have fixed congruence class so there are $\frac{N}{k}$ choices for each of these giving a total factor of $(\frac{N}{k})^{\frac{\eta-m_1}{2}}$. So multiplying these gives the total number of ways to assign indices which is

%\begin{align}
%\left(\frac{N}{k}\right)^{m_1}\cdot \left(\frac{N^2}{k}\right)^{\frac{\eta-m_1}{2}}\cdot \left(\frac{N}{k}\right)^{\frac{\eta-m_1}{2}}=\frac{N^{\frac{3}{2}\eta-\frac{1}{2}m_1}}{k^\eta}
%\end{align}

%So combining all of these together we get that the contribution of this fixed $S_{ab}$ class is the desired result.
%\end{proof}

%\begin{lemma}\label{inequalities}
%For any $0\leq p<m$,
%\begin{align}
%\sum_{i=0}^m(-1)^i\binom{m}{i}i^p &\ = \ 0. \\
%\sum_{i=0}^m(-1)^{m-i}\binom{m}{i}i^m &\ = \ m!.
%\end{align}
%\end{lemma}

%Observe that if $m_1>m$, then by Lemma \ref{Sclasscontribution} the contribution of an $S_{ab}$ class with $m_1$ $a$ block is
%\begin{align}
%&\frac{1}{k}\sum_{\alpha=2n}^{8n}c_\alpha\left(\frac{k}{N^{3/2}}\right)^\alpha\left(\frac{1}{N^m}\sum_{i=0}^m\binom{m}{i}\left(-\frac{N^{3/2}}{k}\right)^{m-i}p(\alpha+i)\left(\frac{N^{\frac{3}{2}(\alpha+i)-\frac{1}{2}m_1}}{k^{\alpha+i}}\right)\right) \\
%&\ = \ \frac{C_{k,m}}{N^{\frac{1}{2}(m_1-m)}}\sum_{\alpha=2n}^{8n}c_\alpha \sum_{i=0}^m\binom{m}{i}(-1)^{m-i}p(\alpha+i) \\
%&\ = \ \frac{C_{k,m}}{N^{\frac{1}{2}(m_1-m)}}\sum_{\alpha=2n}^{8n}c_\alpha \sum_{i=0}^m\binom{m}{i}(-1)^{m-i}\left(\frac{2(\alpha+i)^{m_1}}{m_1}+O\left((\alpha+i)^{m_1-1}\right)\right) \\
%&\ \ell \ \frac{C_{k,m,m_1}}{N^{\frac{1}{2}(m_1-m)}}\sum_{\alpha=2n}^{8n}c_\alpha\alpha^{m_1}
%\end{align}
%Since $f_1^{2n}(x) \ = \ \left(\frac{x(2-x)(x+1)(3-x)}{4}\right)^{2n}$, then $|c_\alpha|\ell C_0^{2n}$ for some $C_0>0$. Moreover, $\alpha\ell\log\log(N)$, then for some $\epsilon>0$ 
%\begin{align}
%\sum_{\alpha=2n}^{8n}c_\alpha \alpha^{m_1}\ell n^{m_1+1}C_0^{2n}\ell (\log\log(N))^{m_1+1}\log(N)\ell N^{1/2(m_1-m)-\epsilon}
%\end{align}

%Hence, as $N\rightarrow 0$, the contribution of $S_{ab}$ class with $m_1>m$ $a$ block and $m_2$ $aa$ block is negligible. Moreover, if $m_1<m$, then the contribution of an $S_{ab}$ class with $m_1$ $a$ block is
%\begin{align}
%&\frac{1}{k}\sum_{\alpha=2n}^{8n}c_\alpha\left(\frac{k}{N^{3/2}}\right)^\alpha\left(\frac{1}{N^m}\sum_{i=0}^m\binom{m}{i}\left(-\frac{N^{3/2}}{k}\right)^{m-i}p(\alpha+i)\left(\frac{N^{\frac{3}{2}(\alpha+i)-\frac{1}{2}m_1}}{k^{\alpha+i}}\right)\right) \\
%&\ = \ \frac{C_{k,m}}{N^{\frac{1}{2}(m_1-m)}}\sum_{\alpha=2n}^{8n}c_\alpha\sum_{i=0}^m\binom{m}{i}(-1)^ip(\alpha+i) \\
%&\ = \ \frac{C_{k,m}}{N^{\frac{1}{2}(m_1-m)}}\sum_{\alpha=2n}^{8n}c_\alpha \sum_{q=0}^{m_1}c_q \alpha^{m_1-q} \sum_{i=0}^m(-1)^i\binom{m}{i} i^q \ = \ 0.
%\end{align}
%Thus, we must have $m_1=m$.

%\begin{theorem}\label{GOE-checkerboard Moments}
%The expected $m$-th moment associated to the empirical blip spectral measure is
%\begin{align}
%\mathbb{E}\left[\mu_{\{A_N,B_N\},1}^{(m)}\right] \ = \ 2\left(\frac{1}{k}\right)^{m+1}\mathbb{E}_k\textup{Tr}C^m
%\end{align}
%\end{theorem}
%\begin{proof}
%By the discussion above, we know that $m_1=m$.

%\begin{align}
%\mathbb{E}\left[\mu_{\{A_N,B_N\},1}^{(m)}\right] &\ = \ \frac{1}{k}\sum_{\alpha=2n}^{8n}c_\alpha\left(\frac{k}{N^{3/2}}\right)^\alpha\frac{1}{N^{m+\frac{1}{2}m}} \sum_{i=0}^m \binom{m}{i}\left(-\frac{N^{3/2}}{k}\right)^{m-i}\frac{2(\alpha+i)^m}{m!}\left(\frac{N^{3/2}}{k}\right)^{\alpha+i}\mathbb{E}_k\textup{Tr}C^{m} \\
%&\ = \ \frac{2}{m!}\left(\frac{1}{k}\right)^{m+1}\mathbb{E}_k\textup{Tr}C^m \sum_{\alpha=2n}^{8n}c_\alpha\sum_{i=0}^m \binom{m}{i}(-1)^{m-i}(\alpha+i)^m \\
%&\ = \ \frac{2}{m!}\left(\frac{1}{k}\right)^{m+1}\mathbb{E}_k\textup{Tr}C^m \sum_{\alpha=2n}^{8n}c_\alpha\sum_{i=0}^m \binom{m}{i}(-1)^{m-i}\sum_{p=0}^m \binom{m}{p}\alpha^{p}i^{m-p} \\
%&\ = \ \frac{2}{m!}\left(\frac{1}{k}\right)^{m+1}\mathbb{E}_k\textup{Tr}C\sum_{\alpha=2n}^{8n}\sum_{p=0}^m\binom{m}{p}c_\alpha\alpha^p\sum_{i=0}^m\binom{m}{i}(-1)^{m-i}i^{m-p}
%\end{align}
%Since the inner sum is 0 if $p>0$ and $m!$ if $p=0$ by Lemma \ref{inequalities} and $f_1^{(2n)}(1)=\sum_{\alpha=2n}^{8n}c_\alpha=1$, then
%\begin{align}
%\mathbb{E}\left[\mu_{\{A_N,B_N\},1}^{(m)}\right] &\ = \ \frac{2}{m!}\left(\frac{1}{k}\right)^{m+1}\mathbb{E}_k\textup{Tr}C^m\sum_{\alpha=2n}^{8n}c_\alpha m! \\
%&\ = \ 2\left(\frac{1}{k}\right)^{m+1}\mathbb{E}_k\textup{Tr}C^m.
%\end{align}
%\end{proof}



%\section{Largest Blip for k-checkerboard and j-checkerboard}

%\begin{definition}[Largest Blip Spectral Measure]
%The largest blip spectral measure for k-checkerboard matrix $A_N$ and j-checkerboard matrix $B_N$ is defined as 
%\begin{align}
%\mu_{A,B,N}=\sum_{\lambda\text{ eigenvalue of }\{A_N,B_N\}}f_{n(N)}\left(\frac{jk\lambda}{2N^2}\right)\delta\left(x-\left(\frac{\lambda-\frac{2}{jk}N^2}{N}\right)\right)
%\end{align}
%where $f_{n(N)}$ is the weight function $x^{2n}(2-x)^{2n}$ and $n(N)=O(\log \log N)$. Also note that we can write the expansion of the weight function as $\sum_{\alpha=2n}^{4n}c_\alpha x^{\alpha}$.
%\end{definition}

%\begin{theorem}
%The $m$th moment of the largest blip spectral measure is 
%\begin{align}
%\E{\mu_{A,B,N}^{(m)}}=\sum_{\substack{m_{1a}+m_{1b}+m_{2a}+m_{2b}=m; \\ m_{1a},m_{1b}\textup{ even}}}\frac{m!2^{m_1/2+2m_2}m_{1a}!!m_{1b}!!}{m_{1a}!m_{1b}!m_{2a}!m_{2b}!}\left(\frac{1}{k}\sqrt{1-\frac{1}{k}}\right)^{m_{1a}+2m_{2a}}\left(\frac{1}{j}\sqrt{1-\frac{1}{j}}\right)^{m_{1b}+2m_{2b}}
%\end{align}
%\end{theorem}

%\begin{proof}
%\textcolor{red}{To-do: define $l$/depends on weight function.} First, by the definition of the largest blip spectral measure and the expansion of the weight function we get
%\begin{align}
%\E{\mu_{A,B,N}^{(m)}}& \ = \ \mathbb{E}\left[\sum_\lambda \sum_{\alpha=2n}^{4nl}c_\alpha\left(\frac{\lambda}{\frac{2}{jk}N^2}\right)^\alpha \left(\frac{\lambda-\frac{2}{jk}N^2}{N}\right)^m\right] \\
%& \ = \ \mathbb{E}\left[\sum_\lambda \sum_{\alpha=2n}^{4nl}c_\alpha\left(\frac{1}{\frac{2}{jk}N^2}\right)^\alpha\frac{1}{N^m} \left(\sum_{i=0}^m\binom{m}{i}\lambda^i\left(-\frac{2}{jk}N^2\right)^{m-i}\right)\right] \\
%& \ = \ \sum_{\alpha=2n}^{4nl}c_\alpha \left(\frac{1}{\frac{2}{jk}N^2}\right)^\alpha \frac{1}{N^m}\sum_{i=0}^m\binom{m}{i}\left(-\frac{2}{jk}N^2\right)^{m-i}\E{\text{Tr}\{A_N,B_N\}^{\alpha+i}}
%\end{align}

%Note that the first equality follows from binomial expansion and the second equality from the eigenvalue trace lemma.

%We can now substitute our result from \ref{j,k-checkerboard trace} for the trace expansion and we find that we only need to consider terms that satisfy $|S|=m_{1a}+m_{1b}+m_{2a}+m_{2b}\leq m$. This is because if $m_{1a}+m_{1b}+m_{2a}+m_{2b}>m$, after plugging in our trace expansion, the power of $N$ would be negative which would mean that the terms go to $0$ as $N\to \infty$. We therefore get 
%\begin{align}
%&\sum_{\alpha=2nl}^{4n}c_\alpha \left(\frac{1}{\frac{2}{jk}N^2}\right)^\alpha \frac{1}{N^m}\sum_{i=0}^m\binom{m}{i}\left(-\frac{2}{jk}N^2\right)^{m-i}\sum_{\substack{m_{1a}+m_{1b}+m_{2a}+m_{2b}=m; \\ m_{1a},m_{1b}\textup{ even}}}\frac{2^{(\alpha+i)-2m_2+m_1/2}(m_{1a})!!(m_{1b})!!}{m_{1a}!m_{1b}!m_{2a}!m_{2b}!} \\
%&\left(\frac{1}{k}\right)^{(\alpha+i)-m_{1a}-2m_{2a}}\left(\frac{1}{j}\right)^{(\alpha+i)-m_{1b}-2m_{2b}}\left(1-\frac{1}{k}\right)^{m_{1a}/2+m_{2a}}\left(1-\frac{1}{j}\right)^{m_{1b}/2+m_{2b}}(\alpha+i)^{m_1+m_2}N^{2(\alpha+i)-|S|}.
%\end{align}

%Now, when $m_1+m_2=|S|<m$, the power of $i$ is less than $m$ so by \ref{inequalities}, we can exclude these cases and only consider the case when $|S|=m$ and ignore all of the lower order terms in $(\alpha+i)^{|S|}$. Combining like terms, we see that all of the powers of $N$ and dependencies on $i$ cancel, reducing to
%\begin{equation*}
%\resizebox{1.0\hsize}{!}{$\sum_{\substack{m_{1a}+m_{1b}+m_{2a}+m_{2b}=m; \\ m_{1a},m_{1b}\textup{ even}}}\frac{m!2^{m+m_1/2-2m_2}m_{1a}!!m_{1b}!!}{m_{1a}!m_{1b}!m_{2a}!m_{2b}!}\left(\frac{1}{k}\right)^{m-m_{1a}-2m_{2a}} \left(\frac{1}{j}\right)^{m-m_{1b}-2m_{2b}} \left(1-\frac{1}{k}\right)^{m_{1a}/2+m_{2a}} \left(1-\frac{1}{j}\right)^{m_{1b}/2+m_{2b}}\sum_{\alpha=2n}^{4n}c_\alpha$.}
%\end{equation*}
%$Lastly, since $f_n(1)=1$, by definition we have that $\sum_{\alpha=2n}^{4n}c_\alpha=1$ which proves our result.
%\end{proof}
\appendix
\section{Moments of anticommutators}\label{AppendixMomentsAnti}
\subsection{Moments of $\ell$ anticommutators}


\begin{proposition}
For sequences $f_0,f_1,\dotsc,f_\ell$ defined such that $f_0(0)=1$, and $f_0(1)=f_1(1)=f_2(1)=\cdots=f_\ell(1)=1$, and with recurrence relations for $m>1$ given by
\begin{align}
&f_0(m)=f_1(m)+ \ell! \sum_{j=1}^{m-1}f_1(j)f_0(m-j),\nonumber\\
&f_k(m)=f_{k+1}(m)+\sum_{\substack{1\leq x_1,x_2\leq m\\ x_1+x_2\leq m}}(\ell-k)!(k-1)!f_{k+1}(x_1)f_{k+1}(x_2)f_{\ell-k-1}(m-x_1-x_2+1)
\end{align}
for any $0<k<\ell-1$, and by
\begin{align}
&f_{\ell-1}(m)=f_\ell(m)+\nonumber\\&\sum_{\substack{0\leq x_1,x_2<m-1\\ x_1+x_2<m-1}} (\ell-1)!(1+(\ell!-1)\cdot \mathbb{1}_{x_1>0})(1+(\ell!-1)\cdot \mathbb{1}_{x_2>0})f_0(x_1)f_0(x_2)f_1(m-x_1-x_2-1),\nonumber
\\
&f_\ell(m)= \ell! \cdot f_0(m-1),
\end{align}
the $2m^{th}$ moment of the $\ell$-anticommutator is 
\begin{align}
M_{2m}=\ell! \cdot f_0(m)
\end{align}
\end{proposition}

\begin{proof}
The proof follows similarly as with the $2$-anticommutator. Let $f_0(m)$ be the number of non-crossing matchings with respect to all $(\ell, 2\ell m)$-configurations starting with $a^{(1)}_{i_1i_2}a^{(2)}_{i_2i_3}\cdots a^{(\ell)}_{i_\ell i_{\ell+1}}$ and let $f_k(m)$ be the number of such matchings where the first $k$ terms are paired with the last $k$ terms in a nested fashion (i.e. for $k=3$, we would have configurations of the form $a^{(1)}_{i_1i_2}a^{(2)}_{i_2i_3}a^{(3)}_{i_3i_4}\cdots a^{(3)}_{i_{2\ell m-2}i_{2\ell m-1}}a^{(2)}_{i_{2\ell m-1}i_{2\ell m}}a^{(1)}_{i_{2\ell m}i_{1}}$ and matchings such that $i_2=i_{2\ell m}$, $i_3=i_{2\ell m-1}$, and $i_4=i_{2\ell m-2}$).

We first find the recurrence relation for $f_0(m)$. To ensure non-crossing matchings, $a^{(1)}_{i_1 i_2}$ must be paired with some $a^{(1)}_{i_{2\ell j} i_{2\ell j+1}}$ with $j\leq m$ (in the case when $j=m$, we identify $2\ell m+1$ as $1$). When $j=m$, the number of non-crossing matchings is simply $f_1(m)$ by definition. When $j<m$, the number of non-crossing matchings within $a^{(1)}_{i_1 i_2}\cdots a^{(1)}_{i_{2\ell j} i_{2\ell j+1}}$ is $f_1(j)$, while the number of non-crossing matchings within the rest of the cyclic product for which we have no restrictions is $\ell!f_0(m-j)$, with the $\ell!$ accounting for different possible arrangements of the first $\ell$ terms. Multiplying these together and summing over all possible $j$'s gives
\begin{align}
f_0(m)=f_1(m)+\ell! \sum_{j=1}^{m-1}f_1(j)f_0(m-j).
\end{align}

Now turning to $f_k(m)$, we look separately at when $0<k<\ell-1$, when $k=\ell-1$, and when $k=\ell$.

When $0<k<\ell-1$, we have either that $a^{(k+1)}_{i_{k+1}i_{k+2}}$ is paired with $a^{(k+1)}_{i_{2\ell m-k}i_{2\ell m-k+1}}$, or that $a^{(k+1)}_{i_{k+1}i_{k+2}}$ is paired with $a^{(k+1)}_{i_{2\ell x_1 -k}i_{2\ell x_1-k+1}}$ and $a^{(n)}_{i_{2\ell m-k}i_{2\ell m-k+1}}$ is paired with $a^{(n)}_{i_{2\ell (m-x_2) + k+1}i_{2\ell(m- x_2) + k+2}}$, where $n\in \{k+1, \dotsc, \ell\}$, with both $x_1, x_2 \geq 1$ and $2\ell x_1-k+1 < 2\ell (m-x_2) + k+1$, or $x_1+x_2 \leq m$. The first case is precisely the definition of $f_{k+1}(m)$. In the second case, the number of non-crossing matchings of terms between $a^{(k+1)}_{i_{k+1}i_{k+2}}$ and $a^{(k+1)}_{i_{2\ell x_1 -k}i_{2\ell x_1-k+1}}$ is $f_{k+1}(x_1)$, the number of non-crossing matchings of terms between $a^{(n)}_{i_{2\ell (m-x_2) + k+1}i_{2\ell(m- x_2) + k+2}}$ and $a^{(n)}_{i_{2\ell m-k}i_{2\ell m-k+1}}$ is $(\ell-k)!f_{k+1}(x_2)$, with the $(\ell-k)!$ accounting for different possible arrangements of the last $\ell$ terms, and the number of non-crossing matchings of terms between $a^{(k+1)}_{i_{2\ell x_1 -k}i_{2\ell x_1-k+1}}$ and $a^{(n)}_{i_{2\ell (m-x_2) + k+1}i_{2\ell(m- x_2) + k+2}}$ is $(k-1)!f_{\ell-k+1}(m-x_1-x_2+1)$. The last statement follows from viewing the $2\ell(m-x_1-x_2)+2k+2$ terms between $a^{(k+1)}_{i_{2\ell x_1 -k}i_{2\ell x_1-k+1}}$ and $a^{(n)}_{i_{2\ell (m-x_2) + k+1}i_{2\ell(m- x_2) + k+2}}$ as $2\ell (m-x_1-x_2+1)$ terms where the $(l-k-1)$ terms on both end are matched to each other and hence fixed, with the $(k-1)!$ accounting for different permutations of the remaining $k+1$ of the first  $\ell$ terms. We sum over all possible $x_1$ and $x_2$'s to get the desired result:
\begin{align}
f_k(m)=f_{k+1}(m)+
 \sum_{\substack{1\leq x_1,x_2\leq m\\ x_1+x_2\leq m}}(\ell-k)!(k-1)!f_{k+1}(x_1)f_{k+1}(x_2)f_{\ell-k-1}(m-x_1-x_2+1).
\end{align}

When $k=\ell-1$, we either have that $a^{(\ell)}_{i_{\ell}i_{\ell+1}}$ is paired with $a^{(\ell)}_{i_{2\ell m - \ell+1}i_{2\ell m - \ell+2}}$, which is simply $f_\ell(m)$ by definition, or that $a^{(\ell)}_{i_{\ell}i_{\ell+1}}$ is paired with $a^{\ell}_{i_{2\ell x_1 +\ell+1}i_{2\ell x_1+\ell+2}}$ and $a^{(n)}_{i_{2\ell m-\ell+1}i_{2\ell m-\ell+2}}$ is paired with $\\a^{(n)}_{i_{2\ell (m-x_2) - \ell}i_{2\ell(m- x_2) - \ell+1}}$, with both $x_1, x_2 \geq 0$. The number of non-crossing matchings of terms between $a^{(\ell)}_{i_{\ell}i_{\ell+1}}$ and $a^{\ell}_{i_{2\ell x_1 +\ell+1}i_{2\ell x_1+\ell+2}}$ is $(1+(\ell!-1)\mathbb{1}_{x_1>0})f_0(x_1)$, the number of non-crossing matchings of terms between $a^{(n)}_{i_{2\ell (m-x_2) - \ell}i_{2\ell(m- x_2) - \ell+1}}$ and $a^{(n)}_{i_{2\ell m-\ell+1}i_{2\ell m-\ell+2}}$ is  $(1+(\ell!-1)\mathbb{1}_{x_2>0})f_0(x_2)$, with a factor of $\ell!$ when either $x_1,x_2 > 0$ due to different possible arrangements of the first $\ell$ terms starting at $a^{(n)}_{i_{\ell+1}i_{\ell+2}}$, where $n \in \{1, \dotsc, \ell-1\}$, and the number of non-crossing matchings of terms between $a^{\ell}_{i_{2\ell x_1 +\ell+1}i_{2\ell x_1+\ell+2}}$ and $a^{(n)}_{i_{2\ell (m-x_2) - \ell}i_{2\ell(m- x_2) - \ell+1}}$ is $(\ell-1)!f_1(m-x_1-x_2-1)$. For the last statement, we view the $2\ell(m-x_1-x_2)-2\ell-2$ terms between $a^{\ell}_{i_{2\ell x_1 +\ell+1}i_{2\ell x_1+\ell+2}}$ and $a^{(n)}_{i_{2\ell (m-x_2) - \ell}i_{2\ell(m- x_2) - \ell+1}}$ as $2\ell(m-x_1-x_2-1)$ terms with the first and last term matched with each other, with $(l-1)!$ accounting for different arrangements of the remaining $l-1$ terms. We once again sum over all possible $x_1$ and $x_2$'s to reach the desired result:
\begin{align}
&f_{\ell-1}(m)=f_\ell(m)+\nonumber
\\
&\sum_{\substack{0\leq x_1,x_2<m-1\\ x_1+x_2<m-1}} (\ell-1)!(1+(\ell!-1)\cdot \mathbb{1}_{x_1>0})(1+(\ell!-1)\cdot \mathbb{1}_{x_2>0})f_0(x_1)f_0(x_2)f_1(m-x_1-x_2-1).
\end{align}

Lastly, when $k=\ell$, with no matching conditions on the terms between the first and last $\ell$ terms, for each possible permutation of the next $\ell$ terms, we have $f_0(m-1)$ non-crossing matchings, amounting to $\ell!\cdot f_0(m-1)$ total non-crossing matchings.

We have now fully defined our recurrences for $f_0(k)$, which represents the number of non-crossing matchings with respect to $(\ell, 2\ell m)$-configurations where the first $\ell$ terms are fixed to be $a^{(1)}_{i_1i_2}a^{(2)}_{i_2i_3}\cdots a^{(\ell)}_{i_\ell i_{\ell+1}}$. Applying any permutation to these $\ell$ terms preserves the non-crossing property of these matchings. Hence, we multiply $f_0(m)$ by $\ell!$ to obtain all possible non-crossing partitions with respect to $(\ell, 2\ell m)$-configurations, and we arrive at the even moments being $M_{2k}=\ell! \cdot f_0(k)$.

% The proof follows in the same way as with the $2$-anticommutator. We get that if two terms are matched together, their indices must sum to $1$ mod $2\ell$. Moreover, $f_k(m)$ counts the number of sequences of length $2m\ell$ such that the first $\ell$ terms are $A_1A_2\cdots A_\ell$ and the first $k$ terms are matched with the last $k$ terms, which must be in reverse order. Then, if $A_{k+1}$ (the $k+1^{\textup{th}}$ term from the front) is matched to the $k+1^{\textup{th}}$ term from the end, the number of such configurations is counted by $f_{k+1}(m)$, which is the first term in our expression for $f_k(m)$. Otherwise, $A_{k+1}$ must be matched to a term in the middle, of index $2\ell x_1-k$, with the number of internal matchings between them given by $f_k(x_1)$. Similarly, the $k+1^{\textup{th}}$ term from the end must be matched to a term with index $2\ell (m-x_2)+k$, leading to $f_k(x_2)$ possibilities. We add an extra factor of $(\ell-k)!$ since the final $\ell-k$ terms can be ordered arbitrarily (with the final $k$ terms fixed due to their being matched with the first $k$ terms). There are $2k+2 \mod{2m}$ terms between $2\ell x_1-k$ and $2\ell (m-x_2)+k+2$, which is equivalent to fixing $\ell-k + 1$ terms on the outside, yielding a factor of $f_{\ell-k+1}(m-x_1-x_2)$. The $k-1$ terms on the inside can be arranged in $(k-1)!$ ways which completes the summand in the second component of $f_k(m)$. Note that the $f_{\ell-1}(m)$ case is slightly different but works in the same way as described in the $2$-anticommutator case. Furthermore, the $f_0$ case works in exactly the same way, thus proving the formula for the recurrence. Observe that $f_0(k)$ represents the number of non-crossing partitions with $2k\ell$ terms where the first $\ell$ terms are fixed to be $A_1A_2\cdots A_\ell$. Applying any permutation to these $l$ terms would preserve the non-crossing property of these partitions and count \textit{all} possible non-crossing partitions with $2k\ell$ terms. We know from earlier lemmas that the even moments are given by precisely this number of non-crossing partitions and thus we arrive at $M_{2k}=\ell! \cdot f_0(k)$.
\end{proof}

\subsection{Bulk Moments of $\{\textup{GOE, }k\textup{-checkerboard}\}$}

\begin{proposition}\label{bulkGOEcheckerboard} The $2m$\textsuperscript{th} bulk moment of $\{\textup{GOE, }k\textup{-checkerboard}\}$ is $M_{2m}=2(1-\frac{1}{k})^mf(m)$, where $f(0)=f(1)=1$, $g(1)=1$, and
\begin{align}
f(m) \ = \ 2\sum_{j=1}^{m-1}g(j)f(m-j) + g(m),
\end{align}
and 
\begin{align}
g(m) \ = \ 2f(m-1) + \sum_{\substack{0\leq x_1,x_2\leq m-2\\ x_1+x_2\leq m-2}}(1+\mathbb{1}_{x_1>0})(1+\mathbb{1}_{x_2>0})f(x_1)f(x_2)g(m-1-x_1-x_2)
\end{align}
\end{proposition}

\begin{proof}
By a result in \cite{Tao1}, the limiting distribution of the bulk of $\{\textup{GOE, }(k,1)\textup{-checkerboard}\}$ is given by the limiting distribution of $\{\textup{GOE, }(k,0)\textup{-checkerboard}\}$. Because in a contributing cyclic product, every term $c_{i_\ell i_{\ell+1}}$ from the $(k, 0)$-checkerboard must be non-weight with the modular restriction $i_\ell\not\equiv i_{\ell+1}\textup{ (mod $k$)}$, then the $2m$\textsuperscript{th} bulk moment of $\{\textup{GOE, }k\textup{-checkerboard}\}$ is essentially the $2m$\textsuperscript{th} moment $\{\textup{GOE,}\\ \textup{GOE}\}$, except that we have to account for all the modular restrictions. Since the $2m$ non-weight terms are paired together, the probability that all the terms from the $(k,0)$-checkerboard are non-weights is $\left(1-\frac{1}{k}\right)^m$. This completes the proof.

%Note that since all of the non-weight elements of the $(m,0)$-checkerboard are independent mean $0$, variance $1$ variables and thus they must be paired similarly to how they are in lemma 2.1 of \cite{split}. So the if we choose the matchings for such an anticommutator the number of ways to choose these is the same as in \ref{GOE-GOE moment recurrence}, however we cannot have any of the weights appearing. If we choose the indices arbitrarily we see that there are $2k$ $b_{i_{j},i_{j+1}}$ terms but they are all paired up so there are $k$ distinct pairs of indices (note that if there were any fewer than $k$ there would be a loss of degrees of freedom), so the probability that none of these $k$ pairs of indices are weights is $\left(1-\frac{1}{m}\right)^k$ since there is a $\frac{1}{m}$ chance that the two arbitrarily chosen indices are equivalent mod $m$.
\end{proof}

\subsection{Bulk Moments of $\{k\textup{-checkerboard, }j\textup{-checkerboard}\}$}
\begin{corollary} The $2m$\textsuperscript{th} bulk moment of $\{k\textup{-checkerboard, }j\textup{-checkerboard}\}$ is $M_{2m}=2\left(1-\frac{1}{k}\right)^m\\ \left(1-\frac{1}{j}\right)^m f(m)$, where
\begin{align}
f(m) \ = \ 2\sum_{j=1}^{m-1}g(j)f(m-j) + g(m)
\end{align}
and 
\begin{align}
g(m) \ = \ 2f(m-1) + \sum_{\substack{0\leq x_1,x_2\leq m-2\\ x_1+x_2\leq m-2}}(1+\mathbb{1}_{x_1>0})(1+\mathbb{1}_{x_2>0})f(x_1)f(x_2)g(m-1-x_1-x_2).
\end{align}
\end{corollary}

\begin{proof}
The proof is essentially the same as the proof of Proposition \ref{bulkGOEcheckerboard}.
\end{proof}

\section{Proof of Multiple Regimes}\label{multipleregimes}
%We use Weyl's inequality to prove that the blip actually exists so then we can use the weight functions to extract the moments of these blips and solve for their respective distributions. First we can extract the mean matrix which is just the mean of every term then we can write every k-checkerboard matrix as the sum of the mean matrix and the perturbation matrix. Then when we expand the product of the checkerboard and GOE matrix we will get eigenvalues on different orders and we can prove that these eigenvalues do not move and are on the correct order. 

In this section, we prove the existence of multiple regimes of eigenvalues for $\{\textup{GOE, }k\textup{-checkerboard}\}$ and $\{k\textup{-checkerbord, }j\textup{-checkerboard}\}$. Our method involves decomposing each checkerboard matrix into the sum of its mean matrix and perturbation matrix and applying Weyl's inequality to bound the eigenvalue of the matrix ensemble in terms of the eigenvalue of its components. For the sake of simplicity, throughout this section we assume that the weight $w=1$ and that $k|N$ for $\{\textup{GOE, }k\textup{-checkerboard}\}$ and $jk|N$ for $k\{\textup{-checkerboard, }j\textup{-checkerboard}\}$.

\begin{definition}[Mean Matrix]
The mean matrix $\overline{A}_N$ of the $k$-checkerboard matrix $A_N=(a_{ij})$ is given by
\begin{align}
\overline{a}_{ij}=
\begin{cases}
0,& \text{if }i\not\equiv j\mod{k}\\
1,& \text{if }i\equiv j\mod{k}.
\end{cases}
\end{align}
We note that the rank of $\overline{A}_N$ is $k$.
\end{definition}

\begin{definition}[Perturbation Matrix]
The perturbation matrix $\Tilde{A}_N$ of the k-checkerboard matrix $A_N=\Tilde{A}_N=(\Tilde{a}_{ij})$ is given by
\begin{align}
\Tilde{a}_{ij}=
\begin{cases}
a_{ij},& \text{if }i\not\equiv j\mod{k}\\
0,& \text{if }i\equiv j\mod{k}.
\end{cases}
\end{align}
%where $a_{ij}=a_{ji}$ and all of the distinct $a_{ij}$ terms are sampled from a distribution with mean $0$ and variance $1$.
\end{definition}

%So clearly by definition of the $k$-checkerboard matrix we see that for the $(k,1)$-checkerboard we can simply write it as $A_N=\overline{A}_N+\Tilde{A}_N$ where $\overline{A}_N$ is a fixed matrix of rank $k$ and $\Tilde{A}_N$ is a random matrix. Also note that $\Tilde{A}_N$ has similar structure to a GOE but with some diagonals set to $0$ so by the commonly used moment method we see that with probability $1-o(1)$ all of the eigenvalues of $\Tilde{A}_N$ are on the order of $\sqrt{N}$. Also note that since $\overline{A}_N$ is fixed we see that it has $k$ eigenvalues on the order of $\frac{N}{k}$.

Thus, we can write the $k$-checkerboard matrix simply as $A_N=\overline{A}_N+\Tilde{A}_N$. As shown in \cite{split},  all the eigenvalues of $\Tilde{A}_N$ are $O(N^{1/2})$. Moreover, $\overline{A}_N$ has $k$ eigenvalues at $\frac{N}{k}$ and $N-k$ eigenvalues at $0$.

\begin{lemma} 
Let $\Tilde{A}_N$ be the perturbation matrix as defined above and $B_N$ an $N\times N$ GOE matrix, then with probability $1-o(1)$, all the eigenvalues of $\{\Tilde{A}_N, B_N\}$ are $O(N)$.
\end{lemma}

\begin{proof}
%The maximum eigenvalue is equal to the operator norm so since $||\Tilde{A}_N||=O(\sqrt{N})$ as discussed earlier and $||B_N||=O(\sqrt{N})$ then $||\Tilde{A}_NB_N||\leq ||\Tilde{A}_N||||B_N||=O(N)$ and similarly $||B_N\Tilde{A}_N||=O(N)$ as well. So by Weyl's inequality we get that $\lambda_N(\Tilde{A}_NB_N+B_N\Tilde{A}_N)\leq \lambda_N(\Tilde{A}_NB_N)+\lambda_N(B_N\Tilde{A}_N)=O(N)$. The argument for the smallest negative eigenvalues follows from considering $-\Tilde{A}_N$ and $-B_N$ and the operator norms of these matrices and gives the result that the smallest of $\Tilde{A}_NB_N+B_N\Tilde{A}_N$ is also $O(N)$.

We know that the maximum eigenvalue of a matrix is equal to the operator norm of the matrix. Since $||\Tilde{A}_N||=O(N^{1/2})$ and $||B_N||=O(N^{1/2})$, then by submultiplicativity of the matrix norm, $||\Tilde{A}_NB_N||\leq ||\Tilde{A}_N||||B_N||=O(N)$. Similarly, $||B_N\Tilde{A_N}||=O(N)$. By Weyl's inequality, $\lambda_N(\{\Tilde{A}_N, B_N\})\leq \lambda_N(\Tilde{A}_NB_N)\\+\lambda_N(B_N\Tilde{A}_N)=O(N)$. The argument for the smallest negative eigenvalues follows from considering $-\Tilde{A}_N$ and $-B_N$.
\end{proof}

\begin{lemma}
Let $\overline{A}_N$ be the mean matrix as defined above and $B_N$ the $N\times N$ GOE matrix, then the largest eigenvalue of $\{\overline{A}_N, B_N\}$ is bounded above by $\frac{4N^{3/2}}{k}$, the smallest eigenvalue is bounded below by $-\frac{4N^{3/2}}{k}$, and there are at least $N-2k$ eigenvalues at 0.
\end{lemma}

\begin{proof}
First, observe that $\textup{rank}(\overline{A}_N B_N)\leq \textup{min}(\textup{rank}(\overline{A}_N), \textup{rank}(B_N))=k$. Similarly, $\textup{rank}(B_N\overline{A}_N)\leq k$. By the subadditivity of rank, $\textup{rank}(\{\overline{A}_N, B_N\})\leq 2k$. Thus, at least $N-2k$ eigenvalues are $0$. For the highest eigenvalues, we see that $||\{\overline{A}_N, B_N\}||\leq 2||\overline{A}_N||||B_N||=2\cdot \frac{N}{k}\cdot 2N^{1/2}=\frac{4N^{3/2}}{k}$. Similarly, the smallest eigenvalue is bounded below by $-\frac{4N^{3/2}}{k}$.
%We know that since $\overline{A}_N$ is a fixed matrix, by its definition we know that it has $k$ eigenvalues at $\frac{1}{k}N$ and all other eigenvalues at $0$. So first we see that the rank of $\overline{A}_NB_N$ is at most $k$ since $\text{rank}(\overline{A}_NB_N)\leq \min(\text{rank}(\overline{A}_N),\text{rank}(B_N))=k$. So also since rank is subadditive we see that $\text{rank}(\overline{A}_NB_N+B_N\overline{A}_N)\leq 2k$ so at least $N-2k$ of these eigenvalues must be $0$. Then we can consider the highest and lowest eigenvalues, we see that the highest eigenvalue of $\overline{A}_NB_N$ is $||\overline{A}_NB_N||$ so by submultiplicativity of the operator norm and the triangle inequality we get that $||\overline{A}_NB_N+B_N\overline{A}_N||\leq ||\overline{A}_NB_N||+||B_N\overline{A}_N||\leq ||\overline{A}_N||||B_N|| + ||B_N||||\overline{A}_N||\leq \frac{4}{k}N^{3/2}$ and similarly we can do this on the negative side by considering $-B_N$ rather than $B_N$ which proves that the smallest eigenvalue is at least $-\frac{4}{k}N^{3/2}$.
\end{proof}
%So this proves that there are $k$ eigenvalues potentially on the order of $N^{3/2}$ on the positive side and $k$ eigenvalues potentially on the order of $N^{3/2}$ on the negative side. We empirically see that it ends up that we are centered at $\pm \frac{1}{k}N^{3/2}$ for both of these blips. Now we can show that the width of these blips is $O(N)$ which shows that they are roughly concentrated and the weight functions we use can control them effectively.

Empirically, we observe that $\overline{A}_N$ has $k$ blip eigenvalues at $\frac{N^{3/2}}{k}+O(N)$ and $k$ blip eigenvalues at $-\frac{N^{3/2}}{k}+O(N)$. By assuming this, we are able to prove the existence of multiple regimes for $\{A_N, B_N\}$, as follows:

\begin{lemma}
Let $A_N$ be a $k$-checkerboard matrix and $B_N$ an $N\times N$ GOE matrix, then $\{A_N,B_N\}$ has a blip containing $k$ eigenvalues at $\frac{N^{3/2}}{k} + O(N)$, a blip containing $k$ eigenvalues at $-\frac{N^{3/2}}{k}+ O(N)$, and $N-2k$ eigenvalues of $O(N)$.
\end{lemma}

\begin{proof}
First note that we can write $\{A_N,B_N\}=\{\overline{A}_N,B_N\} + \{\Tilde{A}_N,B_N\}$. By Weyl's inequality, we see that 
\begin{align}
\lambda_{N-k+1}(\{A_N,B_N\})\ \geq \ \lambda_{N-k+1}(\{\overline{A}_N,B_N\}) + \lambda_1(\{\Tilde{A}_N,B_N\})\ = \ \frac{1}{k}N^{3/2}+O(N)
\end{align}
and 
\begin{align}
\lambda_N(\{A_N,B_N\})\ \leq \ \lambda_N(\{\overline{A}_N,B_N\}) + \lambda_N(\{\Tilde{A}_N,B_N\})\ = \ \frac{1}{k}N^{3/2}+O(N).
\end{align}

So this proves the existence of $k$ blip eigenvalues at $\frac{N^{3/2}}{k}$. Similarly, we can use Weyl's inequality to show the existence of blip eigenvalues at $-\frac{N^{3/2}}{k}$. For the bulk, we see that 
\begin{align}
\lambda_{N-k}(\{A_N,B_N\})\ \leq \ \lambda_{N-k}(\{\overline{A}_N,B_N\}) + \lambda_N(\{\Tilde{A}_N,B_N\})\ = \ O(N),
\end{align}
and 
\begin{align}
\lambda_{k+1}(\{A_N,B_N\})\ \geq \ \lambda_{k+1}(\{\overline{A}_N,B_N\}) + \lambda_1(\{\Tilde{A}_N,B_N\})\ = \ O(N).
\end{align}
This completes the proof for the existence of three different regimes. 
\end{proof}

Now we consider $\{A_N, B_N\}$, where $A_N$ is a $k$-checkerboard matrix and $B_N$ is a $j$-checkerboard matrix. We assume $\textup{gcd}(k,j)=1$, $N\mid kj$. Then we can write $\{A_N,B_N\}=\{\Tilde{A}_N,\Tilde{B}_N\}+\{\overline{A}_N,\Tilde{B}_N\}+\{\Tilde{A}_N,\overline{B}_N\}+\{\overline{A}_N, \overline{B}_N\}$. Similarly, we see that all eigenvalues of $\{\Tilde{A}_N,\Tilde{B}_N\}$ are of $O(N)$. Empirically, $\{\overline{A}_N,\Tilde{B}_N\}$ has $k$ eigenvalues at $\frac{1}{k}\sqrt{1-\frac{1}{j}}N^{3/2}$, $k$ eigenvalues at $-\frac{1}{k}\sqrt{1-\frac{1}{j}}N^{3/2}$, and the remaining $N-2k$ eigenvalues at $0$. Heuristically, this can be seen from the fact that $\overline{A}_N$ has $k$ eigenvalues at $\frac{1}{k}N$ and the eigenvalues of $\Tilde{B}_N$ are bounded above and below by $\pm 2\sqrt{1-\frac{1}{j}}N^{1/2}$. Similarly, empirically $\{\Tilde{A}_N, \overline{B}_N\}$ has $j$ eigenvalues at $\frac{1}{j}\sqrt{1-\frac{1}{k}}N^{3/2}$, $j$ eigenvalues at $-\frac{1}{j}\sqrt{1-\frac{1}{k}}N^{3/2}$, and the remaining $N-2j$ eigenvalues are at 0. For $\{\overline{A}_N, \Tilde{B}_N\}+\{\Tilde{A}_N, \overline{B}_N\}$, we observe that the largest eigenvalue is of $O(N^{3/2})$ but larger than $\frac{1}{k}\sqrt{1-\frac{1}{j}}N^{3/2}$ and $\frac{1}{j}\sqrt{1-\frac{1}{j}}N^{3/2}$, the smallest eigenvalue is of $O(N^{3/2})$ but smaller than $-\frac{1}{k}\sqrt{1-\frac{1}{j}}N^{3/2}$ and $-\frac{1}{j}\sqrt{1-\frac{1}{k}}N^{3/2}$. Furthermore, There are $k-1$ eigenvalues at each of $\pm \frac{1}{k}\sqrt{1-\frac{1}{j}}N^{3/2}$, and $j-1$ eigenvalues at each of $\pm \frac{1}{j}\sqrt{1-\frac{1}{k}}N^{3/2}$, and the remaining $N-2k-2j+3$ eigenvalues of $O(N)$.

%SImilarly the eigenvalues of $\{\Tilde{A}_N,\overline{B}_N\}$ are $\frac{1}{j}\sqrt{1-\frac{1}{k}}N^{3/2}$ for the $j$ highest eigenvalues, $-\frac{1}{j}\sqrt{1-\frac{1}{k}}N^{3/2}$ for the $j$ lowest eigenvalues, and $0$ for the other $N-2j$ eigenvalues. We also see that for $\{\overline{A}_N,\Tilde{B}_N\}+\{\Tilde{A}_N,\overline{B}_N\}$ we get both of the blips where we have one largest blip on the order of $N^{3/2}$ but greater than $\frac{1}{k}N^{3/2}$ and $\frac{1}{j}N^{3/2}$, $k-1$ eigenvalues centered at each of $\pm \frac{1}{k}\sqrt{1-\frac{1}{j}}N^{3/2}$, $j-1$ eigenvalues centered at each of $\pm \frac{1}{j}\sqrt{1-\frac{1}{k}}N^{3/2}$, and all other eigenvalues at $O(N)$.

%\begin{lemma}
%If $\Tilde{A}_N$ and $\Tilde{B}_N$ are perturbation matrices as defined above, then all of the eigenvalues of $\Tilde{A}_N\Tilde{B}_N+\Tilde{B}_N\Tilde{A}_N$ are $O(N)$.
%\end{lemma}

%\begin{proof}
%The maximum eigenvalue is equal to the operator norm so since $||\Tilde{A}_N||=O(\sqrt{N})$ as discussed earlier and $||\Tilde{B}_N||=O(\sqrt{N})$ then $||\Tilde{A}_N\Tilde{B}_N||\leq ||\Tilde{A}_N||||\Tilde{B}_N||=O(N)$ and similarly $||\Tilde{B}_N\Tilde{A}_N||=O(N)$ as well. So by Weyl's inequality we get that $\lambda_N(\Tilde{A}_N\Tilde{B}_N+\Tilde{B}_N\Tilde{A}_N)\leq \lambda_N(\Tilde{A}_N\Tilde{B}_N)+\lambda_N(\Tilde{B}_N\Tilde{A}_N)=O(N)$. The argument for the smallest negative eigenvalues follows from considering $-\Tilde{A}_N$ and $-\Tilde{B}_N$ and the operator norms of these matrices and gives the result that the smallest of $\Tilde{A}_N\Tilde{B}_N+\Tilde{B}_N\Tilde{A}_N$ is also $O(N)$.
%\end{proof}

\begin{lemma}
Let $\overline{A}_N$ and $\overline{B}_N$ be average matrices as defined above, then $\{\overline{A}_N, \overline{B}_N\}$ has 1 eigenvalue exactly at $\frac{2N^2}{jk}$ and $N-1$ eigenvalues at 0.
\end{lemma}
\begin{proof}
Since $j$ and $k$ are relatively prime, then from matrix multiplication, we see that $\overline{A}_N\overline{B}_N$ and $\overline{B}_N\overline{A}_N$ are both the constant matrix where every entry is $\frac{N}{kj}$. Hence, $\{\overline{A}_N, \overline{B}_N\}$ is the constant matrix where every entry is $\frac{2N}{kj}$. Such matrix has 1 eigenvalue exactly at $\frac{2N^2}{kj}$ and $N-1$ eigenvalues at $0$.  
\end{proof}

\begin{lemma}\label{multipleregimeskjcheckerboard}
Let $A_N$ be an $N\times N$ $k$-checkerboard matrix, and $B_N$ an $N\times N$ $j$-checkerboard matrix such that $\gcd(j,k)=1$ and $jk|N$. Assume without loss of generality that $2\leq k<j$. Then the eigenvalues of $\{A_N,B_N\}$ are distributed as follows: 
\begin{enumerate}
\item $1$ eigenvalue at $\frac{2}{jk}N^2+O(N^{3/2})$,
\item $k-1$ eigenvalues at each of $\pm \frac{1}{k}\sqrt{1-\frac{1}{j}}N^{3/2}+O(N)$,
\item $1$ eigenvalue between $\frac{1}{j}\sqrt{1-\frac{1}{k}}N^{3/2}+O(N)$ and $\frac{1}{k}\sqrt{1-\frac{1}{j}}N^{3/2}+O(N)$ and $1$ eigenvalue between $-\frac{1}{k}\sqrt{1-\frac{1}{j}}N^{3/2}+O(N)$ and $-\frac{1}{j}\sqrt{1-\frac{1}{k}}N^{3/2}+O(N)$,
\item $j-2$ eigenvalues at $\pm \frac{1}{j}\sqrt{1-\frac{1}{k}}N^{3/2}+O(N)$,
\item 1 eigenvalue between $O(N)$ (positive) and $\frac{1}{j}\sqrt{1-\frac{1}{k}}N^{3/2}+O(N)$ and 1 eigenvalue between $O(N)$ (negative) and $-\frac{1}{j}\sqrt{1-\frac{1}{k}}N^{3/2}+O(N)$,
\item The remaining $N-2k-2j+1$ eigenvalues of $O(N)$.
\end{enumerate}
%$\frac{1}{j}\sqrt{1-\frac{1}{k}}N^{3/2}+O(N)$ and $\frac{1}{k}\sqrt{1-\frac{1}{j}}N^{3/2}+O(N)$, $j-1$ eigenvalues between $-\frac{1}{k}\sqrt{1-\frac{1}{j}}N^{3/2}$ and $-\frac{1}{j}\sqrt{1-\frac{1}{k}}N^{3/2}+O(N)$, 1 eigenvalue bounded above by 
%and the remaining $N-2k-2j+3$ eigenvalues are $O(N)$.
\end{lemma}

\begin{proof}
By assumption, we have $2\leq k<j$, so $\frac{1}{k}\sqrt{1-\frac{1}{j}}> \frac{1}{j}\sqrt{1-\frac{1}{k}}$. Since $\{A_N,B_N\}=\{\Tilde{A}_N,\Tilde{B}_N\}+\{\overline{A}_N,\Tilde{B}_N\}+\{\Tilde{A}_N,\overline{B}_N\}+\{\overline{A}_N,\overline{B}_N\}$, then
\begin{align}
\lambda_N(\{A_N,B_N\}) &\ \geq \ \lambda_N(\{\overline{A},\overline{B}\}) + \lambda_1(\{\overline{A},\Tilde{B}\} + \{\Tilde{A},\overline{B}\} + \{\Tilde{A},\Tilde{B}\}) \nonumber \\
&\ \geq \ \lambda_N(\{\overline{A},\overline{B}\}) + \lambda_1(\{\overline{A},\Tilde{B}\}) + \lambda_1(\{\Tilde{A},\overline{B}\}) + \lambda_1(\{\Tilde{A},\Tilde{B}\}) \ = \ \frac{2N^2}{jk} + O(N^{3/2}).
\end{align}
This establishes the existence of the largest blip. Then, we establish the existence of the intermediary blip containing $k-1$ eigenvalues at $\frac{1}{k}\sqrt{1-\frac{1}{j}}N^{3/2}+O(N)$:

\begin{alignat}{2}
&\lambda_{N-1}(\{A_N,B_N\}) &&\ \leq \ \lambda_{N-1}(\{\overline{A}_N,\overline{B}_N\})+\lambda_{N}(\{\overline{A},\Tilde{B}\} + \{\Tilde{A},\overline{B}\} + \{\Tilde{A},\Tilde{B}\}) \nonumber \\
 & &&  \ \leq \ \lambda_N(\{\overline{A}_N,\Tilde{B}_N\}+\{\Tilde{A}_N,\overline{B}_N\}) + \lambda_N(\{\Tilde{A}_N,\Tilde{B}_N\}) \ = \ \frac{1}{k}\sqrt{1-\frac{1}{j}}N^{3/2} + O(N),\nonumber
\\
&\lambda_{N-k+1}(\{A_N,B_N\}) && \ \geq \ \lambda_{1}(\{\overline{A}_N,\overline{B}_N)+\lambda_{N-k+1}(\{\overline{A},\Tilde{B}\} + \{\Tilde{A},\overline{B}\} + \{\Tilde{A},\Tilde{B}\}) \nonumber \\
 & &&\ \geq \ 
\lambda_{N-k+1}(\{\overline{A}_N,\Tilde{B}_N\}+\{\Tilde{A}_N,\overline{B}_N\})+\lambda_1(\{\Tilde{A}_N,\Tilde{B}_N)=\frac{1}{k}\sqrt{1-\frac{1}{j}}N^{3/2} + O(N).
\end{alignat}
Next, we show that there is one eigenvalue between $\frac{1}{j}\sqrt{1-\frac{1}{k}}N^{3/2}+O(N)$ and $\frac{1}{k}\sqrt{1-\frac{1}{j}}N^{3/2}+O(N)$ as well as $j-2$ eigenvalues at $\frac{1}{j}\sqrt{1-\frac{1}{k}}N^{3/2}+O(N)$:

\begin{alignat}{2}
&\lambda_{N-k}(\{A_N,B_N\}) &&\ \leq \ \lambda_{N-k+1}(\{\overline{A}_N,\Tilde{B}_N\} + \{\Tilde{A}_N,\overline{B}_N\}) + \lambda_{N-1}(\{\overline{A}_N,\overline{B}_N\}+\{\Tilde{A}_N, \Tilde{B}_N\}) \nonumber \\
& &&\ = \ \frac{1}{k}\sqrt{1-\frac{1}{j}}N^{3/2}+O(N),\nonumber \\
&\lambda_{N-k-1}(\{A_N,B_N\}) &&\ \leq \ \lambda_{N-k}(\{\overline{A}_N,\Tilde{B}_N\} + \{\Tilde{A}_N,\overline{B}_N\}) + \lambda_{N-1}(\{\overline{A}_N,\overline{B}_N\}+\{\Tilde{A}_N, \Tilde{B}_N\}) \nonumber \\
& &&\ = \ \frac{1}{j}\sqrt{1-\frac{1}{k}}N^{3/2}+O(N),\nonumber \\
&\lambda_{N-k-j+2}(\{A_N,B_N\}) &&\ \geq \ \lambda_1(\{\overline{A}_N,\overline{B}_N\}+\{\Tilde{A}_N,\Tilde{B}_N\}) + \lambda_{N-j-k+2}(\{\overline{A}_N,\Tilde{B}_N\} + \{\Tilde{A}_N,\overline{B}_N\}) \nonumber \\
& &&\ = \ \frac{1}{j}\sqrt{1-\frac{1}{k}}N^{3/2} + O(N).
\end{alignat}
By symmetry argument, we can use Weyl's inequality to establish the existence of their negative counterpart. Finally, for the bulk, we have 
\begin{alignat}{2}
&\lambda_{N-j-k+1}(\{A_N,B_N\}) &&\ \leq \ \lambda_{N-j-k+2}(\{\overline{A}_N,\Tilde{B}_N\} + \{\Tilde{A}_N,\overline{B}_N\}) + \lambda_{N-1}(\{\overline{A}_N,\overline{B}_N\}+\{\Tilde{A}_N,\Tilde{B}_N\}) \nonumber \\
& &&\ \leq \ \frac{1}{j}\sqrt{1-\frac{1}{k}}N^{3/2} + O(N),\nonumber \\
&\lambda_{N-j-k}(\{A_N, B_N\}) &&\ \leq \ \lambda_{N-j-k+1}(\{\overline{A}_N, \Tilde{B}_N\}+\{\Tilde{A}_N, \overline{B}_N\})+\lambda_{N-1}(\{\overline{A}_N, \overline{B}_N\}+\{\Tilde{A}_N, \Tilde{B}_N\})\nonumber  \\
& &&\ = \ O(N).
\end{alignat}
By symmetry argument, we can use Weyl's inequality to bound the bulk from the below. This completes the proof.
\end{proof}

Note that in the proof of Lemma \ref{multipleregimeskjcheckerboard}, Weyl's inequality fails to provide an accurate bound on the four eigenvalues between different regimes. Empirically, we observe that among those eigenvalues, the positive ones belong to the regimes corresponding to their lower bound, and the negative ones belong to the regimes corresponding to their upper bound.

\section{Almost Sure Convergence}\label{sec:convergence}
\begin{comment}
\begin{enumerate}
    \item element of $\Omega$ is now $\{A_N\}_{N \in \N}$ where we suppress $N\in\N$ usually.
    \item matrix in $\Omega_N$ is $A_N$
    \item element of $\Omega^\N$ is $\overline{A} = \{A^{(i)}\}_{i \in \N}$ where we suppress $i \in \N$ usually and note that $A^{(i)}$ is actually a sequence. In fact, we have $A^{(i)} \in \Omega$ is a sequence of $N \times N$ matrices for $N$ increasing.
\end{enumerate}
\end{comment}



The traditional way to show weak convergence of empirical spectral measures to a limiting spectral measure (in probability or almost-surely) is to show that the variance (resp. fourth moment) of the $m$\textsuperscript{th} moment, averaged over the $N \times N$ ensemble, is $O(\frac{1}{N})$ (resp. $O\left(\frac{1}{N^2}\right)$). In the case of the blip spectral measure, neither of these methods works properly. However, for a k-checkerboard matrix there are only $k$ eigenvalues in the blip, so each blip spectral measure is just a collection of $k$ isolated delta spikes distributed according to the limiting spectral computed in Theorem \ref{GOE-checkerboard Moments}. As such, for fixed $k$ the variance and fourth moment over the ensemble of the general $m$\textsuperscript{th} moment do not go to $0$ which means we cannot use the general methods. We therefore define a modified spectral measure which averages over the eigenvalues of many matrices in order to extend standard techniques, in particular we link the moments to the moments of the $k\times k$ Gaussian Wigner matrix using similar methods as \cite{split}.

In order to facilitate the proof of the main convergence result (Theorem \ref{thm_as_convergence}) we first introduce some new notation. In all that follows we fix $k$ and suppress $k$-dependence in our notation. Let $\Omega_N$ be the probability space of $N \times N$ k-checkerboard matrices with the natural probability measure. Then we define the product probability space
\begin{equation}
\Omega\ :=\ \prod_{N \in \N} \Omega_N.
\end{equation}
By Kolmogorov's extension theorem (see \cite{Tao2}), this is equipped with a probability measure which agrees with the probability measures on $\Omega_N$ when projected to the $N$\textsuperscript{th} coordinate. Given $\{A_N\}_{N \in \N} \in \Omega$, we denote by $A_N$ the $N \times N$ matrix given by projection to the $N$\textsuperscript{th} coordinate. In what follows, we suppress the subscript $N \in \N$ on elements of $\Omega$, writing them as $\{A_N\}$.

\begin{rem}
\cite{Block Circulant} employs a similar construction using product space, while \cite{Toeplitz} views elements of $\Omega$ as infinite matrices and the projection map $\Omega \rightarrow \Omega_N$ as simply choosing the upper left $N \times N$ minor.
\end{rem}

Previously we treated the $m$\textsuperscript{th} moment of an empirical spectral measure $\mu_{A,N}^{(m)}$ as a random variable on $\Omega_N$, but we may equivalently treat it as a random variable on $\Omega$. To highlight this, we define the random variable $X_{m,N}$ on $\Omega$
\begin{equation}\label{eq_xmn}
X_{m,N}(\{A_N\})\ :=\ \mu_{A_N,N}^{(m)}.
\end{equation}
These have centered $r$\textsuperscript{th} moment
\begin{equation}
X_{m,N}^{(r)}\ :=\ \E{(X_{m,N}-\E{X_{m,N}})^r}.
\end{equation}

Per our motivating discussion at the beginning of this section, because we wish to average over a growing number of matrices of the same size, it is advantageous to work over $\Omega^\N$; this again is equipped with a natural probability measure by Kolmogorov's extension theorem. Its elements are sequences of sequences of matrices, and we denote them by $\overline{A}=\{A^{(i)}\}_{i \in \N}$ where $A^{(i)} \in \Omega$. We now give a more abstract definition of the averaged blip spectral measure.

\begin{defi}\label{def_average_blip_measure}
Fix a function $g: \N \rightarrow \N$. The \textbf{averaged empirical blip spectral measure} associated to $\overline{A} \in \Omega^\N$ is
\begin{equation}
\mu_{N,g,\overline{A}}\ :=\ \frac{1}{g(N)}\sum_{i=1}^{g(N)} \mu_{A_N^{(i)},N}.
\end{equation}
%If these measures converge in any sense to a limit as $N \rightarrow \infty$, we refer to it as the \textbf{limiting averaged blip spectral measure} or the limiting spectral measure when clear from context. \fix{make sure we actually do this somewhere or take it out.-R}
\end{defi}

In other words, we project onto the $N$\textsuperscript{th} coordinate in each copy of $\Omega$ and then average over the first $g(N)$ of these $N \times N$ matrices.

\begin{rem}\label{rem_stupidblocks}
If one wishes to avoid defining an empirical spectral measure which takes eigenvalues of multiple matrices, one may use the (rather contrived) construction of a $\N \times \N$ block matrix with independent $N \times N$ checkerboard matrix blocks.
\end{rem}

Analogously to $X_{m,N}$, we denote by $Y_{m,N,g}$ the random variable on $\Omega^\N$ defined by the moments of the averaged empirical blip spectral measure
\begin{equation}
Y_{m,N,g}(\overline{A})\ :=\ \mu_{N,g,\overline{A}}^{(m)}.
\end{equation}
The centered $r$\textsuperscript{th} moment (over $\Omega^\N$) of this random variable is denoted by $Y_{m,N,g}^{(r)}$.


We now prove almost-sure weak convergence of the averaged blip spectral measures under a growth assumption on $g$. Recall the following definition.

\begin{defi}
A sequence of random measures $\{\mu_N\}_{N \in \N}$ on a probability space $\Omega$ converges \textbf{weakly almost-surely} to a fixed measure $\mu$ if, with probability $1$ over $\Omega^\N$, we have
\begin{equation}
\lim_{N \rightarrow \infty} \int f d\mu_N = \int f d\mu
\end{equation}
for all $f \in \mathcal{C}_b(\R)$ (continuous and bounded functions).
\end{defi}

\begin{thm}\label{thm_as_convergence}
Let $g: \mathbb{N} \rightarrow \mathbb{N}$ be such that there exists an $\delta>0$ for which $g(N) = \omega(N^\delta)$. Then, as $N\to\infty$, the averaged empirical spectral measures $\mu_{N,g,\overline{A}}$ of the k-checkerboard ensemble converge weakly almost-surely to the measure with moments $M_{k,m}=\frac{1}{k} \mathbb{E}\text{Tr}[B^m]$, the limiting expected moments computed in Theorem \ref{GOE-checkerboard Moments}.
\end{thm}
\begin{proof}

For simplicity of notation, we suppress $k$ and denote $M_{k,m}$ by $M_m$. By the triangle inequality, we have
\begin{equation}\label{eqn_tri_ineq_y}
|Y_{m,N,g}-M_m|\ \leq\ |Y_{m,N,g}-\E{Y_{m,N,g}}|+|\E{Y_{m,N,g}}-M_m|.
\end{equation}
From Theorem \ref{GOE-checkerboard Moments}, we know that $\E{X_{m,N}} \to M_m$, and it follows that $\E{Y_{m,N,g}} \to M_m$. Hence to show that $Y_{m,N,g} \to M_m$ almost surely, it suffices to show that $|Y_{m,N,g} - \E{Y_{m,N,g}}|\to 0$ almost surely as $N \rightarrow \infty$.
We show that the limit as $N \rightarrow \infty$ of all moments over $\Omega_N$ of any arbitrary moment of the empirical spectral measure exists, and that we may always choose a sufficiently high moment\footnote{Note the difference between this and the standard technique of, for instance, \cite{Toeplitz}, which uses only the fourth moment.} such that the standard method of Chebyshev's inequality and the Borel-Cantelli lemma gives that $|Y_{m,N,g}-\E{Y_{m,N,g}}|\to 0$. Finally, the moment convergence theorem gives almost-sure weak convergence to the limiting averaged blip spectral measure.

\begin{lem}\label{lem_moments_of_moments}
Let $X_{m,N}$ be as defined in \eqref{eq_xmn}. %the $m$\textsuperscript{th} moment of the empirical spectral measure for an $N\times N$ matrix treated as a random variable over the space $\Omega_N$, t
Then for any $t \in \N$, the $r$\textsuperscript{th} centered moment of $X_{m,N}$ satisfies
\begin{equation}
X_{m,N}^{(r)} \ =\ \E{\left(X_{m,N}-\E{X_{m,N}}\right)^r}\ =\ O_{m,r}(1)
\end{equation}
as $N$ goes to infinity.
%exists.
\end{lem}
\begin{proof}
Firstly, we have
\begin{align}
 \E{\left(X_{m,N}-\E{X_{m,N}}\right)^r}\ &\ =\  \mathbb{E}\left[\sum_{\ell=0}^r \binom{r}{\ell}(X_{m,N})^\ell \left(\E{X_{m,N}}\right)^{r-\ell}\right]\  \nonumber \\
 &\ =\ \ \sum_{\ell=0}^r \binom{r}{\ell}\mathbb{E}\left[(X_{m,N})^\ell\right] \left(\E{X_{m,N}}\right)^{r-\ell}.
\end{align}
From the moments given in Section \ref{GOE-checkerboard Moments}, we have $\E{X_{m,N}}=O_m(1)$ hence $\left(\E{X_{m,N}}\right)^{r-\ell}=O_{m,r,\ell}(1)$ for all $\ell$. As such, it suffices to show that $\E{(X_{m,N})^\ell}=O_{m,\ell}(1)$.
By \eqref{GOE-checkerboard Moments}, we have that

\begin{align}
&\E{X_{m,N,1}}^l \nonumber \\ 
&\ = \ \mathbb{E}\left[ \frac{1}{k_1}\sum_{\alpha = 2n} ^{8n}c_{\alpha} \left(\frac{k}{N^{3/2}}\right)^{\alpha}\left(\frac{1}{N^m}\sum_{i = 0}^{m}\binom{m}{i} \left(-\frac{N^{3/2}}{k}\right)^{m-i}\text{Tr}[\{A_N,B_N\}^{\alpha+i}]\right)^l\right]  \nonumber \\
&\ = \  \mathbb{E}\left[\sum_{\substack{2n \leq \alpha_1 \leq 8n\\ 0 \leq i_1 \leq m}} \sum_{\substack{2n \leq \alpha_2 \leq 8n\\ 0 \leq i_2 \leq m}}\dots\sum_{\substack{2n \leq \alpha_l \leq 8n\\ 0 \leq i_l \leq m}} \frac{1}{N^{m\ell}}\prod_{\nu=1}^lc_{\alpha_v} {\binom{m}{i_v}} (-1)^{m-i_v}\left(\frac{N^{3/2}}{k}\right)^{l-i_v}
\text{Tr}[\{A_N,B_N\}^{\alpha+i_v}] \right] \nonumber \\
&\ = \  \sum_{\substack{2n \leq \alpha_1 \leq 8n\\ 0 \leq i_1 \leq m}} \sum_{\substack{2n \leq \alpha_2 \leq 8n\\ 0 \leq i_2 \leq m}}\dots\sum_{\substack{2n \leq \alpha_l \leq 8n\\ 0 \leq i_l \leq m}} \frac{1}{N^{m\ell}}\prod_{\nu=1}^lc_{\alpha_v} {m \choose i_v} (-1)^{m-i_v}\left(\frac{N^{3/2}}{k}\right)^{l-i_v}\mathbb{E} \left[\prod_{v=1}^l \text{Tr}[\{A_N,B_N\}^{\alpha+i_v}]\right]. \label{eq_finite_moments}
\end{align}
Now, recall that
\begin{equation}
\mathbb{E}\left[\prod_{v=1}^\ell\text{Tr} A^{2n+i_v} \right] \ = \  \sum_{\alpha^1_1,\dots,\alpha^1_{2n+i_1} \leq N} \dots \sum_{\alpha^\ell_1,\dots,\alpha^\ell_{2n+i_\ell}\leq N} \mathbb{E}\left [ \prod_{j=1}^\ell a_{\alpha^j_1,\alpha^j_2}\dots a_{\alpha^j_{2n+i_j},\alpha^j_1} \right].
\end{equation}
We have now reached a combinatorial problem similar to the one we encounter in Section \ref{sec: anticommutator Combinatorics}. For each $j$, since the length of the cyclic product $a_{\alpha^j_1,\alpha^j_2}\dots a_{\alpha^j_{2n+i_j},\alpha^j_1}$ is fixed at $2n+i_j$, we can choose the number of blocks (determining the class), the location of the blocks (determining the configuration), the matchings and indexings. By Lemma \ref{blockslemma} and \ref{Sclasscontribution}, we have that the main contribution from configurations of length $(2n+i_j)$ in $B_j$-class is $\frac{(2n+i_j)^{B_j}}{B_j!}$. By the same arguments made in \S\ref{sec: anticommutator Combinatorics}, the number of ways we can choose the number of blocks having one $a$ and two $a$'s as well as the number of ways to choose matchings across the $\ell$ cyclic products are independent of $N$, $j$'s and $i_j$'s, so for simplicity, we are denoting them as $C$. Finally, the contribution from choosing the indices of all the blocks and $w$'s is $O_k(N^{2n\ell+i_1+\dots+i_\ell-B_1-\dots-B_\ell})$. As such, if $B_1,\dots,B_\ell \geq m$, the total contribution is $O_{m,k}(1)$.
If there exists $B_{j'}<m$, then the overall contribution is
\begin{equation}
CN^{\ell m-B_1-\dots-B_\ell} \prod_{u=1}^\ell \left[\sum_{j_u=0}^{2n} \binom{2n}{j_u} \sum_{i_u=0}^{m+j_u} \binom{m+j_u}{i_u}(-1)^{m-i_u}\frac{(2n+i_u)^{B_u}}{B_u!}\right]\ = \ 0,
\end{equation}
since the sum over $j_u=j'$ is equal to 0 by by Lemma \ref{inequalities}. As such, the total contribution of $\E{X_{m,N}^\ell}$ is simply $O_{m,\ell}(1)$ (suppressing $k$), as desired.
\end{proof}

We apply the following theorem (Theorem $1.2$ of \cite{Fer}) with $X=X_{m,N}-\E{X_{m,N}}$, $s=g(N)$ and $\mu_i=X_{m,N}^{(i)}$.

\begin{thm}\label{thm_turkishguy}
Let $r \in \N$ and let $X_1,\ldots,X_s$ be i.i.d. copies of some mean-zero random variable $X$ with absolute moments $\E{|X|^\ell}<\infty$ for all $\ell \in \N$. Then
\begin{equation}
\mathbb{E}\left[ \left( \sum_{i=1}^s X_i\right)^r \right]\ =\ \sum_{1 \leq m \leq \frac{r}{2}}B_{m,r}(\mu_2,\mu_3,\ldots,\mu_r) \binom{s}{m}
\end{equation}
where the $\mu_i$ are the moments of $X$ and $B_{m,r}$ is a function independent of $s$, the details of which are given in \cite{Fer}.
\end{thm}
We must first show boundedness of the absolute moments of $X_{m,N}$. By Cauchy-Schwarz,
\begin{equation}
\left(\int |x^{2\ell+1}|d\mu_{X_{m,N}}\right)^2\ \leq\ \int |x|^2 d\mu_{X_{m,N}} \cdot \int |x|^{4\ell}d\mu_{X_{m,N}},
\end{equation}
where $\mu_{X_{m,N}}$ is the probability measure on $\Omega$ given by the density of $X_{m,N}$. Since, for fixed $N$, the even moments of $X_{m,N}$ are finite by \eqref{eq_finite_moments}, the previous bound shows that all odd absolute moments are finite as well. Hence Theorem \ref{thm_turkishguy} applies, yielding
\begin{equation}
\mathbb{E}\left[\left(\sum_{i=1}^{g(N)} X_{m,N,i}-\E{X_{m,N,i}}\right)^r\right]\ =\ \sum_{1 \leq m \leq \frac{r}{2}}B_{m,r}(X_{m,N}^{(2)},X_{m,N}^{(3)},\ldots,X_{m,N}^{(r)}) \binom{g(N)}{m}.
\end{equation}
where the $X_{m,N,i}$ are $i$-indexed i.i.d. copies of $X_{m,N}$. By Lemma \ref{lem_moments_of_moments}, for sufficiently high $N$, $X_{m,N}^{(t)}$ are uniformly bounded above by some constant $K$ for $1 \leq t \leq m$, so there exists $C$ such that $\\B_{m,r}(X_{m,N}^{(2)},X_{m,N}^{(3)},\ldots,X_{m,N}^{(r)}) < C$ for all sufficiently large $N$ and for all $1\leq m \leq r/2$. Hence
\begin{equation}
\mathbb{E}\left[\left(\sum_{i=1}^{g(N)} X_{m,N,i}-\E{X_{m,N,i}}\right)^r\right]\ \leq\ \sum_{1\leq m \leq \frac{r}{2}}  C\binom{g(N)}{m}.
\end{equation}
As such, we have
\begin{equation}
Y_{m,N,g}^{(r)} = \frac{1}{g(N)^r} \mathbb{E}\left[\left(\sum_{i=1}^{g(N)} X_{m,N,i}-\E{X_{m,N,i}}\right)^r\right] \leq \sum_{1\leq m \leq \frac{r}{2}} \frac{C}{g(N)^r}\binom{g(N)}{m}  = O\left(\frac{1}{g(N)^{r/2}}\right).
\end{equation}
Since $g(N) = \omega(N^\delta)$, we may choose $r$ sufficiently large so that
\begin{equation}
Y_{m,N,g}^{(r)}\ =\ O\left(\frac{1}{N^2}\right).
\end{equation}
Then by Chebyshev's inequality,
\begin{equation}
\mathbb{P}(|Y_{m,N,g}-\E{Y_{m,N,g}}|>\epsilon)\ \leq\ \frac{\E{(Y_{m,N,g}-\E{Y_{m,N,g}})^r}}{\epsilon^r}\ =\ \frac{Y_{m,N,g}^{(r)}}{\epsilon^r}\ =\ O\left(\frac{1}{N^2}\right).
\end{equation}

We now apply the following (see for example \cite{Can}).

\begin{lem}[Borel-Cantelli]
Let $B_i$ be a sequence of events with $\sum_i \mathbb{P}(B_i)<\infty$. Then
\begin{equation}
\mathbb{P}\left(\bigcap_{j=1}^\infty \bigcup_{\ell=j}^\infty B_\ell\right)\ =\ 0.
\end{equation}
\end{lem}

Define the events
\begin{equation}
B_N^{(m,d,g)}\ := \ \left\{A \in \Omega^\N: |Y_{m,N,g}(A) - \E{Y_{m,N,g}}| \ \geq\ \frac{1}{d}\right\}.
\end{equation}
Then $\mathbb{P}(B_N^{(m,d,g)}) \leq \frac{C_m d^r}{N^2}$, so for fixed $m$, $d$, the conditions of the Borel-Cantelli lemma are satisfied. Hence
\begin{equation}
\mathbb{P}\left(\bigcap_{j=1}^\infty \bigcup_{\ell=j}^\infty B_\ell^{(m,d,g)}\right) \ =\ 0.
\end{equation}
Taking a union of these measure-zero sets over $d \in \N$ we have
\begin{equation}
\mathbb{P}\left(Y_{m,N,g} \neq \E{Y_{m,N,g}} \text{ for infinitely many $N$}\right)\ =\ 0,
\end{equation}
and taking the union over $m \in \Z_{\geq 0}$,
\begin{equation}
\mathbb{P}\left(\exists m \text{ such that }Y_{m,N,g} \neq \E{Y_{m,N,g}} \text{ for infinitely many $N$}\right)\ = \ 0.
\end{equation}
Therefore with probability $1$ over $\Omega^\N$, $|Y_{m,N,g}-\E{Y{m,N,g}}| \to 0$ for each $m$. This, together with \eqref{eqn_tri_ineq_y} and the discussion following it, yields that the moments $\mu_{N,g}^{(m)}=Y_{m,N,g} \to M_m$ almost surely. We now use the following to show almost-sure weak convergence of measures (see for example \cite{Ta}).
%Let $B \subset \Omega$ be the subset of $\Omega$ on which the sequence of empirical spectral measures  Then by standard arguments (see e.g. \fix{cite hammond-miller})

\begin{thm}[Moment Convergence Theorem]\label{thm_moment_convergence}
Let $\mu$ be a measure on $\R$ with finite moments $\mu^{(m)}$ for all $m \in \Z_{\geq 0}$, and $\mu_1,\mu_2,\ldots$ a sequence of measures with finite moments $\mu_n^{(m)}$ such that $\lim_{n\rightarrow \infty} \mu_n^{(m)} = \mu^{(m)}$ for all $m \in \Z_{\geq 0}$. If in addition the moments $\mu^{(m)}$ uniquely characterize a measure, then the sequence $\mu_n$ converges weakly to $\mu$.
\end{thm}


To show Carleman's condition is satisfied for the limiting moments $M_m$, we show that $M_m$ are bounded above by the moments of the Gaussian. The odd moments vanish, and by Theorem \ref{GOE-checkerboard Moments} the even moments are given by
\begin{equation}
M_{2m}\ =\ \frac{2}{k} \E{\text{Tr } A^{2m}}\ =\ \sum_{1 \leq i_1,\ldots,i_{2m} \leq k} 2\E{a_{i_1i_2}a_{i_2i_3}\ldots a_{i_{2m}i_1}},
\end{equation} and as
$\E{a_{i_1i_2}a_{i_2i_3}\ldots a_{i_ni_1}}$ is maximized when all $a_{i_{\ell}i_{\ell+1}}$ are equal,
\begin{equation}
M_{2m}\ \leq\  \sum_{1 \leq i_1,\ldots,i_{2m} \leq k} 2(2m-1)!!\ =\ 2k^{2m}(2m-1)!!.
\end{equation}
These are bounded by the moments of $\mathcal{N}(0,2k)$ so Carleman's condition is satisfied, thus we let $\overline{\mu}$ be the unique measure determined by the moments $M_m$. Choose $\overline{A} \in \Omega^\N$. Then the preceding argument showed that, with probability $1$ over $\overline{A}$ chosen from $\Omega^\N$, all moments $\mu_{N,g,\overline{A}}^{(m)}$ of the measures $\mu_{N,g,\overline{A}}$ converge to $M_m$. Then by Theorem \ref{thm_moment_convergence} the measures $\mu_{N,g,\overline{A}}$ converge weakly to $\overline{\mu}$ with probability $1$, completing the proof.
\end{proof}

\section{Almost-Sure Convergence of the Bulk}

\begin{theorem}\label{checkerboard convergence} For $A_N$ and $B_N$ both $N\times N$ $(k,0)$-checkerboard matrices we get that for any fixed $\ell$
\begin{align}
\lim_{N\to\infty}\text{Var}[\nu_{\{A_N,B_N\}}^{(\ell)}]\ = \ O\left(\frac{1}{N^2}\right).
\end{align}
\end{theorem}

\begin{proof}
We know that by the eigenvalue trace lemma, we have
\begin{align}
&\text{Var}\left[\nu_{\{A_N,B_N\}}^{(\ell)}\right]\ = \ \left| \E{(\nu_{\{A_N,B_N\}^{(\ell)})^2}}-\left[\E{\nu_{\{A_N,B_N\}}}\right]^2\right|\nonumber \\&=\frac{1}{N^{2\ell +2}}\left|\E{\text{Tr}(\{A_N,B_N\}^\ell)^2}-(\E{\text{Tr}\{A_N,B_N\}^{\ell}})^2\right|
\ = \ \frac{1}{N^{2\ell+2}}\sum_{I,I'}\left|\E{\zeta_{I}\zeta_{I'}}-\E{\zeta_I}\E{\zeta_{I'}}\right|
\end{align}

where the $\zeta_I$ and $\zeta_{I'}$ are stand ins for a product $c_{i_1,i_2}c_{i_2,i_3}\cdots c_{i_{2\ell},i_1}$ where every $c$ is $a$ or $b$ and is some expansion of $(AB+BA)^\ell$. So then we know that from the proof in \cite{split} that for any choices of $A$'s and $B$'s this is $O(1/N^2)$ and since we consider $\ell$ as fixed we know that there are $2^\ell$ different configurations which are constant and for each configuration from the paper we know that they are $O_m(1/N^2)$ which means that if we add up all of these different cases and configurations we get that it is still $O(1/N^2)$ which proves the theorem. This theorem proves convergence when combined with Chebyshev's inequality and the Borel-Cantelli lemma.
\end{proof}

By Chebyshev's inequality we get the first inequality and by Theorem \ref{checkerboard convergence} we get that the sum of variances is finite, giving,
\begin{align}
\sum_{N=1}^\infty \text{Pr}\left(\left|\nu_{\{A_N,B_N\}}^{(\ell)}-\E{\nu_{\{A_N,B_N\}}^{(\ell)}}>\epsilon\right|\right)\ \leq \ \frac{1}{\epsilon^2}\sum_{N=1}^\infty \text{Var}(\nu_{\{A_N,B_N\}})\ < \ \infty.
\end{align}
So then by the Borel-Cantelli lemma and Theorem \ref{checkerboard convergence} we get that the moments converge almost surely.

\begin{theorem}\label{PT-PT Convergence} For $A_N$ and $B_N$ both $N\times N$ palindromic Toeplitz matrices we get that for any fixed $\ell$
\begin{align}
\lim_{N\to\infty}\E{|M_m(\{A_N,B_N\})-\E{M_m(\{A_N,B_N\})}|^4}\ = \ O_m\left(\frac{1}{N^2}\right)
\end{align}
\end{theorem}

\begin{proof}
Expanding this yields
\begin{align}
&\E{M_m(\{A_N,B_N\})^4}\ - \ 4\E{M_m(\{A_N,B_N\})^3}\E{M_m(\{A_N,B_N\})}\nonumber\\&\ + \ 6\E{M_m(\{A_N,B_N\})^2}\E{M_m(\{A_N,B_N\})}^2\ - \ 3\E{M_m(\{A_N,B_N\})}\E{M_m(\{A_N,B_N\})}^3
\end{align}

As the odd moments are all $0$ their expected value is always $0$l so we only need to consider even moments. So then we can write the terms as 
\begin{align}
\E{M_{2m}(\{A_N,B_N\})^4}\ = \ \frac{1}{N^{8m+4}}\sum_i\sum_j\sum_k\sum_\ell\E{c_{is}c_{js}c_{k s}c_{\ell s}}
\end{align}
where 
\begin{align}
\E{c_{is}c_{js}c_{\ell s}c_{ks}}=\E{c_{|i_1-i_2|}\cdots c_{|i_{4m}-i_1|}c_{|j_1-j_2|}\cdots c_{|j_{4m}-j_1|}c_{|k_1-k_2|}\cdots c_{|k_{4m}-k_1|}c_{|\ell_1-\ell_2|}\cdots c_{|\ell_{4m}-\ell_1|}},
\end{align}
where the $c$'s are all $a$'s or $b$'s and $a$'s can only match with $a$'s and $b$'s can only match with $b$'s, however, it suffices to allow matches to be free since the upper bound of $O(1/N^2)$ holds in any case. Then we know that from \cite{Toeplitz}, given a fixed expansion of $a$'s and $b$'s of the terms in the binomial expansion are $O(N^{8m+2})$. So then since we know that there are $2^{8m}$ expansions of the anticommutator and $m$ is a fixed constant we get that the total contribution of all of the terms in the anticommutator expansion are $2^{8m}\cdot O(N^{8m+2})=O(N^{8m+2})$ since $2^{8m}$ is a fixed constant.
\end{proof}

\begin{rem}
Note that the proof of Theorem \ref{PT-PT Convergence} also applies when $A_N$ is a palindromic Toeplitz and $B_N$ is a GOE. This is due to the fact that all matchings of a palindromic Toeplitz matrix and GOE anticommutator are valid in the palindromic Toeplitz case, so whenever a degree of freedom is lost in the palindromic Toeplitz and palindromic-Toeplitz anticommutator it is also lost in the palindromic Toeplitz - GOE case since the GOE case is strictly more restricted and having the same indices implies their differences are equal. So the same argument proves convergence for the case where $A_N$ is palindromic Toeplitz and $B_N$ is a GOE.
\end{rem}

\section{Polynomial Weight Functions for Intermediary Blips}\label{intermediaryblip}
In theory, the exact choice of the polynomial weight function shouldn't affect the moment of the intermediary blips, as long as it satisfies all the required conditions. For the sake of completeness, we include here the expression for the weight functions for the intermediary blips of $\{k\textup{-checkerboard}, j\textup{-checkerboard}\}$. Let $w_s=\frac{(-1)^{s+1}}{k}\sqrt{1-\frac{1}{j}}$ for $s\in \{1, 2\}$ and $w_s=\frac{(-1)^{s+1}}{j}\sqrt{1-\frac{1}{k}}$ for $s\in \{3, 4\}$. Then, the weight function for the intermediary blip at $w_sN^{3/2}$ is
\begin{align}
g_s^{2n}(x) \ = \ \frac{x^{2n}\prod_{\substack{i=1; i\neq s}}^4\left(x-\frac{w_i}{w_s}\right)^{2n}\left(x-\frac{w_5\sqrt{N}}{w_s}\right)^{10n}(x-A)^{2n}}{B^{2n}\left(1-\frac{w_5 \sqrt{N}}{w_s}\right)^{10n}(1-A)^{2n}},
\end{align}
where $A=1+\left(1+\sum_{i=1;i\neq s}^4 \frac{1}{1-w_i/w_s}\right)^{-1}$ and $B=\prod_{i=1;i\neq s}^4\left(1-\frac{w_i}{w_s}\right)$. It's clear that $g_s^{2n}$ has zeros of order $2n$ at $0$, $\frac{w_i}{w_s}$ for $i\neq s$, and zero of order $10n$ at $\frac{w_5\sqrt{N}}{w_s}$. We shall prove that $g_s^{2n}$ vanishes at $O\left(\frac{1}{\sqrt{N}}\right)$, $\frac{w_i}{w_s}+O\left(\frac{1}{\sqrt{N}}\right)$ for $i\neq s$, and $\frac{w_5\sqrt{N}}{w_s}+O\left(\frac{1}{\sqrt{N}}\right)$, is equal to $1$ at $1+O\left(\frac{1}{\sqrt{N}}\right)$, and has a critical point at $1$. The key to the proof is the evaluation of the expression $\lim_{N\rightarrow\infty}(1+O(1/\sqrt{N}))^{2n}$. Since $\lim_{y\rightarrow \infty}(1+1/y)^y=e$ and $n\ll \log\log(N)$, then $1\leq\lim_{N\rightarrow\infty}(1+O(1/\sqrt{N}))^{2n}\leq\lim_{N\rightarrow\infty}\left((1+O(1/\sqrt{N}))^{O(\sqrt{N})}\right)^{2n/O(\sqrt{N})}=\lim_{N\rightarrow\infty} e^{2n/O(\sqrt{N})}=1$. Hence, $\lim_{N\rightarrow\infty}(1+O(1/\sqrt{N}))^{2n}\\=1$.

Now, we first evaluate the function at $x=1+O\left(\frac{1}{N}\right)$ as $N\rightarrow\infty$,
\begin{align}
&\lim_{N\rightarrow\infty}g_s^{2n}\left(1+O\left(\frac{1}{\sqrt{N}}\right)\right) \nonumber \\ 
&\ = \ \lim_{N\rightarrow\infty}\left(1+O\left(\frac{1}{\sqrt{N}}\right)\right)^{2n}\cdot \frac{\left(1+O\left(\frac{1}{\sqrt{N}}\right)-\frac{w_5\sqrt{N}}{w_s}\right)^{10n}}{\left(1-\frac{w_5\sqrt{N}}{w_s}\right)^{10n}}\cdot \frac{\left(1+O\left(\frac{1}{\sqrt{N}}\right)-A\right)^{2n}}{(1-A)^{2n}} \nonumber\\
& \cdot \frac{\prod_{i=1;i\neq s}{4}\left(1+O\left(\frac{1}{\sqrt{N}}\right)-\frac{w_i}{w_s}\right)^{2n}}{B^{2n}} \nonumber\\
&\ = \ \lim_{N\rightarrow\infty} \frac{\prod_{i=1;i\neq s}^4\left(1+O\left(\frac{1}{\sqrt{N}}\right)-\frac{w_i}{w_s}\right)^{2n}}{B^{2n}} \ = \ 1.
\end{align}
Note that we repeatedly apply the evaluation $\lim_{N\rightarrow\infty}(1+O(1/\sqrt{N}))^{2n}=1$ above to simplify the expression. As an example, $\left(\frac{1+O(1/\sqrt{N})-A}{1-A}\right)^{2n}=\left(1+\frac{O(1/\sqrt{N})}{1-A}\right)^{2n}=\left(1+O(1/\sqrt{N})\right)^{2n}=1$.

Then, we consider the evaluation of the function at $O\left(\frac{1}{\sqrt{N}}\right)$, and the evaluation of the function at other vanishing points similarly follows,
\begin{align}
&\lim_{N\rightarrow\infty}g_s^{2n}\left(O\left(\frac{1}{\sqrt{N}}\right)\right) \nonumber \\
%&\ = \ \lim_{N\rightarrow\infty}\frac{\left(O\left(\frac{1}{\sqrt{N}}\right)\right)^{2n}\prod_{i=1; i\neq s}^4 \left(O\left(\frac{1}{\sqrt{N}}\right)-\frac{w_i}{w_s}\right)^{2n}\left(O\left(\frac{1}{\sqrt{N}}\right)-\frac{w_5\sqrt{N}}{w_s}\right)^{10n}\left(O\left(\frac{1}{\sqrt{N}}\right)-A\right)^{2n}}{B^{2n}\left(1-\frac{w_5\sqrt{N}}{w_s}\right)^{10n}(1-A)^{2n}} \nonumber \\
&\ = \ \lim_{N\rightarrow\infty}\frac{\prod_{i=1;i\neq s}^4 \left(O\left(\frac{1}{\sqrt{N}}\right)-\frac{w_i}{w_s}\right)^{2n}}{B^{2n}}\cdot \frac{\left(O\left(\frac{1}{\sqrt{N}}\right)-\frac{w_5\sqrt{N}}{w_s}\right)^{10n}}{\left(1-\frac{w_5\sqrt{N}}{w_s}\right)^{10n}}\cdot \frac{\left(O\left(\frac{1}{\sqrt{N}}\right)-A\right)^{2n}}{(1-A)^{2n}} \cdot \left(O\left(\frac{1}{\sqrt{N}}\right)\right)^{2n}
\nonumber \\
&\ = \ \lim_{N\rightarrow\infty} C_N \cdot \left(O\left(\frac{1}{\sqrt{N}}\right)\right)^{2n} \ = \ 0.
\end{align}
Finally, we prove that $g_s^{2n}$ has a critical point at $1$. Using logarithmic derivative, we obtain
\begin{align}
\frac{(g_s^{2n})'(x)}{g_s^{2n}(x)} &\ = \ 2n \cdot \left(\frac{1}{x}+\sum_{i=1;i\neq s}^4\frac{1}{x-\frac{w_i}{w_s}}+\frac{5}{x-\frac{w_5\sqrt{N}}{w_s}}+\frac{1}{x-A}\right)
\end{align}
Since $g_s^{2n}(1)=1$ and $A=1+\left(1+\sum_{i=1;i\neq s}^4\frac{1}{1-\frac{w_i}{w_s}}\right)$, then
\begin{align}
(g_s^{2n})'(1) &\ = \ \lim_{N\rightarrow\infty} 2n \cdot \left(1+\sum_{i=1;i\neq s}\frac{1}{1-\frac{w_i}{w_s}}+\frac{5}{1-\frac{w_5\sqrt{N}}{w_s}}+\frac{1}{1-A}\right) \ = \ \lim_{N\rightarrow\infty}\frac{10n}{1-\frac{w_5\sqrt{N}}{w_s}} \ = \ 0.
\end{align}
Thus, $g_s^{2n}$ is the desired weight function for the intermediary blip at $w_sN^{3/2}$.

%More specifically,
%$$f_s^{2n}(x)\ = \ \begin{cases}
%(1+O(1/N^4))^{2n} &\text{  when  } x = 1 + O(1/\sqrt{N}) \\
%(O(1/\sqrt{N}))^{2n} &\text{  when  } x = O(1/\sqrt{N}) \\
%(O(1/\sqrt{N}))^{2n} &\text{  when  } x = \frac{w_i}{w_s} + O(1/\sqrt{N}) \\
%(O(1/\sqrt{N}))^{10n} &\text{  when  } x = \frac{w_5\sqrt{N}}{w_s} + O(1/\sqrt{N}).
%\end{cases}$$
%Moreover, ${f'}_s^{2n}(1)=O(1/\sqrt{N})$.

%\begin{rem}
%Each of the expressions can be reduced through algebra as you substitute the values and simplify the order that depends on N within the parenthesis. Additionally, 

%After substituting each value into the weight function and simplifying the expression, we can verify the above properties by taking the natural log of the polynomial and evaluating some limits. 
%\end{rem}

\section{Lower even moments of $\{\textup{GOE}, k\textup{-BCE}\}$ and $\{k\textup{-BCE}, k\textup{-BCE}\}$}

We provide here explicit expression of lower even moments of $\{\textup{GOE}, k\textup{-BCE}\}$ and $\{k\textup{-BCE}, k\textup{-BCE}\}$ based on genus expansion formulae, where distributions are rescaled and normalized to have mean zero and variance one. This means that the second method of both distributions are 1. Theoretically, with enough computing power, we should be able to obtain closed form expressions for any even moments for the two distribution. However, in reality, the computation is extremely complicated and time-consuming. Hence, we only provide the moments of $\{\textup{GOE, }k\textup{-BCE}\}$ up to the 10\textsuperscript{th} moment and $\{k\textup{-BCE}, k\textup{-BCE}\}$ up to the 8\textsuperscript{th} moment.

%The explicit computation of the exact $2k$th moment is involved, and a reliable value can be obtained using Matlab. 
%The computation is rather straightforward in the presence of 
%\ref{thm:GOEBC} and \ref{thm:BCBC}. %For each loop, construct a list of all valid configurations or pairings, and add up the summands
%\begin{table}[h]
%\centering
%\caption{Lower Even Moments of $\{$GOE, BC$\}$}
%\renewcommand{\arraystretch}{1.5} % Increase row height
%\begin{tabular}{|>{\centering\arraybackslash}m{1cm}|>{\centering\arraybackslash}m{2cm}|}
%\hline
%\textbf{Moment} & \textbf{Value} \\
%\hline
%Normalized 4\textsuperscript{th} moment & $$\frac{5}{2} + \frac{1}{2m^2}$$ \\
%\hline
%Normalized 6\textsuperscript{th} moment & $$\frac{33}{4} + \frac{19}{4m^2}$$ \\
%\hline
%Normalized 8\textsuperscript{th} moment & $$\frac{249}{8} + \frac{34}{m^2} + \frac{27}{8m^4}$$ \\
%\hline
%Normalized 10\textsuperscript{th} moment & $$\frac{2033}{16} + \frac{875}{4m^2} + \frac{1043}{16m^4}$$ \\
%\hline
%\end{tabular}
%\end{table}

\begin{table}[h]
\centering
\caption{Lower Even Moments of $\{\textup{GOE}, k\textup{-BCE}\}$}
\renewcommand{\arraystretch}{3}
\begin{tabular}{|>{\centering\arraybackslash}m{1.5cm}|>{\centering\arraybackslash}m{4cm}|} 
\hline
Moment & Value \\
\hline
4\textsuperscript{th} & $\dfrac{5}{2}+\dfrac{1}{2k^2}$ \\ 
\hline
6\textsuperscript{th} & $\dfrac{33}{4}+\dfrac{19}{4k^2}$ \\ 
\hline
8\textsuperscript{th} & $\dfrac{249}{8}+\dfrac{34}{k^2}+\dfrac{27}{8k^4}$ \\ 
\hline
10\textsuperscript{th} & $\dfrac{2033}{16}+\dfrac{875}{4k^2}+\dfrac{1043}{16k^4}$ \\
\hline
\end{tabular}
\end{table}
%\footnote{Higher even moments can be computed that sufficient amount of computing power is provided. The 10th and the 8th moment is a reasonable upper limit such that the moment computation terminates in a reasonable time using MATLAB online. }

% Second Table: Normalized Spectral Density of Block Circulant times Block Circulant
%\begin{table}[h!]
%\centering
%\caption{Lower Even Moments of $\{k\textup{-BCE},k\textup{-BCE}\}$}
%\renewcommand{\arraystretch}{1.5} % Increase row height
%\begin{tabular}{|>{\centering\arraybackslash}m{3cm}|>{\centering\arraybackslash}m{2cm}|}
%\hline
%\textbf{Moment} & \textbf{Value} \\
%\hline
%Normalized 4\textsuperscript{th} moment & $$ \frac{10m^4 + 86m^2 + 48}{4m^4 + 8m^2 + 4}$$ \\
%\hline
%Normalized 6\textsuperscript{th} moment & $$\frac{66m^6 + 1890m^4 + 9084m^2 + 3360}{8m^6 + 24m^4 + 24m^2 + 8}$$ \\
%\hline
%Normalized 8\textsuperscript{th} moment & $$ \frac{498m^8 + 33236m^6 + 529634m^4 + 1759064m^2 + 499968}{16m^8 + 64m^6 + 96m^4 + 64m^2 + 16}$$ \\
%\hline
%\end{tabular}
%\end{table}

\begin{table}[h]
\centering
\caption{Lower Even Moments of $\{k\textup{-BCE}, k\textup{-BCE}\}$}
\renewcommand{\arraystretch}{3}
\begin{tabular}{|>{\centering\arraybackslash}m{1.5cm}|>{\centering\arraybackslash}m{10cm}|} 
\hline
Moment & Value \\
\hline
4\textsuperscript{th} & $\dfrac{10k^4+86k^2+48}{4k^4+8k^2+4}$ \\ 
\hline
6\textsuperscript{th} & $\dfrac{66k^6+1890k^4+9084k^2+3360}{8k^6+24k^4+24k^2+8}$ \\ 
\hline
8\textsuperscript{th} & $\dfrac{498k^8+33236k^6+529634k^4+1759064k^2+499968}{16k^8+64k^6+96k^4+64k^2+16}$ \\
\hline
\end{tabular}
\end{table}

\newpage

%\section{Intermediary Blip}

%By a similar argument to \ref{j,k-checkerboard trace}, we can show that the total contribution to $\mathbb{E}[\textup{Tr}\{A_N, B_N\}^\eta]$ of an $S$-class with $m_{1a}$ 1-blocks of $a$, $m_{1b}$ 1-blocks of $b$, $m_{2a}$ 2-blocks of $a$, $m_{2b}$ 2-blocks of $b$ is
%\begin{align*}
%\binom{\frac{\eta-m_1}{2}}{m_1}\binom{\frac{\eta-m_1}{2}}{m_{2a}}\binom{m_1}{m_{1a}}2^{m_{1a+1b}}(m_{1a})!!(m_{1b})!!\left(\frac{1}{k}\sqrt{1-\frac{1}{j}}\right)^{\eta-(m_{1a}+2m_{2a})}\left(\frac{1}{j}\sqrt{1-\frac{1}{k}}\right)^{m_{1a}+2m_{2a}}N^{\frac{3}{2}\eta-\frac{1}{2}(m_{1a}+m_{1b})}
%\end{align*}

%Let $w_1=\frac{1}{k}\sqrt{1-\frac{1}{j}}$ and $w_2=\frac{1}{j}\sqrt{1-\frac{1}{k}}$. 

%%%%%%%%%%%%%%%%%%%%%%%%%%%%%%%%%%%%%%%%%%%%%%%%%%%%%%%%%%
%%%%%%%%%%%%%%%%%%%%%%%%%%%%%%%%%%%%%%%%%%%%%%%%%%%%%%%%%%%%%%%%%%%%%%%%%%%%%%%%%%%%%%%%%%%%%%%%%%%%%%%%%%%%%%%%%%%%%%%%%%%%%%
%%%%%%%%%%%%%%%%%%%%%%%%%%%%%%%%%%%%%%%%%%%%%%%%%%%%%%%%%%%%%%%%%%%%%%%%%%%%%%%%%%%%%%%%%%%%%%%%%%%%%%%%%%%%%%%%%%%%%%%%%%%%%%
%%%%%%%%%%%%%%%%%%%%%%%%%%%%%%%%%%%%%%%%%%%%%%%%%%%%%%%%%%%%%%%%%%%%%%%%%%%%%%%%%%%%%%%%%%%%%%%%%%%%%%%%%%%%%%%%%%%%%%%%%%%%%%
\begin{thebibliography}{BCDHMSTPY} % '2nd argument contains the widest acronym'


\bibitem[NR]{commutator}
A. Nica, and S. Roland, Commutators of free random variables. (1998): 553-592.

\bibitem[BasBo1]{BasBo1}
A. Basak and A. Bose, Balanced random Toeplitz and Hankel matrices, Electronic Comm. in Prob. 15 (2010), 134–148.

\bibitem[BasBo2]{BasBo2}
A. Basak and A. Bose, Limiting spectral distribution of some band matrices, Periodica Mathematica Hun- garica 63 (2011), no. 1, 113–150.

\bibitem[BBDLMSWX]{disco}
K. Blackwell, N. Borde, C. Devlin VI, N. Luntzlara, R. Ma, S. J. Miller, M. Wang, and W. Xu, Distribution of Eigenvalues of Random Real Symmetric Block Matrices (2019), Submitted

\bibitem[BCDHMSTPY]{split}
P. Burkhardt, P. Cohen, J. DeWitt, M. Hlavacek, S.J. Miller, C. Sprunger, Y.N.T. Vu, R.V. Peski, and K. Yang, Random matrix ensembles with split limiting behavior. Random Matrices: Theory and Applications, 7.03 (2018): 1850006.

\bibitem[BLMST]{BLMST}
O. Beckwith, V. Luo, S. J. Miller, K. Shen, and N. Triantafillou, Distribution of eigenvalues of weighted, structured matrix ensembles, Integers: Electronic Journal Of Combinatorial Number Theory 15 (2015), paper A21, 28 pages.

\bibitem[BHS1]{BHS1}
A. Bose, R. S. Hazra, and K. Saha, Patterned random matrices and notions of independence, Technical report R3/2010 (2010), Stat-Math Unit, Kolkata.

\bibitem[BHS2]{BHS2}
A. Bose, R. S. Hazra, and K. Saha, Patterned random matrices and method of moments, in Proceedings of the International Congress of Mathematicians Hyderabad, India, 2010, 2203–2230. (Invited article). World Scientific, Singapore and Imperial College Press, UK.

\bibitem[Can]{Can}
 F.P. Cantelli, "Sulla probabilità come limite della frequenza", Atti Accad. Naz. Lincei 26:1 (1917) pp.39–45.

\bibitem[DFJKRSSW]{Swirl}
T. Dunn , H. Fleischmann, F. Jackson, S. Khunger, S. J. Miller, L. Reifenberg, A. Shashkov, and S. Willis, Limiting Spectral Distributions of Families of Block Matrix Ensembles, The PUMP Journal of Undergraduate Research (5 (2022), 122-147)

\bibitem[Fer]{Fer}
D. Ferge, Moment equalities for sums of random variables via integer partitions and Faà di Bruno’s formula, Turkish Journal of Mathematics 38 (2014), no. 3, 558–575; doi:10.3906/mat-1301-6.

\bibitem[FM]{FM}
F. W. K. Firk and S. J. Miller, Nuclei, Primes and the Random Matrix Connection, Symmetry 1 (2009), 64–105;

\bibitem[GKMN]{GKMN}
L. Goldmakher, C. Khoury, S. J. Miller and K. Ninsuwan, On the spectral distribution of large weighted random regular graphs, to appear in Random Matrices: Theory and Applications. http://arxiv. org/abs/1306.6714.

\bibitem[HJ]{Weyl}
R. Horn and C. Johnson, Matrix Analysis, Cambridge University Press, 1985

\bibitem[HM]{Toeplitz}
C. Hammond and S. J. Miller "Distribution of Eigenvalues of Real Symmetric Toeplitz Matrices" Journal of Theoretical Probability 18 (2005)

\bibitem[KKMSX]{Block Circulant}
G. Kopp et. al, Journal of Theoretical Probability  (26 (2013), no. 4, 1020--1060)

\bibitem[McK]{McK}
B. McKay, The expected eigenvalue distribution of a large regular graph, Linear Algebra Appl. 40 (1981), 203–216.

\bibitem[Me]{Me}
M. Meckes, The spectra of random abelian G-circulant matrices, ALEA Lat. Am. J. Probab. Math. Stat. 9 (2012) no. 2, 435–450.

\bibitem[MMS]{palindromicToeplitz}
A. Massey, S.J. Miller, and J. Sinsheimer. "Distribution of eigenvalues of real symmetric palindromic Toeplitz matrices and circulant matrices." Journal of Theoretical Probability 20 (2007): 637-662.

\bibitem[MS]{MS}
A. J. Mingo, and R. Speicher. \emph{Free probability and random matrices}. Vol. 35. New York: Springer, 2017.

\bibitem[Ta]{Ta}
L. Takacs, A Moment Convergence Theorem, The Amer. Math. Monthly 98 (Oct 1991), no. 8, 742-746

\bibitem[Tao2]{Tao2}
T. Tao, An Introduction to Measure Theory. Graduate Studies in Mathematics. Vol. 126 Providence: American Mathematical Society. p.195. ISBN 978-0-8218-6919-2.

\bibitem[Tao1]{Tao1}
T. Tao, 254a, notes 4: The semi-circular law, https://terrytao.wordpress.com/2010/02/ 02/254a-notes-4-the-semi-circular-law/, Posted:2010-02-02, Accessed:2016-08-04.

\bibitem[Vas]{Vas}
V. Vasilchuk, "On the asymptotic distribution of the commutator and anticommutator of random matrices." Journal of Mathematical Physics 44.4 (2003): 1882-1908.


\bibitem[Wig1]{Wigner1}
E. Wigner, On the statistical distribution of the widths and spacings of nuclear resonance levels, Proc. Cambridge Philo. Soc. 47 (1951), 790–798.

\bibitem[Wig2]{Wigner2}
E. Wigner, Statistical Properties of real symmetric matrices. Pages 174–184 in Canadian Mathematical Congress Proceedings, University of Toronto Press, Toronto, 1957.

\bibitem[Wis]{Wishart}
J. Wishart, The generalized product moment distribution in samples from a normal multivariate population, Biometrika 20 A (1928), 32–52.
\end{thebibliography}



\end{document}
