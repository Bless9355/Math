\documentclass{article}
\usepackage{amsfonts}
\usepackage{amsthm}
\usepackage{amssymb}
\usepackage{amsmath}
\usepackage{graphicx}
\usepackage{subcaption}
\usepackage{xcolor}
\usepackage{mathtools}
\usepackage{ wasysym }
\usepackage{enumerate}
\usepackage{verbatim}


\newcommand{\new}[2]{
    \vspace{2mm}
    \noindent
    \textbf{
    \underline{#1}}
    \textit{{#2}}
    \
    \newline
}

\def\<{{\langle}}
\def\>{{\rangle}}

\DeclarePairedDelimiter\bra{\langle}{\rvert}
\DeclarePairedDelimiter\ket{\lvert}{\rangle}
\DeclarePairedDelimiterX\braket[2]{\langle}{\rangle}{#1\,\delimsize\vert\,\mathopen{}#2}


\newcommand{\textOr}{
    {
        \hspace{5mm}
        \textrm{or}
        \hspace{5mm}
    }
}

\newcommand{\textAnd}{
    {
        \hspace{5mm}
        \textrm{and}
        \hspace{5mm}
    }
}


\newcommand{\textWhere}{
    {
        \hspace{5mm}
        \textrm{where}
        \hspace{5mm}
    }
}



\newcommand{\Ixp}[1]{
    {
        e^{i{#1}}
    }
}



\newcommand{\halfFigure}[1]{
\begin{center}
\includegraphics[width = .5\linewidth]{{#1}}
\end{center}
}

\newcommand{\fullFigure}[1]{
\begin{center}
\includegraphics[width = .9\linewidth]{{#1}}
\end{center}
}

\def\twobytwoMat(#1, #2, #3, #4){
    {
        \begin{bmatrix}
            {#1} & {#2}\\
            {#3} & {#4}
        \end{bmatrix}
    }
}

\def\twobyoneMat(#1, #2){
    {
        \begin{bmatrix}
            {#1}\\
            {#2}
        \end{bmatrix}
    }
}

\def\twobytwoDet(#1, #2, #3, #4){
    {
        \begin{vmatrix}
            {#1} & {#2}\\
            {#3} & {#4}
        \end{vmatrix}
    }
}


\newcommand{\RR}{\mathbb{R}}
\newcommand{\CC}{\mathbb{C}}
\newcommand{\ZZ}{\mathbb{Z}}
\newcommand{\Zpos}{\mathbb{Z}_{pos}}
\newcommand{\NN}{\mathbb{N}}

\newcommand{\Tr}{\textrm{Tr}}


\newcommand{\Eta}{H}

\newtheorem{theorem}{Theorem}
\newtheorem{prop}{Proposition}
\newtheorem{lemma}{Lemma}
\newtheorem{cor}{Corollary}
\newtheorem{remark}{Remark}
\newtheorem{definition}{Definition}
\newtheorem{ex}{Example}
\newtheorem{conj}{Conjecture}
\newtheorem{question}{Question}

\newcommand{\ch}{\text{ch}}

\begin{document}
\begin{center}
    \Large
    \textbf{Algebraically Linked GOEs}

    \large
    Benevolent Tomato
\end{center}

\section{Statement of the Result} 
\begin{definition}[Algebraic Link]
    Let $X, Z$ be independant GOEs of dimension $N$-by-$N$. We define 
    an algebraically linked GOE with link variable $l \in [0, 1]$ 
    as follows. 
    \[
        Y = l X + (1-l) Z
    \]
\end{definition}

\begin{definition}[Spectral Density of the Anticommutator]
    Let $X, Y$ be two GOEs that are not necessarily independant. 
    Let $\Lambda$ denote the set of all eigenvalues of the matrix 
    $XY + YX$. The spectral density of the Anticommutator 
    is defined as the following. 
    \[
    \mu(x) \ := \ 
    \lim_{N \rightarrow \infty} \mathbb{E}\left[
    \frac 1 N \sum_{\lambda \in \Lambda} 
    \delta\left(
        x - \frac \lambda N
    \right)
    \right]
    \]
    
\end{definition}

It is well known that we can compute the $(n)$th moment 
by using the trace. 

\begin{theorem}[Normalized Spectral Density of the Anticommutator]
    Denote the $k$th moment of this probability 
    distribution as $\mu^{(k)}$. Then, 
    \[
    \mu^{(n)} \ = \ \Tr \left(\left[\frac 1 {\sqrt N}(XY+YX)\right]^n\right)
    \].
\end{theorem}

Our result provides a formula for $\mu^{n}$ that involves 
a constant defined via a reasonable recursion. 

\begin{theorem}[Moments of the anticommutator of two algebraically linked GOEs]
    \[ \mu^{(n)} \ = \ 
    \sum_{k = 0}^n \nu_{n, k} (2l)^{k} (1 - l)^{n - k} 
    \] 
\end{theorem}

Here are the moments up to $n = 7$. 

\begin{align*}
\mu^{(1)} &= 2 l \\
\mu^{(2)} &= 2 - 4 l + 10 l^2 \\
\mu^{(3)} &= 18 l - 36 l^2 + 58 l^3 \\
\mu^{(4)} &= 10 - 40 l + 204 l^2 - 328 l^3 + 378 l^4 \\
\mu^{(5)} &= 170 l - 680 l^2 + 2140 l^3 - 2920 l^4 + 2634 l^5 \\
\mu^{(6)} &= 66 - 396 l + 3054 l^2 - 9576 l^3 + 22014 l^4 - 25932 l^5 + 19218 l^6 \\
\mu^{(7)} &= 1666 l - 9996 l^2 + 46830 l^3 - 120680 l^4 + 222558 l^5 - 230412 l^6 + 144946 l^7
\end{align*}


\section{Combinatorial Preliminaries}

\begin{definition}[Special Words]
A \textit{special word} of length \(2k\) is composed of \(k\) blocks, where each block is one of \(\{XX, ZX, XZ\}\). The characteristic of a special word \(w\), denoted by \(\chi(w)\), is the number of blocks \(XX\) used in the word.

For example, when \(k = 3\),
\[
XX \, ZX \, XZ
\]
is an example of a special word of length 6 with characteristic 1.
\end{definition}

\begin{definition}[Set of Special Words]
\(\Eta_{n, k}\) is defined as the set of all special words of length \(2n\) with characteristic \(k\).

For example, if \(n = 2\) and \(k = 1\), then
\[
\Eta_{2, 1} = \{\text{XX ZX}, \, \text{XX XZ}, \, \text{ZX XX}, \, \text{XZ XX}\}.
\]
\end{definition}

\begin{definition}[Valid Pairings]
A \textit{valid pairing} is a partition of the indices of the word into pairs such that each pair contains the same type of letter.

For example, for the word \(XXZZ\), a valid pairing is \(\{\{1, 2\}, \{3, 4\}\}\).
\end{definition}

\begin{definition}[Non-Crossing Pairings]
A \textit{non-crossing pairing} is a valid pairing where for any two pairs \(\{i, k\}\) and \(\{j, l\}\), it is not the case that \(i < j < k < l\).

For example, for the word \(XXZZ\), the pairing \(\{\{1, 2\}, \{3, 4\}\}\) is non-crossing, while the pairing \(\{\{1, 3\}, \{2, 4\}\}\) is crossing because \(1 < 2 < 3 < 4\).
\end{definition}

\begin{definition}[Pairing number of a property $n, k$]
    Call \(\nu_{n, k}\) to be the pairing number of the property $n, k$
    \footnote{Technically, would be accurate to say the the Pairing number 
    of the words with property $n, k$, but the word is clearly implied by the context}
\(\nu_{n, k}\) is defined as the number of valid, non-crossing pairings for all special words in \(\Eta_{n, k}\). To compute \(\nu_{n, k}\):

\begin{enumerate}[(i)]
    \item Consider all special words in \(\Eta_{n, k}\).
    \item For each word, count the number of valid, non-crossing pairings of the indices.
    \item Sum these counts for all words in \(\Eta_{n, k}\).
\end{enumerate}

Mathematically, \(\nu_{n, k}\) is given by:
\[
\nu_{n, k} = \sum_{w \in \Eta_{n, k}} \phi(w)
\]
Where $\phi(w)$ counts valid, non-crossing pairings of $w$. 
\newpage
For example, if \(n = 2\) and \(k = 0\):
\[
\Eta_{2, 0} = \{\text{ZX ZX}, \, \text{ZX XZ}, \, \text{XZ ZX}, \, \text{XZ XZ}\}
\]
\[
\begin{aligned}
&\text{For } \text{ZX ZX}, \text{ no valid non-crossing pairings.} \\
&\text{For } \text{ZX XZ}, \text{ valid non-crossing pairing: } \{\{1, 4\}, \{2, 3\}\}. \\
&\text{For } \text{XZ ZX}, \text{ valid non-crossing pairing: } \{\{1, 4\}, \{2, 3\}\}. \\
&\text{For } \text{XZ XZ}, \text{ no valid non-crossing pairings. } \{\{1, 3\}, \{2, 4\}\}. \\
\end{aligned}
\]
Therefore, \(\nu_{2, 0} = 1 + 1 = 2\).
\end{definition}


\section{Counting Valid Pairings by $\sigma$-recurrences}

In this section, we provide a method to compute Pairing numbers 
with a property $n, k$. We introduce an additional quantity to the property 
of the word, $s$, that denotes the number of $XX$ blocks at the beginning 
of the word. 

\begin{definition}
    Define the pairing number of the property $n, s, k$ as the following. 
    \[
\sigma_{n, s, k} = \sum_{w \in \Eta_{n, s, k}} \phi(w)
\]. 
$\Eta_{n, s, k}$ denotes the set of all words composed of $n$ blocks that have at least
$s$ $XX$ blocks in the beginning of the word, and $k$ blocks of $XY$, $YX$. 
\end{definition}

\begin{theorem}[\(\sigma_{n, s, k}\)]
The auxiliary sequence \(\sigma_{n, s, k}\) is defined with the following initial conditions. 
\begin{enumerate}
    \item \(\sigma_{n, s, k} = 0\) if \(s + k > n\)
    \item \(\sigma_{n, s, 0} = C_n\), where \(C_n\) is the \(n\)-th Catalan number, \(C_n = \frac{1}{n + 1} \binom{2n}{n}\)
    \item \(\sigma_{n, s, 2k + 1} = 0\)
    \item \(\sigma_{n, s, -k} = 0\)
\end{enumerate}

The recurrence relation for \(\sigma_{n, s, 2k}\) is given by:
\begin{equation}
\sigma_{n, s, 2k} = 
\sum_{p = s + 1}^{n} 
\sum_{q = p + 1}^{n}
\sum_{r = 0}^{2k}
\left[
\sigma_{n - q + p, p, r} \cdot \sigma_{q - p - 1, 0, 2k - 2 - r}
+ 
\sigma_{n - q + p - 1, p - 1, r} \cdot \sigma_{q - p, 1, 2k - 2 - r}
\right]
\end{equation}
\end{theorem}

\begin{proof}
    Initial conditions 1, 3, 4 trivially follows from the nature of 
    valid pairings. $s + k \leq n$ in any block. Also, if there are 
    $2k + 1$ blocks of the type $XY, YX$, the number of $Y$'s in the 
    word is odd, and hence there exists no valid pairing. Clearly, 
    the number of $XY, YX$ blocks cannot be negative. 

    Consider initial condition 2. If $k = 0$, then the word is entirely 
    composed of $XX$ blocks, so the number of non-crossing pairngs can 
    be easily counted by the Catalan numbers. This concludes the proof 
    for the four initial conditions. 

    We move on to prove the recurrence relation. Let $p$ be the 
    first occurence of any block that has a $Y$ and $q$ the block in which 
    the $Y$ in the $p$th block matches to. For example, if $(n, s, k) = (5, 1, 2)$, 
    here is an example word with the pairing with $p = 3, q = 5$. 

    \newcommand{\red}[1]{\color{red} {#1} \color{black}}

    \[
    W \ = \
    XX \, XX \, X\red{Y} \, XX \, \red{Y}X
    \]

    Note that the pairing between the $Y$ blocks divide into two types. 
    Type 1 pairing is $XY \, YX$ and Type 2 paring is $YX \, XY$ both in order. 
    Pairing the two $Y$'s split the word into two sub-words, the word 
    outside the $YY$ block and thw word between the $YY$ block. So for the previous example, 

    \[
        W_1 \ = \ XX \, XX \, XX \textAnd W_2 \ = \  XX
    \]
    where $W_1$ is outside the $Y$ pairing and $W_2$ is between the $Y$ pairings. 

    For pairing Type 1, the value of $s$ increases by 1 after the splitting for 
    the outer word. For pairing Type 2, the value of $s$ increases from zero to 1 
    after the split. The inner word and the outer word can be considered 
    independent. The important obsevation to deduce the latter fact is to observe 
    that the pairing number is equivalent for the following two blocks. 
    \[
    X \, [\textnormal{Some Blocks}] \, X
    \]
 \[
    XX \, [\textnormal{Some Blocks}]
    \]

    With these fact in mind, we count the contribution of 
    Type 1 and Type 2 matchings for fixed $p, q$. 
    For Type 1, the contribution is 
    \[
    \sigma_{n - q + p, p, r} \cdot \sigma_{q - p - 1, 0, 2k - 2 - r}
    \]
    For Type 2, the contribution is 
    \[
\sigma_{n - q + p - 1, p - 1, r} \cdot \sigma_{q - p, 1, 2k - 2 - r}
    \]
    This proves the recursive relation 
\[
\sigma_{n, s, 2k} = 
\sum_{p = s + 1}^{n} 
\sum_{q = p + 1}^{n}
\sum_{r = 0}^{2k}
\left[
\sigma_{n - q + p, p, r} \cdot \sigma_{q - p - 1, 0, 2k - 2 - r}
+ 
\sigma_{n - q + p - 1, p - 1, r} \cdot \sigma_{q - p, 1, 2k - 2 - r}
\right]
\]. 
\end{proof}

\begin{remark}[Relation to \(\nu_{n, k}\)]
\(\nu_{n, k}\) is related to \(\sigma_{n, s, k}\) by the following:
\[
\nu_{n, k} = \sigma_{n, 0, n - k}
\]
This means that to compute \(\nu_{n, k}\), we compute \(\sigma_{n, 0, n - k}\) 
using the above initial conditions and recurrence relation. 
\end{remark}

\begin{remark}[Case where $X$'s are allowed to cross]
    When computing the moments for GOE anticommutated 
    with Palindromic Topelitz, we can modify the initial condition as 
    \[
        \sigma_{n, s, 0} \ = \ (2n - 1)!!
    \]
    to obtain the Pairing Number appropriate for this case. 
\end{remark}
\end{document}
