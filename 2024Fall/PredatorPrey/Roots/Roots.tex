\documentclass{article}
\usepackage{amsfonts}
\usepackage{amsthm}
\usepackage{amssymb}
\usepackage{amsmath}
\usepackage{graphicx}
\usepackage{subcaption}
\usepackage{xcolor}
\usepackage{mathtools}
\usepackage{ wasysym }
\usepackage{enumerate}
\usepackage{verbatim}


\numberwithin{equation}{section}
\newcommand{\new}[2]{
    \vspace{2mm}
    \noindent
    \textbf{
    \underline{#1}}
    \textit{{#2}}
    \
}

\def\<{{\langle}}
\def\>{{\rangle}}

\DeclarePairedDelimiter\bra{\langle}{\rvert}
\DeclarePairedDelimiter\ket{\lvert}{\rangle}
\DeclarePairedDelimiterX\braket[2]{\langle}{\rangle}{#1\,\delimsize\vert\,\mathopen{}#2}


\newcommand{\textOr}{
    {
        \hspace{5mm}
        \textrm{or}
        \hspace{5mm}
    }
}

\newcommand{\textAnd}{
    {
        \hspace{5mm}
        \textrm{and}
        \hspace{5mm}
    }
}


\newcommand{\textWhere}{
    {
        \hspace{5mm}
        \textrm{where}
        \hspace{5mm}
    }
}



\newcommand{\Ixp}[1]{
    {
        e^{i{#1}}
    }
}



\newcommand{\halfFigure}[1]{
\begin{center}
\includegraphics[width = .5\linewidth]{{#1}}
\end{center}
}

\newcommand{\fullFigure}[1]{
\begin{center}
\includegraphics[width = .9\linewidth]{{#1}}
\end{center}
}

\def\twobytwoMat(#1, #2, #3, #4){
    {
        \begin{bmatrix}
            {#1} & {#2}\\
            {#3} & {#4}
        \end{bmatrix}
    }
}

\def\twobyoneMat(#1, #2){
    {
        \begin{bmatrix}
            {#1}\\
            {#2}
        \end{bmatrix}
    }
}

\def\twobytwoDet(#1, #2, #3, #4){
    {
        \begin{vmatrix}
            {#1} & {#2}\\
            {#3} & {#4}
        \end{vmatrix}
    }
}


\newcommand{\deriv}[2]{
\frac {d {#1} } {d {#2}}
}

\newcommand{\pderiv}[2]{
\frac {\partial {#1} } {\partial {#2}}
}


\newcommand{\RR}{\mathbb{R}}
\newcommand{\CC}{\mathbb{C}}
\newcommand{\ZZ}{\mathbb{Z}}
\newcommand{\Zpos}{\mathbb{Z}_{pos}}
\newcommand{\NN}{\mathbb{N}}

\newtheorem{theorem}{Theorem}
\newtheorem{proposition}{Proposition}
\newtheorem{lemma}{Lemma}
\newtheorem{corollary}{Corollary}
\newtheorem{remark}{Remark}
\newtheorem{definition}{Definition}
\newtheorem{example}{Example}
\newtheorem{conjecture}{Conjecture}
\newtheorem{question}{Question}

\newcommand{\ch}{\text{ch}}

\begin{document}
\begin{center}
    \Large
    \textbf{Title}

    \large
    Benevolent Tomato
\end{center}





\begin{theorem}
    The characteristic polynomial $\ch(z)$ always has a real root. 
\end{theorem}

\begin{theorem}
    If $f \ \geq \ 1/N$, then $\ch(z)$ has a unique positive, real 
    root that has a magnitude strictly greater than any of the other 
    complex roots. 
\end{theorem}

\begin{proof}
    Consider the polynomial 
    \begin{align}
        h(z) \ := \ (z - 1) \ch_N(z) \ = \ z^{N + 1} - (f+1) z^N + f
    \end{align}
    which has a simpler algebraic expression. 
    We split the 
    polynomial $h(x)$ into two summands, and invoke Rouche's Theorem (\cite{SS03} p91). 
    Let $C_{1 + \epsilon}$ be a circular contour centered at the origin 
    with radius $1 + \epsilon$ for arbitrarily small $\epsilon$. 
    Write 
    \begin{align}
        h(z) \ = \ (z^{N + 1} + f) + (1 + f)z^N 
    \end{align}
    and Taylor expand the two summands at $z = 1$. 
    \begin{align}
        (z^{N + 1} + f) &\ = \ 1 + f + (N + 1) \epsilon \\ 
        z^N(1 + f) &\ = \ (1 + N \epsilon)  (1 + f) \ = \ 1 + f + N(1 + f)\epsilon
    \end{align}
    By assumption, $f \ \geq \ 1/N$, which implies $(N + 1) \ \leq \ N(1 + f)$. 
    The modullus of the two terms along the contour can be compared 
    as follows. 
    \begin{align}
        \left|z^{N + 1} + f\right| \ \leq \ \left|(1 + f)z^N\right|
    \end{align}
    By Rouche's theorem, $h(z)$ has the same number of roots as the 
    term that has a larger modullus in the countour $C_{1 + \epsilon}$, which is the 
    summand $(1-f)z^N$. It is trivial to see that this summand has $N$ 
    roots inside the countour, and by fundamental theorem of algebra, 
    $h(z)$ has $N + 1$ roots. 

    We know that $\ch_N(z)$ is positive somewhere in the interval $[1, \infty)$. 
    We consider the following:
    \begin{equation}
        \ch_N(1) \ =\  1 - fN \ \leq \ 0.  
    \end{equation}
    By the Intermediate Value Theorem, we conclude that the one root outside 
    the unit circle is a positive real value. 
\end{proof}

\begin{theorem}
    If $f \ < \ 1/N$, then all the roots of $\ch_N(z)$ have 
    a modullus strictly less than $1$. 
\end{theorem}

\begin{proof}
    It suffices to show that 
    \begin{equation}
        \widetilde h(z) \ = \ h (1/z) z^{N + 1} \ = \ fz^{N + 1} - (f+1)z+1
    \end{equation} has exactly one root within the unit circle which 
    comes from multiplying \newline $(z - 1)$. Again, consider the countour 
    $C_{1+\epsilon}$ and split $\widetilde h (z)$ into two summands. 
    \begin{align}
        \widetilde h(z) \ = \ (f z^{N + 1} + 1) - (f + 1)z
    \end{align}
    Taylor expand the two summands at $z \ = \ 1$, and notice that under the condition 
    $f \ < \ 1/N$, the second summand has a larger modullus along the 
    contour $C_{1 + \epsilon}$. 
    \begin{align}
        fz^{N + 1} + 1 & \ = \ f(1 + (N + 1)\epsilon) + 1  \\ 
        (f + 1)z & \ = \ (f + 1) (1 + \epsilon)
    \end{align} 
    Clearly, the second summand has one root inside the contour $C_{1 + \epsilon}$, 
    which originates from $(z - 1)$. By Rouche's theorem, $\widetilde h(z)$ 
    has exactly one root inside the unit circle, i.e. $z = 1$, and all 
    other roots have a modullus greater than 1. Consequently, $\ch_N(z) \ = \ h(z)/(z + 1)$ 
    has all of its roots strictly inside the unit circle.  
\end{proof}

\begin{theorem}[Bounds for the dominant eigenvalue]\label{thm:Bound}

 Given that $f \ \geq \ 1/N$, the dominant eigenvalue of $L_f$ of order 
 $N$ is given by 
 \begin{equation}
 1 + f - \frac 1 {N} \ \leq\ \lambda_{\max} \ < \ 1 + f.
 \end{equation}
\end{theorem}

\begin{proof}
    The upper bound is trivial:
    \begin{equation}
        \ch_N(1+f) \ = \ f \ >\  0.
    \end{equation}
    We have $\ch_N(0) \ = \  -f \ < \  0$, and thus by the Intermediate Value Theorem the maximum root is 
    bounded. 

    To obtain the lower bound, we write $f \ =  \ 1/N + \epsilon$ for 
    some $\epsilon \ \geq \  0$. With some algebra listed below
    % \textcolor{red}{\textit{(Miller commented to include this algebra in the appendix; we should do this and cite/refer to it here?)}}
    , we compute $\ch_N(z)$ at the claimed lower 
    bound. If we show that this value is less than zero, the dominating root must be greater than the purported lower bound. We find
    \begin{equation}
        \ch_N\left(
            1 + f - \frac 1 N
        \right)  \ = \ -
        \left(
            1 + f - \frac 1 N
        \right)^N \left[
            \frac {1} {fN - 1}
        \right]
        + \frac {fN} {fN - 1}.
    \end{equation}
    We wish to bound this value by zero. It suffices to show 
    \begin{equation}
        fN - \left(
            1 + f - \frac 1 N
        \right)^N \ \leq\ 0,
    \end{equation}
    which, using the $\epsilon$ substitution, converts to
    \begin{equation}
        1 + N\epsilon - (1 + \epsilon)^N \ \geq \ 0.
    \end{equation}
    Expanding the power term by the binomial theorem, we see that inequality indeed holds. 
\end{proof}


\end{document}