\documentclass[12pt]{article}
\usepackage{amsmath}
\usepackage{amssymb}

\title{Summary of the Hodgkin-Huxley Model Discussion}
\author{}
\date{}

\begin{document}

\maketitle

\section*{Introduction}
The Hodgkin-Huxley model describes the dynamics of ion channels in excitable cells. It uses voltage-dependent rate constants to model transitions between open and closed states of ion channel gates. The model introduces gating variables \(n\), \(m\), and \(h\) to capture the probabilities of activation and inactivation of these channels.

\section*{Key Components}
\begin{itemize}
    \item \(n\): The probability of a potassium activation gate being open. Potassium conductance is given by:
    \[
    g_K = \bar{g}_K n^4,
    \]
    where \(\bar{g}_K\) is the maximal potassium conductance, and \(n^4\) represents the probability of all four \(n\)-gates being open.
    
    \item \(m\): The probability of a sodium activation gate being open. Sodium conductance is given by:
    \[
    g_{Na} = \bar{g}_{Na} m^3 h,
    \]
    where \(m^3\) represents the probability of all three \(m\)-gates being open.
    
    \item \(h\): The probability of the sodium inactivation gate being open (not blocking the channel). It modulates sodium conductance by controlling inactivation.

\end{itemize}

\section*{Gating Dynamics}
The gating variables \(n\), \(m\), and \(h\) evolve according to:
\[
\frac{dx}{dt} = \alpha_x(V)(1 - x) - \beta_x(V)x, \quad x \in \{n, m, h\},
\]
where \(\alpha_x(V)\) and \(\beta_x(V)\) are voltage-dependent rate constants governing the opening and closing transitions.

\section*{Sodium Inactivation Gate (\(h\))}
\begin{itemize}
    \item At rest (\(V\) near resting potential): \(h\) is close to 1, meaning the sodium channel is ready to conduct ions.
    \item During depolarization: \(h\) decreases, inactivating the sodium channel even if the activation gates (\(m\)) remain open.
    \item During repolarization: \(h\) increases back toward 1, resetting the channel.
\end{itemize}

The inactivation gate is essential for:
\begin{itemize}
    \item Terminating sodium current during depolarization.
    \item Enforcing a refractory period to prevent backward signal propagation.
\end{itemize}

\section*{Probabilistic Nature}
The Hodgkin-Huxley model uses deterministic differential equations to describe the time evolution of the gating probabilities. However, these probabilities reflect the average behavior of a large population of channels, capturing the probabilistic transitions between states at the molecular level.

\end{document}
