\documentclass{article}
\usepackage{amsfonts}
\usepackage{amsthm}
\usepackage{amssymb}
\usepackage{amsmath}
\usepackage{graphicx}
\usepackage{subcaption}
\usepackage{xcolor}
\usepackage{mathtools}
\usepackage{ wasysym }
\usepackage{enumerate}
\usepackage{verbatim}
\usepackage{hyperref}


\numberwithin{equation}{section}
\newcommand{\new}[2]{
    \vspace{2mm}
    \noindent
    \textbf{
    \underline{#1}}
    \textit{{#2}}
    \
}

\def\<{{\langle}}
\def\>{{\rangle}}

\DeclarePairedDelimiter\bra{\langle}{\rvert}
\DeclarePairedDelimiter\ket{\lvert}{\rangle}
\DeclarePairedDelimiterX\braket[2]{\langle}{\rangle}{#1\,\delimsize\vert\,\mathopen{}#2}


\newcommand{\textOr}{
    {
        \hspace{5mm}
        \textrm{or}
        \hspace{5mm}
    }
}

\newcommand{\textAnd}{
    {
        \hspace{5mm}
        \textrm{and}
        \hspace{5mm}
    }
}


\newcommand{\textWhere}{
    {
        \hspace{5mm}
        \textrm{where}
        \hspace{5mm}
    }
}



\newcommand{\Ixp}[1]{
    {
        e^{i{#1}}
    }
}



\newcommand{\halfFigure}[1]{
\begin{center}
\includegraphics[width = .5\linewidth]{{#1}}
\end{center}
}

\newcommand{\fullFigure}[1]{
\begin{center}
\includegraphics[width = .9\linewidth]{{#1}}
\end{center}
}

\def\twobytwoMat(#1, #2, #3, #4){
    {
        \begin{bmatrix}
            {#1} & {#2}\\
            {#3} & {#4}
        \end{bmatrix}
    }
}

\def\twobyoneMat(#1, #2){
    {
        \begin{bmatrix}
            {#1}\\
            {#2}
        \end{bmatrix}
    }
}

\def\twobytwoDet(#1, #2, #3, #4){
    {
        \begin{vmatrix}
            {#1} & {#2}\\
            {#3} & {#4}
        \end{vmatrix}
    }
}


\newcommand{\deriv}[2]{
\frac {d {#1} } {d {#2}}
}

\newcommand{\pderiv}[2]{
\frac {\partial {#1} } {\partial {#2}}
}


\newcommand{\RR}{\mathbb{R}}
\newcommand{\CC}{\mathbb{C}}
\newcommand{\ZZ}{\mathbb{Z}}
\newcommand{\Zpos}{\mathbb{Z}_{pos}}
\newcommand{\NN}{\mathbb{N}}

\newtheorem{theorem}{Theorem}
\newtheorem{proposition}{Proposition}
\newtheorem{lemma}{Lemma}
\newtheorem{corollary}{Corollary}
\newtheorem{remark}{Remark}
\newtheorem{definition}{Definition}
\newtheorem{example}{Example}
\newtheorem{conjecture}{Conjecture}
\newtheorem{question}{Question}

\newcommand{\ch}{\text{ch}}

\begin{document}
\begin{center}
    \Large
    \textbf{MATH 374 Final Project Outline}

    \large
    Daniel Son
\end{center}

The final project will be about the topic of \textbf{Fixed Point Theory and its applications to PDEs}. An important 
theorem of the theory is the Cauchy-Peano Theorem which guarantees a solutions for differential equation in the form of 
\begin{align}
    \dot y(t) \ = \ f(y, t) \textAnd y(t_0) = y_0
\end{align}
for minimal conditions over the function $f$. Upon a precursory research, 
it is possible to solve physical problems such as the obstacle problem 
or the bending rod problem using fixed point methods. I intend to give 
a vague justification of the theorem and apply the theorem to demonstrate 
the Euler Lagrange equations. Finally, using the developed methods, 
I will solve the two toy models presented above. 

\section*{References}

\begin{enumerate}
    \item University of Oxford. \emph{Fixed Point Theory Course Materials}. Available at: \url{https://courses.maths.ox.ac.uk/course/view.php?id=170}. Accessed November 24, 2024.
    \item Agarwal, R. P., Meehan, M., \& O’Regan, D. (2001). \emph{Contractions}. In \emph{Fixed Point Theory and Applications} (pp. 1–11). Cambridge Tracts in Mathematics. Cambridge University Press.
    \item Albert, J. (2019). \emph{Physical Applications of Fixed Point Methods in Differential Equations}. University of Chicago. Available at: \url{https://math.uchicago.edu/~may/REU2019/REUPapers/Albert.pdf}. Accessed November 24, 2024.
\end{enumerate}


\end{document}