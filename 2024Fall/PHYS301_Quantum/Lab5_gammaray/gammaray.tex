\documentclass{article}
\usepackage{amsfonts}
\usepackage{amsthm}
\usepackage{amssymb}
\usepackage{amsmath}
\usepackage{graphicx}
\usepackage{subcaption}
\usepackage{xcolor}
\usepackage{mathtools}
\usepackage{ wasysym }
\usepackage{enumerate}
\usepackage{verbatim}
\usepackage{siunitx}
\usepackage{tikz}






\numberwithin{equation}{section}
\newcommand{\new}[2]{
    \vspace{2mm}
    \noindent
    \textbf{
    \underline{#1}}
    \textit{{#2}}
    \
}

\def\<{{\langle}}
\def\>{{\rangle}}

\DeclarePairedDelimiter\bra{\langle}{\rvert}
\DeclarePairedDelimiter\ket{\lvert}{\rangle}
\DeclarePairedDelimiterX\braket[2]{\langle}{\rangle}{#1\,\delimsize\vert\,\mathopen{}#2}


\newcommand{\textOr}{
    {
        \hspace{5mm}
        \textrm{or}
        \hspace{5mm}
    }
}

\newcommand{\textAnd}{
    {
        \hspace{5mm}
        \textrm{and}
        \hspace{5mm}
    }
}


\newcommand{\textWhere}{
    {
        \hspace{5mm}
        \textrm{where}
        \hspace{5mm}
    }
}



\newcommand{\Ixp}[1]{
    {
        e^{i{#1}}
    }
}



\newcommand{\halfFigure}[1]{
\begin{center}
\includegraphics[width = .5\linewidth]{{#1}}
\end{center}
}

\newcommand{\fullFigure}[1]{
\begin{center}
\includegraphics[width = .9\linewidth]{{#1}}
\end{center}
}

\def\twobytwoMat(#1, #2, #3, #4){
    {
        \begin{bmatrix}
            {#1} & {#2}\\
            {#3} & {#4}
        \end{bmatrix}
    }
}

\def\twobyoneMat(#1, #2){
    {
        \begin{bmatrix}
            {#1}\\
            {#2}
        \end{bmatrix}
    }
}

\def\twobytwoDet(#1, #2, #3, #4){
    {
        \begin{vmatrix}
            {#1} & {#2}\\
            {#3} & {#4}
        \end{vmatrix}
    }
}


\newcommand{\deriv}[2]{
\frac {d {#1} } {d {#2}}
}

\newcommand{\pderiv}[2]{
\frac {\partial {#1} } {\partial {#2}}
}


\newcommand{\RR}{\mathbb{R}}
\newcommand{\CC}{\mathbb{C}}
\newcommand{\ZZ}{\mathbb{Z}}
\newcommand{\Zpos}{\mathbb{Z}_{pos}}
\newcommand{\NN}{\mathbb{N}}

\newtheorem{theorem}{Theorem}
\newtheorem{proposition}{Proposition}
\newtheorem{lemma}{Lemma}
\newtheorem{corollary}{Corollary}
\newtheorem{remark}{Remark}
\newtheorem{definition}{Definition}
\newtheorem{example}{Example}
\newtheorem{conjecture}{Conjecture}
\newtheorem{question}{Question}

\newcommand{\ch}{\text{ch}}

\begin{document}
\begin{center}
    \Large
    \textbf{Gamma Ray Emission}

    \large
    Daniel Son
\end{center}


\section{Pair Anhilation Data Analysis}
The data was collected by the MCA overnight, and Professor Doret 
handed over the .spe files containing the data. For each count, 
the corresponding voltage was retrieved by the calibration data 
and the following formula. 
\[
\#1 + \#2 \times (\text{Bin\#} - 1) + \#3 \times (\text{Bin\#} - 1)^2
\]
The calibration constants \(\#1\), \(\#2\), and \(\#3\) are given as follows:
\[
\#1 = 4.942828 \times 10^{1}, \quad \#2 = 2.629989 \times 10^{0}, \quad \#3 = 5.826260 \times 10^{-4}
\]

%Insert data here

If all the coincidece detection indeed comes from pair anhilation of
electron and positrons, the 
emission spectrum must be centered at $511\rm keV$, which is the rest 
energy of the electron. Our observations indicate a peak at $511\rm keV$, 
and we validate that the detection comes from pair anhilation.  



\section{Compton Scattering Analysis}

\subsection{Extacting the Linear Relation}

We wish to use the collected data to confirm Compton scattering, i.e.,
\[
\lambda' - \lambda = \frac{h}{m_e c} (1 - \cos \theta)
\]
where \( f_i = 662 \, \text{keV} \).

We can deduce 
the frequency of the incident and exiting photon by the equation 
$E = hf$. Furthermore,  
we can deduce \( \lambda \) and \( \lambda' \) from this information.

\[
c = \lambda f \Rightarrow \lambda_i = \frac{c}{f_i} \quad \text{and} \quad \lambda_o = \frac{c}{f_o}
\]

Therefore,

\[
\Delta \lambda = \left( \frac{1}{f_i} - \frac{1}{f_o} \right) = \frac{h}{m_e c} (1 - \cos \theta)
\]

Let \( E_e = m_e c^2 \) be the rest energy of the electron.

\[
\left( \frac{1}{h f_i} - \frac{1}{h f_o} \right) = \frac{1}{E_e} (1 - \cos \theta)
\textOr
\left( \frac{1}{E_i} - \frac{1}{E_o} \right) = \frac{1}{E_e} (1 - \cos \theta)
\]

From the collected data, we can compute the two quantities 
inside the parantheses for a linear fit. Also, there is an uncertainty in the measured frequency \( f_o \). To do a linear regression analysis, we must convert the uncertainty of \( f_o \) to that of \( \frac{1}{f_i} - \frac{1}{f_o} \).

The uncertainty of a compound variable is computed by the weighted RMS of the uncertainties. In our case,

\[
\Delta \left( \frac{1}{E_i} - \frac{1}{E_o} \right) = \sqrt{\left( \Delta E_o \right)^2 \left( \frac{1}{E_o^2} \right)^2} = \frac{\Delta E_o}{E_o^2}
\]

\subsection{Slope Prediction}

The slope of the plot $1/E_i - 1/E_0$ verses $1 - \cos(\theta)$ is expected to be 

\[
\frac{1/E_i - 1/E_o}{1 - \cos \theta} \ = \ \frac 1 {m_e c^2} \ = \ \frac 1 {511keV} \ = \ = \boxed{1.96 \cdot 10^{-3} \rm keV^{-1}}
\]

\subsection{Data Analysis}
The observed slope of the fit is 
$-2.14 \cdot 10^-3 \rm kev^{-1}$, which is close to the theoretical slope. 
Moreover, the $p$-value is in the order of $-7$, which indicates that 
the relation is likely linear. 

\end{document}
