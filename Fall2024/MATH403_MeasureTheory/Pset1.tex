\documentclass{article}
\usepackage{amsfonts}
\usepackage{amsthm}
\usepackage{amssymb}
\usepackage{amsmath}
\usepackage{graphicx}
\usepackage{subcaption}
\usepackage{xcolor}
\usepackage{mathtools}
\usepackage{ wasysym }
\usepackage{enumerate}
\usepackage{verbatim}


\newcommand{\new}[2]{
    \vspace{2mm}
    \noindent
    \textbf{
    \underline{#1}}
    \textit{{#2}}
    \
}

\def\<{{\langle}}
\def\>{{\rangle}}

\DeclarePairedDelimiter\bra{\langle}{\rvert}
\DeclarePairedDelimiter\ket{\lvert}{\rangle}
\DeclarePairedDelimiterX\braket[2]{\langle}{\rangle}{#1\,\delimsize\vert\,\mathopen{}#2}


\newcommand{\textOr}{
    {
        \hspace{5mm}
        \textrm{or}
        \hspace{5mm}
    }
}

\newcommand{\textAnd}{
    {
        \hspace{5mm}
        \textrm{and}
        \hspace{5mm}
    }
}


\newcommand{\textWhere}{
    {
        \hspace{5mm}
        \textrm{where}
        \hspace{5mm}
    }
}



\newcommand{\Ixp}[1]{
    {
        e^{i{#1}}
    }
}



\newcommand{\halfFigure}[1]{
\begin{center}
\includegraphics[width = .5\linewidth]{{#1}}
\end{center}
}

\newcommand{\fullFigure}[1]{
\begin{center}
\includegraphics[width = .9\linewidth]{{#1}}
\end{center}
}

\def\twobytwoMat(#1, #2, #3, #4){
    {
        \begin{bmatrix}
            {#1} & {#2}\\
            {#3} & {#4}
        \end{bmatrix}
    }
}

\def\twobyoneMat(#1, #2){
    {
        \begin{bmatrix}
            {#1}\\
            {#2}
        \end{bmatrix}
    }
}

\def\twobytwoDet(#1, #2, #3, #4){
    {
        \begin{vmatrix}
            {#1} & {#2}\\
            {#3} & {#4}
        \end{vmatrix}
    }
}


\newcommand{\RR}{\mathbb{R}}
\newcommand{\CC}{\mathbb{C}}
\newcommand{\ZZ}{\mathbb{Z}}
\newcommand{\Zpos}{\mathbb{Z}_{pos}}
\newcommand{\NN}{\mathbb{N}}

\newtheorem{theorem}{Theorem}
\newtheorem{prop}{Proposition}
\newtheorem{lemma}{Lemma}
\newtheorem{cor}{Corollary}
\newtheorem{remark}{Remark}
\newtheorem{definition}{Definition}
\newtheorem{ex}{Example}
\newtheorem{conj}{Conjecture}
\newtheorem{question}{Question}

\newcommand{\ch}{\text{ch}}

\begin{document}
\begin{center}
    \Large
    \textbf{MATH 403 Pset 1}

    \large
    Daniel Son
\end{center}

\new{Problem} {Section 2.1 Q2}
\begin{enumerate}
    \item[a)] Show that the intervals in the definition of outer measure may be assumed to be closed.
    \item[b)] Show that for any $\delta$, the intervals in the definition of outer measure may be assumed to be open and of length less than $\delta$.
    \item[c)] Show that if $A$ is contained in an interval $K$, then we can assume that all intervals $I_i$ in the cover are contained in $K$.
\end{enumerate}

\begin{proof}[part a)]
Recall the definition of the outer measure. 
\begin{equation}
    \lambda^*(A) \ := \ 
    \inf \bigg\{
        \sum_{s \in S} |s| :
        A \subseteq \bigcup_{s \in S} s, 
        \textnormal{
            S is a collection of intervals
        }
        \bigg\}
\end{equation}
We define a custom outer measure where all intervals are closed. 
\begin{equation}
    \alpha(A) \ := \ 
    \inf \bigg\{
        \sum_{s \in S} |s| :
        A \subseteq \bigcup_{s \in S} s, 
        \textnormal{
            S is a collection of closed intervals
        }
        \bigg\}
\end{equation}
Clealy, $\alpha(A)$ takes the infimum of a more restricted set compared to the regular outer measure. Hence, $\lambda^*(A) \leq \alpha(A)$. It suffices to show $\lambda^*(A) \geq \alpha(A)$ to show that the two measures are indeed equivalent. 

Any fully open or half open interval has a corresponding closed interval that has the same length but includes the original interval. Consider the following. 
\begin{equation}
    (a, b) \subseteq (a, b] \subseteq [a, b] \textAnd
    (a, b) \subseteq [a, b) \subseteq [a, b] 
\end{equation}

Consider a collection of a interval $S$, either closed or open, that is included in the infimum for the outer measure $\lambda^*(A)$. By substituting all the intervals in $S$ with the fully-closed interval, we obtain a collection $\widetilde S$ where all intervals are closed, the sum of the length is preserved, and includes $A$. 
In symbols, that is 
\begin{eqnarray}
    A \ \subseteq \ \bigcup_{s \in \widetilde S} s \textAnd
    \sum_{s \in S} |s| \ = \ \sum_{s \in \widetilde S}|s|.
\end{eqnarray}
The existance of the collection $\widetilde S$ shows that  $\lambda^*(A) \geq \alpha(A)$ which concludes the proof 
\end{proof}


\begin{proof} [part b)]
   We start with a simple observation that we can split any interval $s = (a, b)$ into a collection of intervals $\dot s$ that have a length strictly less than $\delta$ and $\bigcup \dot s = s$. Let $k$ be the greatest integer that satisfies 
   \begin{equation}
    a + k \frac \delta 2 \ < \ b
   \end{equation}
   where $k$'s uniqueness is guaranteed by the well-ordering theorem. Let our collection $\dot s$ be written as follows. 
   \begin{equation}
    \dot s \ := \ \bigg\{
    (a, a + \frac \delta 2), [a + \frac \delta 2, a + \frac \delta 2), \dots, [a + k \frac \delta 2, b)    
    \bigg\}
   \end{equation}
   Apparently $\bigcup \dot s = s$ and the number of collection in $\dot s$ is finite. Also, change the open/closedness of the first and the last interval depending on the open/closedness of the inital interval $s$. 

   Now, define the custom outer measure $\alpha(A)$ as in part a, where this time the length of each interval is restricted to be strictly less than $\delta$. Again, it suffices to show that $\alpha(A) \leq \lambda^*(A)$. For each collection $S$ included in the infimum computation of $\lambda^*(A)$, we define a new collection 
   \begin{equation}
    \dot S \ := \ \{\dot s : s \in S \}
   \end{equation}
   
   $\dot S$ sees witness to $\alpha(A) \leq \lambda^*(A)$
\end{proof}

\begin{proof}[part c)]
    We repeat the procedure in the previous parts. Let the custom measure $\alpha(A)$ to restrict the collections such that the union of the collections are contained in the interval $K$. For collection $S$ associated with $\lambda^*(A)$, we define 
    \begin{equation}
        \dot S \ := \ \{s \cap K: s \in S\}
    \end{equation}
    Since $A \subseteq K$ by assumption, $A \subseteq \bigcup \dot S = K \cap \bigcup S $ so $\alpha(A) \geq \lambda^*(A)$ which concludes the proof. 
\end{proof}

\new{Problem} {Section 2.1 Q7}

Show that the union of countably many null sets is a null set.

\begin{proof}
    For simplicity, we call $S$ the collection of pairwise disjoint 
    intervals that cover the set $A$ 
     \footnote{i.e. $A \subseteq \bigcup S$} as \textbf{coverings}
    and denote its length $|S|$ as the sum of the length of its component intervals. 

    A null set is a set with an outer measure zero. This means, that for any $\epsilon >0$, there exists some covering of the null set with a length shorter than $\epsilon$. 

    Now, we move on to the proof of the proposition. Let $\{A_n\}_{n \in \Zpos}$ be a collection of null sets. For each $A_n$, it is possible to obtain a covering of the set, namely $S_n$ such that $|S_n| < \frac \epsilon {2^n}$. From the collection of coverings 
    $\{S_n\}_{n \in \Zpos}$, we take the union $\bigcup_{n \in \Zpos} \bigcup_{s \in S_n} s$ and disassemble the resulting subset of $\RR$ into disjoint intervals. Call this new covering $S_\epsilon$. For example, if 
    \begin{equation}
        S_1 \ = \ \{(1, 1.5), (4, 6)\}, S_2 \ = \ \{(.5, 2)\}
    \end{equation}
    then the total union of the coverings are 
    \begin{equation}
        \bigcup_{n \in \Zpos} \bigcup_{s \in S_n} s \ = \ 
        (1,2) \cup (4, 6)
    \end{equation}
    and we obtain 
    \begin{equation}
        S_\epsilon \ =\ \{(1, 2), (4, 6)\}
    \end{equation}
    Clearly, the length of the covering $S_\epsilon$ is less than $\epsilon$. 
    \begin{equation}
        |S_\epsilon| \ \leq \ |S_1| + |S_2| + |S_3| + \cdots \ \leq \ 
        \epsilon \left(
            \frac 1 2 + \frac 1 4 + \frac 1 8 \dots 
        \right) \ = \ \epsilon
    \end{equation}
    The union of countable number of countable sets are countable, so there must be a contable number of intervals in the collection $S_\epsilon$
\end{proof}

\end{document}