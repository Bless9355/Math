\documentclass{article}
\usepackage{amsfonts}
\usepackage{amsthm}
\usepackage{amssymb}
\usepackage{wasysym}

\newcommand{\new}[1]{
    \vspace{2mm}
    \noindent
    \textbf{
    \underline{#1}}
}

\def\INT{{\mathcal{O}}}
\def\contradiction{{\lightning}}


\newcounter{problemcnt}
\setcounter{problemcnt}{0}

\newcommand{\Problem}{{
    \vspace{5mm}
    \stepcounter{problemcnt}
    \noindent
    \arabic{problemcnt}. 
}
}

\newcommand{\nProblem}[1]{{
    \noindent
    \setcounter{problemcnt}{{#1}}
    \arabic{problemcnt}. 
}
}

\newcommand{\Proof}{{
    \vspace{2mm}
    \noindent
    \textbf{
    \underline{Proof}}
}
}

\begin{document}
\new{Proposition} The following are equivalent:
\begin{enumerate}
    \item (Division Algorithm) For any $\alpha, \beta \in \INT_K$ 
    there exists a unique integer $q, \gamma$
    that satisfies $\alpha = q\beta + \gamma$
    \item (Approximation) For any $\theta \in K$,
    there exists a K-integer $\kappa \in \INT_K$
    that satisfies $|N(\theta - \kappa)| < 1$
\end{enumerate}

\Proof ($2 \Leftarrow 1$) The idea is to 
approximate the field element $\alpha / \beta$. 
Let $\theta$ be this fraction, and obtain 
$\kappa$ accordingly. By the precision of 
the approximation:

\[
    |N(\alpha/\beta - \kappa)| < 1
\]

With some manipulation:

\[
    |N(\alpha-\kappa\beta)/N(\beta)| < 1
\]

Which in turn, yields:

\[
    |N(\alpha - \kappa\beta)| < |N(\beta)|
\]

$(\alpha - \kappa\beta)$ is in $\INT_K$. Call it 
$\gamma$. By the definition of $\gamma$, 
we also have:

\[
    \alpha = \kappa\beta + \gamma
\]

as desired. 
\checkmark

($2 \Rightarrow 1$) The idea is to multiply by 
some integer c to guarantee $c\theta$ is in 
the ring of integers.  
\qed

\end{document}
