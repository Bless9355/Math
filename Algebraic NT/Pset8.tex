\documentclass{article}
\usepackage{amsfonts}
\usepackage{amsthm}
\usepackage{amssymb}

\newcommand{\new}[1]{
    \vspace{2mm}
    \noindent
    \textbf{
    \underline{#1}}
}

\def\calO{{\mathcal{O}}}


\newcounter{problemcnt}
\setcounter{problemcnt}{0}

\newcommand{\Problem}{{
    \vspace{5mm}
    \stepcounter{problemcnt}
    \noindent
    \arabic{problemcnt}. 
}
}

\newcommand{\nProblem}[1]{{
    \noindent
    \setcounter{problemcnt}{{#1}}
    \arabic{problemcnt}. 
}
}

\newcommand{\Proof}{{
    \vspace{2mm}
    \noindent
    \textbf{
    \underline{Proof}}
}
}

\begin{document}

\Problem
Let K be a number field,
and let $I$ be an ideal in $\calO_k$. 
If $\alpha \in I$, prove that 
$N(\alpha) \in I$. 


\Proof 

Recall the definition of $N(\alpha)$.

\[
    N(\alpha) = 
    \prod_{i = 1}^{n}{\sigma_i(\alpha)}
\]

where each $\sigma_i(\alpha)$ are the 
embedded images of $\alpha$ and 
$n$ is the degree of the extension. 
WLOG, we claim that $\sigma_1(\alpha) = 
\alpha$. If $n = 1$, then the norm of 
$\alpha$ equals to $\alpha$ and the 
theorem becomes trivial. Assume $n>1$. 

It suffices to show that the 
following product is an element of 
$\calO_k$:

\[
P := 
    \prod_{i = 2}^{n}{\sigma_i(\alpha)}
    \]

Notice:
\[
    N(\alpha) = \alpha \cdot P
\]

by strong closure of ideals, 
$\alpha \in I$ implies the product 
implies that the Norm is in $I$. Nonetheless, 
it must be shown that the product 
$P$ is in the ring.

Consider the field polynomial of $\alpha$. 
It must be a polynomial over $\mathbb{Z}$. 
Call it $f(t)$. Write:

\[
    f(t)/(t-\alpha) = 
    \prod_{i = 2}^{n}
    (t - \sigma_i(\alpha))
\]

By polynomial long division, we notice 
that the constant term of the LHS is 
some constant within $\calO_k$. By 
Viete's relation, that constant term 
is exactly the desired product P. 

\qed


\Problem
Show in $\mathbb{Z}[\sqrt{-5}]$ that 
$\sqrt{-5}|(a+b\sqrt{-5})$ iff $5|a$. 
Deduce that $\sqrt{-5}$ is prime in 
$\mathbb{Z}[\sqrt{-5}]$. Hence, conclude 
that the element 5 factors uniquely in this 
ring, even if the ring is not a UFD. 

\Proof
($\Leftarrow$) If $5|a$, there exists some 
element $\xi \in \mathbb{Z}[\sqrt{-5}]$
that satisfies $a = 5\xi$. Thus, 
$a = -\sqrt{-5}\cdot (\sqrt{-5}\xi)$. 

Write:
\[
    a + b\sqrt{-5} = 
    \sqrt{-5}
    (-\xi\sqrt{-5} + b)
\]
which concludes this side of the proof. 

($\Rightarrow$) 
$\sqrt{-5}|(a+b\sqrt{-5})$ implies
$\sqrt{-5}|a$. Again, write $a$ in terms 
of multiples of $\xi$. 
\[
    a = \sqrt{-5}\xi
\]

Multiply by the algebraic conjugate both side.
\[
    a^2 = \sqrt{-5}-(\sqrt{-5})(\xi)(\bar{\xi}) = 5N(\xi)
\]

Looking at the equation in the ring $\mathbb{Z}$,
we conclude that $5|a^2$ and thus $5|a$, for 5 is prime. 

Take any two elements $\alpha, \beta$ in 
the ring $\mathbb{Z}[\sqrt{-5}]$. Assume 
that the product is divisible by 
$\sqrt{-5}$. Write:

\[
    \alpha := a + b\sqrt{-5},
    \beta := c + d\sqrt{-5}
\]
\[
    \alpha\beta = (a + b\sqrt{-5})
    (c + d\sqrt{-5})
\]
\[
    = ac-5bd+\sqrt{-5}(ad+bc)
\]

The real part of the product must be 
divisible by 5, by the proposition that 
we have proven. $5|(ac-5bd)$ so 
$5|ac$. WLOG, $5|a$. Again, by the proposition,
$\sqrt{-5}|\alpha$ as desired. 

Move on to factorize 5. $5 = -(\sqrt{-5})^2$. 
Any irreducible factorization of 5 must 
include two associates of $\sqrt{-5}$. 
Assume we have another factorization that 
has more irreducibles other than these 
two associates. After cancellation, 
we are left with a set of irreducibles 
that multiply up to a unit, which is 
impossible for all irreducibles are nonunits. 
We conclude that the factorization is unique 
up to associates. 




\newpage
\Problem
Consider the ring R = $\mathbb{Z}
[\sqrt{-3}]$. Let 
$I:=<2, 1+\sqrt{-3}>$ be an ideal in R.

\vspace{2mm}
(i) Prove that $I\neq<2>$ in R 

\Proof The element $1 + \sqrt{-3}$ 
is in the ideal $I$, but it is not in 
$<2>$. Assume for a contradiction, 
$1+\sqrt{-3} \in <2>$. For some element 
$\xi \in R$:

\[
    1+\sqrt{-3} = 2\xi
\]

Write $\xi = a+b\sqrt{-3}$ where $a, b$ are 
integers. 

\[
   1+\sqrt{-3} = 2a+2b\sqrt{-3} 
\]

The coefficients must match, so 
$2a = 1$ for some integer $a$. There is 
no integer solution, and we conclude 
that $1+\sqrt{-3}$ is not in $<2>$


(2) Prove that $I^2 = <2>I$

\Proof
Recall the identity:
\[
    <a, b>^2 = <a^2, ab, b^2>
\]

Rewrite the LHS:
\[
    I^2 = <2, 1+\sqrt{-3}>^2
    =<4, 2+2\sqrt{-3}, 1-3+2\sqrt{-3}>
    =<4, 2+2\sqrt{-3}, 2-2\sqrt{-3}>
\]

Notice:
\[
    4-(2+2\sqrt{-3}) = 2-2\sqrt{-3}
\]

We write:
\[
    I^2 = <4, 2+2\sqrt{3}>
\]

Moving on the the RHS:
\[
    <2>I = <2><2, 1+\sqrt{-3}> = <4, 2+2\sqrt{-3}>
\]

We conclude:
\[
    I^2 = <2>I
\]

(iii) Is $R$ a dedekind domain?

\new{Claim} No $R$ is not a dedekind domain.

\Proof First establish two propositions. 
First of all, $2$ is irreducible in the 
ring $R$. Assume for a contradiction that 
$2$ is reducible. Write:

\[
    (a + b\sqrt{-3}) 
    (c + d\sqrt{-3})
    = 2
\]
where $a, b, c, d \in \mathbb{Z}$

Multiplying both sides by the algebraic 
conjugate, we get:

\[
    (a^2+3b^2)(c^2+3d^2) = 2 \cdot
    \bar{2} = 4
\]

If either one of the two terms equal 1, 
then $(a, b) = (\pm1, 0)$ 
or $(c, d) = (\pm1, 0)$
which in case shows that 2 is irreducible. 
Thus, we consider 
$(a^2+3b^2) = 2$. 
However, this equation has no integer solution,
since $|b| < 1$ but then $a^2 = 0$ which has 
no solution. 

The second proposition is that $I$ is 
a proper ideal. It suffices to show 
that $1 \notin I$. Assume $1 \in I$ 
for a contradiction. Write:

\[
    2(a+b\sqrt{-3})+(1+\sqrt{-3})
    (c+d\sqrt{-3}) = 1
\]

Where $a, b, c, d \in \mathbb{Z}$
Comparing coefficients, we deduce:

\[
    2a+c+3d = 1
\]
\[
    2b-c+d = 0
\]
Adding up, we get:
\[
    2a+4d = 1
\]
But the LHS is even while the RHS is odd. We 
have reached a contradiction and indeed I is proper. 

$I$ is not a dedekind domain. The ideal 
$<2>$ is a prime ideal, for $2$ is irreducible. 
Clearly, $<2> \subsetneq <2, 1+\sqrt{-3}> = I$. 
On part(1), we showed that the two ideals are 
distinct. By the proposition established, $I$ 
is a proper Ideal. Thus $<2>$ is a prime ideal 
that is not maximal. $R$ is not a dedekind domain. 
\qed


%============
%Book Problem 6.1
%------------

\Problem
In domain $D$, a principal ideal $<p>$ 
is prime iff $p$ is zero or prime. 

\Proof
$(\Leftarrow)$
If $p = 0$, then the ideal $<p> = {0}$. 
The zero ideal must be prime, for integral 
domians don't have zero divisor. Suppose 
$p$ is prime, but $<p>$ is not a prime ideal. 
It is possible to obtain two elements $a, b \in D$
such that $a, b \notin <p>$ but $ab \in p$.
This means $p|ab$.  
Since $p$ is prime, $p|a$ WLOG. 
This implies $a \in <p>$ which is a contradiction. 

$(\Rightarrow)$
Assume that $<p>$ is a prime ideal, but 
$p$ is nonzero and nonprime. For $p$ is 
nonprime, write:

\[
    pq = ab
\]

for $a, b\in D$ which are nonzero and nonunits, and some 
$q \in D$. Recall that in a domain, 
an ideal is prime iff the domain mod 
ideal is also a domain. Since $<p>$ is 
domain by assumption, the fractional 
domain $D/<p>$ must have no zero divisors. 

We claim that $a + <p>$ and $b+<p>$ are 
zero divisors. Clearly, both of them are 
nonzero by assumption. Write:

\[(a+<p>)(b+<p>) = ab+<p>\]

and notice that $ab\in <p>$ so indeed 
the coset is the zero coset. \qed




\end{document}