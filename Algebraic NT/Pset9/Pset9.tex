\documentclass{article}
\usepackage{amsfonts}
\usepackage{amsthm}
\usepackage{amssymb}
\usepackage{wasysym}
\usepackage{verbatim}

\newcommand{\new}[1]{
    \vspace{2mm}
    \noindent
    \textbf{
    \underline{#1}}
}

\def\calO{{\mathcal{O}}}
\def\ZZ{{\mathbb{Z}}}
\def\_{{\hspace{1mm}}}

\def\contradiction{{\lightning}}



\newcounter{problemcnt}
\setcounter{problemcnt}{0}

\newcommand{\Problem}{{
    \vspace{5mm}
    \stepcounter{problemcnt}
    \noindent
    \arabic{problemcnt}. 
}
}

\newcommand{\nProblem}[1]{
    \vspace{5mm}
    \noindent
    \setcounter{problemcnt}{#1}
    \arabic{problemcnt}. 
    \stepcounter{problemcnt}  
}

\newcommand{\Proof}{{
    \vspace{2mm}
    \noindent
    \textbf{
    \underline{Proof}}
}
}

\newcommand{\textOr}{
    \hspace{5mm}
    \textrm{or}
    \hspace{5mm}
}

\newcommand{\textAnd}{
    \hspace{5mm}
    \textrm{and}
    \hspace{5mm}
}

%Article specific commands for ANT
\newcommand{\<}{{{
    \langle
}}}


\def\>{{{
    \rangle
}}}

\def\ZZ{{\mathbb{Z}}}

\newcommand{\ringInt}{
    {\mathcal{O}}
}

\newcommand{\pideal}{
    {{\mathfrak{p}}}
}


\newcommand{\qideal}{
    {{\mathfrak{q}}}
}

\begin{document}

\Problem
Let $H, I, J$ be nonzero ideals in dedekind domain D. 
Given $HI = HJ$, prove $I = J$. 

\Proof
We show $I \subseteq J$. Then, by symmetry, $J \subseteq I$, 
which shows $I = J$. 

We know that any ideal in a dedekind domain has an inverse ideal. 
The ideal $H$ has some ideal $H'$ such that $H'H = \<\alpha\>$
for some nonzero element $\alpha \in H$. Write:

\[
    H'HI = H'HJ \textOr \<\alpha\>I = \<\alpha\>J 
\]

For any element $i \in I$, we extract $\alpha i = \alpha j$ 
for some $j \in J$. $D$ is a domain, so by cancellation, 
$i = j$. We conclude $I \subseteq J$ and thus $I = J$. \qed

\Problem Let $R:= \ZZ[\sqrt{-3}]$. Also, define an ideal in R, 
$I = \<2, 1+\sqrt{-3}\>$. 

\begin{itemize}
    \item Prove $I \neq \<2\>$
    \item Prove $I^2 = \<2\>I$
    \item Is $R$ a dedekind domain?
\end{itemize}

\new{Solution}
We start with showing that $I$ is not equal to the principal 
ideal generated by $2$. Assume for a contradiction, that 
indeed $I = \<2\>$. Then, it must be $1+\sqrt{-3} \in \<2\>$. 
There must be some element $r \in R$ such that:

\[
    2r = 1+\sqrt{-3} \textOr r = \frac{1+\sqrt{-3}} {2}
\]

by expanding our search to the field of quotients. However, 
$r \notin Z[\sqrt{-3}]$, for the field of quotients is indeed 
a field, and inverses are unique. We reach a contradiction and 
$I \neq \<2\>$

\qed

\vspace{3mm}

We move on to show  $I^2 = \<2\>I$. By ideal algebra:

\[
    \<2, 1+\sqrt{-3}^2\>
    =\<4, 2+2\sqrt{-3}, (1+\sqrt{-3})^2\>
\]
\[
    \<4, 2+2\sqrt{-3}, -2+2\sqrt{-3}\>
    =\<2\>\<2, 1+\sqrt{-3}, -1+\sqrt{-3}\>
\]
Notice that $-1+\sqrt{-3} = 1+\sqrt{-3} - 2$. 
Thus, we conclude:

\[
    I^2 = \<2\>\<2, 1+\sqrt{-3}\> = \<2\>I
\]

as desired. \qed

\vspace{3mm}

Sadly, $R$ is not a dedekind domain. In a dedekind domain, 
ideals cancel out. Thus $I^2 = \<2\>I$ implies $I = \<2\>$, 
which we have proven to be false on the first part. 
\contradiction

\qed

\newpage

\Problem
Prove that $\<3, 1\pm \sqrt{-5}\>$ are prime ideals in the ring 
$\ZZ[\sqrt{-5}]$

\Proof
Denote $I:=\<3, 1+ \sqrt{-5}\>$
Consider the following line of Ideal algebra:

\[
    \<3, 1+\sqrt{-5}\>^2 = \<9, 3+3\sqrt{-5}, -4+2\sqrt{-5}\>
\]

We can add a ring multiple of one entry and add to another generator 
and still get the same ideal. Thus:

\[
    = \<9, 3+3\sqrt{-5} +4 - 2\sqrt{-5}, -4+2\sqrt{-5}\> 
    = \<9, 7+\sqrt{-5}, -4+2\sqrt{-5}\> 
\]

\[
    = \<9, 7+\sqrt{-5}, -4+2\sqrt{-5}-14-2\sqrt{-5}\>
    = \<9, 7+\sqrt{-5}, -18\> 
\]
\[
    = \<9, 7+\sqrt{-5}\> = \<9, -2+\sqrt{-5}\>
    = \<-2+\sqrt{-5}\> = \<2-\sqrt{-5}\>
\]

In fact, this ideal is a prime ideal. This is because the element 
$2-\sqrt{-5}$ is prime in the ring $\ZZ[-5]$. According to the textbook, 
$\ZZ[\sqrt{-5}]$ is indeed a UFD, so it suffices to show that 
$2-\sqrt{-5}$ is irreducible. The element has a norm of 9. 
Assuming that this element has a nonunit divisor, the norm of 
the divisor must necessarily be 3. 

Assume, for some $(a+b\sqrt{-5}) | (2-\sqrt{-5})$:
\[
    N(a+b\sqrt{-5}) = 3 \textAnd a^2+5b^2 = 3
\]

Clearly, there are no integer solutions for $a, b$. 
Hence the element is irreducible, and the principal ideal 
generated by it is also prime. $I^2$ must be prime, but then, 
$I|I^2$. This means, by ideal cancellation, $I = R$. 
(Ideal cancellation is justified for $\ZZ[\sqrt{-5}]$ is a ring 
of integers, and all ring of integers are dedekind domains
). 

We derive a contradiction by demonstrating that 
$I^2$ is proper. If $I = R$, $I^2 = R = \<1\>$. Thus, 
$1 \in \<2-\sqrt{-5}\>$, so the multiplicative 
inverse of $2-\sqrt{-5}$ must be in the ring $R$. 
Again, in the ring of quotients, 

\[
    \frac{1}{2-\sqrt{-5}} = \frac{2+\sqrt{-5}}{9}
\]

and the latter element is clearly not in the ring 
$\ZZ[\sqrt{-5}]$ \contradiction 

\vspace{3mm}
For the ideal $I' := \<3, 1-\sqrt{-5}\>$, it suffices 
to show that $I'^2$ is principal of a nonunit element. 
We can then repeat the argument above. The following 
lines of algebra concludes the proof:


\[
    \<3, 1-\sqrt{-5}\>^2 = \<9, 3-3\sqrt{-5}, -4-2\sqrt{-5}\>
\]\[
    = \<9, 3-3\sqrt{-5} +4 + 2\sqrt{-5}, -4-2\sqrt{-5}\> 
    = \<9, 7-\sqrt{-5}, -4-2\sqrt{-5}\> 
\]

\[
    = \<9, 7-\sqrt{-5}, -4-2\sqrt{-5}-14+2\sqrt{-5}\>
    = \<9, 7-\sqrt{-5}, -18\> 
\]
\[
    = \<9, 7-\sqrt{-5}\> = \<9, -2-\sqrt{-5}\>
    = \<2+\sqrt{-5}\>
\]
\qed


\newpage

\Problem
Let $K:=Q(\sqrt{d})$ be a quadratic field where $d$ is 
squarefree. Suppose $\ringInt_K$ is a UFD. Prove the following:
\begin{itemize}
    \item Let $p$ be a prime in $\ZZ$ where 
    $p|d$. Prove that $p$ 
    is an associate of a square of some prime element 
    in $\ringInt_K$
\end{itemize}

\new{Q1}
We first show that $p$ is not prime, and hence must be reducible. 
Since $p$ divides $d$, we can write:

\[
    (\sqrt{d})^2 = pa
\]

for some integer $a$. Notice that $p \nmid \sqrt{d}$. Otherwise, 
we can write $\sqrt{d} = p\alpha$ for some $\alpha \in \ringInt_K$.
Again, in the field of quotients, $\alpha = \sqrt{d}/p$, but 
this element cannot be in the ring of integers unless $p = 2$.
Moreover, even if $p = 2$, the ring of integers include only the 
element where the parity of the integer part and the irrational part 
match. Thus, $\sqrt{d}$ is always irreducible. 

This factorization sees witness to the fact that 
$p$ is nonprime. $p$ must be reducible in $O_K$. We write:

\[
    p = \alpha \beta
\]

For some $\alpha, \beta \in \ringInt_K$ that is not a unit. 
Taking the norm, we observe that $N(\alpha) = p$ necessarilly. 
Otherwise, one of the two elements will be a unit. From the norm 
statement, we deduce:

\[
    \alpha \bar{\alpha} = p
\]

Consider the case $d \not\equiv 1 (mod \_ 4)$. 
Write out $\alpha  = a+b\sqrt{d}$ for some integer $a, b$. We obtain 
$a^2-b^2d = p$. $p|d$ so $p|a^2$ and $p|a$. For $\alpha$ has a prime norm, it is an irreducible. 
We will show that $p$ is an associate of $\alpha^2$. 

\[
    \frac{\alpha^2}{p} = 
    \frac{(a+b\sqrt{d})^2}{p} = 
    \frac{a^2+b^2d+2ab\sqrt{d}}{p}
\]

$p|a, d$ guarantees that the fraction above is indeed in the ring of 
integers. Finally, we take the norm of this integer to demonstrate 
that it is indeed a unit:

\[
    N(\alpha^2/p) = 
    \left(
        \frac{a^2+b^2d}{p}
    \right)^2
    -4 a^2 b^2d/p^2
\]
\[
    = \frac{(a^2 - b^2d)^2}{p^2}
    = p^2/p^2 = 1
\]

which condludes the proof. 

Consider the case $d \equiv 1 (mod \_ 4)$. 
Write out $\alpha  = (a+b\sqrt{d})/2$ for some integer $a, b$. We obtain 
$a^2-b^2d = 4p$. $p|d$ so $p|a^2$ and $p|a$. For $\alpha$ has a prime norm, it is an irreducible. 
We will show that $p$ is an associate of $\alpha^2$. 

\[
    \frac{\alpha^2}{p} = 
    \frac{(a+b\sqrt{d})^2}{4p} = 
    \frac{a^2+b^2d+2ab\sqrt{d}}{4p}
\]

$p|a, d$ guarantees that the fraction above is indeed in the ring of 
integers. Finally, we take the norm of this integer to demonstrate 
that it is indeed a unit:

\[
    N(\alpha^2/p) = 
    \left(
        \frac{a^2+b^2d}{4p}
    \right)^2
    -a^2 b^2d/(4p^2)
\]
\[
    = \frac{(a^2 - b^2d)^2/16}{p^2}
    = p^2/p^2 = 1
\]

which condludes the proof.
\qed

\newpage

\begin{itemize}
    \item Let $p$ be an odd prime and $d$ a square mod $p$. 
    $p$ is a multiple of two distinct primes. 
\end{itemize}

\new{Q2}
Write $d = r^2 (mod\_ p)$ where $r$ is some 
nonzero positive integer less than $p$. For $p$ is 
an odd integer, we have $gcd(2r, p) = 1$. By Bezout's identity, 
extract integers $n, m$ that satisfies:

\[
    2rn + pm = 1
\]

Factorize the prime ideal generated by the prime $p$. Consider:

\[
    \<p, \sqrt{d} + r\>\<p, \sqrt{d} - r\>
    = \<p^2, p(\sqrt{d}+r), p(\sqrt{d} - r), d - r^2\>
\]

By the condition on $d$, $(d-r^2)/p$ must be an integer. Write:

\[
    \<p\>\<p, \sqrt{d}+r, \sqrt{d}-r, (d-r^2)/p\>
    =
    \<p\>\<p, 2r, \sqrt{d}+r, (d-r^2)/p\>
\]

By Bezout's identity, it is possible to obtain a unit from 
a $\ZZ$ combination of $p$ and $2r$. The latter ideal simplifies 
to the whole ring. Thus:

\[
    \<p, \sqrt{d} + r\>\<p, \sqrt{d} - r\>
    = \<p\>
\]

Still, it remains to show that the two ideals involved in this 
factorization are both proper. Assume for a contradiction that 
the right ideal is indeed the whole ring. Consequently:

\[
    \<p, \sqrt{d}+r\> = \<p\> \textAnd 
    \sqrt{d}+r \in \<p\>
\]

There must exist some element $\alpha \in \ringInt_K$ that satisfies:

\[
   p\alpha = \sqrt{d}+r
\]

Observing this equation in the factor ring:
\[
    \alpha = \frac{\sqrt{d}+r}{p}
\]
Clearly, this element is not in the ring of integers. A similar 
argument applies to the other ideal. 

Since $\ringInt_K$ is known to be a UFD, it is a PID. The two 
generators of the ideals $\<p, \sqrt{d} + r\>\<p, \sqrt{d} - r\>$
are both non-units. The product of the generators must be $p$. 
Hence, $p$ reducible. 

\[
    p = \alpha \beta
\]

For some $\alpha, \beta \in \ringInt_K$ that is not a unit. 
Taking the norm, we observe that $N(\alpha) = p$ necessarilly. 
Otherwise, one of the two elements will be a unit. 

Start with $d \not\equiv 1(mod\_4)$. 
From the norm 
statement, we deduce:

\[
    \alpha \bar{\alpha} = p
    \textAnd 
    a^2-db^2 = p
\]

where $\alpha = a+b\sqrt{d}$. 

It suffices to show that $\alpha, \bar{\alpha}$ are not 
associates of each other. We extend our search to the field of 
quotients. If the two elements are associates, 
$\alpha / \bar{\alpha}$ must yield a unit in the ring. However 
computation shows that this element is not even in the ring:

\[
    \frac{\alpha}{\bar{\alpha}}
    = \frac{a+b\sqrt{d}}{a-b\sqrt{d}}   
    = \frac{a+b\sqrt{d}}{a-b\sqrt{d}}   
    \frac{a+b\sqrt{d}}{a+b\sqrt{d}}
    =
    \frac{a^2+b^2d+2ab\sqrt{d}}{a^2-b^2d}
\]
\[
    = 1+\frac{2b^2d+2ab\sqrt{d}}{a^2-b^2d}
    = 1+\frac{2b^2d+2ab\sqrt{d}}{p}
\]
For this element to be in the ring of integers, 
$p|2b^2d$ by looking at the rational part (this is in $\ZZ$). 
This implies $p|b$, but then, $p|a$. Recall:

\[
a^2-db^2 = p
\]

By dividing both sides by $p$, we obtain, $p|1$, a contradiction. 
\contradiction

We can repeat the process for $p \equiv 1 (mod\_4)$. The 
division relation is mostly exploited for odd p, and it is not 
hard to deduce a contradiction using a similar argument. 
\qed

\newpage

\begin{itemize}
    \item If $p$ is an odd prime, and $d$ is not a square of 
    mod p, it is guaranteed that $p$ is prime in the ring 
    $\ringInt_K$. 
\end{itemize}


\new{Q3}
Start with the case $k \not\equiv (mod\_4)$. Assume $p$ to 
be reducible. Repeating the norm argument, we derive 
some element $\alpha \in \ringInt_K$ such that $N(\alpha) = p$. 
Expand $\alpha := a+b\sqrt{d}$ for integers $a, b$. Write:

\[
    a^2-b^2d = p
\]

We claim $p\nmid b$. Otherwise, $p|a$ and dividing out $p$, 
\[
    p(a/p)^2-p(b/p)^2d=1
\]
which in turn implies $p|1$, a contradiction. 

$p$ is an odd prime. Hence, in $\ZZ$, $gcd(p, b) = 1$. There 
is a modular inverse of $b$ in mod p. In other words, there exists 
$b' \in \ZZ$ such that $bb' \equiv = 1(mod \_ p)$. 

Reconsider the norm equation in mod p. 
\[
    a^2 - b^2d \equiv 0 (mod\_p)
\]
\[
    a^2 \equiv b^2d (mod\_p)
\]
\[
    (ab')^2 \equiv (bb')^2d \equiv d (mod\_p)
\]

Oh, but d cannot be a square mod p. We have reached a contradiction. 
\contradiction
Now let $k \equiv (mod\_4)$. Assume $p$ to 
be reducible. Repeating the norm argument, we derive 
some element $\alpha \in \ringInt_K$ such that $N(\alpha) = p$. 
Expand $\alpha := a+b\sqrt{d}$ for integers $a, b$. Write:

\[
    (a^2-b^2d)/2 = p 
    \textOr
a^2-b^2d = 2p 
\]

We claim $p\nmid b$. Otherwise, $p|a$ and dividing out $p$, 
\[
    p(a/p)^2-p(b/p)^2d=2
\]
which in turn implies $p|2$, a contradiction. 

$p$ is an odd prime. Hence, in $\ZZ$, $gcd(p, b) = 1$. There 
is a modular inverse of $b$ in mod p. In other words, there exists 
$b' \in \ZZ$ such that $bb' \equiv = 1(mod \_ p)$. 

Reconsider the norm equation in mod p. 
\[
    a^2 - b^2d \equiv 0 (mod\_p)
\]
\[
    a^2 \equiv b^2d (mod\_p)
\]
\[
    (ab')^2 \equiv (bb')^2d \equiv d (mod\_p)
\]

Oh, but d cannot be a square mod p. We have reached a contradiction.
\qed

\newpage
\new{Q4}



\begin{itemize}
    \item Let $2 \nmid d$. When is 2 a prime, square of a prime, or a product of two primes?
\end{itemize}



\new{Solution}

Consider this product of ideals:

\[
    \<2, \sqrt{d}+1\>\<2, \sqrt{d} - 1\>\]\[
    =\<4, 2\sqrt{d}+2, 2\sqrt{d} - 2, d-1\>
\]
\[
    = \<4, 4, 2\sqrt{d}-2, d-1\>
\]
\[
    = \<4, 2\sqrt{d}-2, d-1\>
\]

Notice that since $d$ is odd, $d-1$ must be even. Write:
\[
    = \<2\>\<2, \sqrt{d}-1, (d-1)/2\>
\]

And also notice:
\[
     \<2, \sqrt{d}+1\> = \<2, \sqrt{d} - 1\>
\]

We claim:
\[
     \<2, \sqrt{d} - 1\>^2 = \<2\>\<2, \sqrt{d}-1, (d-1)/2\>
\]

If $d \equiv 3 (mod\_4)$, then $(d-1)/2$ is odd. The above equation 
condenses to:
\[
    \<2, \sqrt{d} - 1\>^2 = \<2\>
\]
For $\ringInt_K$ is a UFD hence a PID, 
\[
    \<\alpha\>^2 = \<\alpha^2\> = \<2\>
\]
Thus, $2 = \alpha^2$ up to associates. Taking the norm, 
$4 = N(\alpha)^2$ and $N(\alpha) = 2$ necessarily. $\alpha$ has 
a prime norm, so it must be prime. Thus, $p$ is an associate of 


If $d \equiv 1(mod\_4)$, $(d-1)/2$ is even. 
Consider the following claims:


\new{Claim 1}
If $d \equiv 5 (mod\_8)$ then 2 is prime. 

\Proof We assume for a contradiction that there is some d where 
2 is reducible. Again, by the norm argument, we obtain an 
element $\alpha \in \ringInt_K$ where $N(\alpha) = 2$. 
$d \equiv (mod\_4)$ so write $\alpha = \frac{a+b\sqrt{d}}{2}$
for some integer $a, b$. Taking the norm:

\[
    N(\alpha) = \frac{a^2-b^2d}{4} = 2
\]
\[
    a^2-b^2d = 8
\]

It is convinient to remember that the quadratic residue of 
8 is $0, 1, 4$. Taking mod 5 of the equation:

\[
    a^2-5b^2 \equiv 0 (mod\_8)
    \textAnd
    a^2 \equiv 5b^2 (mod\_8)
\]

Trying all the slurs of possibilities for the residue of $b^2$, 
we claim $a^2 \equiv b^2 \equiv 0 (mod\_8)$. This in turn implies 
that $a, b$ are multiples of 4. Back to the original equation:


\[
    16(a/4)^2-16(b/4)^2d = 8
    \textAnd
    2(a/4)^2-2(b/4)^2d=1
\]

The equation implies $2|1$, a contradiction. \contradiction \qed

\new{Claim 1}
If $d \equiv 1 (mod\_8)$ then 2 can be expressed as 
a product of two distinct primes. 

\Proof Observe that $8|(d-1)$. Thus $(d-1)/4$ is even. Consider 
the following lines of ideal algebra:
\[
    \big\langle2, \frac{\sqrt{d}+1}{2}\big\rangle
\big\langle2, \frac{\sqrt{d}-1}{2}\big\rangle
=
\big\langle4, \sqrt{d}+1, \sqrt{d}-1, \frac{d-1}{4}\big\rangle
\]
\[
    \<2\>\big\langle2, \frac{\sqrt{d}+1}{2}, \frac{\sqrt{d}-1}{2}, \frac{d-1}{8}\big\rangle
=\<2\>
    \]

The last line follows by subtraction. The difference of 
the second and the third entry is a unit. 

The two products in the first entry are both not divisible by 2. 
Again, looking in the field of quotients, we notice that 
$(\sqrt{d}\pm 1)/4$ are both in the field but not the ring. 

The two ideals involved in the factorization of $\<2\>$ are both 
proper. If one of the two are non-propper, one of the 
ideals must equal to $\<2\>$. 

Take 
$(\sqrt{d}\pm1)/2$ in one of the factor rings.
This element must be in the principal ideal generated by 2, so 
we write:
\[
    2\alpha = (\sqrt{d}\pm1)/2
\]
We obtain, for some element 
$\alpha \in \ringInt_K$:
\[
    \alpha = \frac{\sqrt{d}\pm 1}{4}
\]

Which is in the field of quotients, but not in the ring of integers. 
Thus, the two rings are proper and this shows that 2 is reducible. 

We also claim that the two ideals cannot be equal to each other. 
If the two ideals equal to some ideal, say $I$, then write:

\[
    \frac{\sqrt{d}\pm1}{2} \in I
\]

and thus their difference, which is a unit, must be in $I$. 
However, $I$ is proper as shown above, and cannot contain a unit. 

Combining the results, we write, for some nonunit $\alpha, \beta \in 
\ringInt_K$:

\[
    \<\alpha\>\<\beta\> = \<2\>
\]

and necessarily, $\alpha\beta = 2$. Taking the norms, we obtain 
$N(\alpha)N(\beta) = 4$ and the norm of both $\alpha, \beta$ must 
be 2, which is prime. We have factorized 2 into two primes, 
and the two generators are distinct, which shows $\alpha \neq \beta$. 
\qed


\newpage
\new{Book 5.8} Let $\pideal, \qideal$ be distinct prime ideals in 
a dedekind domain $\ringInt_K$. Prove that $\pideal + \qideal = \ringInt_K$
and $\pideal \qideal = \pideal \cap \qideal$. 

\Proof 
Start with the first statement. Assume for a contradiction that 
$\pideal + \qideal$ is a proper ideal. All proper ideals are 
contained in a maximal ideal. Let $M$ be the maximal ideal containing 
the sum of the two prime ideals. $M$ is maximal, hence prime. 

$\pideal, \qideal \subseteq \pideal + \qideal \subseteq M$, so 
$M|\pideal, \qideal$. Since $\pideal$ and $\qideal$ is a prime ideal, 
and since the factorization of ideals are unique, $M = \pideal = \qideal$. 
This contradicts the fact that the two prime ideals are unique. 
\checkmark. 

Now, show the second statement. Denote the intersect of the two 
ideals as $I$. By construction, $I$ is included in both 
$\pideal, \qideal$. These two ideals contain $I$. Hence, $\pideal|I$ and $\qideal|I$. 
Again, ideals factor uniquely in dedekind domains and $\ringInt_K$ 
is a dedekind domain. Ergo, $\pideal \qideal | I$ and we deduce 
$\pideal \qideal \supseteq I$. 

By strong closure of ideals, $\pideal\qideal \subseteq \pideal$. 
Write any element in the product ideal as:

\[
    \alpha = \sum_{i = 1}^{N} p_iq_i 
\]

where each $p_i, q_i$ are elements of $\pideal$ and $\qideal$ for 
all $1\leq i \leq N$. Each summand is in $\pideal$, and the closure 
of $\pideal$ under addition guarantees $\alpha \in \pideal \qideal$. 

By symmetry, $\pideal \qideal \subseteq \qideal$. Thus, $\pideal \qideal \subseteq \pideal \cap \qideal = I$
We have shown containment both ways. $\pideal \qideal = \pideal \cap \qideal$

\qed

\new{Book 5.12} In the ring $\ZZ[\sqrt{-5}]$, find all the ideals 
that contain the element $6$. 

\new{Solution}
In class, we decomposed the principal ideal $\<6\>$ as:

\[
    \<6\> = \<-2+\sqrt{-5}\>^2\<3+\sqrt{-5}\>\<3-\sqrt{-5}\>
\]

Ideals factor uniquely into prime ideals for dedekind domains. 
$\ZZ[\sqrt{-5}]$ is the ring of integers of the quadratic field 
where $d = -5 \equiv 3 (mod\_4)$. In order for an ideal to include 
6, it must include $\<6\>$ by strong closure, and hence the ideal 
must divide the ideal $\<6\>$. Let $\mathcal{F}$ be the family of 
all ideals that contain 6. We conclude:

\[
    \mathcal{F} = 
    \{
        \<-2+\sqrt{-5}\>^a\<3+\sqrt{-5}\>^b\<3-\sqrt{-5}\>^c
        |
        a \leq 2, b \leq 1, c \leq 1, a,b,c \in \mathbb{N} \cup {0}
    \}
\]
\qed


\begin{comment}
\new{Proposition} In the ring $\ZZ[\sqrt{-5}]$, elements with norm 
4, 6, 9 are irreducible. 

\Proof
If such an element is not prime, there must exist a nonunit divisor 
that has a norm of 2 or 3. Assume there exists some ring element 
$\alpha = a+b\sqrt{-5}$ that has a norm of 2 or 3. It must be:

\[
    a^2+b^2 = 2 \textOr a^2+b^2=3
\]

Both equations have no integer solutions. \qed

\new{Proposition} The only factorization of 6 in the ring 
$\ZZ[\sqrt{-5}]$ is:

\[
    6 = 2\cdot 3 \textOr 6 = (1+\sqrt{-5})(1-\sqrt{-5})
\]

\Proof
Compute $N(6) = 36$. The possible nonunit divisors of 36 are:

\[n = 2, 3, 4, 6, 9, 12, 18, 36\]

Let $n$ be the norm of some divisor of 6. We can 
eliminate $n = 2, 3$ for no element can have such a norm 
in the ring $\ZZ[\sqrt{-5}]$. For $n \geq 6$, the symmetry will 
guarantee that the norm of the other divisor will fall into 
the values of $n$ that are less than 6. 

We are left with:

\[
    n = 4, 6
\]

$n = 4$ implies that the element in question is exclusively 
$2$. $n = 6$ implies the element is one of $1 \pm \sqrt{-5}$
By the first proposition, these elements along with their 
corresponding 6-conjugates are also irreducible. \qed

\new{Corollary 1}
The divisors of 6 in $\ZZ[\sqrt{-5}]$ are:

\[
    \{2, 3, 1\pm\sqrt{-5}\}
\]

\new{Corollary 2}
The ideals in $\ZZ[{\sqrt{-5}}]$ are those 
which contain $\{2, 3, 1\pm\sqrt{-5}\}$ in its generators. 

\Proof 
$\ZZ[\sqrt{-5}]$ is the ring of integers of the quadratic field 
where 

$d = -5 \equiv 3 (mod\_4)$. The ring of integers of any number 
fields are noetherian, and thus all ideals are finitely generated. 
It is trivial to see that if one of the generators in the ideal 
is the divisor of 6, the ideal contains 6. \qed 

\new{Example} Some(but not all) Ideals of $\ZZ[\sqrt{-5}]$ that include 6
\[
    \<2\>, \<3\>, \<2, 1+\sqrt{-5}\>, \<2, \sqrt{-5}\>
\]
and the list goes on. 
\end{comment}



\end{document}