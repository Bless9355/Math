\documentclass{article}
\usepackage{amsfonts}
\usepackage{amsthm}
\usepackage{amssymb}
\usepackage{wasysym}


\newcommand{\new}[1]{
    \vspace{2mm}
    \noindent
    \textbf{
    \underline{#1}}
}

\def\calO{{\mathcal{O}}}
\def\ZZ{{\mathbb{Z}}}
\def\_{{\hspace{1mm}}}

\def\contradiction{{\lightning}}



\newcounter{problemcnt}
\setcounter{problemcnt}{0}

\newcommand{\Problem}{{
    \vspace{5mm}
    \stepcounter{problemcnt}
    \noindent
    \arabic{problemcnt}. 
}
}

\newcommand{\nProblem}[1]{
    \vspace{5mm}
    \noindent
    \setcounter{problemcnt}{#1}
    \arabic{problemcnt}. 
    \stepcounter{problemcnt}  
}

\newcommand{\Proof}{{
    \vspace{2mm}
    \noindent
    \textbf{
    \underline{Proof}}
}
}

\newcommand{\textOr}{
    \hspace{5mm}
    \textrm{or}
    \hspace{5mm}
}

\newcommand{\textAnd}{
    \hspace{5mm}
    \textrm{And}
    \hspace{5mm}
}

%Article specific commands for ANT
\newcommand{\<}{{
    \langle
}}


\def\>{{
    \rangle
}}

\def\ZZ{{\mathbb{Z}}}

\newcommand{\ringInt}{
    {\mathcal{O}}
}

\newcommand{\pideal}{
    {{\mathfrak{p}}}
}


\newcommand{\qideal}{
    {{\mathfrak{q}}}
}

\begin{document}

\Problem
Let $H, I, J$ be nonzero ideals in dedekind domain D. 
Given $HI = HJ$, prove $I = J$. 

\Proof
We show $I \subseteq J$. Then, by symmetry, $J \subseteq I$, 
which shows $I = J$. 

We know that any ideal in a dedekind domain has an inverse ideal. 
The ideal $H$ has some ideal $H'$ such that $H'H = \<\alpha\>$
for some nonzero element $\alpha \in H$. Write:

\[
    H'HI = H'HJ \textOr \<\alpha\>I = \<\alpha\>J 
\]

For any element $i \in I$, we extract $\alpha i = \alpha j$ 
for some $j \in J$. $D$ is a domain, so by cancellation, 
$i = j$. We conclude $I \subseteq J$ and thus $I = J$. \qed

\Problem Let $R:= \ZZ[\sqrt{-3}]$. Also, define an ideal in R, 
$I = \<2, 1+\sqrt{-3}\>$. 

\begin{itemize}
    \item Prove $I \neq \<2\>$
    \item Prove $I^2 = \<2\>I$
    \item Is $R$ a dedekind domain?
\end{itemize}

\new{Solution}
We start with showing that $I$ is not equal to the principal 
ideal generated by $2$. Assume for a contradiction, that 
indeed $I = \<2\>$. Then, it must be $1+\sqrt{-3} \in \<2\>$. 
There must be some element $r \in R$ such that:

\[
    2r = 1+\sqrt{-3} \textOr r = \frac{1+\sqrt{-3}} {2}
\]

by expanding our search to the field of quotients. However, 
$r \notin Z[\sqrt{-3}]$, for the field of quotients is indeed 
a field, and inverses are unique. We reach a contradiction and 
$I \neq \<2\>$

\qed

\vspace{3mm}

We move on to show  $I^2 = \<2\>I$. By ideal algebra:

\[
    \<2, 1+\sqrt{-3}^2\>
    =\<4, 2+2\sqrt{-3}, (1+\sqrt{-3})^2\>
\]
\[
    \<4, 2+2\sqrt{-3}, -2+2\sqrt{-3}\>
    =\<2\>\<2, 1+\sqrt{-3}, -1+\sqrt{-3}\>
\]
Notice that $-1+\sqrt{-3} = 1+\sqrt{-3} - 2$. 
Thus, we conclude:

\[
    I^2 = \<2\>\<2, 1+\sqrt{-3}\> = \<2\>I
\]

as desired. \qed

\vspace{3mm}

Sadly, $R$ is not a dedekind domain. In a dedekind domain, 
ideals cancel out. Thus $I^2 = \<2\>I$ implies $I = \<2\>$, 
which we have proven to be false on the first part. 
\contradiction

\qed

\newpage

\Problem
Prove that $\<3, 1\pm \sqrt{-5}\>$ are prime ideals in the ring 
$\ZZ[\sqrt{-5}]$

\Proof
Denote $I:=\<3, 1+ \sqrt{-5}\>$
Consider the following line of Ideal algebra:

\[
    \<3, 1+\sqrt{-5}\>^2 = \<9, 3+3\sqrt{-5}, -4+2\sqrt{-5}\>
\]

We can add a ring multiple of one entry and add to another generator 
and still get the same ideal. Thus:

\[
    = \<9, 3+3\sqrt{-5} +4 - 2\sqrt{-5}, -4+2\sqrt{-5}\> 
    = \<9, 7+\sqrt{-5}, -4+2\sqrt{-5}\> 
\]

\[
    = \<9, 7+\sqrt{-5}, -4+2\sqrt{-5}-14-2\sqrt{-5}\>
    = \<9, 7+\sqrt{-5}, -18\> 
\]
\[
    = \<9, 7+\sqrt{-5}\> = \<9, -2+\sqrt{-5}\>
    = \<-2+\sqrt{-5}\> = \<2-\sqrt{-5}\>
\]

In fact, this ideal is a prime ideal. This is because the element 
$2-\sqrt{-5}$ is prime in the ring $\ZZ[-5]$. According to the textbook, 
$\ZZ[\sqrt{-5}]$ is indeed a UFD, so it suffices to show that 
$2-\sqrt{-5}$ is irreducible. The element has a norm of 9. 
Assuming that this element has a nonunit divisor, the norm of 
the divisor must necessarily be 3. 

Assume, for some $(a+b\sqrt{-5}) | (2-\sqrt{-5})$:
\[
    N(a+b\sqrt{-5}) = 3 \textAnd a^2+5b^2 = 3
\]

Clearly, there are no integer solutions for $a, b$. 
Hence the element is irreducible, and the principal ideal 
generated by it is also prime. $I^2$ must be prime, but then, 
$I|I^2$. This means, by ideal cancellation, $I = R$. 
(Ideal cancellation is justified for $\ZZ[\sqrt{-5}]$ is a ring 
of integers, and all ring of integers are dedekind domains
). 

We derive a contradiction by demonstrating that 
$I^2$ is proper. If $I = R$, $I^2 = R = \<1\>$. Thus, 
$1 \in \<2-\sqrt{-5}\>$, so the multiplicative 
inverse of $2-\sqrt{-5}$ must be in the ring $R$. 
Again, in the ring of quotients, 

\[
    \frac{1}{2-\sqrt{-5}} = \frac{2+\sqrt{-5}}{9}
\]

and the latter element is clearly not in the ring 
$\ZZ[\sqrt{-5}]$ \contradiction 

\vspace{3mm}
For the ideal $I' := \<3, 1-\sqrt{-5}\>$, it suffices 
to show that $I'^2$ is principal of a nonunit element. 
We can then repeat the argument above. The following 
lines of algebra concludes the proof:


\[
    \<3, 1-\sqrt{-5}\>^2 = \<9, 3-3\sqrt{-5}, -4-2\sqrt{-5}\>
\]\[
    = \<9, 3-3\sqrt{-5} +4 + 2\sqrt{-5}, -4-2\sqrt{-5}\> 
    = \<9, 7-\sqrt{-5}, -4-2\sqrt{-5}\> 
\]

\[
    = \<9, 7-\sqrt{-5}, -4-2\sqrt{-5}-14+2\sqrt{-5}\>
    = \<9, 7-\sqrt{-5}, -18\> 
\]
\[
    = \<9, 7-\sqrt{-5}\> = \<9, -2-\sqrt{-5}\>
    = \<2+\sqrt{-5}\>
\]
\qed


\newpage

\new{Book 5.8}
Let $\pideal$ and $\qideal$ be

\end{document}