\documentclass{article}
\usepackage{amsfonts}
\usepackage{amsthm}
\usepackage{amssymb}
\usepackage{amsmath}
\usepackage{wasysym}
\usepackage{verbatim}
\usepackage{enumerate}
\usepackage{xcolor}

\newcommand{\new}[1]{
    \vspace{2mm}
    \noindent
    \textbf{
    \underline{#1}}
}

\def\calO{{\mathcal{O}}}
\def\ZZ{{\mathbb{Z}}}
\def\_{{\hspace{1mm}}}

\def\contradiction{{\lightning}}



\newcounter{problemcnt}
\setcounter{problemcnt}{0}

\newcommand{\Problem}{{
    \vspace{5mm}
    \stepcounter{problemcnt}
    \noindent
    \arabic{problemcnt}. 
}
}

\newcommand{\nProblem}[1]{
    \vspace{5mm}
    \noindent
    \setcounter{problemcnt}{#1}
    \arabic{problemcnt}. 
    \stepcounter{problemcnt}  
}

\newcommand{\Proof}{{
    \vspace{2mm}
    \noindent
    \textbf{
    \underline{Proof}}
}
}

\newcommand{\textOr}{
    \hspace{5mm}
    \textrm{or}
    \hspace{5mm}
}

\newcommand{\textAnd}{
    \hspace{5mm}
    \textrm{and}
    \hspace{5mm}
}

\newcommand{\textWhere}{
    \hspace{10mm}
    \textrm{where}
    \hspace{3mm}
}

\newcommand{\textForany}{
    \hspace{10mm}
    \textrm{for any }
    \hspace{3mm}
}



\def\twobytwoMat(#1, #2, #3, #4){
    {
        \begin{bmatrix}
            {#1} & {#2}\\
            {#3} & {#4}
        \end{bmatrix}
    }
}

\def\twobytwoDet(#1, #2, #3, #4){
    {
        \begin{vmatrix}
            {#1} & {#2}\\
            {#3} & {#4}
        \end{vmatrix}
    }
}

%Article specific commands for ANT
\newcommand{\<}{{{
    \langle
}}}


\def\>{{{
    \rangle
}}}

\def\ZZ{{\mathbb{Z}}}

\newcommand{\ringInt}{
    {\mathcal{O}}
}

\newcommand{\pideal}{
    {{\mathfrak{p}}}
}


\newcommand{\qideal}{
    {{\mathfrak{q}}}
}

\begin{document}
    \begin{center}
        \LARGE
        All Ideals in Dedekind Domains 
        
        can be generated by two generators

        \Large 
        presented with CRT for Rings
    \end{center}

    The goal of this note is to prove that any ideal in a 
    Dedekind domain can be generaged by two generators. In 
    fact, the first generator can be chosen by random as 
    long as it is nonzero. Some basic notation of product 
    rings are presented along with a proof of the CRT. Then, 
    we prove the theorem in question. 

    Assume all rings to be commutative. 

    \new{Proposition} Cartesian product of rings are also rings

    Let $R, S$ be rings. From the cartesian product $R\times S$, 
    define addition and multiplication as follows:

    \[
        (r, s) \cdot (\bar{r}, \bar{s})
        = (r\bar{r}, s\bar{s})
    \]
  \[
        (r, s) + (\bar{r}, \bar{s})
        = (r+\bar{r}, s+\bar{s})
    \]
    The following binary operations along with the set $R \times S$ 
    forms a ring. 

    \Proof
    It is trivial to show closure under subtraction and 
    multiplication. Write:

    \[
        (r, s) - (\bar{r}, \bar{s}) = (r - \bar{r}, s - \bar{s})
        \in R \times S 
    \]

    The line of algebra is justified by the fact that $R, S$ are rings. 
    Similarly, the set is closed under multiplicaiton. Associativity 
    and distributivity follows from the ringness of $R, S$. 
    \qed 
    
    \new{Theorem} Chinese Remainder Theorem for Rings 

    Let $I, J$ be ideals of a ring $R$. Suppose 
    $I + J = R$. The following sets 
    are isomorphic:

    \[
        R/I\cup J \approxeq R/I \times R/J
    \]

    \new{Proof}
    Define a ring homeomorphism:
    \[
        \varphi: R \rightarrow R/I \times R/J
    \]
    It suffices to show that $\varphi$ is surjective along with:
    \[
        \ker(\varphi) = I \cup J
    \]
    
    Then, the isomorphism can be proved by the first isomorphsm theorem 
    for rings. 

    Define:
    \[
        \varphi(a) = (a + I, a + J)
    \]
    Take any two cosets $x + I, y + J$. By the condition 
    $R = X + Y$, it is possible to rewrite elements $x, y$ as:
    
    \[
        x = x_i + x_j \textAnd y = y_i + y_j
    \]
    where $x_i, y_i \in I$ and $x_j, y_j \in J$.
    It is possible to construct an element in $R$ that 
    maps into the two randomly selected cosets. Namely:

    \[
        \varphi(x_j + y_i) = (x_j + y_i + I,x_j + y_i + J)
        =(x_j + I, y_i + J) \]\[= (x_i + x_j + I, y_i + y_j + J) 
        = (x + I, y + J)        
    \]

    This shows surjectivity. Move on to prove that the kernel 
    of $\varphi$
    is $I \cup J$. $\varphi(a) = (I, J)$ implies 
    $a \in I, a \in J$ so $a \in I \cap J$. Also, 
    choose arbitrary $b \in I \cap J$. $\varphi(b) = (I, J)$. 
    This shows that $\ker(\varphi) = I\cap J$. \qed

    \new{Remark} The result can be generalized to multiple ideals. 
    That is, for ideals $I_i$ that satisfy:
    \[
        \sum_{i = 1}^n I_i = R 
    \]
    \[
        R/(\bigcap_{i = 1}^n I_i) \approxeq R / I_1 \times R/I_2 \cdots \times R/I_n
    \]
    A simple proof can be written by induction, and is left 
    for the reader as an exercise. 

    \vspace{5mm}

    Now we are ready to present the main theorem and proof. 

    \new{Theorem} Any ideal in a Dedekind domain can 
    be generated by two geneators. In fact, the first generator 
    can be any nonzero element in the ring. 

    \new{Proof} Let $D$ be the Dedekind domain and $I$ be an 
    ideal in $D$. If the ideal $I$ is principal, the theorem holds. 
    Choose any nonzero element $\alpha \in I$. 

    The principal ideal generated by $\alpha$ is in the ideal $I$. 
    Inclusion of an ideal implies that the large ideal divides the 
    small ideal. In symbols:

    \[
        \<\alpha\> \subsetneq I
        \implies 
        I | \<\alpha\>
    \]
    
    We know that 
    ideals uniquely factor into prime ideals. In light 
    of the divisibility relation above, we write 
    the prime factorization of $I$ and $\<\alpha\>$:

    \newcommand\ppow[2]{{
        \pideal_{#2}^{{#1}_{#2}}
        }
    }
    \[
        I = \ppow{e}{1} \ppow{e}{2} \cdots \ppow{e}{t}
    \]
    \[
        \<\alpha\> = \ppow{f}{1} \ppow{f}{2} \cdots \ppow{f}{t} 
        Q_1 Q_2 \cdots Q_l
    \]

    Where the the indicies $e, f$ satisfy $e_i \leq f_i$ 
    for any $1 \leq i \leq t$. $Q_i$ are ideals that are 
    not in the family of ideals $\{\pideal_1 \dots \pideal_n\}$
    but are not necessarily distinct. 
    
    We construct a $\beta$ that satisfies:
    \[
        \ppow{e}{i} || I
        \textForany 
        1 \leq i \leq t
    \]
    \[
        Q_i \nmid I 
        \textForany 
        1 \leq i \leq l
    \]

    As long as $I$ is not principle, there must exist an 
    ideal $Q_i$. Otherwise, if $\<\alpha\>$ was in the form of:

\[
        \<\alpha\> = \ppow{f}{1} \ppow{f}{2} \cdots \ppow{f}{t} 
    \]

    Then $\<\alpha\> = I$, and otherwise the principal ideal will 
    be greater than the original ideal, which is a contradiction. 

    We recognize:

    \[
        \ppow{e}{1} + \cdots + \ppow{e}{t} + Q_1+\cdots + Q_l = D
    \]

    This is because the ideal $Q_1$ is distinct from $\ppow{e}{1}$, 
    and the gcd of the ideal is the whole ring. By the Chinese 
    Remainder Theorem for Rings, it is possible to construct 
    an isomorphism between:

    \[
        R / (\ppow{e}{1} \cap \cdots \cap
        \ppow{e}{t}\cap Q_1 \cap \cdots \cap Q_l)
        \approxeq 
        R / \ppow{e}{1} \times  \cdots \times 
        R/\ppow{e}{t} \times
        R/Q_1 \cap \times \cap R/Q_l
    \]
    
    Also, we know exactly what the isomorphism is. We choose a 
    series of elements $\beta_i$ from the set  
    $\ppow{e}{i}\setminus \pideal^{e_{1}+1}$. Consider 
    an element is the right product ring:

    \[
        (\beta_1+\ppow{e}{1}, \dots, \beta_t+ \ppow{e}{t} 
        , 1+ Q_1, \dots, 1+Q_l)
    \]

    As a consequence of CRT, it is possible to draw a single 
    element $\beta$ such that:

    \[
        \beta + \ppow{e}{i} = \beta_i + \ppow{e}{i} 
        \textForany 1 \leq i \leq t \textAnd 
    \]
\[
    \beta + Q_j = 1+Q_j \textForany 1 \leq j \leq l
\]

We deduce 
$\beta - \beta_i \in \ppow{e}{i}$ and $\beta - 1 \in Q_j$. 
The first statemeint implies $\beta \in \ppow{e}{1}$. If 
$\beta \in \pideal_i^{e_i+1}$, we reach $\beta_i \in \pideal_i^{e_i+1}$, 
a contradiction. Likewise, $\beta \in Q_j$ implies $1 \in Q_j$ 
which contradicts the fact that $Q_j$ is proper. 

From $\beta \in \ppow{e}{i}$, we deduce $\<\beta\> \subseteq \ppow{e}{i}$ 
which implies $\ppow{e}{i}| \<\beta\>$. The prime ideal $\pideal_i$ 
divides $\<\beta\>$ exactly $e_i$ times. Otherwise, $\pideal_i^{e_i+1} | \<\beta\>$
and $\beta$ will be in the ideal  $\pideal_i^{e_i+1}$ which we have shown 
not to be true. Also, $Q_j \nmid \<\beta\>$ by a similar argument. 

Thus:

\[
    \<\alpha, \beta\> = \<\alpha\> + \<\beta\> 
    = \gcd(\<\alpha\>, \<\beta\>)
\]
\[
    = \ppow{e}{1} \ppow{e}{2} \cdots \ppow{e}{t} = I 
\]

\qed

\end{document}