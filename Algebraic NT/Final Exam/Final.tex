\documentclass{article}
\usepackage{amsfonts}
\usepackage{amsthm}
\usepackage{amssymb}
\usepackage{amsmath}
\usepackage{wasysym}
\usepackage{verbatim}
\usepackage{enumerate}
\usepackage{xcolor}

\newcommand{\new}[1]{
    \vspace{2mm}
    \noindent
    \textbf{
    \underline{#1}}
}

\def\calO{{\mathcal{O}}}
\def\ZZ{{\mathbb{Z}}}
\def\_{{\hspace{1mm}}}

\def\contradiction{{\lightning}}



\newcounter{problemcnt}
\setcounter{problemcnt}{0}

\newcommand{\Problem}{{
    \vspace{5mm}
    \stepcounter{problemcnt}
    \noindent
    \arabic{problemcnt}. 
}
}

\newcommand{\nProblem}[1]{
    \vspace{5mm}
    \noindent
    \setcounter{problemcnt}{#1}
    \arabic{problemcnt}. 
    \stepcounter{problemcnt}  
}

\newcommand{\Proof}{{
    \vspace{2mm}
    \noindent
    \textbf{
    \underline{Proof}}
}
}

\newcommand{\textOr}{
    \hspace{5mm}
    \textrm{or}
    \hspace{5mm}
}

\newcommand{\textAnd}{
    \hspace{5mm}
    \textrm{and}
    \hspace{5mm}
}

\newcommand{\textWhere}{
    \hspace{5mm}
    \textrm{where}
    \hspace{5mm}
}

\newcommand{\polymod}{
    \hspace{3mm}
    \textrm{poly mod}
}



\def\twobytwoMat(#1, #2, #3, #4){
    {
        \begin{bmatrix}
            {#1} & {#2}\\
            {#3} & {#4}
        \end{bmatrix}
    }
}

\def\twobytwoDet(#1, #2, #3, #4){
    {
        \begin{vmatrix}
            {#1} & {#2}\\
            {#3} & {#4}
        \end{vmatrix}
    }
}

%Article specific commands for ANT
\newcommand{\<}{{{
    \langle
}}}


\def\>{{{
    \rangle
}}}

\def\ZZ{{\mathbb{Z}}}

\newcommand{\ringInt}{
    {\mathcal{O}}
}

\newcommand{\pideal}{
    {{\mathfrak{p}}}
}


\newcommand{\qideal}{
    {{\mathfrak{q}}}
}
\def\QQ{\mathbb{Q}}

\begin{document}
\Problem 
Let $K = \QQ(\sqrt{3}, \sqrt{7})$. Find the degree of the extension 
$K/\QQ$. That is, compute $[K:\QQ]$. 

\new{Solution} 
Start with finding the primitive element of the field $K$. The primitive 
element theorem dictates that any number field must have a primitive 
element. We claim that the primitive element is $\sqrt{7}+\sqrt{3}$. 
We wish to show $\QQ(\sqrt{3}, \sqrt{7}) = \QQ(\sqrt{3} + \sqrt{7})$. 
It suffices to show that each of the generators are included in 
the other field. We prove the following to relations:

\[
    \sqrt{7}+\sqrt{3} \in \QQ(\sqrt{7}, \sqrt{3})
    \textAnd 
    \sqrt{7}, \sqrt{3} \in \QQ(\sqrt{7}+\sqrt{3})
\]

The first statement follows trivially from the additive closure 
of the field $\QQ(\sqrt{7}, \sqrt{3})$. For the second statement, 
exploit the existance of multiplicative inverses. We know 
that the following element exists in $\QQ(\sqrt{7}+\sqrt{3})$. 

\[
    4\cdot(\sqrt{7}+\sqrt{3})^{-1}
    = \frac{4}{\sqrt{7}+\sqrt{3}}
    = \frac{4(\sqrt{7}-\sqrt{3})}{(\sqrt{7}+\sqrt{3}) (\sqrt{7}-\sqrt{3})}
    = \sqrt{7} - \sqrt{3}
\]

We have obtained that the sum and difference of $\sqrt{3}, \sqrt{7}$ 
are both in the field $\QQ(\sqrt{7}, \sqrt{3})$. Thus, 

\[
    \frac{(\sqrt{7} + \sqrt{3})+(\sqrt{7} - \sqrt{3})}{2} 
    = \sqrt{7}
    \in \QQ(\sqrt{7}, \sqrt{3})
\]
\[
    \sqrt{7} - (\sqrt{7} - \sqrt{3}) = \sqrt{3} \in \QQ(\sqrt{7}, \sqrt{3})
\]

\newcommand{\The}{{\theta}}

Thus, $K = \QQ(\sqrt{7}+ \sqrt{3})$. The degreee of the extension 
$K/\QQ$ equals to the minimum polynomial of the generator. In search 
for a minimum polynomil, we first observe some relations around 
the generator. 
For convinience, let $\The := \sqrt{7} + \sqrt{3}$. We have shown 
above that:

\[
    4/\The = \sqrt{7}-\sqrt{3} = \The - 2\sqrt{3}
\]

\[
    \The - 4/\The = 2\sqrt{3} 
    \textAnd 
    (\The - 4/\The)^2 = 12
\]
\[
    \The^2-8+16/\The^2 = 6
\]
\[
    \The^4 -20\The^2 + 16 = 0
\]

We purport that the polynomial $\The^4 -20\The^2 + 16 $ 
is irrecucible in $\ZZ$. Apply the mod 5 test to show that the 
equation has no linear factors. Assuming the polynomial has 
at least one lienar factor, the following must hold:

\[
    \The^4 + 1 \equiv 0 \mod 5
    \textOr 
    \The^4 \equiv -1 \mod 5
\]

With some basic arithmetic, we verify that $-1$ is not a 
quintic residue of $5$. Hence there are no linear factors. 

For the quintic polynomial has no lienar factors, it must 
factor into two quadratic polynomials assuming it is reducible. 
We write:

\[
    (\The^2+ a\The + b)(\The^2 -a\The + c) =   \The^4 -20\The^2 + 16
\]

where $a, b, c \in \ZZ$

The coefficient of $\The$ of the two linear factors must differ by 
a factor of $-1$ for their product to have no terms of $\The^3$. 
Furthermore, comparing the coefficient of $\The$, we observe 
$ac -ab = a(c-b)= 0$. Either $a = 0$ or $c-b = 0$. If $a = 0$, 
applying the substitution $x := \The^2$, we write:

\[
    (x+b)(x+c) = x^2-20x+16
\]

By the quadratic formula, $b = 10 \pm \sqrt{84}$. But then, 
$b \not\in \ZZ$. This is a contradiction. 

We revert to the only alternative $c-b = 0$ or $c = b$. 
\[
  (\The^2+ a\The + b)(\The^2 -a\The + b)
  =
  (\The^2+ b + a\The) (\The^2+ b - a\The)  
  = (\The^2 + b)^2 - a^2\The ^2
  = \The^4 -20\The^2 + 16
\]

So $b^2 = 16$ and $b = \pm 4$. $2b - a^2 = -20$ so 
either $8-a^2 = -20$ or $-8 - a^2 = -20$. $a^2$ must 
either be 28 or 12. In both cases, $a$ is not an integer. 
The polynomial must be irreducible. 

Thus:
\[
    [K:Q] = 4
\]
\qed

\newpage 


\Problem 
Prove that the class number of $\QQ(\sqrt{-13})$ is 2. 

\new{Solution} 
I have decided to write this proof in hand due to 
the excessive amount of simple arithmetic. Proof 
attatched at the end of the solutions. 
\newpage

%---------Q3-----------
\Problem 
Find all integer solutions to the equation 
$x^2+13 = y^3$.

\new{Solution}
We begin with some divisibility relations of $x, y$ in $\ZZ$. 
$13 \nmid x$ and $y$ must be odd. To show $13 \nmid x$, assume 
$13 | x$. From $x^2+13 = y^3$, we observe that $13$ divides the 
LHS so it must also divide the RHS, hence $13|y$. However, $13$ 
divides the LHS exactly once while it divides the RHS thrice. 
We reach a contradiction and conclude $13 \nmid x$. 

It is possible to show $y$ odd in a similar manner. Assume $y$ 
to be even. $x$ must be odd. 
\[y^3 \equiv 0 \mod 4\]
 but 
 \[
    x^2 + 13 \equiv 1 \textOr 2 \mod 4
 \]
so
\[
    x^2+13 \not\equiv y^3 \mod 4
\]
and $y$ must be odd. 

Factor the equation in the ring of integers of the field $\QQ(\sqrt{-13})$. 
\[
    (x-\sqrt{-13})(x+\sqrt{-13}) = y^3
\]
Passing up to the ideals:
\[
    \<x-\sqrt{-13}\>\<x+\sqrt{-13}\> = \<y\>^3
\]

The two ideals are coprime. Assume for a contradiction that 
some prime ideal $\pideal$ divides both ideals. Division implies 
inclusion, so we write:

\[
    \<x-\sqrt{-13}\>, \<x+\sqrt{-13}\> \subseteq \pideal
    \textAnd 
    x \pm \sqrt{-13} \in \pideal
\]

Thus:
\[
    2x, 2\sqrt{-13} \in \pideal
\]
By strong closure of ideals, we deduce that 
$-2\sqrt{-13}^2 = 26$ is an element of $\pideal$. Since 
$13 \nmid x$,

\[
    \gcd(2x, 26) = 2 \textAnd 
    (2x)s+26t = 2
\]
where the latter equation obtained from Bezouts identity. 
$s, t$ are integers. We write $2 \in \pideal$. 

Moreover, $\pideal | \<y\>^3$ and thus $\pideal | \<y\>$ 
and $y \in \pideal$. We have shown previously that $y$ is 
odd. Thus, $y - 2v = 1$ for some integer $v$. This implies 
$1 \in \pideal$ and $\pideal = \ringInt_K$, a contradiction. 
\contradiction

Ideals factor uniquely into prime ideals in any Dedekind Domain. 
Recall the equation:
\[
    \<x-\sqrt{-13}\>\<x+\sqrt{-13}\> = \<y\>^3
\]
Where the two ideals on the left are coprime. It must be:

\[
    \<x-\sqrt{-13}\> = I^3
\]
for some ideal $I$. In light of the class group, the left ideal 
is principal so it belongs to the principal ideal group. $I^3$ must 
also belong to the principal class. We know that the class number of 
$\QQ(\sqrt{-13})$ is 2. If $I$ is not in the principal class, so will 
$I^3$ not be in the principal class. Hence, $I$ must be principal. 

$-13 \equiv 3 \mod 4$ so the integers of this field are in the form 
of $a+b\sqrt{-13}$ for integers $a,b$. The generator must $I$ must 
also be an integer in the field $\QQ(\sqrt{-13})$. Write:

\[
    \<x-\sqrt{-13}\> = \<a+b\sqrt{-13}\>^3 
    = \<(a+b\sqrt{-13})^3\>
\]

The generators of a principal ideal must differ by a factor 
of a unit. We know that the only units of an imaginary quadriatic 
field is $\pm1$. So:

\[
    x-\sqrt{-13} = \pm (a+b\sqrt{-13})^3
    = \pm\bigg([a^3-39ab^2] + \sqrt{-13} [3a^2b -13b^3]\bigg)
\]

Comparing the coefficient of $\sqrt{-13}$:
\[
    3a^2b-13b^3 = \pm 1
\]

$b$ divides the LHS so it must divide the RHS. Thus, $b = \pm 1$. 
Plugging, in, we obtain $3a^2-13 = \pm1$ or $3a^2 = 13\pm1$. The only 
integer solutions to the equation are $a = \pm2$. Comparing the integer 
coefficient of the generator equation:

\[
    x = \pm (a^3-39ab^2) = \pm a (a^2-39) = \pm 70
\]

Each of $x =\pm70$ both yield $y = 17$. There exists no other solution. 

\[
\boxed{(x, y) = (\pm70, 17)}
\]

%--------Q4-----------
\Problem
Prove that if a prime number $p$ is totally ramified in a field 
$K$ and not ramified in the field $L$, then $K \cap L = \QQ$. 

\Proof 
Note that the field $M := K \cap L$ is a field extension of $\QQ$. 
Take any two elements $a, b \in M$. $a - b \in K, L$ for $K, L$ are 
both fields. $a-b \in K \cap L = M$. The multiplicative closure 
and existance of inverses
can be shown in a similar fashion. $\QQ \subseteq K$ and 
$\QQ \subseteq L$ so $\QQ \in L$. 

For $p$ is unramified in the extension $L$, it cannot be 
ramified in the smaller field $M$. We conclude that in the 
ring $\ringInt_M$, $\<p\>\ZZ$ is a prime ideal. 

Note that $\pideal \cap \ringInt_M$ is a prime ideal, for 
the intersect of a prime ideal in a smaller ring is also 
a prime ideal. If the intersect is not a prime ideal, 
two elements outside the prime ideal will multiply to 
be included in the prime ideal, and the same two elements 
will contradict the primeness of the larger ideal, leading 
to a contradiction. 

Also, $\pideal|\pideal^n$ so $\pideal|\<p\>\ZZ$ meaning $p \in \pideal$. 
Thus, $\<p\>\ZZ \subseteq \pideal$, and $\<p\>\ZZ \subseteq \ringInt_M$. Hence, 
$\<p\>\ZZ \subseteq \pideal \cap \ringInt_M$. Inclusion implies division 
for ideals in Dedikind domains, so $\<p\>\ZZ|\pideal \cap \ringInt_M$. 
The latter ideal is assumed to be prime, and we have previously shown 
that the principal ideal generated by $p$ is prime in the ring $\ringInt_M$. 
The two ideals must equal to each other, or in symbols:

\[
    \pideal \cap \ringInt_M = \<p\>\ZZ
\]

Since $p$ is unramified in $M$, we write:

\[
    f( \pideal \cap \ringInt_M|p) = 1
    \textOr
    [\ringInt_M/p\ZZ : \ZZ/p\ZZ] = 1
\]

It must be $\ringInt_M = \ZZ$ which in turn implies 
$M = \QQ$. \qed

\Problem 
Let $K := \QQ(\theta)$ where $\theta$ is a root of the polynomial 
$t^3 + t + 1$. 

i) Prove that $\ringInt_K = \ZZ[\theta]$. 
We claim that the set $\{1, \theta, \theta^2\}$ is an integral basis 
of the field $K$. Theorem 2.17 of the book states that if the 
discriminant of a field is squarefree, the $\QQ$ basis of $K$ 
must be an integral basis of $\ringInt_K$. We know that the set 
$\{1, \theta, \theta^2\}$ is a $\QQ$ basis of the field $K$. Also, 
from HW4, we have derived a formula for discirminants of cubic fields. 

The discriminant of the power basis in the field $\QQ(\theta)$ where $\theta$ where 
$\theta$ is a root of $t^3+at+b$ is:

\[
    \Delta [1, \theta, \theta^2] = -4a^3 - 27b
\]

So for the field $K$, the discriminant is:

\[
    \Delta[1, \theta, \theta^2] = -4-27 = -31
\]

And this value is squarefree. By theorem 2.17, we conclude that 
the power set is indeed a integral basis. The $\ZZ$ span of the 
set  $\{1, \theta, \theta^2\}$ is the set $\ZZ[\theta]$. Thus, 
$\ringInt_K = \ZZ[\theta]$. \qed

ii) provide all the possible decomposition of the principal 
ideal generated by p. 

\new{Solution} The integers of $K$ are elements of $\ZZ[\theta]$. 
Theorem 10.1 of the textbook dictates exactly how the prime ideal 
$\<p\>$ must decompose. Say that minimum polynomial of $\theta$, which 
is $t^3+t+1$, factorize into some product of irreducible $\ZZ_p$ polynomials. 
Write:
\[
    t^3+t+1 = \prod_{i = 1}^{n} f_i(t)^{e_i}
\]
where $n \leq 3$ and all $f_i(t) \in \ZZ_p[t]$ are irreducible. 
Then we know that the factorization of $\<p\>$ is given as:

\[
    \<p\> = \prod_{i = 1}^{n} \pideal_i^{e_i}
\textWhere \pideal_i = \<p, f_i(i)\>
\]

We observe that $t^2 + t + 1$ is irreducible in $\ZZ_2$.
This can be easily shown by the mod 2 test. The cubic 
has no linear factors, and hence is irrecucible.  
Also, $t^2+t+1 = (t^2+t+2)(t-1)$ in $\ZZ_3$, and the 
factors are all irredicible. Irreducibility of 
each terms can also easily be computed by the mod 3 test. 

By Theroem 10.1, write:

\[
    \<2\> \textrm{   Prime}
    \textAnd 
    \<3\> = \<3, \theta - 1\>\<3, \theta^2 + \theta + 2\>
\]

These two examples show that ideal $\<p\>$
can either be a prime ideal or factor into two distinct prime ideals. 
The distinctness also comes from Theorem 10.1. 

It can be easily demonstrated that $\<p\>$ cannot be decomposed into a 
square of some prime ideal. Assume for a contradiction:

\[
    \pideal^2 = \<p\>
\]

Then by taking the norm both sides:

\[
    N(\pideal^2) = N(\<p\>) = p^3
\]
since $[K:Q] = 3$. Let $v$ denote $N(\pideal)$. We obtain $v^2 = p^3$ 
for integer $v$. $p$ is prime and no integer solutions exists, hence a 
contradiction. \contradiction

We assume that $\<p\>$ can decompose into some three distinct 
prime ideals. 

Consider the case that $\<p\>$ decomposes 
into $\pideal_1^2\pideal$. By Theorem 10.1, $t^3+t+1$ must factorize 
in the form of $(t-a)^2(t-b)$ over $\ZZ_p$. 
Write:

\[
    (t-a)^2(t-b) = t^3 +t+1 \polymod p
\]

\[
    x^3-(2a+b)x+a^2+2ab-a^2b = t^3 + t + 1 \polymod p 
\]
Comparing coefficients:
\[
    2a + b = 0 \textAnd a^2+2ab=1 \textAnd -a^2b = 1 \mod p
\]
So $b=-2a$ and $a^2 - 4a^2 = 1$ or $-3a^2 = 1$ or $(-3)a^2 = 1$. 
$a^2(-b) = 1$ and $\ZZ_p$ is a field so the mutiplicative inverse of 
$a^2$ must be unique. $b = 3$ and $a = -b2^{-1} = -3(p+1)/2$. It 
is safe to assume $p$ odd, for we have already ruled out the 
case $p = 2$. 

Plugging in $a = -3(p+1)/2 = (p-3)/2$, $b = 3$ to $a^2b = -1$ yields $p = 31$. 
We conclude:

\[
    (t+17)^2(t-3) = t^3+t+1 \polymod 31
\]

Thus:

\[
    \<31\> = \<31, \theta +17\>^2\<31, \theta -3\>
\]

It is impossible for $\<p\>$ to completely ramify in $K$. If so, again 
by Thm 10.1, we obtain:

\[
    (t-n)^3 = t^3+t+1 \polymod p 
\]

And by comparing coefficients, $3n = 0$, $3n^2=1$, $n^3 = 1$ mod p. 
$(3n)n = (0)n = 1$ so $0 = 1$ mod p, and we reach a contradiction. 

To summarize, all possible decomposition of $\<p\>$ in K are:

\[
\<p\> = \pideal \textOr \pideal_1 \pideal_2 
 \textOr \pideal^2_1\pideal_2 \textOr \pideal_1\pideal_2\pideal_3
\]

\qed

\Problem 
Let $K$ be any number field and $\ringInt_K$ be the ring of integers. 
Prove the following:

i) If $\pi$ is irreducible in $\ringInt_K$ and $\<\pi\>$ is not 
a prime ideal, then $\<\pi\>$ is not contained in any prime ideal.

\Proof
Assume for a contradiction, that $\pi$ is irreducible and $\<\pi\>$ 
is not prime but there exists a prime principal ideal $\<a\>$ that includes  
$\<\pi\>$. Inclusion implies division for ideals in a Dedekind domain. Hence:

\[
    \<a\>|\<\pi\> \textAnd \<\pi\> = I\<a\>
\]
for some ideal $I$. Passing the equation up to the class group:

\[
    [\<\pi\>] = [I][\<a\>] \textAnd [I] = [\<1\>]
\]

So $I$ must be principal. Let $I := \<b\>$. We have 
$\<\pi\> = \<b\>\<a\> = \<ba\>$. The generators of the same 
principal ideal must differ by a unit. For some unit $\mu$:

\[
    \pi = \mu ab 
\]
And we are given that $\pi$ is irreducible. Necessarily, 
$a$ or $b$ must be a unit. If $a$ is a unit, $\<a\>$ will not 
be a proper principal ideal. If $b$ is a unit, $\<pi\> = \<a\>$ 
and $\<\pi\>$ will be prime. \contradiction \qed

ii) Suppose $h_k = 2$. $\pi$ is irreducible in $\ringInt_K$ and 
$\<\pi\>$ is not a prime ideal. Prove that $\<p\>$ decomposes 
into two not necessarily distinct ideals. 

\Proof Decompose the ideal $\<\pi\>$ into a product of prime ideals. 

\[
    \<\pi\> = \prod_{i = 1}^{t}\pideal_i^{e_i}
\]
By part i), we deduce that none of $\pideal_i$ can be principal. 
All of the prime ideals are in the same class group; the non-principal 
group. However, the product of all the prime ideals up to multiplicity, 
form a principal ideal. Relabeling the factors, write:

\[
    \<\pi\> = \prod_{i = 1}^t \pideal_i \qideal_i = \prod_{i = 1}^t\<\alpha_i\>
    = 
    \bigg\langle
    \prod_{i = 1}^t\alpha_i
    \bigg\rangle
\]
where $\pideal_i\qideal_i = \<\alpha_i\>$. For $\pi$ is irreducible, 
without loss of generality, $\alpha_1$ is an associate of $\pi$, 
and all other $\alpha_i$'s are units. For $i > 1$, by assumption, $\pideal_i\qideal_i = \<\alpha_i\> = 
\ringInt_K$. This implies $\pideal|\ringInt_K$, a contradiction. 
Thus, the product simplifies to the first pair of which $\alpha_1$ is an 
associate of $\pi$. In symbols:

\[
    \<\pi\> = \pideal \qideal
\]

\qed

\new{Remark} It follows that two ideals $\pideal, \qideal$ 
are both non-principal. 

iii) Prove that if $h_k = 2$, then the factorization 
of any element has the same number of irreducibles. 

\Proof 

Assume for a contradiction that there exists 
some element that has two factorizations that have different 
number of irreducibles associated with itself. Let $\alpha$ 
be an element in $\ringInt_K$ be such an element with minimum 
number of irreducibles in its shortest factorization. Write:

\[
    \alpha = \beta_1 \beta_2 \cdots \beta_n = 
    \gamma_1 \gamma _2 \cdots \gamma_m
\]
where the beta-factorization is the shortest factorization 
(meaning least irreducible factors) of $\alpha$, so $n<m$.  
None of the beta-gamma pairs can be associates. Ohterwise, 
by cancellation, we obtain an element with a shorter minimal
factorization. 

Passing the equation to ideals:

\[
    \<\alpha\> = \<\beta_1\>\<\beta_2\> \cdots \<\beta_n\> 
    =
    \<\gamma_1\>\<\gamma_2\> \cdots \<\gamma_m\>
\]
We claim that none of the principal ideals generated by 
betas or gammas are prime. Assume for a contradiction, WLOG, that 
$\<\beta_1\>$ was prime. Ideals factor uniquely, so WLOG 
$\<\beta_1\>|\<\gamma_1\>$. If $\<\gamma_1\>$ is not prime, we 
reach a contradiction from part i). So $\<\gamma_1\>$ is prime 
and it must be equal to $\<\beta_1\>$. This implies that 
the generators must differ by a factor of associates, but previously 
we assumed that none of the beta-gamma pairs are associates. \contradiction 

From part ii) we know that each principal ideal generated by 
an irreducible must factor into two prime ideals if it is not prime. 
The beta-ideal multiples decomposes into $2n$ prime ideals while 
the gamma-ideal multiples decompose into $2m$ prime ideals. Ideal 
factorization is unique, so it must be $2n = 2m$, but $n < m$. \contradiction 

So the number of irreducibles in a factorization is invariant. 
\qed

\color{blue}
neat!

\color{black}

\newpage

\new{Bonus} Let ideal $I$ be any ideal in the ring of integers 
of $K := \QQ(\sqrt{d})$. Prove that if $N(I)|d$, then $I^2$ is principal. 

\Proof 
It suffices to prove for all prime ideals $\pideal$. The reason is 
it is possible to decompose $I$ into a product of prime ideals. 
Since $d$ is squarefree, so is $N(I)$. We write:

\[
    I = \prod_{i = 1}^t \pideal_n
\]
Note that each prime ideal can have maximum multiplicity of 1. 
If a prime ideal divided $I$ more than once, the norm of the prime ideal 
will divide $I$ more than twice. $N(\pideal)$ is a power of some prime, 
so $N(I)$ will not be squarefree, leading to a contradiction. 

Focus on a prime ideal $\pideal$ which satisfies $N(\pideal)|d$. 
The norm must be some prime $p$ that divides $d$. Recall that 
the norm of an ideal is also an element of the ideal (Theorem 5.14-b). Thus, 
$p \in \pideal$ an $\<p\> \subseteq \pideal$ and $\pideal | \<p\>$. 

In a quadratic field, we know the prime decomposition of the principal 
ideal $\<p\>$. For the case where $p|d$, $\<p\> = \<p, \sqrt{d}\>^2$. 
It must be the case that $\pideal = \<p, \sqrt{d}\>$. Then, $\pideal^2 = \<p\>$, 
which concludes the proof. 
\qed



\end{document}