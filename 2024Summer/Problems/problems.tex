\documentclass{article}
\usepackage{amsfonts}
\usepackage{amsthm}
\usepackage{amssymb}
\usepackage{amsmath}
\usepackage{graphicx}
\usepackage{subcaption}
\usepackage{xcolor}
\usepackage{mathtools}
\usepackage{ wasysym }
\usepackage{enumerate}
\usepackage{verbatim}



\newcommand{\new}[2]{
    \vspace{2mm}
    \noindent
    \textbf{
    \underline{#1}}
    \textit{{#2}}
    \
    \newline
}

\def\<{{\langle}}
\def\>{{\rangle}}

\DeclarePairedDelimiter\bra{\langle}{\rvert}
\DeclarePairedDelimiter\ket{\lvert}{\rangle}
\DeclarePairedDelimiterX\braket[2]{\langle}{\rangle}{#1\,\delimsize\vert\,\mathopen{}#2}


\newcommand{\textOr}{
    {
        \hspace{5mm}
        \textrm{or}
        \hspace{5mm}
    }
}

\newcommand{\textAnd}{
    {
        \hspace{5mm}
        \textrm{and}
        \hspace{5mm}
    }
}


\newcommand{\textWhere}{
    {
        \hspace{5mm}
        \textrm{where}
        \hspace{5mm}
    }
}



\newcommand{\Ixp}[1]{
    {
        e^{i{#1}}
    }
}



\newcommand{\halfFigure}[1]{
\begin{center}
\includegraphics[width = .5\linewidth]{{#1}}
\end{center}
}

\newcommand{\fullFigure}[1]{
\begin{center}
\includegraphics[width = .9\linewidth]{{#1}}
\end{center}
}

\def\twobytwoMat(#1, #2, #3, #4){
    {
        \begin{bmatrix}
            {#1} & {#2}\\
            {#3} & {#4}
        \end{bmatrix}
    }
}

\def\twobyoneMat(#1, #2){
    {
        \begin{bmatrix}
            {#1}\\
            {#2}
        \end{bmatrix}
    }
}

\def\twobytwoDet(#1, #2, #3, #4){
    {
        \begin{vmatrix}
            {#1} & {#2}\\
            {#3} & {#4}
        \end{vmatrix}
    }
}


\newcommand{\RR}{\mathbb{R}}
\newcommand{\CC}{\mathbb{C}}
\newcommand{\ZZ}{\mathbb{Z}}
\newcommand{\Zpos}{\mathbb{Z}_{pos}}
\newcommand{\NN}{\mathbb{N}}
\newcommand{\EE}{\mathbb{E}}

\newtheorem{theorem}{Theorem}
\newtheorem{prop}{Proposition}
\newtheorem{lemma}{Lemma}
\newtheorem{cor}{Corollary}
\newtheorem{remark}{Remark}
\newtheorem{definition}{Definition}
\newtheorem{ex}{Example}
\newtheorem{conj}{Conjecture}
\newtheorem{question}{Question}

\newcommand{\ch}{\textnormal{ch}}
\newcommand{\tr}{\textnormal{tr}}

\begin{document}
\begin{center}
    \Large
    \textbf{Notes as of Aug 22}

    \large
    Benevolent Tomato
\end{center}

\section{Free Probability}
We have proved two theorems regarding the expected trace of GOE's

\new{Theorem 7}{Even moments of the GOE are Catalan Numbers}
\begin{eqnarray}
    \EE[\tr(X^n)] \ = \ c_n \ = \  \begin{cases}
        0 & 2 \nmid k \\ 
        C_{k/2} & 2 | k
    \end{cases}
\end{eqnarray}

\new{Theorem 10}{Product of the centered moments converge to zero}
Suppose $X_1, X_2, \dots, X_m$ are independent $N$-by-$N$ GOEs. As 
$N \rightarrow \infty$, the following value converges to zero. 

\begin{eqnarray}
    \EE \left[
        \tr\left(
            X_1^{r_1} - c_{r_1}
        \right)\cdots 
\left(
            X_m^{r_m} - c_{r_m}
        \right)
    \right] \ = \ 0
\end{eqnarray}

We can generalize the concept of expected trace. Let $\varphi$ 
denote the expected trace in an abstract sense. We call the function 
to be a \textbf{state function} under two additional conditions. 
\begin{eqnarray}
    \varphi(1) \ = \ 1 \nonumber \\
    \varphi(a* a) \ \geq a \hspace{5mm} \forall a \in \mathcal{A} 
\end{eqnarray}

The second condition is necessary to make $\mathcal A$ a $*-$algebra. 

Motivated by the space of all random matricies, we define a probability 
space $(\mathcal A, \varphi)$ where $\mathcal A$ is some unital algebra. 
The state function links the algebra element to some complex number. 

A $*-$ probability space $(\mathcal A, \varphi)$ is \textbf{faithful} 
if 
\begin{equation}
    \varphi(x*x) = 0 \ \textnormal{iff} \ x = 0 
\end{equation}
ity space $(\mathcal A, \varphi)$ is \textbf{non-degenerate} 
if 
\begin{eqnarray}
    \varphi(yx) = 0 \ \forall y \in \mathcal A \ \rightarrow \ x = 0\nonumber\\
    \varphi(xy) = 0 \ \forall y \in \mathcal A \ \rightarrow \ x = 0 
\end{eqnarray}
Also, by the means of induction, it is possible to deduce the following. 

\vspace{3mm}
\new{Proposition 13}{Recovering the global state from local states}
Suppose $\mathcal A_1, \mathcal A_2, \dots, \mathcal A_m$ are 
free subalgebras of a $*-$probability space $\mathcal A$. The local 
state functions $\varphi|_{\mathcal A_i}$ for $1 \leq i \leq m$ determines 
the global state function $\varphi$

\section{Putnam Problems}
\new{1991B6}{Hyperbolic sine inequality}
Let $a, b$ be positive numbers. Find the largest $c$, in terms of $a, b$ that 
satisfies the inequality
\begin{equation}
    a^x b^{1-x} \ \leq a \frac {
        \sinh(ux)
    } {\sinh(u)}
    + b \frac{
        \sinh(u (1-x))
    }{\sinh(u)}
\end{equation}

Apply the substitution $v:=e^u$ and $r:= a/b$. Then, guess an appropriate 
value of $c$ where the inequality turns to an equality. Afterwards, show that 
any value of $c$ greater than the guessed value has a $x$ that bears 
a witness to the failiture of the inequality. 

\new{1991B3}{Using the Postage Stamp Theorem}
Given $a, b \in \mathbb Z_{pos}$, we know that for any equation 
\begin{equation}
    ax + by \ = \ s
\end{equation}
where $\textnormal{gcd}(a, b) = 1$, there exists some $N > 0$ where 
there always exists a solution for the equation for any $s \geq N$. 
We can prove this by considering the residues of the set 
\begin{equation}
    \{0, a, 2a, \dots, (b-1) a\}
\end{equation}
which must be a compete set of residues mod b. 

\new{1991B2} {Cauchy's Lemma}
If the function $f:\RR \rightarrow \RR$ is an additive continuous automorphism of 
the reals, that is for all $a, b \in \RR$ 
\begin{equation}
    f(a + b) \ =\ f(a) + f(b)
\end{equation}
$f(x)$ must be in the form of 
\begin{equation}
    f(x) \ = \ cx
\end{equation}
\begin{proof}
Applying the additive automorphism, it is easy to verify that this must be 
true for all the rationals. Thus, the function $f(x) - cx$ must vanish 
at all the rationals, and by continuity, at all the reals. 
\end{proof}

\end{document}