\documentclass{article}
\usepackage{amsfonts}
\usepackage{amsthm}
\usepackage{amssymb}
\usepackage{amsmath}
\usepackage{graphicx}
\usepackage{subcaption}
\usepackage{xcolor}
\usepackage{mathtools}
\usepackage{ wasysym }


\newcommand{\new}[1]{
    \vspace{2mm}
    \noindent
    \textbf{
    \underline{#1}}
}

\def\calO{{\mathcal{O}}}
\def\th{{\theta}}
\def\_{{\hspace{1mm}}}
\def\<{{\langle}}
\def\>{{\rangle}}

\DeclarePairedDelimiter\bra{\langle}{\rvert}
\DeclarePairedDelimiter\ket{\lvert}{\rangle}
\DeclarePairedDelimiterX\braket[2]{\langle}{\rangle}{#1\,\delimsize\vert\,\mathopen{}#2}



\newcounter{problemcnt}
\setcounter{problemcnt}{0}

\newcommand{\Problem}{{
    \vspace{5mm}
    \stepcounter{problemcnt}
    \noindent
    \arabic{problemcnt}. 
}
}

\newcommand{\nProblem}[1]{
    \vspace{5mm}
    \noindent
    \setcounter{problemcnt}{#1}
    \arabic{problemcnt}. 
}


\newcommand{\Proof}{{
    \vspace{2mm}
    \noindent
    \textbf{
    \underline{Proof}}
}
}

\newcommand{\textOr}{
    {
        \hspace{5mm}
        \textrm{or}
        \hspace{5mm}
    }
}

\newcommand{\textAnd}{
    {
        \hspace{5mm}
        \textrm{and}
        \hspace{5mm}
    }
}

\newcommand{\Ixp}[1]{
    {
        e^{i{#1}}
    }
}



\newcommand{\halfFigure}[1]{
\begin{center}
\includegraphics[width = .5\linewidth]{{#1}}
\end{center}
}

\newcommand{\fullFigure}[1]{
\begin{center}
\includegraphics[width = .9\linewidth]{{#1}}
\end{center}
}

\def\twobytwoMat(#1, #2, #3, #4){
    {
        \begin{bmatrix}
            {#1} & {#2}\\
            {#3} & {#4}
        \end{bmatrix}
    }
}

\def\twobyoneMat(#1, #2){
    {
        \begin{bmatrix}
            {#1}\\
            {#2}
        \end{bmatrix}
    }
}

\def\twobytwoDet(#1, #2, #3, #4){
    {
        \begin{vmatrix}
            {#1} & {#2}\\
            {#3} & {#4}
        \end{vmatrix}
    }
}



\begin{document}
\begin{center}
\LARGE
PHYS 314 HW7

\Large
Daniel Son
\end{center}

\new{Q2 no-cloning theorem}
a) Consider a qualtum controlled-NOT gate. 
This gate seems to copy the states for 
\[
    \ket{\psi} = \ket{0}, \ket{1}
\]. 
Does this gate violate the no-cloning theorem?

\new{Solution}
No, the no-cloning theorem introduced in Townsend 
tells us that there does not exist a unitary operator 
the copies a general quantum state. The c-NOT gate 
successfully clones the $\ket{0}, \ket{1}$ state, 
but it fails for an entangled state, for example 
\[
    \ket{\psi} = \frac 1 {\sqrt 2} (\ket{0}+\ket{1})
\]. An attempt to copy $\ket{\psi}$ through the c-NOT 
gate results in a state 
\[
\begin{bmatrix}
    \frac 1 {\sqrt 2}
    \\
    0
    \\0
    \\
\frac 1 {\sqrt 2}
\end{bmatrix}
\]
The correct copy must result in a state 
\[
    \twobyoneMat(\frac 1 {\sqrt 2}, \frac 1 {\sqrt 2})
    \otimes 
    \twobyoneMat(\frac 1 {\sqrt 2}, \frac 1 {\sqrt 2})
    = 
\begin{bmatrix}
    1/2 \\ 
    1/2 \\ 
    1/2 \\ 
    1/2
\end{bmatrix}
\]

And clearly the two states do not match which 
leads to a contradiction. 
\hfill \lightning 

\vspace{.5cm}
b) By using the method of Quantum Teleportation, Alice 
can send a quantum state exactly by using entanglement 
and sending two classical bits. Now, assume Bob recieved 
a cubit from Alice and Bob made a measurement. How much 
information about $\{\theta, \phi\}$ can Bob retrieve 
from this experiemnt?

\new{Solution}
Suppose Bob recieves a state 
\[
    \ket{\psi} = \twobyoneMat(\cos(\theta/2), {e^{i \phi} \sin(\theta / 2) })
\]
We can retrieve the probability that $\ket{\psi}$ 
will collapse to either $\ket{0}$ or $\ket{1}$. 
\[
    P(0) = \cos^2(\theta/2 )
    \textAnd 
    P(1) = \sin^2(\theta/2)
\]
Depending on Bob's measurement, we can claim that the 
probability that $\ket{\psi}$ will collapse to the 
measured state is more likely. If Bob measures 1, then 
it is likely that 
\[
    \theta \in [\frac \pi 4, \frac {3\pi} 4] 
    \cup  [\frac {5\pi} 4, \frac {7\pi} 4]
\]
. This method does not allow us to make any claims about the 
phase $\phi$. 
\vspace{.5cm}

c) What if Bob is allowed many duplicates of the same 
qubit?

\new{Solution}
It would be possible to narrow down the exact value of 
$\theta$. Still, it would be impossible to recover 
the value of $\phi$



\end{document}

