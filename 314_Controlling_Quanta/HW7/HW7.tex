\documentclass{article}
\usepackage{amsfonts}
\usepackage{amsthm}
\usepackage{amssymb}
\usepackage{amsmath}
\usepackage{graphicx}
\usepackage{subcaption}
\usepackage{xcolor}
\usepackage{mathtools}
\usepackage{ wasysym }


\newcommand{\new}[1]{
    \vspace{2mm}
    \noindent
    \textbf{
    \underline{#1}}
}

\def\calO{{\mathcal{O}}}
\def\th{{\theta}}
\def\_{{\hspace{1mm}}}
\def\<{{\langle}}
\def\>{{\rangle}}

\DeclarePairedDelimiter\bra{\langle}{\rvert}
\DeclarePairedDelimiter\ket{\lvert}{\rangle}
\DeclarePairedDelimiterX\braket[2]{\langle}{\rangle}{#1\,\delimsize\vert\,\mathopen{}#2}



\newcounter{problemcnt}
\setcounter{problemcnt}{0}

\newcommand{\Problem}{{
    \vspace{5mm}
    \stepcounter{problemcnt}
    \noindent
    \arabic{problemcnt}. 
}
}

\newcommand{\nProblem}[1]{
    \vspace{5mm}
    \noindent
    \setcounter{problemcnt}{#1}
    \arabic{problemcnt}. 
}


\newcommand{\Proof}{{
    \vspace{2mm}
    \noindent
    \textbf{
    \underline{Proof}}
}
}

\newcommand{\textOr}{
    {
        \hspace{5mm}
        \textrm{or}
        \hspace{5mm}
    }
}

\newcommand{\textAnd}{
    {
        \hspace{5mm}
        \textrm{and}
        \hspace{5mm}
    }
}

\newcommand{\Ixp}[1]{
    {
        e^{i{#1}}
    }
}



\newcommand{\halfFigure}[1]{
\begin{center}
\includegraphics[width = .5\linewidth]{{#1}}
\end{center}
}

\newcommand{\fullFigure}[1]{
\begin{center}
\includegraphics[width = .9\linewidth]{{#1}}
\end{center}
}

\def\twobytwoMat(#1, #2, #3, #4){
    {
        \begin{bmatrix}
            {#1} & {#2}\\
            {#3} & {#4}
        \end{bmatrix}
    }
}

\def\twobyoneMat(#1, #2){
    {
        \begin{bmatrix}
            {#1}\\
            {#2}
        \end{bmatrix}
    }
}

\def\twobytwoDet(#1, #2, #3, #4){
    {
        \begin{vmatrix}
            {#1} & {#2}\\
            {#3} & {#4}
        \end{vmatrix}
    }
}



\begin{document}
\begin{center}
\LARGE
PHYS 314 HW7

\Large
Daniel Son
\end{center}

\new{Q1 Classical Circuits}

\newcommand{\NAND}{\textrm{NAND}}

a) Express the OR gate in terms of AND and XOR. 

Adding up the entries of the truth table yields 
the desired result. Thus,
\[
    a \vee b = 
    (a \wedge b) \oplus (a \oplus b) 
\]

b) Express the XOR gate in terms NOT, AND, OR gates 
\[
    a \oplus b 
    = 
    [\neg(a \wedge b)] \wedge (a \vee b)
\]

c) Express AND, OR, XOR, solely in terms of NAND and FANOUT. 

We first establish the NOT gate. 
\[
    \neg a = \NAND(1, a)
\]

The AND gate is a negation of the NAND gate. 
\[
    a \wedge b = \NAND(1, \NAND(a, b))
\]

The OR gate can be easily deduced by DeMorgan's Law. 
\[
    a \vee b = \NAND(\NAND(1, a), \NAND(1, b))
\]

In part b, we have shown how to write an XOR gate 
with NOT, AND, OR gates. Thus, write 
\[
    a \oplus b = \NAND(a, b) \wedge \NAND(\NAND(1, a), \NAND(1, b))
\]
\[
    = \NAND(1, \NAND(\NAND(a, b), \NAND(\NAND(1, a), \NAND(1, b))))
\]

\new{Q2 no-cloning theorem}
a) Consider a qualtum controlled-NOT gate. 
This gate seems to copy the states for 
\[
    \ket{\psi} = \ket{0}, \ket{1}
\]. 
Does this gate violate the no-cloning theorem?

\new{Solution}
No, the no-cloning theorem introduced in Townsend 
tells us that there does not exist a unitary operator 
the copies a general quantum state. The c-NOT gate 
successfully clones the $\ket{0}, \ket{1}$ state, 
but it fails for an entangled state, for example 
\[
    \ket{\psi} = \frac 1 {\sqrt 2} (\ket{0}+\ket{1})
\]. An attempt to copy $\ket{\psi}$ through the c-NOT 
gate results in a state 
\[
\begin{bmatrix}
    \frac 1 {\sqrt 2}
    \\
    0
    \\0
    \\
\frac 1 {\sqrt 2}
\end{bmatrix}
\]
The correct copy must result in a state 
\[
    \twobyoneMat(\frac 1 {\sqrt 2}, \frac 1 {\sqrt 2})
    \otimes 
    \twobyoneMat(\frac 1 {\sqrt 2}, \frac 1 {\sqrt 2})
    = 
\begin{bmatrix}
    1/2 \\ 
    1/2 \\ 
    1/2 \\ 
    1/2
\end{bmatrix}
\]

And clearly the two states do not match which 
leads to a contradiction. 
\hfill \lightning 

\vspace{.5cm}
b) By using the method of Quantum Teleportation, Alice 
can send a quantum state exactly by using entanglement 
and sending two classical bits. Now, assume Bob recieved 
a cubit from Alice and Bob made a measurement. How much 
information about $\{\theta, \phi\}$ can Bob retrieve 
from this experiemnt?

\new{Solution}
Suppose Bob recieves a state 
\[
    \ket{\psi} = \twobyoneMat(\cos(\theta/2), {e^{i \phi} \sin(\theta / 2) })
\]
We can retrieve the probability that $\ket{\psi}$ 
will collapse to either $\ket{0}$ or $\ket{1}$. 
\[
    P(0) = \cos^2(\theta/2 )
    \textAnd 
    P(1) = \sin^2(\theta/2)
\]
Depending on Bob's measurement, we can claim that the 
probability that $\ket{\psi}$ will collapse to the 
measured state is more likely. If Bob measures 1, then 
it is likely that 
\[
    \theta \in [\frac \pi 4, \frac {3\pi} 4] 
    \cup  [\frac {5\pi} 4, \frac {7\pi} 4]
\]
. To find the value of $\phi$, repeat the measurement with 
respect to a different axis. 
\vspace{.5cm}

c) What if Bob is allowed many duplicates of the same 
qubit?

\new{Solution}
It would be possible to narrow down the exact value of 
$\theta$. Still, it would be impossible to recover 
the value of $\phi$



\new{Q4} The controlled Z-gate has the following matrix form. 

\[
    Z_c := 
    \begin{pmatrix}
    1 &0 &0 &0 \\
    0 &1 &0& 0 \\
    0 &0& 1& 0\\
    0 &0& 0& -1
\end{pmatrix}
\]

\new{a)} Show that, up to a global phase, this gate 
can be generated by the time evolution operator of 
the following Hamiltonian. 

\[
    \hat H 
    := \hbar
    \omega_1 (Z \otimes I + I \otimes Z) 
    + \hbar \omega_2 Z \otimes Z
\]
for some appropriate value $\omega_1 t, \omega_2 t$. 

\new{Solution}
We wish to accomplish the identity 
\[
    \exp \left(
        \frac {\hat H t} {i \hbar }
    \right) 
    = e^{i\phi} Z_c
\]
for some phase factor $\phi$. 
Expanding both sides into matricies, we 
rewrite the identity as follows. 

\[
    \begin{split}
    \begin{pmatrix}
        \exp \left(
            - \frac{i t}{\hbar} (2\omega_1 + \omega_2)
        \right) & & &
        \\
        & \exp \left(
            \frac{it}{\hbar} \omega_2
        \right) & &
        \\ & &
\exp \left(
            \frac{it}{\hbar} \omega_2
        \right) &
        \\ & & &
        \exp \left(
            - \frac{i t}{\hbar} (-2\omega_1 + \omega_2)
        \right)
    \end{pmatrix}
    \\=
    \begin{pmatrix}
        e^{i\phi} &&&\\
        &e^{i\phi} &&\\
        &&e^{i\phi} &\\
        &&&-e^{i\phi} 
    \end{pmatrix}
\end{split}
\]
Comparing the diagonal entries, we obtain three 
distinct equations. Upon inspection, we 
guess a solution. 
\[
    \boxed{
    \omega_1 t = \frac \pi 4 
    \textAnd 
    \omega_2 t = -\frac \pi 4
    }
\]
where $\phi = -\frac \pi 4$. 

\new{b} 
The Hadamard Gate is defined as 
\[
    H := 
    \frac 1 {\sqrt{2}}
    \begin{pmatrix}
        1& 1 \\
        1 & -1
    \end{pmatrix}
\]
Note that the Hadamard gate can be considered 
as an active change of the basis vectors. 
\[
    \hat z \mapsto \hat x \textAnd 
    \hat x \mapsto \hat z
\]
The transformation goes both directions. Thus, 
the Hadamard gate can be considered a "flip"
of $\hat{x}$ and $\hat{z}$. Show that 
a $X_c$ gate is equivalent to $HZ_cH$

\new{Solution} 
If the control channel has a state $\ket{0}$, the 
bottom gate will perform the operation $H^2 = I$. 
It suffices to show $HZH = X$. 
Plug the equation into mathematica. 
\halfFigure{HadamardSwap.png}
\hfill \qed


\new{c)} 
Recall the Hamiltonian for the hyperfine problem. 
\[
    \hat H := 
    \omega_1 (X \otimes X + Y \otimes Y + Z \otimes Z)
\]
evolution of this Hamiltonian for time $t = \pi\hbar / (4\omega_1 )$
corresponds to the swap gate. Prove and justify this claim. 

\new{Solution}
The swap gate is 
\[
    \begin{pmatrix}
        1 & 0 & 0 & 0 \\
        0 & 0 & 1 & 0 \\
        0 & 1 & 0 & 0 \\
        0 & 0 & 0 & 1
    \end{pmatrix}
\]
The time evolution operator of the Hamiltonian is 
\[
    \hat U(t) = \exp \left(
        \frac{\hat H t}{i \hbar}
    \right)
\]
Simplify the argument of the matrix exponential. 
Also, remember the condition of $t$. 
\[
    \frac{\hat H t}{i \hbar} 
    = 
    \frac {
        \omega_1 t (X \otimes X + Y \otimes Y + Z \otimes Z)
    } {i \hbar}
    = 
    -\frac{i\pi}4 (X \otimes X + Y \otimes Y + Z \otimes Z)
\]
Thus, 
\[
    \hat U(t) = \exp \left(
    -\frac{i\pi}4 (X \otimes X + Y \otimes Y + Z \otimes Z)
    \right)
\]
which can be computed by plugging in to Mathematica. 
\fullFigure{SwapGate.png}

Up to a global phase of $e^{-i\pi / 4}$, the 
time evolution operator is a swap gate. 


\new{Q5 BSM and Teleportation}

\new{a} Figure 4.11 of KLM shows a circuit that 
conducts a Bell State Measurement. Prove this claim. 
\halfFigure{BSMGate.png}

The physical system of this two wire quantum circuit 
can be descirbed as a tensor product of two Hilbert spaces. 
Let $\ket 0 _u, \ket 1_u$ be the basis for the upper 
channel and $\ket 0_d, \ket 1_d$ be the basis for the lower 
channel. The combined state has four bases, namely 
\begin{align*}
    \ket{00}:= \ket0_u \ket 0_d && \ket{01} := \ket 0_u \ket 1_d \\
    \ket{10}:= \ket 1_u \ket 0_d && \ket{11}:= \ket 1_u \ket 1_d
\end{align*}

The circuit first applies the Hadamard gate 
to the first channel, then applies a controlled-X 
gate to the second channel which is controlled by the first 
channel. The operator of this circuit is 
\[
    B:= (H \otimes I)X_c
\]

Considering the operator $B$ as an passive linear transformation, 
we find the basis of the quantum state after 
the signal passes the gate. We must obtain the inverse of $B$. 

\[
    B^\dagger = X_c^\dagger (H\otimes I)^\dagger 
    = X_c (H\otimes I) = 
    \frac 1 {\sqrt{2}}
    \begin{pmatrix}
        1 & 0 & 1 & 0 \\
        0 & 1 & 0 & 1 \\ 
        0 & 1 & 0 & -1 \\ 
        1 & 0 & -1 & 0
    \end{pmatrix}
\]

The four column vectors represent the basis of this passive transform. 

\begin{align*}
    \frac 1 {\sqrt 2} (\ket{00}+\ket{11}) = \ket{\Phi^+}
    && \frac 1 {\sqrt 2}(\ket{01} + \ket{10}) = \ket{\Psi^+}
    \\
    \frac 1{\sqrt 2} (\ket{00} - \ket{11}) = \ket{\Phi^-}
    && \frac 1 {\sqrt 2} (\ket {01} - \ket{10}) = \ket{\Psi^-}
\end{align*}
These four bases are exactly the Bell Bases. 
\hfill \qed

c) Suppose Alice witheld the result of the classical measurement 
after the measurement was made. Describe Bob's state. 

\new{Solution}
All four Bell basis states are equally likely. Hence, the solution 
is the classical average of all the four Bell states. 

\[
    \ket \phi = 
    \frac 1 4 (\ket {\Psi^+} + \ket{\Psi ^-} + \ket{\Phi ^+}+\ket {\Phi^+})
\]
\[
    \boxed{
    = \frac 1 {2\sqrt 2} (\ket{00} + \ket {01})    
    }
\]

\end{document}

