\documentclass{article}
\usepackage{amsfonts}
\usepackage{amsthm}
\usepackage{amssymb}
\usepackage{amsmath}
\usepackage{graphicx}
\usepackage{subcaption}
\usepackage{xcolor}

\usepackage{mathtools}
\DeclarePairedDelimiter\bra{\langle}{\rvert}
\DeclarePairedDelimiter\ket{\lvert}{\rangle}
\DeclarePairedDelimiterX\braket[2]{\langle}{\rangle}{#1\,\delimsize\vert\,\mathopen{}#2}





\newcommand{\new}[1]{
    \vspace{2mm}
    \noindent
    \textbf{
    \underline{#1}}
}

\def\calO{{\mathcal{O}}}
\def\th{{\theta}}
\def\_{{\hspace{1mm}}}
\def\<{{\langle}}
\def\>{{\rangle}}


\newcounter{problemcnt}
\setcounter{problemcnt}{0}

\newcommand{\Problem}{{
    \vspace{5mm}
    \stepcounter{problemcnt}
    \noindent
    \arabic{problemcnt}. 
}
}

\newcommand{\nProblem}[1]{
    \vspace{5mm}
    \noindent
    \setcounter{problemcnt}{#1}
    \arabic{problemcnt}. 
}


\newcommand{\Proof}{{
    \vspace{2mm}
    \noindent
    \textbf{
    \underline{Proof}}
}
}

\newcommand{\textOr}{
    {
        \hspace{5mm}
        \textrm{or}
        \hspace{5mm}
    }
}

\newcommand{\textAnd}{
    {
        \hspace{5mm}
        \textrm{and}
        \hspace{5mm}
    }
}


\newcommand{\Ixp}{
    {
        \textrm{Ixp}
    }
}




\newcommand{\halfFigure}[1]{
\begin{center}
\includegraphics[width = .5\linewidth]{{#1}}
\end{center}
}

\newcommand{\fullFigure}[2]{
\begin{center}
\includegraphics[width = .9\linewidth]{{#1}}
\end{center}
}

\def\twobytwoMat(#1, #2, #3, #4){
    {
        \begin{bmatrix}
            {#1} & {#2}\\
            {#3} & {#4}
        \end{bmatrix}
    }
}

\def\twobyoneMat(#1, #2){
    {
        \begin{bmatrix}
            {#1}\\
            {#2}
        \end{bmatrix}
    }
}

\def\twobytwoDet(#1, #2, #3, #4){
    {
        \begin{vmatrix}
            {#1} & {#2}\\
            {#3} & {#4}
        \end{vmatrix}
    }
}
\begin{document}
\begin{center}
\LARGE
PHYS 314 Formula Sheet

\Large
Daniel Son
\end{center}

\new{Singlet States and Superposition}

The singlet state is defined as follows.

\[
    \frac{1}{\sqrt{2}}
    (\ket{z_+}\ket{z_-} - \ket{z_-} \ket{z_+})
\]

Computing the probability outcomes of the singlet state 
into all the possible output states, we determine that 
the particle never collapses to the up-up or down-down state. 
This means that the spin of one particle decides the spin of the other. 
This is called \textit{Superposition}. 

\new{Raising and Lowering Operators}
The raising and the lowering operators are conventionally 
defined on the \textbf{z-axis}. The operator bumps up 
the state by one. Here is the definition along with an example. 

\[
    \hat{S}_+ := \hat{S}_x + i\hat{S}_y
    \textAnd 
    \hat{S}_- := \hat{S}_x - i\hat{S}_y
\]

\[
    \hat{S}_+ \ket{s, j} = 
    \sqrt{s(s + 1) - j(j + 1)}
    \ket{s, j + 1}
\]

Also, spin operators are Hermitian, so $\hat{S_+}^\dag = \hat{S_-}^\dag$

\new{Product of spins to Product of Raising/Lowering Ops}
The sum of the square of all the spin operators of each direction 
is a natural operator with importance. It is possible to 
express this operator, which is dependant on all three axes, into 
a sum of three products that involve only the z-axis operators. 
The tensor product of vectors/operators behave nicely. In light of 
this fact with the definition of the raising/lowering operator, we 
conclude 
\[
    2\hat{S_{1x}}\hat{S_{2x}}+
    2\hat{S_{1y}}\hat{S_{2y}} + 
    2\hat{S_{1z}}\hat{S_{2z}} = 
    \hat{S}_{1+}\hat{S}_{2-} + \hat{S}_{1-}\hat{S}_{2+} + 2\hat{S_{1z}}\hat{S_{2z}} 
\]

\new{Applications}
It is possible to express a higher order spin, say spin-3/2, as a 
tensor product of two lower spins, spin-1 and spin-1/2. The idea 
is to take the lowest and the highest value of the spin states, 
and to apply the total momentum operator $\hat{J}^2 := (\hat{J}_1 + \hat{J}_2)^2$.
The operator $\hat{J_1}\hat{J_2}$ can be expressed entirely in terms 
of the z-axis operators. It might be tempting to simply apply each 
of the operators separately, but the base states are eigenvectors 
of $\hat{J_i}^2$.  
\[
    \hat{J_1}\hat{J_2} =     2\hat{J_{1x}}\hat{J_{2x}}+
    2\hat{J_{1y}}\hat{J_{2y}} + 
    2\hat{J_{1z}}\hat{J_{2z}} 
\]

Refer to Townsend Appendix B for details. In conclusion, this 
equation will show that the tensor product of the states spin-1/2 
and spin-1 will yield all the states in spin-3/2 and spin-1/2. 

Also, for entangled states, the raising and the lowering operator is 
defined as follows. 
\[
    \hat{S}_+ := \hat{S}_{1+}\cdot 1_2 + 1_1\cdot \hat{S}_{2+} 
    \textAnd 
    \hat{S}_- := \hat{S}_{1-}\cdot 1_2 + 1_1\cdot\hat{S}_{2-}
\]
Where $S$ is defined to be the entangled spin of two particles. 
For an intuitive justification, consider the rotation generator. 
The sum of the two operators are the terms that survive as $d\theta \rightarrow 0$. 

\new{Types of Operators and their properties}
We have the spin operator $\hat{S} := \hat{S}_x + \hat{S}_y + \hat{S}_z$. 
The nice property is that the magnitude of the combined spin $\hat{S}^2$ 
commutes with individual axis spin. That is:

\[
    [\hat{S}^2, \hat{S}_i] = 0
\]

We can take the tensor prouct of two entangled particles. That is 
\[ \hat{S}_1 \cdot \hat{S}_2 :=
\hat{S}_{1x}\hat{S}_{2x}+
    \hat{S}_{1y}\hat{S}_{2y} + 
    \hat{S}_{1z}\hat{S}_{2z}
\]
\halfFigure{TownSend_5_27.png}
Note that any state is a eigenvector of $\hat{S}^2$. 
Refer to Tonwsend CH5 for more information. 

\new{Hyperfine energy structure of the Hydrogen Atom}
A hydrogen atom can be considered as as a system of two 
spin-1/2 particles, proton and the electron. The Hamiltonian 
of the system is defined as $\hat{H} = \frac{2A}{\hbar^2}\hat{S}_1\cdot \hat{S}_2$. 
With some computation, we recognize that there are two energy 
eigenvalues and four energy eigenvectors. 

\[
    \hat{H} = 
    \begin{bmatrix}
        A/2 & 0 & 0 & 0 \\
        0 & -A/2& A & 0 \\
        0 & A & -A/2& 0 \\
        0 & 0 & 0 &A/2
    \end{bmatrix}
\]




The higher energy eigenvalue is $A/2$, and the 
corresponding states are the two product states (top/bottom) and 
the entangled state which has the same signs. In symbols, 

\[
    \lambda = \frac{A}{2} \textAnd \ket{\psi} = \ket{z_+}\ket{z_+}, 
    \ket{z_-}\ket{z_-}, 
    \frac{1}{\sqrt{2}}(\ket{z+}\ket{z-} + \ket{z_-}\ket{z_+})
\]

The lower energy eigenvalue is $-3A/2$, and the 
corresponding state is the singlet state. In symbols, 


\[
    \lambda = -\frac{3A}{2} \textAnd \ket{\psi} = 
    \frac{1}{\sqrt{2}}(-\ket{z+}\ket{z-} + \ket{z_-}\ket{z_+})
\]
Remember that $\frac{2}{\hbar^2}S_1S_2$ has the eigenvalue of $1/2, -3/2$.  
\end{document}

