\documentclass{article}
\usepackage{amsfonts}
\usepackage{amsthm}
\usepackage{amssymb}
\usepackage{amsmath}
\usepackage{graphicx}
\usepackage{subcaption}
\usepackage{xcolor}

\usepackage{mathtools}
\DeclarePairedDelimiter\bra{\langle}{\rvert}
\DeclarePairedDelimiter\ket{\lvert}{\rangle}
\DeclarePairedDelimiterX\braket[2]{\langle}{\rangle}{#1\,\delimsize\vert\,\mathopen{}#2}





\newcommand{\new}[1]{
    \vspace{2mm}
    \noindent
    \textbf{
    \underline{#1}}
}

\def\calO{{\mathcal{O}}}
\def\th{{\theta}}
\def\_{{\hspace{1mm}}}
\def\<{{\langle}}
\def\>{{\rangle}}


\newcounter{problemcnt}
\setcounter{problemcnt}{0}

\newcommand{\Problem}{{
    \vspace{5mm}
    \stepcounter{problemcnt}
    \noindent
    \arabic{problemcnt}. 
}
}

\newcommand{\nProblem}[1]{
    \vspace{5mm}
    \noindent
    \setcounter{problemcnt}{#1}
    \arabic{problemcnt}. 
}


\newcommand{\Proof}{{
    \vspace{2mm}
    \noindent
    \textbf{
    \underline{Proof}}
}
}

\newcommand{\textOr}{
    {
        \hspace{5mm}
        \textrm{or}
        \hspace{5mm}
    }
}

\newcommand{\textAnd}{
    {
        \hspace{5mm}
        \textrm{and}
        \hspace{5mm}
    }
}


\newcommand{\Ixp}{
    {
        \textrm{Ixp}
    }
}




\newcommand{\halfFigure}[1]{
\begin{center}
\includegraphics[width = .5\linewidth]{{#1}}
\end{center}
}

\newcommand{\fullFigure}[2]{
\begin{center}
\includegraphics[width = .9\linewidth]{{#1}}
\end{center}
}

\def\twobytwoMat(#1, #2, #3, #4){
    {
        \begin{bmatrix}
            {#1} & {#2}\\
            {#3} & {#4}
        \end{bmatrix}
    }
}

\def\twobyoneMat(#1, #2){
    {
        \begin{bmatrix}
            {#1}\\
            {#2}
        \end{bmatrix}
    }
}

\def\twobytwoDet(#1, #2, #3, #4){
    {
        \begin{vmatrix}
            {#1} & {#2}\\
            {#3} & {#4}
        \end{vmatrix}
    }
}
\begin{document}
\begin{center}
\LARGE
PHYS 314 Formula Sheet

\Large
Daniel Son
\end{center}


\LARGE
\noindent
\textbf{Part 1 Formalism}
\normalsize

\new{Eigenvalues of the Spin Operators}
It is a convention to denote the basis state in 
terms of the z-axis. A particle of spin-s with 
spin value m is denoted as 
\[
    \ket{\phi} = \ket{s, m}
\]

Also, this notation provides insight of the 
eigenvalues of the spin operators. 

\[
    \hat{S}_z\ket{s, m} = m \ket{s, m}
    \textAnd 
    \hat{S}^2\ket{s, m} = \sqrt{s(s+1)} \ket{s, m}
\]

\new{Behavior of the Spin operators under commutation}
We are interested in mainly two types of spin operators. 
One is the spin operator with repect to each axis, 
and the other is the total angular spin. The following 
relations hold. 

\[
    [\hat{S}_x, \hat{S}_y] = i\hbar \hat{S}_z
    \textAnd 
    [\hat{S}_z, \hat{S}^2] = 0
\]

In other words, the three axes $x, y, z$ behave well 
under the spin operator, and the total angular 
spin commutes with the total angular spin. 

\new{Singlet States and Superposition}

The singlet state is defined as follows.

\[
    \frac{1}{\sqrt{2}}
    (\ket{z_+}\ket{z_-} - \ket{z_-} \ket{z_+})
\]

Computing the probability outcomes of the singlet state 
into all the possible output states, we determine that 
the particle never collapses to the up-up or down-down state. 
This means that the spin of one particle decides the spin of the other. 
This is called \textit{Superposition}. 

\new{Raising and Lowering Operators}
The raising and the lowering operators are conventionally 
defined on the \textbf{z-axis}. The operator bumps up 
the state by one. Here is the definition along with an example. 

\[
    \hat{S}_+ := \hat{S}_x + i\hat{S}_y
    \textAnd 
    \hat{S}_- := \hat{S}_x - i\hat{S}_y
\]

\[
    \hat{S}_+ \ket{s, j} = 
    \sqrt{s(s + 1) - j(j + 1)}
    \ket{s, j + 1}
\]

Also, spin operators are Hermitian, so $\hat{S_+}^\dag = \hat{S_-}^\dag$

\new{Product of spins to Product of Raising/Lowering Ops}
The sum of the square of all the spin operators of each direction 
is a natural operator with importance. It is possible to 
express this operator, which is dependant on all three axes, into 
a sum of three products that involve only the z-axis operators. 
The tensor product of vectors/operators behave nicely. In light of 
this fact with the definition of the raising/lowering operator, we 
conclude 
\[
    2\hat{S_{1x}}\hat{S_{2x}}+
    2\hat{S_{1y}}\hat{S_{2y}} + 
    2\hat{S_{1z}}\hat{S_{2z}} = 
    \hat{S}_{1+}\hat{S}_{2-} + \hat{S}_{1-}\hat{S}_{2+} + 2\hat{S_{1z}}\hat{S_{2z}} 
\]

\new{Applications}
It is possible to express a higher order spin, say spin-3/2, as a 
tensor product of two lower spins, spin-1 and spin-1/2. The idea 
is to take the lowest and the highest value of the spin states, 
and to apply the total momentum operator $\hat{J}^2 := (\hat{J}_1 + \hat{J}_2)^2$.
The operator $\hat{J_1}\hat{J_2}$ can be expressed entirely in terms 
of the z-axis operators. It might be tempting to simply apply each 
of the operators separately, but the base states are eigenvectors 
of $\hat{J_i}^2$.  
\[
    \hat{J_1}\hat{J_2} =     2\hat{J_{1x}}\hat{J_{2x}}+
    2\hat{J_{1y}}\hat{J_{2y}} + 
    2\hat{J_{1z}}\hat{J_{2z}} 
\]

Refer to Townsend Appendix B for details. In conclusion, this 
equation will show that the tensor product of the states spin-1/2 
and spin-1 will yield all the states in spin-3/2 and spin-1/2. 

Also, for entangled states, the raising and the lowering operator is 
defined as follows. 
\[
    \hat{S}_+ := \hat{S}_{1+}\cdot 1_2 + 1_1\cdot \hat{S}_{2+} 
    \textAnd 
    \hat{S}_- := \hat{S}_{1-}\cdot 1_2 + 1_1\cdot\hat{S}_{2-}
\]
Where $S$ is defined to be the entangled spin of two particles. 
For an intuitive justification, consider the rotation generator. 
The sum of the two operators are the terms that survive as $d\theta \rightarrow 0$. 

\new{Types of Operators and their properties}
We have the spin operator $\hat{S} := \hat{S}_x + \hat{S}_y + \hat{S}_z$. 
The nice property is that the magnitude of the combined spin $\hat{S}^2$ 
commutes with individual axis spin. That is:

\[
    [\hat{S}^2, \hat{S}_i] = 0
\]

We can take the tensor prouct of two entangled particles. That is 
\[ \hat{S}_1 \cdot \hat{S}_2 :=
\hat{S}_{1x}\hat{S}_{2x}+
    \hat{S}_{1y}\hat{S}_{2y} + 
    \hat{S}_{1z}\hat{S}_{2z}
\]
\halfFigure{TownSend_5_27.png}
Note that any state is a eigenvector of $\hat{S}^2$. 
Refer to Tonwsend CH5 for more information. 

\new{Hyperfine energy structure of the Hydrogen Atom}
A hydrogen atom can be considered as as a system of two 
spin-1/2 particles, proton and the electron. The Hamiltonian 
of the system is defined as $\hat{H} = \frac{2A}{\hbar^2}\hat{S}_1\cdot \hat{S}_2$. 
With some computation, we recognize that there are two energy 
eigenvalues and four energy eigenvectors. 

\[
    \hat{H} = 
    \begin{bmatrix}
        A/2 & 0 & 0 & 0 \\
        0 & -A/2& A & 0 \\
        0 & A & -A/2& 0 \\
        0 & 0 & 0 &A/2
    \end{bmatrix}
\]




The higher energy eigenvalue is $A/2$, and the 
corresponding states are the two product states (top/bottom) and 
the entangled state which has the same signs. 
These three states are referred as the \textbf{triplet states}. 
In symbols, 

\[
    \lambda = \frac{A}{2} \textAnd \ket{\psi} = \ket{z_+}\ket{z_+}, 
    \ket{z_-}\ket{z_-}, 
    \frac{1}{\sqrt{2}}(\ket{z+}\ket{z-} + \ket{z_-}\ket{z_+})
\]

The lower energy eigenvalue is $-3A/2$, and the 
corresponding state is the \textbf{singlet state}. In symbols, 


\[
    \lambda = -\frac{3A}{2} \textAnd \ket{\psi} = 
    \frac{1}{\sqrt{2}}(-\ket{z+}\ket{z-} + \ket{z_-}\ket{z_+})
\]
Remember that $\frac{2}{\hbar^2}S_1S_2$ has the eigenvalue of $1/2, -3/2$.

The singlet corresponds to spin-0 and the triplet corresponds to 
spin-1. 

\new{Dirac's Spin Exchange} 
The Hamiltonian $\hat{H} := \vec\sigma_1\cdot \vec\sigma_2$
can be expressed as 
\[
    \vec\sigma_1\cdot \vec\sigma_2 = 
    2P_{\textrm{spin exchange}} - 1
\]

\new{Computing the Bra-ket of product vectors}
When dealing with the EPR paradox, it is necessary 
to find the product of to entangled states. We proceed
with the principle that spins don't interact with each 
other in a product state. Suppose

\[
    \ket{x+, x+} := \ket{x+}_1 \ket{x+}_2
    \textAnd 
    \ket{y+, y+} := \ket{y+}_1 \ket{y+}_2
\]

We compute the probability that the first state 
collapses to the second state. 

\[
    P = |\braket{y+,y+}{x+,x+}|^2
    = |\braket{y+}{x+} \braket{y+}{x+}|^2
    = \frac 1 2 \frac 1 2 = \frac 1 4
\]

\new{Spin probability in terms of angles between Bloch vectors}
Consider two states $\ket{a+}, \ket{b+}$ which has a bloch 
vector oriented towards $\vec a, \vec b$. The probability that 
one of them will collapse to the other is given as follows. 
\[
    P(a+ \rightarrow b+) = \braket{b+}{a+}^2 = 
    \cos^2(\theta_{ab})
\]
and $\theta_{ab}$ is the angle between the two vectors. 
This can be derived by fixing one of the vectors to 
be the z-axis, then applying the bloch vector formula. 

\[
    \ket{n} = 
    \begin{bmatrix}
        \sin(\theta / 2)\\
        \Ixp(\phi) \cos(\theta /2)
    \end{bmatrix}
\]

\new{The Bloch vector and Rotations}

The bloch vector is \textbf{a unit vector} that describes 
the direction of the spin. To obtain the physical projection 
of the spin, the bloch vector must be multiplied by $\hbar /2$. 
In symbols, that is, 
\[
    \langle\hat{S}_x\rangle = 
    \frac \hbar 2 r_x
\]
$r_x$ refers to the x-component of the Bloch vector. In terms of 
opertators, 
\[
    \hat{S}_x = \frac \hbar 2 
     \sigma_x
\]

And the total projection operator generates the rotation around 
the axis. That is, 
\[
    \hat{R}_x(\theta) = \exp{\left(
        \frac \hbar 2 \sigma_x
        \cdot
    \frac \theta {i \hbar}\right) }
    = 
    \exp \left(
        -i \frac \theta 2 \sigma_x
    \right)
\]
Computing this matrix exponential, we conclude a following 
shorthand for the rotational operators. 
\begin{equation}
\begin{split}
    \hat{R}_x(\theta) = \cos(\theta/2) - i \sin(\theta/2) \sigma_x 
    \\  = \cos(\theta/2) + \frac 2 {i\hbar} \sin(\theta/2) \hat{S}_x
\end{split}
\end{equation}
This result generalizes to other axes. 

Also, it is useful to remember that rotation along the x-axis 
by a angle of $\pi$ results in $(a, b) \mapsto(b, a)$. For the 
y-axis, the same rotation results in $(a, b) \mapsto (-b, a)$.; 

\new{Schrodinger's Eq}
The time evolution of a quantum state is 
descirbed by the Hamiltonian $\hat H$. 

\[
    i\hbar 
    \frac {\partial} {\partial t} \ket{\psi} 
    = \hat H \ket{\psi}
\]

The time evolution opertator is an operator 
generated by the Hamiltonian. 

\[
    \hat U(t) = 
    \exp \left(
        \frac {\hat H t}{i \hbar}
    \right)
\]

\newpage

\Large
\noindent
\textbf{Part 2 Quantum Computing}
\normalsize 

\new{Introducing quantum circits}
\fullFigure{Qcircuit.png}

Here is a sample of a quantum circuit. Unlike traditional circuits 
where each of the wires take a value of $0, 1$, a quantum circuit 
can take some entangled state $\ket{\phi}$. 

We assume that all quantum gates are \textbf{unitary, or reversible}. Without proof, 
we acknowledge that any unitary operators can be 
decomposed into rotations. Moreover, any non-unitary operator 
can be emulated by \textbf{ancilla qbits}, which are wires with predetermined 
values. 

\new{Some basic gate definitions}

Each quantum gate can be represented by an operator. We 
define some important quantum gates. Here are a class 
of gates defined to be the Pauli gates. 

\[
    \begin{split}
        I := \twobytwoMat(1, 0, 0, 1) 
        \hspace{2cm} 
        X := \sigma_x = \twobytwoMat(0, 1, 1, 0)
        \\
        Y := \sigma_y = \twobytwoMat(0, -i, i, 0)
        \hspace{2cm}
        Z := \sigma_z = \twobytwoMat(1, 0, 0, -1)
    \end{split}
\]

Note that square of any of these operators results in the 
identity operator. Moreover, for any operator $A$ that satisfies 
$A^2 = I$, 
\[
    \Ixp(Ax) = \cos(x) I + i\sin(x) A
\]

The rotation gates are defined by the rotation opertaors. 
\[
    \hat R_z(\theta) := \exp(-\frac {i \theta Z}  2)
\]

Here is a list of gate definitions. 
\fullFigure{Quantum_Logic_Gates.png}

\new{Property of the Pauli Gates with respect to rotations}
Take any rotational operator $\hat R(\theta)$. The operator 
\[
    X \hat R(\theta) X^{-1} = X \hat R(\theta) X
\]
can be considered as the same rotation in the different base. 
Assume the rotation to be orthogonal with respect to the 
Pauli matrix. 
Note that this change of base results in the same type of rotation 
in the opposite magnitude. For example, 
\[
    X \hat R_y(\theta) X = \hat R_y(-\theta)
\]
This can be proved by substitution. 

Also, when the rotation is parallel to the base, then 
the rotation is preserved. For example, 
\[
    X \hat R_x(\theta) X = \hat R_x(\theta)
\]

\new{The Hermitian Gate}
The Hermitian gate is unique to quantum circuits. It can 
be considered as a change of base opertator. 
\[
    H := \frac 1 {\sqrt 2} \twobytwoMat(1, 1, 1, -1)
\]

Note that the Hermitian gate swaps the base $\hat x$ and 
$\hat z$

\new{Decomposing unitary gates}
Without proof, we present that any unitary operator $\hat U$ 
can be written as follows.
\[
    \hat U 
    = 
    e^{i\alpha} \hat R_l (\beta) \hat R_m (\gamma) \hat R_l (\delta)
\]
This decomposition holds for any nonparallel axis $l, m$ and for 
some real numbers $\alpha, \beta, \gamma, \delta$. 

Set $l = z, m = y$ for convinience. Also, we observe
\[
    X \hat R_z (\theta)  X = \hat R_z (-\theta)
    \textAnd
    X \hat R_y (\theta)  X = \hat R_y (-\theta)
\]. A simple proof can be done by direct substitution. 
Nicely decomposing the rotations, it is possible to decompose 
$\hat U$ as 
\[
    \hat U = e^{i\alpha} AXBXC
\]
for some $A, B, C$ that satisfies $ABC = I$. 

\new{Controlled Gates}
Imagine a multiqubit gate where there is one control bit 
and one or more manipulated qbits. The gate performs operation 
$U$ on the manipulated qbits when the control bit is $\ket{1}$. 
We call such gates c-U gates. It is depicted in the diagram as 
follows. 

\halfFigure{cUgate.png}


\new{Postulates of quantum computing}
We present three Postulates that sheds insight 
of quantum computing. 

\new{Evolution Postulate}

For a closed quantum system, the time evolution of 
a quantum state is described by a unitary operator. 
Suppose state $\ket\psi$ turns to $\ket \psi '$ in some 
amount of time. We write 
\[
    \ket \psi ' = \hat U \ket \psi
\]

\new{Composition of System Postulate}

Consider two physical systems described by Hilbert 
spaces $\mathcal{H}_1$, $\mathcal{H}_2$. Let $\ket \psi$ 
and $\ket \phi$ be the state of each of the physical systems. 
The combined physical system is described by $\mathcal{H}_1 \otimes \mathcal{H}_2$, 
and the state of this system is 
\[
    \ket \phi \otimes \ket \psi
\]

\new{Measurement Postulate}

Pure states occur rarely in reality. A realistic quantum system 
is not closed. But, it can be expressed as a mixed state using 
classical probability. 
Let $B := \{\ket{\phi_i}\}$ be an orthonormal basis for the 
Hilbert Space $\mathcal{H}$. Consider a mixed state 
\[
    \ket \psi := \sum_i a_i \ket {\psi_i} 
\]
If this state is measured with respect to the basis $B$, the 
state collapses to $\ket {\psi_i}$ by a probability of $a_i^2$

\end{document}


