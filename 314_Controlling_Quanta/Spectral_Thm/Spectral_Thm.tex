\documentclass{article}
\usepackage{amsfonts}
\usepackage{amsthm}
\usepackage{amssymb}
\usepackage{amsmath}
\usepackage{graphicx}
\usepackage{subcaption}

\newcommand{\new}[1]{
    \vspace{2mm}
    \noindent
    \textbf{
    \underline{#1}}
}

\def\calO{{\mathcal{O}}}
\def\th{{\theta}}
\def\_{{\hspace{1mm}}}
\def\<{{\langle}}
\def\>{{\rangle}}


\newcounter{problemcnt}
\setcounter{problemcnt}{0}

\newcommand{\Problem}{{
    \vspace{5mm}
    \stepcounter{problemcnt}
    \noindent
    \arabic{problemcnt}. 
}
}

\newcommand{\nProblem}[1]{
    \vspace{5mm}
    \noindent
    \setcounter{problemcnt}{#1}
    \arabic{problemcnt}. 
}


\newcommand{\Proof}{{
    \vspace{2mm}
    \noindent
    \textbf{
    \underline{Proof}}
}
}

\newcommand{\textOr}{
    {
        \hspace{5mm}
        \textrm{or}
        \hspace{5mm}
    }
}

\newcommand{\textAnd}{
    {
        \hspace{5mm}
        \textrm{and}
        \hspace{5mm}
    }
}

\begin{document}
\begin{center}
\LARGE
A Short Proof of the Spectral Theorem

\Large
Daniel Son
\end{center}

\normalsize


\begin{enumerate}
    \item Introduction
    \item Eigenvalues and Eigenvectors
    \item Diagonalization
    \item Orthogonal Matricies
    \item Proof of the Spectral Theorem
    \item References
\end{enumerate}

\vspace{15mm}

\textbf{1. Introduction}

The goal of this paper is to present a short proof of the Spectral 
theorem. Basic understanding of matrix and bases are assumed. 

\vspace{15mm}

\textbf{2. Eigenvalues and Eigenvectors}

Let $A$ be a square matrix of order n where entries are complex numbers. 
It might be the case that
 for some nonzero column vector $\vec{x}$ and some complex scalar $\lambda$, 
the following holds:
\[
    A \vec{x} = \lambda \vec{x}
\]

We call $\lambda$ to be the \textbf{Eigenvalue} of A, and $\vec{x}$ 
the \textbf{Eigenvector} corresponding to the scalar $\lambda$. 
Notice that if $\vec{x}$ is the Eigenvector of $\lambda$, so is $k\vec{x}$
for some arbitrary scalar k. 

Also, we establish the following proposition:

\new{Proposition 2.1} Number of Eigenvalues

For square matrix $A$ of order $n$, there are maximum $n$ distinct 
Eigenvalues. 

\proof 
Write:
\[
    A\vec{x} = \lambda I \vec{x} 
    \textAnd 
    (A - \lambda I) \vec{x} = 0
\]
This equation holds only if the matrix $A - \lambda I$ is singular. 
In order for this to be possible, the determinant of $A - \lambda I$ 
must be zero. In fact, the determinant is some polynomial of degree $n$. 
By the fundamental theorem of algebra, there are $n$ distinct roots 
for this polynomial. $\qed$

Note that the polynomial obtained from the determinant is called 
the \textbf{characteristic polynomial} of matrix $A$. 

To build up the spectral theorem, we are interested in a specific class 
of square matricies. Define the \textbf{spectrum} of matrix $A$ 
to be the set of its Eigenvalues. Nice properties occur if the spectrum 
of the matrix includes exactly $n$ distinct elements, where $n$ 
is the order of $A$. 

\new{Proposition 2.2} Distinct Eigenvalues imply linear independence of Eigenvectors 

Let $A$ be a square matrix of order $n$ which has a spectrum of 
$n$ distinct eigenvalues. For each eigenvalue, choose one nonzero 
eigenvector to create a set of vectors. This set is linearly independant. 

\proof Assume for a contradiction and suppose the eigenvectors 
$\{x_1, ... x_n\}$ are linearly dependant and that 
the spectrum is distinct. By the definition of 
linear dependance, write:

\[
    c_1 \vec{x_1}+c_2\vec{x_2}+ ... c_n\vec{x_n} = 0
\]

and without loss of generality, assume $c_n \neq 0$. 
Let $\lambda_i$ denote the eigenvalue corresponding to the 
eigenvector $\vec{x_i}$. 
Multiply the whole equation, first by the matrix $A$ and 
then by the scalar $\lambda_1$. Write:

\[
    c_1A\vec{x_1} + ... c_nA\vec{x_n} = 0
    \textAnd
    c_1\lambda_1\vec{x_1} + ... c_n\lambda_n\vec{x_n} = 0
\]

\[
    c_1\lambda_1\vec{x_1} + ... c_n\lambda_1\vec{x_n} = 0
\]

Subtract the top equation from the bottom to obtain:

\[
    c_2( \lambda_2 - \lambda_1 ) \vec{x_2}
    + ... 
c_n( \lambda_n - \lambda_1 ) \vec{x_n} = 0
\]

Applying a series of such manipulations, we reach:

\[
    c_n\prod_{i = 1}^{n - 1} (\lambda_{i + 1} - \lambda_i) \vec{x_n} = 0
\]

For we assumed $c_n \neq 0$, it must be the case that 
one of the terms of the product must be zero, implying that 
the spectrum is not distinct. We reach a contradiction. 
\qed

\vspace{15mm}

\textbf{3. Diagonalization}

We continue our discourse with matrix diagonalization. We 
first introduce the definition of diagonal matricies 
and a shorthand notation.

\new{Definition 3.1} Diagonal Matricies

A square matrix is called to be diagonal if all of its entries 
other than the elements in its main diagonal (that is the 
entries where $r = c$) are zero. Also, we denote:

\[
    diag\{\lambda_1, ... \lambda_n\} 
    = 
  \begin{bmatrix}
    \lambda_1 & & \\
    & \ddots & \\
    & & \lambda_{n}
  \end{bmatrix}
\]

We now introduce diagonalization

\new{Definition 3.2} Diagonalization 

Let $A$ be a square matrix of order $n$. If there exists a 
non-singular matrix $S$ and a diagonal matrix $D$ such that:

\[
    A = SDS^{-1}
\]

we call the expansion as a diagonalization of matrix $A$. 

Such an expansion is not guaranteed to exist. Nonetheless, 
if one exists, we call matrix $A$ to be \textbf{non-defective}. 

In fact, the propertiy of eigenvectors guarantee a nice technique 
for diagonalization for matricies with distinct spectrum. 

\new{Theorem 3.3} Eigendecomposition

Let matrix $A$ be a square matrix with distinct spectrum. 
$A$ is non-defective and its decomposition is $S\Lambda S^{-1}$ 
where $S$ is a matrix generated by stacking column vector 
representation of the eigenvectors. 

\proof 
By proposition 2.2, we establish that the $n$ eigenvectors are 
linearly independant. Thus, by uniqueness of dimensions, the 
eigenvectors form a basis. Let $\bar{A}$ be the linear transformation 
corresponding to the matrix $A$. That means, $A$ is the matrix of 
the transform $\bar{A}$ with respect to the standard basis of the 
space $\mathbb{C}^n$. Obtain $\Lambda$ by taking the matrix of $\bar{A}$
with respect to the basis ${x_i}$, the eigenvectors. Note:
\[
    \Lambda = diag\{\lambda_1, ... \lambda_n\}
\]

A simple base conversion will allow us to rewrite $A$ in terms 
of a product involving $\Lambda$. Let $S$ be the matrix that 
converts the base from the elementary base ${x_i}$ to the eigenvector 
base ${e_i}$. In other words, $S$ is the matrix of the base $x_i$ 
with respect to the basis $e_i$. 
We deduce:

\[
    A = S \Lambda S^{-1}
    \textAnd 
    S = [\vec{x_1}, ... \vec{x_n}]
\]

where $\vec{x_i}$ is the column vector of $x_i$ with respect to 
the standard basis. \qed

\vspace{15mm}


\textbf{4. Orthogonal matricies}
We wish to obtain a simpler expression for the matrix $S^{-1}$ 
presented above. To continue this endeavor, we consider inner 
product of vectors. 

\new{Definition 4.1} Inner Product of vectors 

Let $a, b$ be vectors in an n-dimensional space with 
elementary base $\{e_i\}$. If 

\[
    a = \sum_{i = 1}^{n} a_ie_i \textAnd 
    b = \sum_{i = 1}^{n} b_ie_i 
\]

then the inner product of the two vector is defined as:

\[
    a\cdot b = \sum_{i = 1}^{n} a_ib_i
\]

An ordered set of vectors $\{v_1, ... v_n\}$ 
is called to be \textbf{orthonormal} if:

\[
    v_i\cdot v_j = \delta _{ij}
\]

Where $\delta_{ij}$ is the Kronecker delta defined to 
be 1 if $i = j$ and 0 otherwise. Orthonormal bases 
exhibit an interesting property. By merely unfolding 
the definitions introduced above, we recognize that:

\[
    \sum_{k = 1}^{n}v_{ik}v_{jk} = \delta_{ij}
\]

Consider the matrix:

\[
    V:= [\vec{v_1} ... \vec{v_n}] \textAnd V^T
\]

The summation above implies that the product of the sum of the 
two matricies equals the identity matrix. $V$ is invertible. Hence:

\[
    VV^T = I \textAnd V^T = V^{-1}
\]

We have established:

\new{Theorem 4.2} The matrix of an orthonormal base has an inverse 
which is equal to the transpose of the matrix. 

\vspace{15mm}

\textbf{5. Proof of the Spectral Theorem}
We have now prepared all the machinery necessary to prove the theorem. 

\new{Theorem 5.1} Spectral Theorem 
Let $A$ be a similar matrix, that is $A^T = A$, with a distinct spectrum. 
$A$ is non-defective, and its diagonalization is given as:

\[
    A = S \Lambda S^T
\]

\proof Invoking theorem 3.3, it suffices to show that $S^{-1}$ = $S^T$. 
By theorem 4.2, showing that the eigenbasis $\{x_i\}$ is orthonormal 
yields the desired result. 

From part 1, recall that the eigenvector 
can be scaled by an arbitrary scalar. Hence, it is possible to normalize 
any eigenvector such that $x_i\cdot x_i = 1$ for all $i\leq n$. 
To show normality, consider two different eigenvectors $x_i$ and 
$x_j$ with eigenvalues $\lambda_i \neq \lambda_j$. 

Let $\vec{x_i}$, $\vec{x_j}$ be the column vectors of the two eigenvectors 
with respect to the standard basis. Write:

\[
    (A\vec{x_i})^T\vec{x_j} = \lambda_i(\vec{x_i}^T \cdot \vec{x_j}) 
    \textAnd 
   (A\vec{x_i})^T\vec{x_j}  = \vec{x_i}^TA^T\vec{x_j} 
   =\vec{x_i}^TA\vec{x_j}
\]

Some more manipulation on the right equation shows:

\[
    \vec{x_i}^TA\vec{x_j} = \vec{x_i}^T\lambda_j\vec{x_j}
    =\lambda_j(\vec{x_i}^T\cdot\vec{x_j})
\]

Using the associative property of matrix multiplication along with 
$A^T = A$, we write:

\[
    \lambda_i(\vec{x_i}^T\cdot\vec{x_j}) =\lambda_j(\vec{x_i}^T\cdot\vec{x_j}) 
    \textOr 
    (\lambda_i - \lambda_j)(\vec{x_i}^T\cdot\vec{x_j}) = 0
\]

And the spectrum of $A$ is distinct. Thus we arrive at:

\[
    \vec{x_i}^T\cdot\vec{x_j} = 0
\]

for $i \neq j$. This proves that the eigenbasis is orthonormal, which 
in turn proves the theorem. 
\qed

\vspace{15mm}

\textbf{6. References} 

http://mlwiki.org/index.php/Eigendecomposition

http://mlwiki.org/index.php/Diagonalization

\end{document}