\documentclass{article}
\usepackage{amsfonts}
\usepackage{amsthm}
\usepackage{amssymb}
\usepackage{amsmath}
\usepackage{graphicx}
\usepackage{subcaption}
\usepackage{xcolor}
\usepackage{mathtools}
\usepackage{ wasysym }


\newcommand{\new}[1]{
    \vspace{2mm}
    \noindent
    \textbf{
    \underline{#1}}
}

\def\calO{{\mathcal{O}}}
\def\th{{\theta}}
\def\_{{\hspace{1mm}}}
\def\<{{\langle}}
\def\>{{\rangle}}

\DeclarePairedDelimiter\bra{\langle}{\rvert}
\DeclarePairedDelimiter\ket{\lvert}{\rangle}
\DeclarePairedDelimiterX\braket[2]{\langle}{\rangle}{#1\,\delimsize\vert\,\mathopen{}#2}



\newcounter{problemcnt}
\setcounter{problemcnt}{0}

\newcommand{\Problem}{{
    \vspace{5mm}
    \stepcounter{problemcnt}
    \noindent
    \arabic{problemcnt}. 
}
}

\newcommand{\nProblem}[1]{
    \vspace{5mm}
    \noindent
    \setcounter{problemcnt}{#1}
    \arabic{problemcnt}. 
}


\newcommand{\Proof}{{
    \vspace{2mm}
    \noindent
    \textbf{
    \underline{Proof}}
}
}

\newcommand{\textOr}{
    {
        \hspace{5mm}
        \textrm{or}
        \hspace{5mm}
    }
}

\newcommand{\textAnd}{
    {
        \hspace{5mm}
        \textrm{and}
        \hspace{5mm}
    }
}

\newcommand{\Ixp}[1]{
    {
        e^{i{#1}}
    }
}



\newcommand{\halfFigure}[1]{
\begin{center}
\includegraphics[width = .5\linewidth]{{#1}}
\end{center}
}

\newcommand{\fullFigure}[1]{
\begin{center}
\includegraphics[width = .9\linewidth]{{#1}}
\end{center}
}

\def\twobytwoMat(#1, #2, #3, #4){
    {
        \begin{bmatrix}
            {#1} & {#2}\\
            {#3} & {#4}
        \end{bmatrix}
    }
}

\def\twobyoneMat(#1, #2){
    {
        \begin{bmatrix}
            {#1}\\
            {#2}
        \end{bmatrix}
    }
}

\def\twobytwoDet(#1, #2, #3, #4){
    {
        \begin{vmatrix}
            {#1} & {#2}\\
            {#3} & {#4}
        \end{vmatrix}
    }
}



\begin{document}
\begin{center}
\LARGE
PHYS 314 HW9

\Large
Daniel Son
\end{center}

\new{Problem 2}

The Quantum Fourier Transform is defined as follows. 
\[
    QFT_m\ket{x} := \frac 1 {\sqrt m} \sum _{y = 0}^{m - 1} e^{2\pi i xy / m} \ket {y}
\]

\new{a)} Verify the inverse QFT of which the formula is given as 
\[
    (QFT_m)^{-1} \ket{x} = \frac 1 {\sqrt m} 
    \sum_{y = 0}^{m - 1} e^{-2\pi i xy / m} \ket{y} 
\]

\new{Solution} 
Naively, we wish to show, for any quantum state vector $\ket{x}$, 
\[
    (QFT_m)^{-1} QFT_n \ket{x} = \ket {x}
\]

Using the linearity of quantum operators, we deduce 
that the equation is equivalent to the following expression. 

\[
    \frac 1 m \sum_{y, z < m} e^{2\pi i y (x-z)/m}\ket{z} = \ket {x}
\]

We will separate the double sum in terms of the basis vectors 
$\ket{z}$ and show that its coefficient equals to zero unless 
$z = x$. Rewrite the LHS as follows. 
\[
    \frac 1 m \sum_{y, z < m} e^{2\pi i y (x-z)/m}\ket{z} 
    = \sum_{z < m} \xi_z \ket{z}
\]
where 
\[
    \xi_z := \frac 1 m \sum_{y < m } e^{2\pi i y(x - z)/m}
\]

We claim that $\xi_z = 1$ if and only if $z = x$ and the 
term is zero otherwise. 

\[
    \xi_x = \frac 1 m \sum_{y < m } e^{2\pi i y(0)/m} 
    = \frac m m = 1
\]

If $z \neq x$, then the complex exponentials would be 
a sum of a unit vectors that are rotated by a phase. Geometrically, 
they must sum to zero. Suppose $k = x-z \neq 0$. In symbols, 
\[
    \xi_z = \frac 1 m \sum_{y < m} e^{2\pi i y (k/m)} = 0
\]

In conclusion, 
\[
    (QFT_m)^{-1}QFT_m \ket{x} = \sum_{z_m} \xi_z \ket{z} = \ket{x}
\]
as desired. 

\hfill \qed

\new{b)} Concisely describe $(QFT_m)^2$. How about 
$(QFT_m)^4$. 

\new{Solution} We proceed with the same steps we used for 
the previous problem. The coefficient of $\ket{z}$ of the 
transform simplifes to the following expression. 
\[
    \xi_z = \frac 1 m \sum_{y < m} e^{2\pi i y(x+z)/m} 
\]
If $2\pi \frac {x + z} m$ is a nonzero phase, the sum 
equals zero by geometry in the complex plane. Otherwise, 
if the phase is zero, then the coefficient sums up to 1. 
The phase is zero if and only if $z = m - x$. Thus, 
\[
    (QFT_m)^2 \ket{x} = \ket{m - x}
\]

So the double Quantum Fourier Transform inverts the state 
$\ket x$ with respect to the mean. 

Consequently, 
\[
    (QFT_m)^4 \ket{x} = (QFT_m)^2\ket{m-x} = \ket{x}
\]

\new{b-2)} Find the Eigenvalues of $QFT_m$

\new {Solution} Let $\lambda$ be an eigenvalue of $(QFT_m)$. 
\[
    (QFT_m)^4  = \lambda ^4 I = I
\]
This implies 
\[
    \lambda^4 = 1
\]
So 
\[
    \lambda \in \{1, i, -1, -i\}
\]
We observe that if $m = 2$, then the eigenvalues are $\pm 1$. 
Otherwise, we observe that all the eigenvalues show up. 

\halfFigure{QFT_ev.png}

\new{Problem 3}
\fullFigure{QFT_TF.png}
Prove this identity 

\new{Solution}
Before showing that this equation is true, first consider 
what it means. The equation tells us that an QFT of a n-qbit 
system can be considered as a superposition of 
rotations of each individual quantum states. 
The degree of rotation is determined by the rank of the 
bit among the entire collection of bits along with the 
frequency $\omega$ in question. 

Label the rank of the n-qubits as $1, 2, \dots, n$. 
When taking the tensor product of all the qubits, the 
rank of the qubit determines the magnitude in which 
it will affect the digit space. For example, consider 
a 3-qbit system and suppose there are two states 
$\ket{0}\ket 0 \ket1$ and $\ket{1} \ket 0 \ket {1}$.
In digit space, these two states correspond to 
$\ket{1}, \ket{5}$ respectively. 
The difference
 between the two states is the first qubit which has a rank of 1.
 The rank 1 qubit affects the state by magnitude $4$. 
 
In light of the relation between rank and magnitude 
differenece, foil the right tensor product of the given identity. 

\newcommand{\compKet}[1]{
 \left(
        \frac{\ket{0} +e^{2\pi i (2^{#1})\omega}\ket 1} {\sqrt 2}
    \right)
}

\begin{equation}
    \begin{split}
    \left(
        \frac{\ket{0} +e^{2\pi i (2^{n - 1})\omega}\ket 1} {\sqrt 2}
    \right)
    \otimes 
    \compKet{n - 2}
    \\
    \otimes 
    \cdots \otimes \compKet{0}
      =  \frac 1 {\sqrt{2^n}} \sum_{y < 2^n} \xi_y \ket{y}
    \end{split}
\end{equation}

Focus on one basis state $\ket{y}$. $y$, in digits, is some 
nonnegative integer less than $2^n$. Write $\ket{y}$ as 
\[
    y = 2^{n - b_1} + 2^{n - b_2} + \cdots + 2^{n - b_m}
\]
assuming that $y$ has $m$ states that are $1$ in binary notation. 
We also assume that $b_k$ is a strictly increasing interger sequence 
where all entries are bounded between 1 and $n$
Inspecting the tensor product, we notice that the coefficient 
of $\ket{y}$ is 
\[
    \xi_y = e^{2\pi i 2^{n - b_1}\omega} e^{2\pi i 2^{n - b_2}\omega} \cdots e^{2\pi i 2^{n - b_m}\omega}
\]
\[
    = e^{2\pi i 2^n (2^{-b_1} + 2^{-b_2} + \cdots + 2^{-b_m}) \omega}
    = e^{2\pi i y \omega}
\]
And plugging in the coefficient to the sum, we conclude 
\begin{equation}
    \begin{split}
    \left(
        \frac{\ket{0} +e^{2\pi i (2^{n - 1})\omega}\ket 1} {\sqrt 2}
    \right)
    \otimes 
    \compKet{n - 2}
    \\
    \otimes 
    \cdots \otimes \compKet{0}
      =  \frac 1 {\sqrt{2^n}} \sum_{y < 2^n} e^{2\pi i y \omega} \ket{y}
    \end{split}
\end{equation}

\hfill \qed

\new{b}
We know that the inverse QTF reverts a quantum state 
correctly to its corresponding state if the frequency value 
is a multiple of $1/2^n$. The inverse QTF is also called the phase 
estimation algorithm. Let $\omega$ be some frequency that 
is not a integer multiple of $1/2^n$. Compute $p(x)$, the probability 
that the quantum state will collapse to $\ket x$ after the phase estimation 
algorithm. 

\new{Solution}
Consider state $\ket \psi$, which corresponds to sending through 
the state $\omega$ through the regular QFT. 

\[
    \ket \psi 
    := 
    \frac 1 {\sqrt {2^n}} 
    \sum_{y < 2^n} 
    e^{2\pi i \omega y} \ket{y}
\]
Where $\omega =  .x_1 x_2 \dots x_n \dots $. 

Recall that the inverse QTF for pure states is defined as follows. 
\[
    (QFT_{2^n})^{-1} \ket{x}:= 
    \frac 1 {\sqrt 2^n} 
    \sum _{y < 2^n} 
    e^{-2\pi i y x/2^n }  \ket y
\]  

Apply the inverse QFT to state $\psi$ and 
exploit linearity. 
\[
    (QFT_{2^n})^{-1} \ket \psi 
     = 
    \frac 1 {\sqrt {2^n}} 
    \sum_{y < 2^n} 
    e^{2\pi i \omega y /2^n} (QFT_{2^n})^{-1} \ket y
\]
\[
    = 
    \frac 1 {2^n} 
    \sum_{y, z< 2^n} 
    e^{2\pi i y (\omega - z/2^n)} \ket z 
    := 
    \sum_{z< 2^n} 
    \xi_z \ket z 
\]
We wish to compute the coefficient $\xi_z$. Write 
\[
    \xi_z = \frac 1 {2^n} \sum_{y < 2^n} e^{2\pi i y (\omega - z/2^n)}
\]
This sum can be computed by the geometric series formula with ease. 
\[
    \xi_z  =  \frac 1 {2^n} \frac {
        e^{2\pi i (\omega - z/ 2^n)} - 1
    }
    {e^{2\pi i (\omega - z/2^n)} - 1}
\]
Ignore the phase and apply the Euler's formula. 
\[
    |\xi_z| = \frac 1 {2^n} \frac {\sin(\pi (\omega - z / 2^n) 2^n)}
    {\sin(\pi (\omega - z / 2^n))}
    = 
\frac 1 {2^n} \frac {\sin(\pi (2^n\omega - z) )}
    {\sin(\pi (\omega - z / 2^n))}
\]
Use Born's probability rule to compute $p(x)$. 
\[
    p(x) = |\xi_z|^2 = 
\frac 1 {2^{2n}} \frac {\sin^2(\pi (2^n\omega - z) )}
    {\sin^2(\pi (\omega - z / 2^n))}
\]

\hfill \qed

\new{c)} Here is a plot of $p(x)$ for $n = 3, 8, 10$ 
where $\omega  = .7$. 

\halfFigure{neq3.png}
\halfFigure{neq8.png}
\halfFigure{neq10.png}

\end{document}