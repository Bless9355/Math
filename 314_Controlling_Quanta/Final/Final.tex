\documentclass{article}
\usepackage{amsfonts}
\usepackage{amsthm}
\usepackage{amssymb}
\usepackage{amsmath}
\usepackage{graphicx}
\usepackage{subcaption}
\usepackage{xcolor}
\usepackage{mathtools}
\usepackage{ wasysym }
\usepackage{enumerate}


\newcommand{\new}[1]{
    \vspace{2mm}
    \noindent
    \textbf{
    \underline{#1}}
}

\def\calO{{\mathcal{O}}}
\def\th{{\theta}}
\def\_{{\hspace{1mm}}}
\def\<{{\langle}}
\def\>{{\rangle}}

\DeclarePairedDelimiter\bra{\langle}{\rvert}
\DeclarePairedDelimiter\ket{\lvert}{\rangle}
\DeclarePairedDelimiterX\braket[2]{\langle}{\rangle}{#1\,\delimsize\vert\,\mathopen{}#2}



\newcounter{problemcnt}
\setcounter{problemcnt}{0}

\newcommand{\Problem}{{
    \vspace{5mm}
    \stepcounter{problemcnt}
    \noindent
    \arabic{problemcnt}. 
}
}

\newcommand{\nProblem}[1]{
    \vspace{5mm}
    \noindent
    \setcounter{problemcnt}{#1}
    \arabic{problemcnt}. 
}


\newcommand{\Proof}{{
    \vspace{2mm}
    \noindent
    \textbf{
    \underline{Proof}}
}
}

\newcommand{\textOr}{
    {
        \hspace{5mm}
        \textrm{or}
        \hspace{5mm}
    }
}

\newcommand{\textAnd}{
    {
        \hspace{5mm}
        \textrm{and}
        \hspace{5mm}
    }
}


\newcommand{\textWhere}{
    {
        \hspace{5mm}
        \textrm{where}
        \hspace{5mm}
    }
}



\newcommand{\Ixp}[1]{
    {
        e^{i{#1}}
    }
}



\newcommand{\halfFigure}[1]{
\begin{center}
\includegraphics[width = .5\linewidth]{{#1}}
\end{center}
}

\newcommand{\fullFigure}[1]{
\begin{center}
\includegraphics[width = .9\linewidth]{{#1}}
\end{center}
}

\def\twobytwoMat(#1, #2, #3, #4){
    {
        \begin{bmatrix}
            {#1} & {#2}\\
            {#3} & {#4}
        \end{bmatrix}
    }
}

\def\twobyoneMat(#1, #2){
    {
        \begin{bmatrix}
            {#1}\\
            {#2}
        \end{bmatrix}
    }
}

\def\twobytwoDet(#1, #2, #3, #4){
    {
        \begin{vmatrix}
            {#1} & {#2}\\
            {#3} & {#4}
        \end{vmatrix}
    }
}


\newcommand{\RR}{\mathbb{R}}
\newcommand{\CC}{\mathbb{C}}

\begin{document}
\begin{center}
\LARGE
PHYS 314 Final Project

\Large
Daniel Son, Sean Mealey
\end{center}

\normalsize

The linear Lie Group $SU(4)$ describes a quantum gate with two channels. 
The goal of this paper is to understand the structure of this group. 
For the first half of the paper, we will review the definition of Lie Groups 
and Algebra which will lay down the foundation to understand the killing form. 
The killing form will provide a method to verify if a Cartan decompositon 
exists. 

\section{Lie groups and Lie Algebras}


In this section, we introduce definition and facts to 
understand the nature of Lie groups and Lie algebras. 
The definitions and facts should be considered as motivations, 
and we will refrain from providing a full proof of the statments. 


A \textbf{group} is defined to be a set equipped with a binary operation. 
For the set to be called a group, the binary operation must satisfy 
closure, associativity. A group must have an identity (i.e. $e \in G$ s.t. 
$a e = a$ for all $a \in G$), and a corresponding 
inverse for every group element, i.e. for all $a \in G$, there exists $a^{-1} \in G$ s.t. 
$a a^{-1} = e$. 

We introduce the concept of a \textbf{group representation}. 
Crudely, a group representation can be considered as a matrix snapshot 
of each group element. \footnote{Refer to A.Zee p89}
Let $G$ be a group, and $M_n$ be the group of all n by n matricies 
with complex entries. 
Formally, a map $\Gamma: G \rightarrow M_n$ is a representation if 
\[
    \Gamma(a \cdot b) = \Gamma(a) \cdot \Gamma(b)
\]
for all elements $a, b \in G$. The operation between the matricies 
are defined as the natural matrix multiplication. The representation 
preserves the group operation. 

We have the tools to consider distances between two group elements. 
Let $\Gamma$ be a representation with degree 2 of group $G$. Any two elements 
$a, b$ in $G$ will have a corresponding matrix in $M_2$.
For simplicity, assume the representation maps the group 
elements to matrices with real entries. 
Write out 
the representations of $a, b$. 

\[
    \Gamma(a) = \twobytwoMat(a_{11}, a_{12}, a_{21}, a_{22})
    \textAnd
    \Gamma(b) = \twobytwoMat(b_{11}, b_{12}, b_{21}, b_{22})
\]

\newcommand{\squareterm}[1]{
    (a_{#1} - b_{#1})^2
}


\newcommand{\compsquareterm}[1]{
    |a_{#1} - b_{#1}|^2
}



The natural way to define the distance between two 2 by 2 matrices is 
to take the Euclidean distance by considering the matrix 
as a tuple of four numbers. \footnote{
    If the matrix entries are complex, we can replace the 
    difference of each entries with the modulus of the difference, 
    e.g. $\compsquareterm{ij}$ instead of $\squareterm{ij}$
}
\[
    d(\Gamma(a), \Gamma(b)) := 
    \sqrt{  
        \squareterm{11} + 
        \squareterm{12} + 
        \squareterm{21} + 
        \squareterm{22}
    }
\].

We generalize this notion of distance to n by n matricies. 
Generalize $\Gamma$ to be a representation of degree $n$. 
For group elements $a, b \in G$, define the \textbf{distance} between 
$a, b$ as follows. 
\[
    d(a, b) := 
    \sqrt{
    \sum_{i, j} \squareterm{ij}
    }
\]

We finally have the tools to define a linear lie group. Let $G$ 
be a group with a representation $\Gamma$. Suppose the degree of 
the representation is n. $G$ is called a linear lie group if it 
satisfies the following three properties. 
\footnote{Refer to Cornwell p36-37}

\new{(A) Parameterization around the identity}

Let $C_\delta(I)$ be all the points in $G$ that is within a distance 
$\delta$ from the identity. Call this the $\delta$-ball around the identity. 
There must exist an injective mapping from the $\delta$-ball to 
some tuple of real numbers, $\mathbb{R}^b$. Also, 
the identity matrix must correspond to the zero tuple. In symbols, 
\[
    prm(\Gamma(E)) = prm(I_n) = (0, 0, \dots, 0) 
\]
where map $prm:M_n \rightarrow \mathbb{R}^b$ is the parameterization function. 
$E$ denotes the identity function in the lie group. 
The integer $b$ is called 
the degree of the linear lie group. 

\new{
(B) Dense Parameterization
}

Consider the space $\mathbb{R}^b$. The $\delta$-ball in the 
group $G$ will fall into some points in this space. The goal of 
this definition is to make the parameterization bijective for 
some domain.

There must exist some positive real number $\eta$ in which 
the $\eta$-ball in the $\mathbb{R}^b$ centered at the zero tuple 
$(0, \dots, 0)$ in which all points fall under the image of the 
$\delta$-ball under the parameterization. In other words, for every point 
$(x_1, x_2, \dots, x_b)$, there exists some point $A \in C_\delta(I)$ such that 
\[
    prm(A) = (x_1, x_2, \dots, x_b)
\]

This implies that around some wierd neighborhood of some point in the 
linear Lie group, the parameterization to the real tuples are bijective. 

\new{
(C) Analytic Parameterization
}

Now that there is a one-to-one correspondence betwen 
the real tuples and some complex matricies, we can consider 
a map from the real tuples to n by n matricies. 
We slightly abuse notation. Let 
\[
    A = prm(g) = (x_1, x_2, \dots, x_b)
    \]
Define 
\[
    \Gamma (x_1, x_2, \dots, x_b) = A
\]

$A$ is an n by n matrix. Hence, each matrix element can be considered 
as separate. That is, for each entry we can define
\[
    \Gamma _{ij}(x_1, x_2, \dots, x_b) = A_{ij}
\]

We dictate that each of the $\Gamma_{ij}$ must be analytic. 

The linear Lie group can be better understood in conjunction with 
the corresponding Lie algebra. The correspondence between the 
Lie algebra and Lie groups are denoted by the matrix exponention. 
The matrix exponential is defined as 
\[
    \exp(A) := I + A + \frac {A^2} {2!} + \cdots 
    = \sum_{k = 0} ^ {\infty} \frac {A^k} {k!}
\]

We list out some useful facts about matrix exponentials. 

\new{Facts}
\begin{enumerate}
\item Let $\{\lambda_1, \dots, \lambda_n\}$ be the set of 
eigenvalues of $A$, then the eigenvalues of $\exp(A)$ are 
$\{\exp(\lambda_1), \dots, \exp(\lambda_n)\}$

\item The determinant of the matrix exponential is the expnential 
of the trace. In symbols, 
\[
    \det(\exp(A)) = e^{tr(A)}
\]. 

\item
The exponential map is bijective and continuous around the identity. 

\item
\[
    \exp(A) \exp(-A) = I
\]

\item (Campbell-Baker-Housdorff)

Let $A, B$ be two square matricies. Let the commutator be defined as 
\[
    [A, B] := AB - BA
\]
Suppose a square matrix $C$ satisfies the following identity. 
\[
    \exp(A)\exp(B) = \exp(C)
\]
Then, 
\[
    C = A + B + \frac 1 2 [A, B] + \frac 1 {12} ([a, [a, b]] + [b, [b, a]]) \cdots
\]
\end{enumerate}

We continue our discourse to understand the correlation between 
linear Lie groups and Lie algebras. 
Suppose $A(t)$ to be a curve within the linear Lie group $G$.
That is, the map $A$ takes a real value $t$ and maps it to 
some element in the $G$. Also, suppose that for any real values $s, t$, 
\[
    A(s + t) = A(s) A(t)
\]

It is straightforward to observe that $A(t)$ is an abelian group 
with an identity $E = A(0)$. Our goal is to identify a quintessential 
property of all such curves and to generate them. 

We present the following theorem. 
\footnote{Refer to Sternberg p234-235}

\new{Theorem} All curves that converts scalar addition to 
matrix multiplication are characterized by the exponential map. 
In symbols, 
\[
    A(t) = \exp(at) \textWhere a = \frac d {dt} A(t)
\]



\proof 
It is straightforward to see that given a matrix $a$, the exponential 
map generates a nice curve. Write the following. 
\[
    A(t) A(s) = \exp(at)\exp(as) = \exp(a (t + s)) = A(t + s)
\]

Now, take any nice curve $A(t)$. It suffices to show, that for all 
$t \in \mathbb R$, 
\[
    A(t) \exp(-at) = I
\]
Clearly, if t is zero, $A(0) = I$ and $\exp(-a0) = I$ so the 
equality holds. 

The strategy is to take the derivative of the LHS with respect to 
$t$ vanishes everywhere. Beforehand, consider the derivatinve 
of $A(t)$. 
\[
    \frac d {dt} A(t) 
    = \lim_{s \rightarrow 0} \frac {A(t + s) - A(t)} s 
\]
Since the curve is nice it is possible to factor out $A(t)$ from 
the first summand. 
\[\lim_{s \rightarrow 0} \frac {A(t)(A(s) - A(0))} s 
    = A(t) \bigg[ \frac d {dt} A(t) \bigg] _{t = 0} = a A(t)
\]

Now take the desired derivative. 
\[
    \frac d {dt} A(t) \exp(-at) = 
    \left(
         \frac d {dt} A(t)
    \right) \exp(-at)
    + 
    A(t) (-a) \exp(-at)
\]
\[
    = a A(t) \exp(-at) - a A(t) \exp(-at) = 0
\]

\hfill 
\qed

The theorem suggests that the derivatives of the linear Lie group 
around the identity generates the Lie group. The derivatives of 
the group around the identity form an algebra. 

For now, we losely define the \textbf{Lie algebra} as a vector space of linear 
lie Groups equipped with a \textbf{Lie bracket}. The exponential map of 
the lie algebra generates the lie group. For square matricies, 
the lie bracket is defined as the commutator. Also, the common 
convention is to denote the Lie group with a capital letter such 
as $G$ and the Lie algebra as a lower case letter $\mathfrak g$. 

Let $a, b \in \mathfrak g$. We list out a motivation for why 
the lie bracket of the two elements, 
$[a, b]$, must also be in the Lie algebra. For $\mathfrak g$ 
generates $G$ by the exponential map, 
\[
    \exp(at) , \exp(bt) \in G
\]
$G$ is a group, and thus 
\[
    \exp(at) \exp(bt) = \exp(C(t)) \in G
\]
By the Campbell-Baker-Housdorff expansion, we can 
find an expression for $C(t)$. 
\[
    C(t) = at + bt + \frac 1 2 [at, bt] + O(t^3)
    = t\left(
        a + b + \frac t 2 [a, b] + O(t^2)
    \right)
\]
Let $C(t) := C'(t) t$. The exponential of $C(t)$ is generated 
by $C'(t)$. For small enough $t$, we claim that the element 
\[
    a + b + \frac t 2 [a, b]
\]
must be in the subalgebra up to a leading term of $t$. By 
the closure of vector spaces, the lie bracket $[a, b]$ must be 
inside the lie algebra. 



\section{Structure Constants and Cartan Decomosition}

We wish to use Cartan Decomposition in order to decompose 
the group $SU(4)$ into a direct sum of two subgroups. In order 
to obtain a decomposition, we will use Cartan Decomposition. 
The Cartan decomposition comes with nice properties, but to 
guarantee that a Lie group has a Cartan decomposition, we 
must understand structure constants and Killing forms. 

Recall that a lie algebra is indeed a vector space. 
This means that the algebra must have a basis. Let $\mathfrak g$ 
be a lie algebra with basis vectors $X_1, X_2 , \dots X_n$. 

Also, the Lie bracket is bilinear. That is, 
\[
    [\alpha X + \beta Y, Z] = \alpha [X, Z] + \beta[Y, Z]
\]
along with 
\[
    [X, Y] = -[Y, X]
\]
for any elements $x, y, z$ in the algebra $\mathfrak g$ and 
any scalars $\alpha, \beta$. 
\footnote{It is straightforward to check that this 
property holds for the typical commutator bracket $[X, Y] = XY - YX$}

These property imply that the lie bracket of any two elements 
in the algebra can be represented by the elements 
\[
    [X_i, X_j] \textWhere i \neq j
\]. Also the Lie bracket guarantees closure, and the Lie bracket 
of the basis elements must also be a linear combination of the basis. 
In symbols, 
\[
    [X_i, X_j] = C_{ij}^1X_1 + C_{ij}^2X_2 \dots C_{ij}^nX_n = \sum_{k \leq n} C_{ij}^k X_k
\]

So the $n^3$ coefficients $C_{ij}^k$ entirely determine the behavior 
of the Lie bracket, and therefore determines the structure of the algebra. 
We call these coefficients the \textbf{structure coefficients}. 

Now, we introduce the Killing form of the algebra. 
\footnote{We present a complicated definition, 
but our interest is to apply the mathematical result, 
and hence we ignore the motivation behind the definition. }
Define a linear transform $L(Y)$ as follows. 

\[
    L\{X_i, X_j\}(Y) := [X_i, [X_j, Y]]
\]

For this transform is linear, it must have a matrix with respect to 
the base $\{X_1, \dots X_n\}$. To construct the matrix, apply 
the basis elements to the transform. 

\[
    L\{X_i, X_j\}(X_c) =  [X_i, [X_j, X_c]] 
    = \left[X_i, \sum_{s} C_{jc}^s X_s\right]
\]
Invoking bilinearity, 
\[
    = \sum_s C_{jc}^s[X_i, X_s]
    = \sum_s \sum _r C_{jc}^s C_{is}^r X_r
\]
Thus, the entry at the rth row and the cth column of the matrix of L 
will be 
\[
    \sum_s C_{jc}^s C_{is}^r  =  \sum_s C_{is}^r C_{jc}^s
\]

Define a scalar function $B$ as follows. Also, invoke the previous 
result to directly compute $B$. 
\[
    B(X_i, X_j) = tr(L\{X_i, X_j\}) = \sum_k L_{kk} = 
    \sum_k \sum_s C_{is}^k C_{jk}^s
\]
The function $B$ also defines a square matrix. This matrix is called 
the killing form. Explicitly, the matrix is
\[
    \begin{bmatrix}
        B(1, 1) &  B(1, 2) & \cdots & B(1, n)\\
        B(2, 1) &  B(2, 2) & \cdots & B(2, n)\\
        \vdots & \vdots & \vdots & \vdots \\
        B(n, 1) &  B(n, 2) & \cdots & B(n, n)\\
    \end{bmatrix}
\]

Without proof, we claim that if the determinant of the killing 
form is nonzero, then there exists a Cartan decomposition, 
which will be introduced shortly. 

\newpage

\section{References}
\begin{enumerate}[ {[}1{]} ]
\item J. Zhang, J. Vala, S. Sastry, and K. B. Whaley, “Geometric theory of nonlocal two-qubit operations,” Physical Review A, vol. 67, no. 4, Apr. 2003, doi: https://doi.org/10.1103/physreva.67.042313.

\item J. F. Cornwell, Group Theory in Physics. Academic Press, 1997.

\item Shlomo Sternberg, Group theory and physics. Cambridge: Cambridge University Press, 2003.

\item A. Zee, Group Theory in a Nutshell for Physicists. Princeton University Press, 2016.
‌
\end{enumerate}

\end{document}