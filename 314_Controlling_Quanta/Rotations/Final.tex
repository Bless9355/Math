\documentclass{article}
\usepackage{amsfonts}
\usepackage{amsthm}
\usepackage{amssymb}
\usepackage{amsmath}
\usepackage{graphicx}
\usepackage{subcaption}
\usepackage{xcolor}
\usepackage{mathtools}
\usepackage{ wasysym }


\newcommand{\new}[1]{
    \vspace{2mm}
    \noindent
    \textbf{
    \underline{#1}}
}

\def\calO{{\mathcal{O}}}
\def\th{{\theta}}
\def\_{{\hspace{1mm}}}
\def\<{{\langle}}
\def\>{{\rangle}}

\DeclarePairedDelimiter\bra{\langle}{\rvert}
\DeclarePairedDelimiter\ket{\lvert}{\rangle}
\DeclarePairedDelimiterX\braket[2]{\langle}{\rangle}{#1\,\delimsize\vert\,\mathopen{}#2}



\newcounter{problemcnt}
\setcounter{problemcnt}{0}

\newcommand{\Problem}{{
    \vspace{5mm}
    \stepcounter{problemcnt}
    \noindent
    \arabic{problemcnt}. 
}
}

\newcommand{\nProblem}[1]{
    \vspace{5mm}
    \noindent
    \setcounter{problemcnt}{#1}
    \arabic{problemcnt}. 
}


\newcommand{\Proof}{{
    \vspace{2mm}
    \noindent
    \textbf{
    \underline{Proof}}
}
}

\newcommand{\textOr}{
    {
        \hspace{5mm}
        \textrm{or}
        \hspace{5mm}
    }
}

\newcommand{\textAnd}{
    {
        \hspace{5mm}
        \textrm{and}
        \hspace{5mm}
    }
}

\newcommand{\Ixp}[1]{
    {
        e^{i{#1}}
    }
}



\newcommand{\halfFigure}[1]{
\begin{center}
\includegraphics[width = .5\linewidth]{{#1}}
\end{center}
}

\newcommand{\fullFigure}[1]{
\begin{center}
\includegraphics[width = .9\linewidth]{{#1}}
\end{center}
}

\def\twobytwoMat(#1, #2, #3, #4){
    {
        \begin{bmatrix}
            {#1} & {#2}\\
            {#3} & {#4}
        \end{bmatrix}
    }
}

\def\twobyoneMat(#1, #2){
    {
        \begin{bmatrix}
            {#1}\\
            {#2}
        \end{bmatrix}
    }
}

\def\twobytwoDet(#1, #2, #3, #4){
    {
        \begin{vmatrix}
            {#1} & {#2}\\
            {#3} & {#4}
        \end{vmatrix}
    }
}


\newcommand{\RR}{\mathbb{R}}
\newcommand{\CC}{\mathbb{C}}

\begin{document}
\begin{center}
\LARGE
PHYS 314 Final Project

\Large
Daniel Son
\end{center}

We wish to better understand the symmetry between 
rotations in 3D space and the two dimentional complex 
vector space. 

We start our venture with defining a peculiar map from 
$\RR^3$ to $GL(\CC, 2)$. Consider $\vec{x} \in \RR^3$ 
and write $\vec x = (x_1, x_2, x_3)^T$. We define the three 
Pauli matricies as follows. 

\[
    \sigma_x := \twobytwoMat(0, 1, 1, 0) 
    \hspace{3mm}
    \sigma_y := \twobytwoMat(0, -i, i, 0)
    \hspace{3mm} 
    \sigma_z := \twobytwoMat(1, 0, 0, -1)
\]

Now, consider the following mapping. 
\[
    \vec x \rightarrow \sigma_x x_1 + \sigma_y x_2 + \sigma_y x_3
\]

In matrix form the mapping can be rephrased as follows. 
\[
    \begin{bmatrix}
        x_1 \\ x_2 \\ x_3
    \end{bmatrix}
    \rightarrow 
    \begin{bmatrix}
        x_3 & x_1 - ix_2 \\
        x_1 + ix_2 & -x_3
    \end{bmatrix}
\]

A nice property of this mapping is that the norm of 
$\vec{x}$ is the negative determinant of the mapped square matrix. 

\[
    \begin{vmatrix}
        x_3 & x_1 - ix_2 \\
        x_1 + ix_2 & -x_3
    \end{vmatrix}
    = 
    -x_3^2 - |x_1 - ix_2|^2 = - (x_1^2 + x_2^2 + x_3^2)
\]

For the rest of this paper, we wil refer to a 3 dimentional 
vector interchangably with its square matrix form. 

\new{Rotation Group in 3D and SO(3)}

Consider the set comprised of rotations in 3 dimensions. 
We first verify that this set is a group under composition. 
Combining any two rotations will result in another rotation, 
so the set is closed under composition. The identity 
rotation is preserving the original coordinate system as it is. 
Applying a different rotation after or before the "do nothing" 
rotation preserves the other rotation, so there exists an identity. 
 If a axis coordinate 
is rotated, it can be rotated back to the original coordinate 
system, so there exists an inverse. Indeed the set of 
all rotations form a group. 

In the context of describing the relationship between this 
rotation group and other groups, it is useful to establish 
a formal structure. Thus, we establish the group $SO(3)$, or the 
the orthogonal group of dimension 3. $SO(3)$ is a set of all 
3x3 matricies that satisfy 
\[
    A^\intercal A = A A^\intercal = I
\]
It is easy to verify that indeed $SO(3)$ is a group under 
matrix multiplication.  

Let $A$ be any element of $SO(3)$. We recognize that each of 
its colomn vectors have length 1. In the matrix expansion, 
the multiplication of the ith row and ith column must yield 1. 
Also, the dot product between any two distinct column vectors 
must yield zero. Hence, $A$ resembles some coordinate conversion 
from the original coordinate system to an arbitrary orthonormal 
coordinate system. 

Any rotation must correspond to an active change of basis 
from the original coordinate to some orthonormal coordinate. 
Hence, $SO(3)$ describes the group of rotations. 


\new{Group Action and Conjugation}

We establish some mathematical background. Let $G$ be a 
group, and $M$ be a set. Assume we are given some mapping 
$\cdot :G\times M \rightarrow M$ that has two properties. 
For all $m \in M$ and $g, f \in G$, 
\[
    (g f) \cdot m = g \cdot (f \cdot m)
\]
\[
    e \cdot m = m
\]
In words, the multiplication between elements in $G, M$ 
are defined so that associativity holds and identity is preserved. 
If the two properties hold, we say that $G$ is a 
\textbf{group action} on the set $M$. 

For any set element $m \in M$, there would be a set of 
elements that fix $m$. That is, $g\cdot m = m$. We call 
the set of element with this property the \textbf{isotropic group}. 
In symbols, 
\[
    G_m := \{g \in G| g \cdot m = g\}
\]
We leave it as an exercise for the reader to verify that 
the isotropic group is indeed a subgroup of $G$. 

A group action can be defined on the group $G$ itself. That is, 
the set $M$ can be $G$. We are interested in a particular action, 
\[
    \phi(g) a = g a g^{-1}
\]
where $g, a$ are both group elements of $G$. We use the 
symbol $\phi$ to differentiate between group action 
and the set.
This particular action is called 
\textbf{conjugation}. 
It might be the case that the element $a$ is not restricted 
to the group $G$

\new{Conjugation of SU(2) as a rotation}

Let set $M$ be the set of all  
unit vectors in $\RR^3$ represented in square matricies. 
Let G be the group $SU(2)$. This
is the group of complex 2x2 matricies that are unitary. In symbols, 
\[
    SU(2) := \{A \in \CC^{2X2}|A^\dagger A =AA^\dagger= I\}
\]
The binary operation is defined by matrix multiplication. 
Since $M$ is also a set of 2x2 matricies, it is natrual 
to define the operation between $M$ and 
$SU(2)$ as matrix multiplication. 

Define a group action by conjugation. 
\[
    \phi(U) M := U M U^{-1} 
\]



%Conjugacy mapping is an homomorphism
%This homomorphism is onto, or endomorphic
%Lagrange's Thm
%The kernel 

\end{document}