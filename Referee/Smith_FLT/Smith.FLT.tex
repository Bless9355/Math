\documentclass{article}
\usepackage{amsfonts}
\usepackage{amsthm}
\usepackage{amssymb}
\usepackage{amsmath}
\usepackage{graphicx}
\usepackage{subcaption}
\usepackage{xcolor}
\usepackage{mathtools}
\usepackage{ wasysym }
\usepackage{enumerate}
\usepackage{verbatim}


\newcommand{\new}[1]{
    \vspace{2mm}
    \noindent
    \textbf{
    \underline{#1}}
}

\def\calO{{\mathcal{O}}}
\def\th{{\theta}}
\def\_{{\hspace{1mm}}}
\def\<{{\langle}}
\def\>{{\rangle}}

\DeclarePairedDelimiter\bra{\langle}{\rvert}
\DeclarePairedDelimiter\ket{\lvert}{\rangle}
\DeclarePairedDelimiterX\braket[2]{\langle}{\rangle}{#1\,\delimsize\vert\,\mathopen{}#2}



\newcounter{problemcnt}
\setcounter{problemcnt}{0}

\newcommand{\Problem}{{
    \vspace{5mm}
    \stepcounter{problemcnt}
    \noindent
    \arabic{problemcnt}. 
}
}

\newcommand{\nProblem}[1]{
    \vspace{5mm}
    \noindent
    \setcounter{problemcnt}{#1}
    \arabic{problemcnt}. 
}


\newcommand{\Proof}{{
    \vspace{2mm}
    \noindent
    \textbf{
    \underline{Proof}}
}
}

\newcommand{\textOr}{
    {
        \hspace{5mm}
        \textrm{or}
        \hspace{5mm}
    }
}

\newcommand{\textAnd}{
    {
        \hspace{5mm}
        \textrm{and}
        \hspace{5mm}
    }
}


\newcommand{\textWhere}{
    {
        \hspace{5mm}
        \textrm{where}
        \hspace{5mm}
    }
}



\newcommand{\Ixp}[1]{
    {
        e^{i{#1}}
    }
}



\newcommand{\halfFigure}[1]{
\begin{center}
\includegraphics[width = .5\linewidth]{{#1}}
\end{center}
}

\newcommand{\fullFigure}[1]{
\begin{center}
\includegraphics[width = .9\linewidth]{{#1}}
\end{center}
}

\def\twobytwoMat(#1, #2, #3, #4){
    {
        \begin{bmatrix}
            {#1} & {#2}\
            {#3} & {#4}
        \end{bmatrix}
    }
}

\def\twobyoneMat(#1, #2){
    {
        \begin{bmatrix}
            {#1}\
            {#2}
        \end{bmatrix}
    }
}

\def\twobytwoDet(#1, #2, #3, #4){
    {
        \begin{vmatrix}
            {#1} & {#2}\
            {#3} & {#4}
        \end{vmatrix}
    }
}


\newcommand{\RR}{\mathbb{R}}
\newcommand{\CC}{\mathbb{C}}

\begin{document}
\begin{center}
    \Large
    \textbf{Report for Smith's proof of Fermat's Last Theorem}

    \large
    Daniel Son
\end{center}
\begin{comment}
The proof has a gap in page 2, equation (6). Smith claims that 
the inequality
\[
    (k+1)^n < (z_{y, k } + 1)^n
\]
implies the inequality 
\[
    y^n + (k+1)^n < (z_{y, k } + 1)^n
\]

Set $n = 3, z_{y, k} = 5, k = 4, y = 100$. Clearly the first 
statement holds. 
\[
    4^3 < 6^3
\]
Nonetheless, the second statement fails.
\[
    100^3 + 4^3 \nless 6^3 
\]\end{comment}

In page 2, from equations 7-10, Smith implicitly assumes 
that both $z_{y + 1, k}$ and $z_{y, k+1}$ are both in $\mathbb{Z}$.
In order to reach the conclusion $\alpha = \beta$, he assumes 
$z_{k, y} + \alpha , z_{k, y} + \beta \in \mathbb{Z}$.  
Thus, what Smith has proved is that there are no 
solutions to the system 
\begin{eqnarray*}
    (y+1)^n + k^n = x^n \\
    y^n + (k+1)^n = x^n
\end{eqnarray*}. 

Moreover, the implication of $\alpha = \beta$ is absurd. 
If the assumption holds, $z_{y+1,k} = z_{y, k}$. Note that $z_{y+1,k} \neq z_{y, k}$. This can verified 
by computing $z$ directly. We have an explicit formula 
for $z_{y, k}$.  
\[z_{y, k} = \sqrt[n]{y^n + k^n}\]

Compare $z_{10, 4}$ and $z_{11, 4}$ for $n = 5$
\halfFigure{zval.png}

And clearly the two values do not match. 





\hfill \lightning
\end{document}